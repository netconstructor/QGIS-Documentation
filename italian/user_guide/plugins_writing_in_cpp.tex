% vim: set textwidth=78 autoindent:

\section{Scrivere un plugin QGIS in C++}\label{cpp_plugin}

% when the revision of a section has been finalized, 
% comment out the following line:
% \updatedisclaimer

In questa sezione viene fornito un'esercitazione per principianti per scrivere un semplice plugin QGIS in C++. E' basata su un workshop tenuto dal Dr. Marco Hugentobler. 

I plugin QGIS C++ sono librerie collegate dinamicamente (.so or .dll). Sono legate a QGIS al runtime quando richiesto nel Gestore dei Plugin ed estendono la funzionalità di QGIS. hanno accesso al GUI di QGIS e possono essere distinti in core ed ed esterni.

Tecnicamente il Gestore dei Plugin QGIS cerca nella directory lib/qgis per tutti i file
.so e li carica quando è lanciato. Quando viene chiuso sono scaricati di nuovo, eccetto quelli con la casella vistata. Per i plugin caricati nuovi,il \method{classFactory} metodo crea un'istanza della classe del plugin e il \method{initGui} metodo del plugin è chiamato a mostrare gli elementi GUI nel menu plugin e nella barra degli strumenti. La funzione  \method{unload()} del plugin è usata per rimuovere gli elementi GUI allocati e la classe stessa del plugin è rimossa usando il distruttore di classe. Per elencare i plugin, ognuno deve avere alcune funzioni esterne 'C' per descrizione e ovviamente il metodo
\method{classFactory}.

\subsection{Perchè C++ e circa le licenze}

QGIS stesso è scritto in C++, così è sensato scrivere anche i plugin in C++. E' un linguaggio di programmazione orientato all'oggetto (object-oriented programming, OOP) che è visto da molti sviluppatori come un linguaggio preferenziale per creared applicativi su larga scala.

I plugins QGIS C++ usano funzionalità di librerie libqgis*.so. Dato che hanno licenze GNU GPL, anche i plugins QGIS C++  devono avere licenze GPL.
Questo significa che si possono usare i plugin per qualsiasi scopo e non si è costretti a pubblicarli. Se si vuole pubblicarli comunque, devono essere pubblicati sotto le condizinni della licenza GPL.

\subsection{Programmare un plugin QGIS C++ in 4 passi}

Il plugin di esempio è un convertitore di punti ed è intenzionalmente mantenuto semplice. Il plugin cerca lo layer vettoriale attivo in QGIS, converte tutti i vertici delle caratteristiche del layer a caratteristiche puntiformi mantenendo gli attributi e finalmente scrive le caratteristiche puntiformi in un file di testo delimitato. Il nuovo layer può quindi essere caricato in QGIS usando il plugin di testo delimitato (vedere la Sezione
\ref{label_dltext}).

\minisec{Passo 1: Far riconoscre il plugin al gestore dei plugin}

Come primo passo si creano i file \filename{QgsPointConverter.h} e
\filename{QgsPointConverter.cpp}. Poi si aggiungono metodi virtuali ereditati da QgisPlugin (ma si lasciano vuoti per ora), si creano i necessari metodi esterni 'C' e un file .pro, che è un meccanismo Qt per creare facilmente Makefiles.
Quindi si compila i sorgenti, si sposta la libreria compilata nell cartella plugin e si carica con il gestore dei plugin QGIS.

\textbf{a) Creare un nuovo file pointconverter.pro e aggiungere}:

\begin{verbatim}
#base directory of the qgis installation
QGIS_DIR = /home/marco/src/qgis

TEMPLATE = lib
CONFIG = qt
QT += xml qt3support
unix:LIBS += -L/$$QGIS_DIR/lib -lqgis_core -lqgis_gui
INCLUDEPATH += $$QGIS_DIR/src/ui $$QGIS_DIR/src/plugins  $$QGIS_DIR/src/gui \
	       $$QGIS_DIR/src/raster $$QGIS_DIR/src/core $$QGIS_DIR 
SOURCES = qgspointconverterplugin.cpp
HEADERS = qgspointconverterplugin.h
DEST = pointconverterplugin.so
DEFINES += GUI_EXPORT= CORE_EXPORT=
\end{verbatim}

\textbf{b) Creare un nuovo file qgspointconverterplugin.h e aggiungere}:

\begin{verbatim}
#ifndef QGSPOINTCONVERTERPLUGIN_H
#define QGSPOINTCONVERTERPLUGIN_H

#include "qgisplugin.h"

/**A plugin that converts vector layers to delimited text point files.
 The vertices of polygon/line type layers are converted to point features*/
class QgsPointConverterPlugin: public QgisPlugin
{
  public:
  QgsPointConverterPlugin(QgisInterface* iface);
  ~QgsPointConverterPlugin();
  void initGui();
  void unload();
  
  private:
  QgisInterface* mIface;
};
#endif
\end{verbatim}

\textbf{c) Creare un nuovo file qgspointconverterplugin.cpp e aggiungere}:

\begin{verbatim}
#include "qgspointconverterplugin.h"

#ifdef WIN32
#define QGISEXTERN extern "C" __declspec( dllexport )
#else
#define QGISEXTERN extern "C"
#endif

QgsPointConverterPlugin::QgsPointConverterPlugin(QgisInterface* iface): mIface(iface)
{
}

QgsPointConverterPlugin::~QgsPointConverterPlugin()
{
}

void QgsPointConverterPlugin::initGui()
{
}

void QgsPointConverterPlugin::unload()
{
}

QGISEXTERN QgisPlugin* classFactory(QgisInterface* iface)
{
  return new QgsPointConverterPlugin(iface);
}

QGISEXTERN QString name()
{
  return "point converter plugin";
}

QGISEXTERN QString description()
{
  return "A plugin that converts vector layers to delimited text point files";
}

QGISEXTERN QString version()
{
  return "0.00001";
}

// Return the type (either UI or MapLayer plugin)
QGISEXTERN int type()
{
  return QgisPlugin::UI;
}

// Delete ourself
QGISEXTERN void unload(QgisPlugin* theQgsPointConverterPluginPointer)
{
  delete theQgsPointConverterPluginPointer;
}
\end{verbatim}

\minisec{Passo 2: Creare un'icona, un pulsante e un menu per il plugin}

Questo passo include l'aggiunta di un puntatore all' oggetto QgisInterface nella classe plugins. Quindi si crea un QAction e una funzione di richiamo (slot), si aggiunge al 
QGIS GUI usando QgisIface::addToolBarIcon() and QgisIface::addPluginToMenu()
e alla fine si rimuove il QAction nel metodo \method{unload()}.

\textbf{d) Aprire qgspointconverterplugin.h di nuovo e estendere il contenuto esistente a}:

\begin{verbatim}
#ifndef QGSPOINTCONVERTERPLUGIN_H
#define QGSPOINTCONVERTERPLUGIN_H

#include "qgisplugin.h"
#include <QObject>

class QAction;

/**A plugin that converts vector layers to delimited text point files.
 The vertices of polygon/line type layers are converted to point features*/
class QgsPointConverterPlugin: public QObject, public QgisPlugin
{
  Q_OBJECT

 public:
  QgsPointConverterPlugin(QgisInterface* iface);
  ~QgsPointConverterPlugin();
  void initGui();
  void unload();
  
 private:
  QgisInterface* mIface;
  QAction* mAction;
  
   private slots:
   void convertToPoint();
};

#endif
\end{verbatim}

\textbf{e) Aprire qgspointconverterplugin.cpp di nuovo e estendere il contenuto esistente a}:

\begin{verbatim}
#include "qgspointconverterplugin.h"
#include "qgisinterface.h"
#include <QAction>

#ifdef WIN32
#define QGISEXTERN extern "C" __declspec( dllexport )
#else
#define QGISEXTERN extern "C"
#endif

QgsPointConverterPlugin::QgsPointConverterPlugin(QgisInterface* iface): \
    mIface(iface), mAction(0)
{

}

QgsPointConverterPlugin::~QgsPointConverterPlugin()
{

}

void QgsPointConverterPlugin::initGui()
{
  mAction = new QAction(tr("&Convert to point"), this);
  connect(mAction, SIGNAL(activated()), this, SLOT(convertToPoint()));
  mIface->addToolBarIcon(mAction);
  mIface->addPluginToMenu(tr("&Convert to point"), mAction);
}

void QgsPointConverterPlugin::unload()
{
  mIface->removeToolBarIcon(mAction);
  mIface->removePluginMenu(tr("&Convert to point"), mAction);
  delete mAction;
}

void QgsPointConverterPlugin::convertToPoint()
{
  qWarning("in method convertToPoint");
}

QGISEXTERN QgisPlugin* classFactory(QgisInterface* iface)
{
  return new QgsPointConverterPlugin(iface);
}

QGISEXTERN QString name()
{
  return "point converter plugin";
}

QGISEXTERN QString description()
{
  return "A plugin that converts vector layers to delimited text point files";
}

QGISEXTERN QString version()
{
  return "0.00001";
}

// Return the type (either UI or MapLayer plugin)
QGISEXTERN int type()
{
  return QgisPlugin::UI;
}

// Delete ourself
QGISEXTERN void unload(QgisPlugin* theQgsPointConverterPluginPointer)
{
  delete theQgsPointConverterPluginPointer;
}
\end{verbatim}


\minisec{Step 3: Read point features from the active layer and write to text file}

To read the point features from the active layer we need to query the current
layer and the location for the new text file. Then we iterate through all
features of the current layer, convert the geometries (vertices) to points,
open a new file and use QTextStream to write the x- and y-coordinates
into it.

\textbf{f) Open qgspointconverterplugin.h again and extend existing content to}

\begin{verbatim}
class QgsGeometry;
class QTextStream;

private:

void convertPoint(QgsGeometry* geom, const QString& attributeString, \
		  QTextStream& stream) const;
void convertMultiPoint(QgsGeometry* geom, const QString& attributeString, \
		  QTextStream& stream) const;
void convertLineString(QgsGeometry* geom, const QString& attributeString, \
		  QTextStream& stream) const;
void convertMultiLineString(QgsGeometry* geom, const QString& attributeString, \
		  QTextStream& stream) const;
void convertPolygon(QgsGeometry* geom, const QString& attributeString, \
		  QTextStream& stream) const;
void convertMultiPolygon(QgsGeometry* geom, const QString& attributeString, \
		  QTextStream& stream) const;
\end{verbatim}

\textbf{g) Open qgspointconverterplugin.cpp again and extend existing content to}:

\begin{verbatim}
#include "qgsgeometry.h"
#include "qgsvectordataprovider.h"
#include "qgsvectorlayer.h"
#include <QFileDialog>
#include <QMessageBox>
#include <QTextStream>

void QgsPointConverterPlugin::convertToPoint()
{
  qWarning("in method convertToPoint");
  QgsMapLayer* theMapLayer = mIface->activeLayer();
  if(!theMapLayer)
    {
      QMessageBox::information(0, tr("no active layer"), \
      tr("this plugin needs an active point vector layer to make conversions \ 
          to points"), QMessageBox::Ok);
      return;
    }
  QgsVectorLayer* theVectorLayer = dynamic_cast<QgsVectorLayer*>(theMapLayer);
  if(!theVectorLayer)
    {
      QMessageBox::information(0, tr("no vector layer"), \
      tr("this plugin needs an active point vector layer to make conversions \
          to points"), QMessageBox::Ok);
      return;
    }
  
  QString fileName = QFileDialog::getSaveFileName();
  if(!fileName.isNull())
    {
      qWarning("The selected filename is: " + fileName);
      QFile f(fileName);
      if(!f.open(QIODevice::WriteOnly))
      {
	QMessageBox::information(0, "error", "Could not open file", QMessageBox::Ok);
	return;
      }
      QTextStream theTextStream(&f);
      theTextStream.setRealNumberNotation(QTextStream::FixedNotation);

      QgsFeature currentFeature;
      QgsGeometry* currentGeometry = 0;

      QgsVectorDataProvider* provider = theVectorLayer->dataProvider();
      if(!provider)
      {
          return;
      }

      theVectorLayer->select(provider->attributeIndexes(), \
      theVectorLayer->extent(), true, false);

      //write header
      theTextStream << "x,y";
      theTextStream << endl;

      while(theVectorLayer->nextFeature(currentFeature))
      {
	 QString featureAttributesString;
      
        currentGeometry = currentFeature.geometry();
        if(!currentGeometry)
        {
            continue;
        }

        switch(currentGeometry->wkbType())
        {
            case QGis::WKBPoint:
            case QGis::WKBPoint25D:
                convertPoint(currentGeometry, featureAttributesString, \
		theTextStream);
                break;

            case QGis::WKBMultiPoint:
            case QGis::WKBMultiPoint25D:
                convertMultiPoint(currentGeometry, featureAttributesString, \
		theTextStream);
                break;

            case QGis::WKBLineString:
            case QGis::WKBLineString25D:
                convertLineString(currentGeometry, featureAttributesString, \
		theTextStream);
                break;

            case QGis::WKBMultiLineString:
            case QGis::WKBMultiLineString25D:
                convertMultiLineString(currentGeometry, featureAttributesString \
		theTextStream);
                break;

            case QGis::WKBPolygon:
            case QGis::WKBPolygon25D:
                convertPolygon(currentGeometry, featureAttributesString, \
		theTextStream);
                break;

            case QGis::WKBMultiPolygon:
            case QGis::WKBMultiPolygon25D:
                convertMultiPolygon(currentGeometry, featureAttributesString, \
		theTextStream);
                break;
        }
      }
    }
}

//geometry converter functions
void QgsPointConverterPlugin::convertPoint(QgsGeometry* geom, const QString& \
attributeString, QTextStream& stream) const
{
    QgsPoint p = geom->asPoint();
    stream << p.x() << "," << p.y();
    stream << endl;
}

void QgsPointConverterPlugin::convertMultiPoint(QgsGeometry* geom, const QString& \
attributeString, QTextStream& stream) const
{
    QgsMultiPoint mp = geom->asMultiPoint();
    QgsMultiPoint::const_iterator it = mp.constBegin();
    for(; it != mp.constEnd(); ++it)
    {
        stream << (*it).x() << "," << (*it).y();
        stream << endl;
    }
}

void QgsPointConverterPlugin::convertLineString(QgsGeometry* geom, const QString& \
attributeString, QTextStream& stream) const
{
    QgsPolyline line = geom->asPolyline();
    QgsPolyline::const_iterator it = line.constBegin();
    for(; it != line.constEnd(); ++it)
    {
        stream << (*it).x() << "," << (*it).y();
        stream << endl;
    }
}

void QgsPointConverterPlugin::convertMultiLineString(QgsGeometry* geom, const QString& \
attributeString, QTextStream& stream) const
{
    QgsMultiPolyline ml = geom->asMultiPolyline();
    QgsMultiPolyline::const_iterator lineIt = ml.constBegin();
    for(; lineIt != ml.constEnd(); ++lineIt)
    {
        QgsPolyline currentPolyline = *lineIt;
        QgsPolyline::const_iterator vertexIt = currentPolyline.constBegin();
        for(; vertexIt != currentPolyline.constEnd(); ++vertexIt)
        {
            stream << (*vertexIt).x() << "," << (*vertexIt).y();
            stream << endl;
        }
    }
}

void QgsPointConverterPlugin::convertPolygon(QgsGeometry* geom, const QString& \
attributeString, QTextStream& stream) const
{
    QgsPolygon polygon = geom->asPolygon();
    QgsPolygon::const_iterator it = polygon.constBegin();
    for(; it != polygon.constEnd(); ++it)
    {
        QgsPolyline currentRing = *it;
        QgsPolyline::const_iterator vertexIt = currentRing.constBegin();
        for(; vertexIt != currentRing.constEnd(); ++vertexIt)
        {
            stream << (*vertexIt).x() << "," << (*vertexIt).y();
            stream << endl;
        }
    }
}

void QgsPointConverterPlugin::convertMultiPolygon(QgsGeometry* geom, const QString& \
attributeString, QTextStream& stream) const
{
    QgsMultiPolygon mp = geom->asMultiPolygon();
    QgsMultiPolygon::const_iterator polyIt = mp.constBegin();
    for(; polyIt != mp.constEnd(); ++polyIt)
    {
        QgsPolygon currentPolygon = *polyIt;
        QgsPolygon::const_iterator ringIt = currentPolygon.constBegin();
        for(; ringIt != currentPolygon.constEnd(); ++ringIt)
        {
            QgsPolyline currentPolyline = *ringIt;
            QgsPolyline::const_iterator vertexIt = currentPolyline.constBegin();
            for(; vertexIt != currentPolyline.constEnd(); ++vertexIt)
            {
                stream << (*vertexIt).x() << "," << (*vertexIt).y();
                stream << endl;
            }
        }
    }
}
\end{verbatim}

\minisec{Step 4: Copy the feature attributes to the text file}

At the end we extract the attributes from the active layer using
QgsVectorDataProvider::fieldNameMap(). For each feature we extract the field
values using QgsFeature::attributeMap() and add the contents comma separated
behind the x- and y-coordinates for each new point feature. For this step
there is no need for any furter change in \filename{qgspointconverterplugin.h} 

\textbf{h) Open qgspointconverterplugin.cpp again and extend existing content
to}:

\begin{verbatim} 
#include "qgspointconverterplugin.h"
#include "qgisinterface.h"
#include "qgsgeometry.h"
#include "qgsvectordataprovider.h"
#include "qgsvectorlayer.h"
#include <QAction>
#include <QFileDialog>
#include <QMessageBox>
#include <QTextStream>

#ifdef WIN32
#define QGISEXTERN extern "C" __declspec( dllexport )
#else
#define QGISEXTERN extern "C"
#endif

QgsPointConverterPlugin::QgsPointConverterPlugin(QgisInterface* iface): \
mIface(iface), mAction(0)
{

}

QgsPointConverterPlugin::~QgsPointConverterPlugin()
{

}

void QgsPointConverterPlugin::initGui()
{
  mAction = new QAction(tr("&Convert to point"), this);
  connect(mAction, SIGNAL(activated()), this, SLOT(convertToPoint()));
  mIface->addToolBarIcon(mAction);
  mIface->addPluginToMenu(tr("&Convert to point"), mAction);
}

void QgsPointConverterPlugin::unload()
{
  mIface->removeToolBarIcon(mAction);
  mIface->removePluginMenu(tr("&Convert to point"), mAction);
  delete mAction;
}

void QgsPointConverterPlugin::convertToPoint()
{
  qWarning("in method convertToPoint");
  QgsMapLayer* theMapLayer = mIface->activeLayer();
  if(!theMapLayer)
    {
      QMessageBox::information(0, tr("no active layer"), \
      tr("this plugin needs an active point vector layer to make conversions \
          to points"), QMessageBox::Ok);
      return;
    }
  QgsVectorLayer* theVectorLayer = dynamic_cast<QgsVectorLayer*>(theMapLayer);
  if(!theVectorLayer)
    {
      QMessageBox::information(0, tr("no vector layer"), \
      tr("this plugin needs an active point vector layer to make conversions \
          to points"), QMessageBox::Ok);
      return;
    }
  
  QString fileName = QFileDialog::getSaveFileName();
  if(!fileName.isNull())
    {
      qWarning("The selected filename is: " + fileName);
      QFile f(fileName);
      if(!f.open(QIODevice::WriteOnly))
      {
	QMessageBox::information(0, "error", "Could not open file", QMessageBox::Ok);
	return;
      }
      QTextStream theTextStream(&f);
      theTextStream.setRealNumberNotation(QTextStream::FixedNotation);

      QgsFeature currentFeature;
      QgsGeometry* currentGeometry = 0;

      QgsVectorDataProvider* provider = theVectorLayer->dataProvider();
      if(!provider)
      {
          return;
      }

      theVectorLayer->select(provider->attributeIndexes(), \
      theVectorLayer->extent(), true, false);

      //write header
      theTextStream << "x,y";
      QMap<QString, int> fieldMap = provider->fieldNameMap();
      //We need the attributes sorted by index.
      //Therefore we insert them in a second map where key / values are exchanged
      QMap<int, QString> sortedFieldMap;
      QMap<QString, int>::const_iterator fieldIt = fieldMap.constBegin();
      for(; fieldIt != fieldMap.constEnd(); ++fieldIt)
      {
        sortedFieldMap.insert(fieldIt.value(), fieldIt.key());
      }

      QMap<int, QString>::const_iterator sortedFieldIt = sortedFieldMap.constBegin();
      for(; sortedFieldIt != sortedFieldMap.constEnd(); ++sortedFieldIt)
      {
          theTextStream << "," << sortedFieldIt.value();
      }

      theTextStream << endl;

      while(theVectorLayer->nextFeature(currentFeature))
      {
        QString featureAttributesString;
         const QgsAttributeMap& map = currentFeature.attributeMap();
         QgsAttributeMap::const_iterator attributeIt = map.constBegin();
         for(; attributeIt != map.constEnd(); ++attributeIt)
         {
            featureAttributesString.append(",");
            featureAttributesString.append(attributeIt.value().toString());
         }


        currentGeometry = currentFeature.geometry();
        if(!currentGeometry)
        {
            continue;
        }

        switch(currentGeometry->wkbType())
        {
            case QGis::WKBPoint:
            case QGis::WKBPoint25D:
                convertPoint(currentGeometry, featureAttributesString, \
		theTextStream);
                break;

            case QGis::WKBMultiPoint:
            case QGis::WKBMultiPoint25D:
                convertMultiPoint(currentGeometry, featureAttributesString, \
		theTextStream);
                break;

            case QGis::WKBLineString:
            case QGis::WKBLineString25D:
                convertLineString(currentGeometry, featureAttributesString, \
		theTextStream);
                break;

            case QGis::WKBMultiLineString:
            case QGis::WKBMultiLineString25D:
                convertMultiLineString(currentGeometry, featureAttributesString \
		theTextStream);
                break;

            case QGis::WKBPolygon:
            case QGis::WKBPolygon25D:
                convertPolygon(currentGeometry, featureAttributesString, \
		theTextStream);
                break;

            case QGis::WKBMultiPolygon:
            case QGis::WKBMultiPolygon25D:
                convertMultiPolygon(currentGeometry, featureAttributesString, \
		theTextStream);
                break;
        }
      }
    }
}

//geometry converter functions
void QgsPointConverterPlugin::convertPoint(QgsGeometry* geom, const QString& \
attributeString, QTextStream& stream) const
{
    QgsPoint p = geom->asPoint();
    stream << p.x() << "," << p.y();
    stream << attributeString;
    stream << endl;
}

void QgsPointConverterPlugin::convertMultiPoint(QgsGeometry* geom, const QString& \
attributeString, QTextStream& stream) const
{
    QgsMultiPoint mp = geom->asMultiPoint();
    QgsMultiPoint::const_iterator it = mp.constBegin();
    for(; it != mp.constEnd(); ++it)
    {
        stream << (*it).x() << "," << (*it).y();
        stream << attributeString;
        stream << endl;
    }
}

void QgsPointConverterPlugin::convertLineString(QgsGeometry* geom, const QString& \
attributeString, QTextStream& stream) const
{
    QgsPolyline line = geom->asPolyline();
    QgsPolyline::const_iterator it = line.constBegin();
    for(; it != line.constEnd(); ++it)
    {
        stream << (*it).x() << "," << (*it).y();
        stream << attributeString;
        stream << endl;
    }
}

void QgsPointConverterPlugin::convertMultiLineString(QgsGeometry* geom, const QString& \
attributeString, QTextStream& stream) const
{
    QgsMultiPolyline ml = geom->asMultiPolyline();
    QgsMultiPolyline::const_iterator lineIt = ml.constBegin();
    for(; lineIt != ml.constEnd(); ++lineIt)
    {
        QgsPolyline currentPolyline = *lineIt;
        QgsPolyline::const_iterator vertexIt = currentPolyline.constBegin();
        for(; vertexIt != currentPolyline.constEnd(); ++vertexIt)
        {
            stream << (*vertexIt).x() << "," << (*vertexIt).y();
            stream << attributeString;
            stream << endl;
        }
    }
}

void QgsPointConverterPlugin::convertPolygon(QgsGeometry* geom, const QString& \
attributeString, QTextStream& stream) const
{
    QgsPolygon polygon = geom->asPolygon();
    QgsPolygon::const_iterator it = polygon.constBegin();
    for(; it != polygon.constEnd(); ++it)
    {
        QgsPolyline currentRing = *it;
        QgsPolyline::const_iterator vertexIt = currentRing.constBegin();
        for(; vertexIt != currentRing.constEnd(); ++vertexIt)
        {
            stream << (*vertexIt).x() << "," << (*vertexIt).y();
            stream << attributeString;
            stream << endl;
        }
    }
}

void QgsPointConverterPlugin::convertMultiPolygon(QgsGeometry* geom, const QString& \
attributeString, QTextStream& stream) const
{
    QgsMultiPolygon mp = geom->asMultiPolygon();
    QgsMultiPolygon::const_iterator polyIt = mp.constBegin();
    for(; polyIt != mp.constEnd(); ++polyIt)
    {
        QgsPolygon currentPolygon = *polyIt;
        QgsPolygon::const_iterator ringIt = currentPolygon.constBegin();
        for(; ringIt != currentPolygon.constEnd(); ++ringIt)
        {
            QgsPolyline currentPolyline = *ringIt;
            QgsPolyline::const_iterator vertexIt = currentPolyline.constBegin();
            for(; vertexIt != currentPolyline.constEnd(); ++vertexIt)
            {
                stream << (*vertexIt).x() << "," << (*vertexIt).y();
                stream << attributeString;
                stream << endl;
            }
        }
    }
}

QGISEXTERN QgisPlugin* classFactory(QgisInterface* iface)
{
  return new QgsPointConverterPlugin(iface);
}

QGISEXTERN QString name()
{
  return "point converter plugin";
}

QGISEXTERN QString description()
{
  return "A plugin that converts vector layers to delimited text point files";
}

QGISEXTERN QString version()
{
  return "0.00001";
}

// Return the type (either UI or MapLayer plugin)
QGISEXTERN int type()
{
  return QgisPlugin::UI;
}

// Delete ourself
QGISEXTERN void unload(QgisPlugin* theQgsPointConverterPluginPointer)
{
  delete theQgsPointConverterPluginPointer;
}

\end{verbatim}

\subsection{Further information}

As you can see, you need information from different sources to write QGIS C++
plugins. Plugin writers need to know C++, the QGIS plugin interface as
well as Qt4 classes and tools. At the beginning it is best to learn from
examples and copy the mechanism of existing plugins. 

There is a a collection of online documentation that may be usefull for
QGIS C++ programers:

\begin{itemize}
\item QGIS Plugin Debugging: \url{http://wiki.qgis.org/qgiswiki/DebuggingPlugins}
\item QGIS API Documentation: \url{http://svn.qgis.org/api_doc/html/}
\item Qt documentation: \url{http://doc.trolltech.com/4.3/index.html}
\end{itemize}

