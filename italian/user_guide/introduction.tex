%  !TeX  root  =  user_guide.tex
\pagestyle{scrheadings}
\chapter{Introduzione al GIS}\label{label_intro}

% when the revision of a section has been finalized, 
% comment out the following line:
%\updatedisclaimer

Un Sistema Informativo Geografico (Geographical Information System,
GIS)(\cite{mitchel05}\footnote{Questo capitolo è stato scritto da Tyler
Mitchell (\url{http://www.oreillynet.com/pub/wlg/7053}) e usato con Licenza Creative Commons. Tyler è l'autore di 
\textit{Web Mapping Illustrated}, pubblicato da O'Reilly, 2005.})
è un insieme di programmi che permettono di creare, visualizzare,
interrogare e analizzare dati geospaziali. I dati geospaziali riportano
informazioni inerenti la posizione geografica di un oggetto. Questo
spesso implica l'uso di coordinate geografiche, quali valori di latitudine
e longitudine. Dato spaziale è un altro termine comunemente usato,
così come lo sono: dato geografico, dato GIS, mappa, location,
coordinate e geometrie spaziali.

Le applicazioni che usano dati geospaziali eseguono varie funzioni.
Quella più conosciuta e facilmente compresa è la produzione di mappe.
I programmi per la realizzazione di mappe impiegano i dati geospaziali
e li rappresentano in una forma che sia visibile, tipicamente su uno
schermo o stampati su carta. Si possono presentare mappe statiche
(una semplice immagine) o mappe dinamiche che sono personalizzate
dall'utente che ne usufruisce attraverso un'applicazione desktop o una pagina web.

Molte persone ritengono erroneamente che le applicazioni geospaziali
si riducano alla produzione di mappe, ma l'analisi geospaziale del
dato è un'altra primaria funzione di tali programmi. Alcune tipiche
analisi che possono essere condotte sono:

\begin{enumerate}
\item distanze tra posizioni geografiche
\item misurazione dell'area (ad. es. metri quadrati) di una certa regione
geografica
\item quali elementi geografici si sovrappongono ad altri
\item determinazione dell'entità della sovrapposizione tra elementi
\item il numero di posizioni comprese entro una certa distanza da un'altra
\item e così via... 
\end{enumerate}

Tutto ciò può apparire riduttivo, eppure può essere applicato in innumerevoli
maniere nelle più disparate discipline. Il risultato dell'analisi può essere
mostrato su una mappa, ma è spesso tabulato in un report a supporto
di decisioni gestionali.

La recente diffusione di servizi basati sulla posizione lascia intravedere
l'introduzione di ogni sorta di ulteriori caratteristiche, ma molte
di esse saranno basate sulla combinazione di mappatura e analisi.
Per esempio i cellulari tracciano la loro posizione geografica. Con
un programma adatto, il telefono può dire che tipo di ristoranti possono
essere raggiunti a piedi dalla propria posizione entro una certa distanza.
Per quanto questa sia un applicazione recentissima della tecnologia geospaziale,
è essenzialmente basata su analisi geospaziale del dato e presentazione
dei risultati.

\section{Perché tutto questo è una novità?}\label{label_whynew}

Bene, in realtà non è così. Ci sono molti nuovi dispositivi hardware
che consentono servizi geospaziali in modalità mobile. Sono disponibili
anche molti programmi geospaziali open source, ma l'esistenza di hardware
e software finalizzati all'uso del dato geospaziale non sono per niente una
novità. I ricevitori GPS (Global Positioning System, GPS) sono diventati
molto comuni, ma sono usati in diverse industrie da più di un decennio.
In maniera analoga, programmi per la mappatura e l'analisi sono stati
un preminente mercato commerciale, focalizzato principalmente su industrie
nell'ambito della gestione delle risorse naturali.

Quello che è nuovo è il modo in cui le recenti tecnologie hardware e software 
sono usate e da chi sono usate. Gli utenti tradizionali di applicazioni
per la mappatura e l'analisi erano analisti GIS fortemente specializzati
o tecnici della mappatura digitale istruiti ad usare strumenti tipo
CAD. Attualmente le capacità di calcolo dei personal computer e il software
open source (Open Source Software, OSS) hanno consentito ad un esercito
di hobbysti, professionisti, sviluppatori web, ecc. d'interagire
con il dato geospaziale. La curva di apprendimento si è abbassata.
I costi sono diminuiti. La quantità di tecnologia geospaziale è cresciuta.

Come sono archiviati i dati geospaziali? In estrema sintesi, attualmente ci sono
due tipi di dati geospaziali di uso comune. Ciò in aggiunta
al tradizionale dato in forma tabellare, che è comunque diffusamente
impiegato dalle applicazioni geospaziali.

\subsection{Dati raster}\label{label_rasterdata}

Uno dei tipi di dato geospaziale è detto dato raster o semplicemente "raster": 
esempi di dati raster sono le immagini satellitari o le foto aeree. 
Le ombreggiature altimetriche o i modelli digitali di terreno (Digital Elevation Model, DEM)
sono anch'essi rappresentati come dati raster. Qualunque elemento di mappa può essere
rappresentato come dato raster, ma ci sono delle limitazioni.

Un raster è una griglia regolare fatta di celle o, nel caso di semplici
immagini, di pixels. Ha un numero fisso di righe e colonne.
Ogni cella ha un valore numerico e una certa dimensione geografica
(ad es. 30x30 metri).

Più raster sovrapposti sono utilizzati per rappresentare immagini che utilizzano
più di un colore. Anche le immagini satellitari sono un esempio
di dati in "bande" multiple. Ogni banda è essenzialmente un livello sovrapposto
al precedente dove vengono salvati i valori della lunghezza della luce. Come è
facile immaginare, un raster di grosse dimensioni occupa maggiore spazio su disco.
Un raster con celle piccole può fornire maggior dettaglio ma richiede anche più spazio.
L'abilità sta nel trovare il giusto compromesso tra la dimensione della cella
ai fini dell'archiviazione e la dimensione della cella ai fini analitici
o di mappatura.

\subsection{Dati vettoriali}\label{label_vectordata}

Anche i dati vettoriali vengono usati nelle applicazioni geospaziali.
Chi ha seguito corsi di trigonometria o di geometria dovrebbe già essere a conoscenza dei vettori.
In termini semplici, i vettori sono un metodo per descrivere
una posizione utilizzando un insieme di coordinate. Ogni coordinata
si riferisce ad una posizione geografica utilizzando un sistema di valori y e x.

Si può pensare a tutto ciò in riferimento ad un piano cartesiano,
i diagrammi di scuola che mostravano un asse x e un asse y. Potreste
averli usati per diagrammare risparmi decrescenti per la pensione
o interessi crescenti per il mutuo, ma gli stessi concetti sono alla
base dell'analisi e della mappatura del dato geospaziale.

Ci sono vari modi per rappresentare le coordinate: questa è l'area di studio dei
sistemi di proiezione cartografiche.

I dati vettoriali si presentano in tre forme, di complessità crescente
e basate sulla forma precedente.

\begin{enumerate}
\item Punti - Una coppia di coordinate (x y) rappresenta una precisa posizione geografica
\item Linee - Coordinate multiple (x1 y1, x2 y2, x3 y4, ... xn yn) collegate
tra loro in un certo ordine. Equivale a disegnare una linea dal punto (x1 y1)
al punto (x2 y2) e così via. Queste parti fra ogni punto sono considerate segmenti.
Hanno una lunghezza ed ad essi si può attribuire una direzione basata sull'ordine
dei punti. Tecnicamente, una linea è data da una singola coppia di coordinate (x1,y1; x2,y2)
collegate insieme; una polilinea è costituita da linee multiple collegate insieme.
\item Poligoni - Quando le linee sono collegate tra loro da più di due punti,
con l'ultimo punto coincidente con il primo, viene definito un poligono.
Triangoli, cerchi, rettangoli ecc. ecc. sono tutti poligoni. La caratteristica
principale dei poligoni è che essi racchiudono un'area. 
\end{enumerate}
