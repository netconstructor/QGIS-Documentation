\section{Moduli degli strumenti GRASS}\label{appdx_grass_toolbox_modules}

% when the revision of a section has been finalized, 
% comment out the following line:
% \updatedisclaimer

La shell di GRASS avviabile dagli strumenti GRASS fornisce accesso a praticamente tutti (oltre 300) moduli GRASS in modalità riga di comando. Per offrire un ambiente di lavoro più confortevole per l'utente, all'incirca 200 di questi moduli e funzionalità sono disponibili con interfaccia grafica.

\subsection{Moduli negli strumenti GRASS per l'importazione e l'esportazione di dati}\index{GRASS!toolbox!modules}

Questa sezione elenca tutti i moduli con interfaccia grafica presenti tra gli strumenti GRASS per importare ed esportare dati nella location e mapset impostato.

\begin{table}[ht]
\centering
\caption{Strumenti GRASS: Moduli per l'importazione di dati}\medskip
 \begin{tabular}{|p{4cm}|p{12cm}|}
  \hline \multicolumn{2}{|c|}{\textbf{Moduli per l'importazione di dati tra gli strumenti GRASS}} \\ 
  \hline \textbf{Nome del modulo} & \textbf{Scopo} \\
  \hline r.in.arc & Converte un file raster in formato ESRI ARC/INFO ascii (GRID) in un livello raster (binario)\\
  \hline r.in.ascii & Converte un file raster in formato testo ASCII in un livello raster (binario) \\
  \hline r.in.aster & Georeferenzia, rettifica, e importa immagini Terra-ASTER imagery e relativi DEM usando il comango gdalwarp \\
  \hline r.in.gdal &  Importa raster supportati da GDAL in formato raster binario GRASS \\
  \hline r.in.gdal.loc &  Importa raster supportati da GDAL in formato raster binario GRASS e crea una location su misura per la visualizzazione del dato \\
  \hline r.in.gridatb & Importa file GRIDATB.FOR (TOPMODEL) in formato raster GRASS \\
  \hline r.in.mat  & Importa file binari MAT-File(v4) in formato raster GRASS  \\
  \hline r.in.poly  &  Crea mappe raster da file ascii descriventi poligoni/linee presenti nella directory attuale \\
  \hline r.in.srtm  & Importa file SRTM HGT in GRASS \\
  \hline i.in.spotvgt & Importa file SPOT VGT NDVI in mappe raster \\
  \hline v.in.dxf & Importa livelli vettoriali in formato DXF \\
  \hline v.in.e00 & Importa livelli vettoriali ESRI E00 \\
  \hline v.in.garmin & Importa vettori da unità/formati gps usando il comando gpstrans \\
  \hline v.in.gpsbabel & Importa vettori da unità/formati gps usando il comando gpsbabel \\
  \hline v.in.mapgen & Importa vettoriali in formato MapGen or MatLab in GRASS \\
  \hline v.in.ogr & Importa livelli vettoriali OGR/PostGIS \\
  \hline v.in.ogr.loc & Importa livelli vettoriali OGR/PostGIS e crea una location su misura per la visualizzazione dei dati \\
  \hline v.in.ogr.all & Importa tutti i livelli vettoriali OGR/PostGIS presenti in una certa sorgente dati \\
  \hline v.in.ogr.all.loc & Importa tutti i livelli vettoriali OGR/PostGIS presenti in una certa sorgente dati e crea una location su misura per la visualizzazione del dato \\
\hline
\end{tabular}
\end{table}

\begin{table}[ht]
\centering
\caption{Strumenti GRASS: Moduli per per l'esportazione di dati}\medskip
 \begin{tabular}{|p{4cm}|p{12cm}|}
  \hline \multicolumn{2}{|c|}{\textbf{Moduli per l'esportazione di dati tra gli strumenti GRASS}} \\ 
  \hline \textbf{Nome modulo} & \textbf{Scopo} \\
  \hline r.out.gdal.gtiff & Esporta livelli raster in formato Geo TIFF \\
  \hline r.out.arc & Converte un raster in un file ESRI ARCGRID \\
  \hline r.gridatb & Esporta raster GRASS in file GRIDATB.FOR (TOPMODEL) \\
  \hline r.out.mat & Esporta un raster GRASS verso il formato binario MAT-File \\
  \hline r.out.bin & Esporta un raster GRASS in una matrice binaria \\
  \hline r.out.png & Esporta raster GRASS come immagini non georeferenziate in formato immagine PNG \\
  \hline r.out.ppm & Converte una mappa raster GRASS in formato immagine PPM alla risoluzione impostata per la region in GRASS \\
  \hline r.out.ppm3 & Converte 3 livelli raster (R,G,B) GRASS in formato immagine PPM alla risoluzione impostata per la region in GRASS \\
  \hline r.out.pov & Converte un raster in un campo di altezze per POVRAY\\
  \hline r.out.tiff & Esporta una mappa raster GRASS verso un'immagine in formato TIFF a 8/24bit alla risoluzione impostata per la region in GRASS \\
  \hline r.out.vrml &  Esporta una mappa raster in formato Virtual Reality Modeling Language (VRML)\\
  \hline v.out.ogr & Esporta layer vettoriali in vari formati supportati dalla libreria OGR \\
  \hline v.out.ogr.gml & Esporta layer vettoriali verso il formato GML \\
  \hline v.out.ogr.postgis & Esporta layer vettoriali verso vari formati (tramite la libreria OGR) \\
  \hline v.out.ogr.mapinfo & Esporta livelli vettoriali verso il formato Mapinfo \\
  \hline v.out.ascii & Converte un raster binario di GRASS verso il formato vettoriale GRASS ASCII  \\
  \hline v.out.dxf & Converte un vettoriale GRASS verso il formato DXF  \\
\hline
\end{tabular}
\end{table}

\subsection{Moduli negli strumenti GRASS per la conversione tra tipi di dato}

Questa Sezione elenca tutti i moduli con interfaccia grafica negli strumenti GRASS per convertire dati raster a vettoriali e viceversa nella location e mapset GRASS selezionati.

\begin{table}[ht]
\centering
\caption{Strumenti GRASS: moduli per la conversione tra tipi di dato}\medskip
 \begin{tabular}{|p{4cm}|p{12cm}|}
  \hline \multicolumn{2}{|c|}{\textbf{Moduli per la conversione tra tipi di dato tra gli strumenti GRASS}} \\
  \hline \textbf{Nome modulo} & \textbf{Scopo} \\
  \hline r.to.vect.point & Converte un raster in un vettoriale di punti \\
  \hline r.to.vect.line & Converte un raster in un vettoriale di linee \\
  \hline r.to.vect.area & Converte un raster in un vettoriale di aree \\
  \hline v.to.rast.constant & Converte un vettore ad un raster usando un valore costante per le categorie\\
  \hline v.to.rast.attr & Converte un vettore in un raster  usando gli attributi per le categorie \\
\hline
\end{tabular}
\end{table}

\subsection{Moduli negli strumenti GRASS per la definizione della regione e l'impostazione della proiezione}

Questa Sezione elenca tutti i moduli con interfaccia grafica negli strumenti GRASS per la gestione e la modifica della regione del mapset attuale e per configurare la proiezione.

\begin{table}[ht]
\centering
\caption{Strumenti GRASS: moduli per la gestione della regione e della proiezione}\medskip
 \begin{tabular}{|p{4cm}|p{12cm}|}
  \hline \multicolumn{2}{|c|}{\textbf{Moduli per la gestione della regione e della proiezione tra gli strumenti GRASS}} \\
  \hline \textbf{Nome modulo} & \textbf{Scopo} \\
  \hline g.region.save & Salva la regione corrente con nome \\
  \hline g.region.zoom & Restringi la regione corrente fino a quando non incontra celle non vuote di una data mappa raster \\
  \hline g.region.multiple.raster & Imposta la regione in modo che combaci con l'estensione di più raster \\
  \hline g.region.multiple.vector & Imposta la regione in modo che combaci con l'estensione di più vettoriali \\
  \hline g.proj.print & Stampa le informazioni di proiezione per la location corrente \\
  \hline g.proj.geo & Stampa le informazioni di proiezione di un file georeferenziato (raster, vettore o immagine)\\
  \hline g.proj.ascii.new & Stampa le informazioni di proiezione di un file ASCII georeferenziato contenente la descrizione della proiezione in formato WKT \\
  \hline g.proj.proj & Stampa le informazioni di proiezione da un file descrittivo di proiezione in formato PROJ.4 \\
  \hline g.proj.ascii.new & Stampa informazioni di proiezione da un file ASCII georeferenziato contenente la descrizione della proiezione in formato WKT e crea una nuova location basata su queste informazioni e sull'estensione di questo file \\
  \hline g.proj.geo.new & Stampa le informazioni di proiezione di un file georeferenziato (raster, vettore o immagine) e crea una nuova location basata su queste informazioni e sull'estensione di questo file \\
  \hline g.proj.proj.new & Stampa le informazioni di proiezione da un file descrittivo di proiezione in formato PROJ.4 e crea una nuova location basata su queste informazioni e sull'estensione di questo file \\
  \hline m.cogo & Una semplice utility per convertire misure di direzione e distanze in coordinate e viceversa. Viene assunto un sistema di coordinate cartesiano \\
\hline
\end{tabular}
\end{table}

\clearpage

\subsection{Moduli negli strumenti GRASS per le operazioni su dati raster}

Questa Sezione elenca tutti i moduli con interfaccia grafica negli strumenti GRASS per l'effettuazione di operazioni su dati raster salvati nella location e nel mapset selezionato.

\begin{table}[ht]
\centering
\caption{Strumenti GRASS: moduli per le operazioni su dati raster}\medskip
 \begin{tabular}{|p{4cm}|p{12cm}|}
  \hline \multicolumn{2}{|c|}{\textbf{Moduli per le operazioni su dati raster tra gli strumenti GRASS}} \\
  \hline \textbf{Nome modulo} & \textbf{Scopo} \\
  \hline r.compress & Comprime e decomprime mappe raster \\
  \hline r.region.region & Imposta i limiti della regione di un raster a quella di default \\
  \hline r.region.raster & Imposta i limiti della regione di un raster a quelli dell'estensione di una mappa raster esistente \\
  \hline r.region.vector & Imposta i limiti della regione di un raster a quelli di una mappa vettoriale esistente \\
  \hline r.region.edge & Imposta i limiti della regione di un raster ai valori impostati (n-s-e-w) \\
  \hline r.region.alignTo & Imposta la regione del raster in modo che abbia la risoluzione spaziale e i limiti di una mappa di riferimento indicata \\
  \hline r.null.val & Trasforma il valore di cella specificato in valore nullo \\
  \hline r.null.to & Assegna il valore specificato alle celle nulle \\
  \hline r.quant & Algoritmo per produrre il file dei quantili di una mappa con valori a virgola mobile \\
  \hline r.resamp.stats & Ricampiona una mappa raster usando il metodo dell'aggregazione statistica \\
  \hline r.resamp.interp & Ricampiona una mappa raster per mezzo dell'interpolazione dei valori \\
  \hline r.resample & Ricampiona una mappa raster verso una nuova risoluzione spaziale precedentemente impostata \\
  \hline r.resamp.rst & Reinterpola e esegue l'analisi topografica per mezzo del metodo "regularized spline" con i parametri "tension" e "smoothing" \\
  \hline r.support & Crea, rigenera o modifica i file di supporto di una mappa raster \\
  \hline r.support.stats & Aggiorna le statistiche della mappa raster \\
  \hline r.proj & Riproietta una mappa raster dalla location originale indicata a quella corrente \\
\hline
\end{tabular}
\end{table}

\begin{table}[ht]
\centering
\caption{Strumenti GRASS: moduli per la gestione del colore dei dati raster}\medskip
 \begin{tabular}{|p{4cm}|p{12cm}|}
  \hline \multicolumn{2}{|c|}{\textbf{Moduli per la gestione del colore dei dati raster tra gli strumenti GRASS}} \\
  \hline \textbf{Nome modulo} & \textbf{Scopo} \\
  \hline r.colors.table & Imposta la tabella colori del raster a quella selezionata tra quelle predefinite \\
  \hline r.colors.rules & Imposta la tabella colori del raster in base alle regole impostate \\
  \hline r.colors.rast & Imposta la tabella colori del raster usando le impostazioni di un raster esistente \\
  \hline r.blend & Miscela le componenti di colore di due mappe raster del valore impostato \\
  \hline r.composite & Miscela le componenti di colore rosso, verde e blu per ottenere un singolo file raster a colori \\
  \hline r.his & Genera mappe raster separate per le componenti rosso, verde e blu combinando i valori di tonalità (hue), intensità (intensity), e saturazione (saturation) (metodo his) di una mappa raster in input specificata dall'utente \\
\hline
\end{tabular}
\end{table}

\begin{table}[ht]
\centering
\caption{Strumenti GRASS: moduli per le operazioni di processamento spaziale di dati raster}\medskip
 \begin{tabular}{|p{4cm}|p{12cm}|}
  \hline \multicolumn{2}{|c|}{\textbf{Moduli per l'analisi geospaziale di dati raster tra gli stumenti di GRASS}} \\
  \hline \textbf{Nome modulo} & \textbf{Scopo} \\
  \hline r.buffer & Crea un buffer di dimensione impostata sui valori del raster \\
  \hline r.mask & Crea una maschera (MASK) per limitare l'estensione spaziale sulla quale eseguire le operazioni sul raster \\
  \hline r.mapcalc & Esegue operazioni geospaziali sul raster in modo algebrico \\
  \hline r.mapcalculator & Procedura guidata e semplificata per l'esecuzione di operazioni algebriche geospaziali su mappe raster \\
  \hline r.neighbors & Analisi raster di tipo "neighbors", con creazione di una nuova mappa raster i cui valori in una certa posizione sono funzione (media, mediana, moda, minimo, massimo...) di un certo numero di valori (size) scelti dall'utente nelle vicinanze di quella posizione \\
  \hline v.neighbors & Creazione di una mappa raster in cui i valori di una certa cella sono funzione dei valori degli attributi assegnati a punti o centroidi presenti nelle vicinanze della medesima entro un certo raggio di ricerca impostato dall'utente \\
  \hline r.cross & Creazione di una nuova mappa raster che contenga un numero di categorie pari alle singole combinazioni di categorie dei raster in input \\
  \hline r.series & Creazione di una mappa raster in cui ogni cella assume il valore stabilito dalla funzione impostata che opera sui valori di cella delle mappe scelte dall'utente \\
  \hline r.patch & Creazione di una singola mappa raster mediante "collage" di altre mappe raster singole \\
  \hline r.statistics & Fornisce le statistichedi un layer "cover" sulla base delle sue relazioni con un layer "base" stabilite dalla funzione specificata \\
  \hline r.cost & Creazione di una mappa raster contenente il costo cumulativo conseguente lo spostamento tra diverse posizioni geografiche su una mappa raster i cui valori di cella rappresentino costi \\
  \hline r.drain & Crea una mappa raster contenente una linea di flusso sulla base dei valori contenuti in una mappa di elevazione \\
  \hline r.shaded.relief & Crea mappe ombreggiate \\
  \hline r.slope.aspect.slope & Genera mappe di acclività da modelli digitali di elevazione \\
  \hline r.slope.aspect.aspect & Genera mappe di orientazione dei versanti da modelli digitali di elevazione \\
  \hline r.param.scale & Estrae i parametri morfometrici di un'area da un modello digitale di elevazione \\
  \hline r.texture & Genera una mappa raster con le caratteristiche tessiturali di una mappa raster (prima serie di indici) \\
  \hline r.texture.bis & Genera una mappa raster con le caratteristiche tessiturali di una mappa raster (seconda serie di indici) \\
  \hline r.los & Analisi di visibilità su mappe raster \\
  \hline r.clump & Raggruppa celle contigue in un'unica categoria \\
  \hline r.grow & Genera una mappa raster in cui le aree contigue vengono accresciute di una cella rispetto alla mappa di partenza \\
  \hline r.thin & Assottiglia gli elementi lineari di una mappa raster \\
\hline
\end{tabular}
\end{table}

\begin{table}[ht]
\centering
\caption{Strumenti GRASS: moduli per la gestione di superfici}\medskip
 \begin{tabular}{|p{4cm}|p{12cm}|}
  \hline \multicolumn{2}{|c|}{\textbf{Moduli per la gestione di superfici tra gli strumenti di GRASS}} \\
  \hline \textbf{Nome modulo} & \textbf{Scopo} \\
  \hline r.random & Crea una mappa di punti vettoriali campionando una mappa raster in modo casuale \\
  \hline r.random.cells & Genera celle casuali legate tra loro da relazioni spaziali \\
  \hline v.kernel & Genera una mappa raster di densità sulla base di un vettoriale di punti usando un Gaussian kernel \\
  \hline r.contour & Produce una mappa vettoriale delle isoipse a partire da una mappa raster con l'intervallo specificato \\
  \hline r.contour2 & Produce una mappa vettoriale delle isoipse elencate dall'utente a partire da una mappa raster \\
  \hline r.surf.fractal & Crea una superificie frattale con dimensione del frattale specificata \\
  \hline r.surf.gauss & Crea una mappa raster della deviazione gaussiana la cui media e deviazione standard è impostata dall'utente \\
  \hline r.surf.random & Genera una mappa raster delle deviazioni random il cui intervallo è impostato dall'utente \\
  \hline r.bilinear & Interpolazione bilineare di mappe raster \\
  \hline v.surf.bispline & Interpolazione bicubica o "bilinear spline" con regolarizzazione di Tykhonov \\
  \hline r.surf.idw & Interpolazione di superfici 3d con metodo idw (inverse distance weighted) \\
  \hline r.surf.idw2 & Altra opzione di generazione di superfici 3d con metodo idw \\
  \hline r.surf.contour & Generazione di superfici 3d da isoipse raster \\
  \hline v.surf.idw & Generazione di superfici 3d per interpolazione di valori di un vettoriale (metodo idw) \\
  \hline v.surf.rst & Generazione di superici 3d per interpolazione di valori di un vettoriale (metodo Regularized Spline Tension, RST) \\
  \hline r.fillnulls & Riempie le aree prive di dati in un raster usando l'interpolazione con il modulo v.surf.rst \\
\hline
\end{tabular}
\end{table}

\begin{table}[ht]
\centering
\caption{Strumenti GRASS: moduli per la modifica delle categorie e per le etichette di dati raster}\medskip
 \begin{tabular}{|p{4cm}|p{12cm}|}
  \hline \multicolumn{2}{|c|}{\textbf{Moduli per le categorie e le etichette tra gli strumenti GRASS}} \\
  \hline \textbf{Nome modulo} & \textbf{Scopo} \\
  \hline r.reclass.area.greater & Riclassifica una mappa raster con parcelle di estensioni in ettari superiori al valore stabilito dall'utente \\
  \hline r.reclass.area.lesser &  Riclassifica una mappa raster con parcelle di estensioni in ettari inferiori al valore stabilito dall'utente \\
  \hline r.reclass & Riclassifica un raster usando un file di regole preformattato dall'utente \\
  \hline r.recode & Ricodifica una mappa raster \\
  \hline r.rescale & Riscala gli intervalli dei valori di categoria di una mappa raster \\
\hline
\end{tabular}
\end{table}

\begin{table}[ht]
\centering
\caption{GRASS Toolbox: Hydrologic modelling modules}\medskip
 \begin{tabular}{|p{4cm}|p{12cm}|}
  \hline \multicolumn{2}{|c|}{\textbf{Hydrologic modelling modules in the GRASS
  Toolbox}} \\
  \hline \textbf{Nome modulo} & \textbf{Scopo} \\
  \hline r.carve & Takes vector stream data, transforms it to raster, and
  subtracts depth from the output DEM \\
  \hline r.fill.dir & Filters and generates a depressionless elevation map
  and a flow direction map from a given elevation layer \\
  \hline r.lake.xy & Fills lake from seed point at given level \\
  \hline r.lake.seed & Fills lake from seed at given level \\
  \hline r.topidx & Creates a 3D volume map based on 2D elevation and value
  raster maps \\
  \hline r.basins.fill & Generates a raster map layer showing watershed
  subbasins \\
  \hline r.water.outlet & Watershed basin creation program \\
\hline
\end{tabular}
\end{table}

\begin{table}[ht]
\centering
\caption{GRASS Toolbox: Reports and statistic analysis modules}\medskip
 \begin{tabular}{|p{4cm}|p{12cm}|}
  \hline \multicolumn{2}{|c|}{\textbf{Reports and statistic analysis modules in the GRASS Toolbox}} \\
  \hline \textbf{Nome modulo} & \textbf{Scopo} \\
  \hline r.category & Prints category values and labels associated with
  user-specified raster map layers \\
  \hline r.sum & Sums up the raster cell values \\
  \hline r.report & Reports statistics for raster map layers \\
  \hline r.average & Finds the average of values in a cover map within areas
  assigned the same category value in a user-specified base map \\
  \hline r.median & Finds the median of values in a cover map within areas
  assigned the same category value in a user-specified base map \\
  \hline r.mode & Finds the mode of values in a cover map within areas
  assigned the same category value in a user-specified base map.reproject
  raster image \\
  \hline r.volume & Calculates the volume of data clumps, and produces a
  GRASS vector points map containing the calculated centroids of these clumps \\
  \hline r.surf.area & Surface area estimation for rasters \\
  \hline r.univar & Calculates univariate statistics from the non-null cells
  of a raster map \\
  \hline r.covar & Outputs a covariance/correlation matrix for user-specified
  raster map layer(s)\\
  \hline r.regression.line & Calculates linear regression from two raster
  maps: y = a + b * x \\
  \hline r.coin & Tabulates the mutual occurrence (coincidence) of categories
  for two raster map layers\\
\hline
\end{tabular}
\end{table}

\clearpage

\subsection{GRASS Toolbox vector data modules}

This Section lists all graphical dialogs in the GRASS Toolbox to work with
and analyse vector data in a currently selected GRASS location and mapset.

\begin{table}[ht]
\centering
\caption{GRASS Toolbox: Develop vector map modules}\medskip
 \begin{tabular}{|p{4cm}|p{12cm}|}
  \hline \multicolumn{2}{|c|}{\textbf{Develop vector map modules in the GRASS
  Toolbox}} \\
  \hline \textbf{Nome modulo} & \textbf{Scopo} \\
  \hline v.build.all & Rebuild topology of all vectors in the mapset \\
  \hline v.clean.break & Break lines at each intersection of vector map \\
  \hline v.clean.snap & Cleaning topology: snap lines to vertex in threshold \\
  \hline v.clean.rmdangles & Cleaning topology: remove dangles \\
  \hline v.clean.chdangles & Cleaning topology: change the type of boundary
  dangle to line \\
  \hline v.clean.rmbridge & Remove bridges connecting area and island or 2
  islands \\
  \hline v.clean.chbridge & Change the type of bridges connecting area and
  island or 2 islands \\
  \hline v.clean.rmdupl & Remove duplicate lines (pay attention to
  categories!) \\
  \hline v.clean.rmdac & Remove duplicate area centroids \\
  \hline v.clean.bpol & Break polygons. Boundaries are broken on each
  point shared between 2 and more polygons where angles of segments are
  different \\
  \hline v.clean.prune & Remove vertices in threshold from lines and
  boundaries \\
  \hline v.clean.rmarea & Remove small areas (removes longest boundary with
  adjacent area) \\
  \hline v.clean.rmline & Remove all lines or boundaries of zero length \\
  \hline v.clean.rmsa & Remove small angles between lines at nodes \\
  \hline v.type.lb & Convert lines to boundaries \\
  \hline v.type.bl & Convert boundaries to lines \\
  \hline v.type.pc & Convert points to centroids \\
  \hline v.type.cp & Convert centroids to points \\
  \hline v.centroids & Add missing centroids to closed boundaries  \\
  \hline v.build.polylines & Build polylines from lines \\
  \hline v.segment & Creates points/segments from input vector lines and
  positions \\
  \hline v.to.points & Create points along input lines \\
  \hline v.parallel & Create parallel line to input lines \\
  \hline v.dissolve & Dissolves boundaries between adjacent areas \\
  \hline v.drape & Convert 2D vector to 3D vector by sampling of elevation
  raster\\
  \hline v.transform & Performs an affine transformation on a vector map \\
  \hline v.proj & Allows projection conversion of vector files \\
  \hline v.support & Updates vector map metadata \\
  \hline generalize & Vector based generalization \\
\hline
\end{tabular}
\end{table}

\begin{table}[ht]
\centering
\caption{GRASS Toolbox: Database connection modules}\medskip
 \begin{tabular}{|p{4cm}|p{12cm}|}
  \hline \multicolumn{2}{|c|}{\textbf{Database connection modules in the GRASS
  Toolbox}} \\
  \hline \textbf{Nome modulo} & \textbf{Scopo} \\
  \hline v.db.connect & Connect a vector to database \\
  \hline v.db.sconnect & Disconnect a vector from database \\
  \hline v.db.what.connect & Set/Show database connection for a vector \\
\hline
\end{tabular}
\end{table}

\begin{table}[ht]
\centering
\caption{GRASS Toolbox: Change vector field modules}\medskip
 \begin{tabular}{|p{4cm}|p{12cm}|}
  \hline \multicolumn{2}{|c|}{\textbf{Change vector field modules in the GRASS
  Toolbox}} \\
  \hline \textbf{Nome modulo} & \textbf{Scopo} \\
  \hline v.category.add & Add elements to layer (ALL elements of the selected
  layer type!)\\
  \hline v.category.del & Delete category values \\
  \hline v.category.sum & Add a value to the current category values \\
  \hline v.reclass.file & Reclass category values using a rules file \\
  \hline v.reclass.attr & Reclass category values using a column attribute
  (integer positive) \\
\hline
\end{tabular}
\end{table}

\begin{table}[ht]
\centering
\caption{GRASS Toolbox: Working with vector points modules}\medskip
 \begin{tabular}{|p{4cm}|p{12cm}|}
  \hline \multicolumn{2}{|c|}{\textbf{Working with vector points modules in the GRASS Toolbox}} \\
  \hline \textbf{Nome modulo} & \textbf{Scopo} \\
  \hline v.in.region & Create new vector area map with current region extent \\
  \hline v.mkgrid.region & Create grid in current region \\
  \hline v.in.db & Import vector points from a database table containing
  coordinates \\
  \hline v.random & Randomly generate a 2D/3D GRASS vector point map \\
  \hline v.kcv & Randomly partition points into test/train sets \\
  \hline v.outlier & Romove outliers from vector point data \\
  \hline v.hull & Create a convex hull \\
  \hline v.delaunay.line & Delaunay triangulation (lines) \\
  \hline v.delaunay.area & Delaunay triangulation (areas) \\
  \hline v.voronoi.line & Voronoi diagram (lines) \\
  \hline v.voronoi.area & Voronoi diagram (areas) \\
\hline
\end{tabular}
\end{table}

\begin{table}[ht]
\centering
\caption{GRASS Toolbox: Spatial vector and network analysis modules}\medskip
 \begin{tabular}{|p{4cm}|p{12cm}|}
  \hline \multicolumn{2}{|c|}{\textbf{Spatial vector and network analysis modules in the GRASS
  Toolbox}} \\
  \hline \textbf{Nome modulo} & \textbf{Scopo} \\
  \hline v.extract.where & Select features by attributes \\
  \hline v.extract.list & Extract selected features \\
  \hline v.select.overlap & Select features overlapped by features in another
  map\\
  \hline v.buffer & Vector buffer \\
  \hline v.distance & Find the nearest element in vector 'to' for elements in
  vector 'from'. \\
  \hline v.net.nodes & Create nodes on network \\
  \hline v.net.alloc & Allocate network\\
  \hline v.net.iso & Cut network by cost isolines \\
  \hline v.net.salesman & Connect nodes by shortest route (traveling
  salesman) \\
  \hline v.net.steiner & Connect selected nodes by shortest tree (Steiner
  tree) \\
  \hline v.patch & Create a new vector map by combining other vector maps \\
  \hline v.overlay.or & Vector union \\
  \hline v.overlay.and & Vector intersection \\
  \hline v.overlay.not & Vector subtraction \\
  \hline v.overlay.xor & Vector non-intersection \\
\hline
\end{tabular}
\end{table}

\begin{table}[ht]
\centering
\caption{GRASS Toolbox: Vector update by other maps modules}\medskip
 \begin{tabular}{|p{4cm}|p{12cm}|}
  \hline \multicolumn{2}{|c|}{\textbf{Vector update by other maps modules in the GRASS
  Toolbox}} \\
  \hline \textbf{Nome modulo} & \textbf{Scopo} \\
  \hline v.rast.stats & Calculates univariate statistics from a GRASS raster
  map based on vector objects\\
  \hline v.what.vect & Uploads map for which to edit attribute table \\
  \hline v.what.rast & Uploads raster values at positions of vector points to
  the table \\
  \hline v.sample & Sample a raster file at site locations \\
\hline
\end{tabular}
\end{table}

\begin{table}[ht]
\centering
\caption{GRASS Toolbox: Vector report and statistic modules}\medskip
 \begin{tabular}{|p{4cm}|p{12cm}|}
  \hline \multicolumn{2}{|c|}{\textbf{Vector report and statistic modules in the GRASS
  Toolbox}} \\
  \hline \textbf{Nome modulo} & \textbf{Scopo} \\
  \hline v.to.db & Put geometry variables in database \\
  \hline v.report & Reports geometry statistics for vectors \\
  \hline v.univar & Calculates univariate statistics on selected table column
  for a GRASS vector map \\
  \hline v.normal & Tests for normality for points\\
\hline
\end{tabular}
\end{table}

\clearpage

\subsection{GRASS Toolbox imagery data modules}

This Section lists all graphical dialogs in the GRASS Toolbox to work with
and analyse imagery data in a currently selected GRASS location and mapset.

\begin{table}[ht]
\centering
\caption{GRASS Toolbox: Imagery analysis modules}\medskip
 \begin{tabular}{|p{4cm}|p{12cm}|}
  \hline \multicolumn{2}{|c|}{\textbf{Imagery analysis modules in the GRASS
  Toolbox}} \\
  \hline \textbf{Nome modulo} & \textbf{Scopo} \\
  \hline i.image.mosaik & Mosaic up to 4 images \\
  \hline i.rgb.his & Red Green Blue (RGB) to Hue Intensity Saturation (HIS)
  raster map color transformation function \\
  \hline i.his.rgb & Hue Intensity Saturation (HIS) to Red Green Blue (RGB)
  raster map color transform function \\
  \hline i.landsat.rgb & Auto-balancing of colors for LANDSAT images \\
  \hline i.fusion.brovey & Brovey transform to merge multispectral and
  high-res pancromatic channels \\
  \hline i.zc & Zero-crossing edge detection raster function for image
  processing \\
  \hline i.mfilter &  \\
  \hline i.tasscap4 & Tasseled Cap (Kauth Thomas) transformation for
  LANDSAT-TM 4 data \\
  \hline i.tasscap5 & Tasseled Cap (Kauth Thomas) transformation for
  LANDSAT-TM 5 data \\
  \hline i.tasscap7 & Tasseled Cap (Kauth Thomas) transformation for
  LANDSAT-TM 7 data \\
  \hline i.fft & Fast fourier transform (FFT) for image processing \\
  \hline i.ifft & Inverse fast fourier transform for image processing \\
  \hline r.describe & Prints terse list of category values found in a raster
  map layer \\
  \hline r.bitpattern & Compares bit patterns with a raster map \\
  \hline r.kappa & Calculate error matrix and kappa parameter for accuracy
  assessment of classification result \\
  \hline i.oif & Calculates optimal index factor table for landsat tm bands \\
\hline
\end{tabular}
\end{table}

\clearpage

\subsection{GRASS Toolbox database modules}

This Section lists all graphical dialogs in the GRASS Toolbox to manage, 
connect and work with internal and external databases. Working with spatial 
external databases is enabled via OGR and not covered by these modules.

\begin{table}[ht]
\centering
\caption{GRASS Toolbox: Database modules}\medskip
 \begin{tabular}{|p{4cm}|p{12cm}|}
  \hline \multicolumn{2}{|c|}{\textbf{Database management and analysis modules in the GRASS
  Toolbox}} \\
  \hline \textbf{Nome modulo} & \textbf{Scopo} \\
  \hline db.connect & Sets general DB connection mapset \\
  \hline db.connect.schema & Sets general DB connection mapset with a schema \\
  \hline v.db.reconnect.all & Reconnect vector to a new database \\
  \hline db.login & Set user/password for driver/database \\
  \hline db.in.ogr & Imports attribute tables in various formats \\
  \hline v.db.addtable & Create and add a new table to a vector \\
  \hline v.db.addcol & Adds one or more columns to the attribute table
  connected to a given vector map \\
  \hline v.db.dropcol & Drops a column from the attribute table connected to
  a given vector map\\
  \hline v.db.renamecol & Renames a column in a attribute table connected to
  a given vector map\\
  \hline v.db.update\_const & Allows to assign a new constant value to a
  column \\
  \hline v.db.update\_query & Allows to assign a new constant value to a
  column only if the result of a query is TRUE \\
  \hline v.db.update\_op & Allows to assign a new value, result of operation
  on column(s), to a column in the attribute table connected to a given map\\
  \hline v.db.update\_op\_query & Allows to assign a new value to a column,
  result of operation on column(s), only if the result of a query is TRUE \\
  \hline db.execute & Execute any SQL statement \\
  \hline db.select & Prints results of selection from database based on SQL \\
  \hline v.db.select & Prints vector map attributes \\
  \hline v.db.select.where & Prints vector map attributes with SQL \\
  \hline v.db.join & Allows to join a table to a vector map table \\
  \hline v.db.univar & Calculates univariate statistics on selected table
  column for a GRASS vector map \\
\hline
\end{tabular}
\end{table}

\clearpage

\subsection{GRASS Toolbox 3D modules}

This Section lists all graphical dialogs in the GRASS Toolbox to work with 3D 
data. GRASS provides more modules, but they are currently only available using 
the GRASS Shell.

\begin{table}[ht]
\centering
\caption{GRASS Toolbox: 3D Visualization}\medskip
 \begin{tabular}{|p{4cm}|p{12cm}|}
  \hline \multicolumn{2}{|c|}{\textbf{3D visualization and analysis modules in the GRASS
  Toolbox}} \\
  \hline \textbf{Nome modulo} & \textbf{Scopo} \\
  \hline nviz & Open 3D-View in nviz\\
\hline
\end{tabular}
\end{table}

\subsection{GRASS Toolbox help modules}

The GRASS GIS Reference Manual offers a complete overview of the available 
GRASS modules, not limited to the modules and their often reduced functionalities 
implemented in the GRASS Toolbox. 

\begin{table}[ht]
\centering
\caption{GRASS Toolbox: Reference Manual}\medskip
 \begin{tabular}{|p{4cm}|p{12cm}|}
  \hline \multicolumn{2}{|c|}{\textbf{Reference Manual modules in the GRASS
  Toolbox}} \\
  \hline \textbf{Nome modulo} & \textbf{Scopo} \\
  \hline g.manual & Display the HTML manual pages of GRASS \\
\hline
\end{tabular}
\end{table}




