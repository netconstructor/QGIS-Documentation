% vim: set textwidth=78 autoindent:

\section{Premessa}\label{label_forward}
\pagenumbering{arabic}
\setcounter{page}{1}

% when the revision of a section has been finalized, 
% comment out the following line:
% \updatedisclaimer

Benvenuti nel meraviglioso mondo dei Sistemi Informativi Geografici (Geographical Information Systems, GIS). Quantum GIS (QGIS) è un Sistema
Informativo Geografico a codice aperto (Open Source). Il progetto
è nato nel maggio 2002 ed è stato ospitato su SourceForge
nel giugno dello stesso anno. Abbiamo lavorato duramente per rendere
il software GIS (che è tradizionalmente un software proprietario costoso)
una valida prospettiva per chiunque avesse disponibilità di un Personal
Computer. QGIS gira attualmente su molte piattaforme Unix (incluso ovviamente Linux!), su Windows,
e OS X. QGIS è sviluppato in Qt (\url{http://www.trolltech.com})
e C++. Ciò fa sì che QGIS appaia reattivo all'uso e piacevole e
semplice da usare grazie all'interfaccia grafica (graphical user interface,
GUI).

QGIS si prefigge lo scopo di essere un GIS facile da usare, in grado
di fornire funzioni e caratteristiche di uso comune. L’obiettivo iniziale
era di fornire un visualizzatore di dati GIS, ma attualmente QGIS ha oltrepassato questo
punto nel suo sviluppo, ed è usato da molti per il loro lavoro quotidiano nel campo GIS.
QGIS supporta nativamente un considerevole numero di formati raster e vettoriali,
il supporto a nuovi formati può essere facilmente inserito mediante
plugin (si veda l'Appendice \ref{appdx_data_formats} per una lista
completa dei formati attualmente supportati).

QGIS è rilasciato con licenza GNU General Public License (GPL). Lo
sviluppo di QGIS con questa licenza implica che si possa ispezionare
e modificare il codice sorgente in modo da garantirvi di avere sempre accesso
ad un programma GIS esente da costi di licenza e modificabile liberamente secondo le
vostre esigenze. Dovresti aver ricevuto una copia completa della licenza
con la tua copia di QGIS, puoi comunque trovarla nell'Appendice \ref{gpl_appendix}.



\begin{Tip}\caption{\textsc{Documentazione aggiornata}}\index{documentazione}
\qgistip{La versione più recente di questo documento è sempre reperibile 
all'indirizzo \url{http://download.osgeo.org/qgis/doc/manual/} o nell'area documentazione
del sito di QGIS all'indirizzo \url{http://qgis.osgeo.org/documentation/}
}
\end{Tip}

\subsection{Caratteristiche}\label{label_majfeat}

QGIS offre molte funzioni GIS di uso comune sia nativamente che mediante plugin. È possibile offrire
una panoramica iniziale raggruppandole sinteticamente
in sei categorie.

\minisec{Visualizzazione di dati}

Possono essere visualizzati e sovrapposti dati vettoriali e raster
in diversi formati e proiezioni senza necessità di conversioni verso un formato
comune interno. Tra i formati supportati sono inclusi:

\begin{itemize}
\item tabelle PostgreSQL con estensione spaziale usando PostGIS, formati vettoriali
%\footnote{i formati di database supportati da OGR come Oracle o MySQL non sono 
%ancora supportati in QGIS.}
supportati dalla libreria OGR installata, inclusi gli
shapefile ESRI e i formati MapInfo, SDTS e GML (vedere l'Appendice \ref{appdx_ogr} per la lista completa),
\item formati raster e immagine supportati dalla libreria GDAL (Geospatial
Data Abstraction Library) installata, come GeoTiff, Erdas Img., ArcInfo
Ascii Grid, JPEG, PNG (vedere l'Appendice \ref{appdx_gdal} per la lista completa),
\item Database SpatiaLite (vedere Sezione \ref{label_spatialite}),
\item formati raster GRASS e dati vettoriali da database GRASS (location/mapset), 
\item dati spaziali forniti da servizi di mappa online conformi agli standard
OGC quali Web Map Service (WMS) o Web Feature Service (WFS)
\item dati OpenStreetMap.
\end{itemize}

\minisec{Esplorazione dei dati e creazione delle mappe} 

Possono essere composte mappe ed esplorati interattivamente dati spaziali
tramite un interfaccia grafica amichevole. Tra i molti strumenti utili disponibili
nell'interfaccia grafica ci sono:
\begin{itemize}
\item riproiezione al volo
\item compositore di mappa
\item pannello vista panoramica 
\item segnalibri geospaziali 
\item identifica/seleziona elementi
\item modifica/vedi/cerca attributi
\item etichettatura elementi
\item cambio della simbologia sia raster che vettoriale
\item aggiunta reticolo su un nuovo layer tramite il plugin fTools
\item decorazione della mappa con freccia del nord, barra di scala ed etichetta di copyright
\item salva e ricarica progetti
\end{itemize}

\minisec{Creazione, modifica, gestione ed esportazione di dati}

Possono essere creati, modificati, gestiti ed esportati dati vettoriali
in molteplici formati. I dati raster devono essere importati in GRASS
per poter essere editati ed esportati in altri formati. QGIS offre
le seguenti funzioni:
\begin{itemize}
\item strumenti per digitalizzare formati supportati da OGR e layer vettoriali GRASS
\item creazione e modifica di shapefile e layer vettoriali GRASS 
\item georeferenziazione di immagini con l'apposito plugin 
\item strumenti GPS per l'importazione ed esportazione del formato GPX e conversione di altri formati GPS
al formato GPX o down/upload dei dati direttamente da unità GPS (su Linux, usb: è stato aggiunto 
alla lista dei dispositivi GPS)
\item OSM!
\item creazione di layer PostGIS da shapefile con il plugin SPIT
\item migliorata la gestione di tabelle PostGIS
\item gestione di tabelle degli attributi di dati vettoriali con il nuovo plugin Tabella attributi (vedi Sezione 
\ref{sec:attribute table}) o il plugin Gestione Tabelle
\item salvare schermate come immagini georeferenziate
\item visualizzare e modificare dati OpenStreetMap
\end{itemize}

\minisec{Analisi di dati} 

Possono essere eseguite analisi spaziali di dati PostgreSQL/PostGIS e di altri
formati supportati da OGR per mezzo del plugin python fTools. QGIS
offre attualmente strumenti per l'analisi, il campionamento, il geoprocessamento,
la gestione delle geometrie e del database di dati vettoriali. Possono
inoltre essere usati gli strumenti GRASS integrati, che includono
l'intera gamma delle funzioni di GRASS di oltre 300 moduli (si veda la Sezione \ref{sec:grass}).

\minisec{Pubblicazione di mappe su internet}

QGIS può essere usato per esportare dati in un mapfile che può essere
pubblicato su internet mediante un webserver sul quale sia installato
UMN MapServer. QGIS può anche essere impiegato come client WMS o
WFS e come server WMS. 

\minisec{Estensione delle funzioni di QGIS per mezzo di plugin}

QGIS può essere adattato a particolari esigenze grazie all'architettura
estensibile per mezzo di plugin. QGIS fornisce le librerie che possono
essere usate per creare i plugin. Possono addirittura essere creati
nuovi programmi in C++ o Python.

\begin{enumerate}
\item Aggiunta Layer a testo delimitato (Caricamento e visualizzazione di files di testo delimitato contenenti coordinate x,y)
\item Cattura delle Coordinate (Cattura delle coordinate del Mouse in differenti Sistemi di Riferimento)
\item Decorazioni (etichetta Diritti d’autore, Freccia nord e Barra di scala)
\item Diagramma sovrapposto (Piazzamento di diagrammi sopra layers vettoriali)
\item Convertitore Dxf2Shp (Converte dal formato vettoriale DXF al formato vettoriale Shape)
\item Strumenti GPS (Caricamento e importazione dati GPS)
\item GRASS (Integrazione GIS con GRASS)
\item Georeferenziatore GDAL (Aggiunge le informazioni sulla proiezione al raster mediante GDAL)
\item Plugin per l'interpolazione (interpolazione basata sui vertici di un layer vettoriale)
\item Etichettamento (Etichettamento rapido su layers vettoriale)
\itemEsportazione verso MapServer (Esporta un progetto QGIS in un file map di Mapserver)
\item Convertitore di Layer OGR (Trasforma layer vettoriali tra vari formati)
\item Plugin OpenStreetMap (Visualizzatore e Editor per dati OpenStreetMap)
\item Supporto per i GeoRaster Oracle
\item Installatore plugin Python (Plugin Installer)
\item Stampa rapida (Stampa una mappa con pochi comandi)
\item Analisi della morfologia terrestre basata su raster
\item SPIT (Importazione Shapefile verso PostgreSQL/PostGIS)
\item Plugin WFS (Aggiunge dei layers WFS nella vista di QGIS)
\item eVIS (Strumento per la visualizzazione di eventi)
\item fTools (Strumenti per l'analisi e la gestione di dati vettoriali)
\item Console Python (Accesso all'ambiente di gestione QGIS)
\item Installatore dei plugin in python
\end{enumerate}

\minisec{Plugins addizionali in python}

QGIS offre un numero crescente di plugin esterni in Python forniti
dalla comunità. Questi plugin sono ospitati sul repository ufficiale
PyQGIS e si possono installare facilmente usando il Gestore di
Plugin Python (si veda la Sezione \ref{sec:plugins}).

\minisec{Novità nella versione ~\CURRENT} 

Queste solo le principali aggiunte e miglioramenti:
\begin{itemize}
 \item Significativo aumento di prestazioni della tabella degli attributi
 \item Barra degli strumenti Editing Avanzato
 \item Configurazione di scorciatoie dalla finestra principale
 \item Unione di geometrie 
 \item Plugin eVis
 \item Plugin OSM 
 \item Nuova shell GRASS
 \item Il Compositore di stampe può esportare in PDF
\end{itemize}

