%  !TeX  root  =  user_guide.tex  
\mainmatter
\pagestyle{scrheadings}
\addchap{Introduzione}\label{label_forward}
\pagenumbering{arabic}
\setcounter{page}{1}

% when the revision of a section has been finalized, 
% comment out the following line:
% \updatedisclaimer

Benvenuti nel meraviglioso mondo dei Sistemi Informativi Geografici (Geographical 
Information Systems, GIS). Quantum GIS (QGIS) è un Sistema Informativo Geografico 
a codice aperto (Open Source). Il progetto è nato nel maggio 2002 ed è stato 
ospitato su SourceForge nel giugno dello stesso anno. Abbiamo lavorato duramente 
per rendere il software GIS (che è tradizionalmente un software proprietario costoso) 
una valida prospettiva per chiunque avesse disponibilità di un Personal Computer. 
QGIS gira attualmente su molte piattaforme Unix (incluso ovviamente Linux!), su 
Windows, e OS X. QGIS è sviluppato in Qt  (\url{http://qt.nokia.com}) e C++. Ciò 
fa sì che QGIS appaia reattivo all’uso e piacevole e semplice da usare grazie 
all’interfaccia grafica (graphical user interface, GUI).

QGIS si prefigge lo scopo di essere un GIS facile da usare, in grado di fornire 
funzioni e caratteristiche di uso comune. L’obiettivo iniziale era di fornire un 
visualizzatore di dati GIS, ma attualmente QGIS ha oltrepassato questo punto nel 
suo sviluppo, ed è usato da molti per il loro lavoro quotidiano nel campo GIS. 
QGIS supporta nativamente un considerevole numero di formati raster e vettoriali, 
il supporto a nuovi formati può essere facilmente esteso mediante plugin (si veda 
l’Appendice \ref{appdx_data_formats} per una lista completa dei formati attualmente 
supportati).

QGIS è rilasciato con licenza GNU General Public License (GPL). Lo sviluppo di QGIS 
con questa licenza implica che si possa esaminare e modificare il codice sorgente, 
in modo da garantirvi di avere sempre accesso ad un programma GIS esente da costi 
di licenza e modificabile liberamente secondo le vostre esigenze. Dovreste aver 
ricevuto una copia completa della licenza con la vostra copia di QGIS, potete 
comunque trovarla nell’Appendice \ref{gpl_appendix}.  

\begin{Tip}\caption{\textsc{Documentazione aggiornata}}\index{documentation}
La versione più recente di questo documento è sempre reperibile all’indirizzo
\url{http://download.osgeo.org/qgis/doc/manual/}, o nell’area documentazione 
del sito di QGIS all’indirizzo \url{http://qgis.osgeo.org/documentation/}
\end{Tip}

\addsec{Caratteristiche}\label{label_majfeat}

\qg QGIS offre molte funzioni GIS di uso comune, sia nativamente che mediante 
plugin. È possibile offrire una panoramica iniziale raggruppandole sinteticamente 
in sei categorie.

\minisec{Visualizzazione di dati}

Si possono visualizzare e sovrapporre dati vettoriali e raster in diversi formati e proiezioni senza
necessità di conversioni verso un formato comune interno. Tra i formati supportati sono inclusi:

\begin{itemize}[label=--]
\item  Tabelle PostgreSQL con estensione spaziale usando PostGIS, formati vettoriali 
%\footnote{Formati di database gestibili tramite OGR come Oracle o 
%mySQL non sono al momento supportati in Qgis.}
supportati dalla libreria OGR installata, inclusi gli Shapefile ESRI e i formati MapInfo, 
SDTS e GML (vedere l’Appendice \ref{appdx_ogr} per la lista completa).
\item  formati raster e immagine supportati dalla libreria GDAL (Geospatial Data 
Abstraction Library) installata, come GeoTiff, Erdas Img., ArcInfo Ascii Grid, 
JPEG, PNG (vedere l’Appendice \ref{appdx_gdal} per la lista completa).
\item Database SpatiaLite (vedere Sezione \ref{label_spatialite}) 
\item Formati raster GRASS e dati vettoriali da database GRASS (location/mapset),
see Section \ref{sec:grass}.
\item Dati spaziali forniti da servizi di mappa online conformi agli standard OGC quali Web Map
Service (WMS) o Web Feature Service (WFS), vedere Sezione \ref{working_with_ogc}.
\item Dati OpenStreetMap (vedere Sezione \ref{plugins_osm}).
\end{itemize}

\minisec{Esplorazione dei dati e creazione di mappe} 

Si possono comporre mappe ed esplorare interattivamente dati spaziali tramite una gradevole 
interfaccia grafica. Tra i molti strumenti utili disponibili nell’interfaccia grafica sono inclusi:

\begin{itemize}[label=--]
\item riproiezione al volo
\item compositore di mappa
\item pannello vista panoramica
\item segnalibri geospaziali
\item identifica/seleziona elementi
\item modifica/visualizza/ricerca attributi
\item etichettatura elementi
\item cambio della simbologia sia raster che vettoriale
\item aggiunta reticolo su un nuovo layer tramite il plugin fTools
\item decorazione della mappa con freccia del nord, barra di scala ed etichetta di copyright
\item salva e ricarica progetti
\end{itemize}

\minisec{Creazione, modifica, gestione ed esportazione di dati}

Possono essere creati, modificati, gestiti ed esportati dati vettoriali in molteplici formati. 
I dati rasterdevono essere importati in GRASS per poter essere modificati ed esportati in 
altri formati. QGIS offre le seguenti funzioni:

\begin{itemize}[label=--]
\item strumenti per digitalizzare formati supportati da OGR e layer vettoriali GRASS
\item creazione e modifica di shapefile e layer vettoriali GRASS
\item georeferenziazione di immagini con l’apposito plugin
\item strumenti GPS per l’importazione ed esportazione del formato GPX e conversione di altri formati
GPS al formato GPX o down/upload dei dati direttamente da unità GPS (su Linux, usb: è
stato aggiunto alla lista dei dispositivi GPS)
\item visualizzazione e modifica dei dati OpenStreetMap
\item creazione di layer PostGIS da shapefile grazie al plugin SPIT 
\item gestione migliorata delle tabelle PostGIS
\item gestione di tabelle degli attributi di dati vettoriali con il nuovo plugin Tabella attributi (vedi
Sezione \ref{sec:attribute table}) o il plugin Gestione Tabelle
\item salvare le schermate come immagini georeferenziate
\end{itemize}

\minisec{Analisi di dati} 

Possono essere eseguite analisi spaziali di dati PostgreSQL/PostGIS e di altri
formati supportati da OGR per mezzo del plugin python ftools. QGIS
offre attualmente strumenti per l'analisi, il campionamento, il geoprocessamento,
la gestione delle geometrie e del database di dati vettoriali. Possono
inoltre essere usati gli strumenti GRASS integrati, che includono
l'intera gamma delle funzioni di GRASS di oltre 300 moduli (si veda la Sezione \ref{sec:grass}).

\minisec{Pubblicazione di mappe su internet}

QGIS può essere usato per esportare dati in un mapfile che può essere
pubblicato su internet mediante un webserver sul quale sia installato
UMN MapServer. QGIS può anche essere impiegato come client WMS o
WFS e come server WMS. 

\minisec{Estensione delle funzioni di QGIS per mezzo di plugins}

QGIS, grazie alla sua architettura estensibile, può essere adattato a particolari 
esigenze. QGIS fornisce le librerie che possono essere usate per creare i plugins, o 
addirittura per creare nuovi programmi in C++ o Python.

\minisec{Plugin preinstallati}

\begin{enumerate}
\item Aggiungi layer testo delimitato (caricamento e visualizzazione di file di testo delimitato
contenenti coordinate x,y).
\item Cattura coordinate (cattura le coordinate del mouse in differenti Sistemi di Riferimento).
\item Decorazioni (etichetta diritti d’autore, freccia nord e barra di scala).
\item Diagramma sovrapposto (disegna diagrammi sopra layer vettoriali).
\item Convertitore Dxf2Shp (converte dal formato vettoriale DXF al formato vettoriale Shape).
\item Strumenti GPS (caricamento e importazione dati GPS).
\item GRASS (integrazione del GIS GRASS).
\item Georeferenziatore raster GDAL (aggiunge le informazioni sulla proiezione mediante GDAL).
\item Plugin per l’interpolazione (interpolazione basata sui vertici di un layer vettoriale).
\item Esportazione verso MapServer (esporta un progetto QGIS in un file map di Mapserver).
\item Convertitore di Layer OGR (conversione di layer vettoriali in vari formati).
\item Plugin OpenStreetMap (Visualizzatore e editor per dati OpenStreetMap).
\item Supporto per i GeoRaster di Oracle Spatial.
\item Plugin Installer (installatore di plugin Python).
\item Stampa rapida (stampa una mappa con pochi comandi).
\item Analisi della morfologia terrestre basata su raster.
\item SPIT (importazione di shapefile verso PostgreSQL/PostGIS).
\item Plugin WFS (Aggiunta di layer WFS nella vista di QGIS).
\item eVIS (strumento per la visualizzazione di eventi).
\item fTools (Strumenti per l’analisi e la gestione di dati vettoriali).
\item Console Python (Accesso all’ambiente di gestione QGIS).
\item Strumenti GDAL.
\end{enumerate}

\minisec{Plugin Python esterni}

La comunità di QGIS offre un numero crescente di plugin esterni in python. Questi plugin sono 
ospitati sul repository ufficiale PyQGIS e si possono installare facilmente usando il 
Gestore di Plugin Python (si veda la Sezione \ref{sec:plugins}).

\subsubsection{Novità della versione \CURRENT} 

Questa versione fa parte della nostra serie allo "stato dell'arte", e come tale contiene nuove 
caratteristiche ed estende l'interfaccia di programmazione rispetto a QGIS 1.0.x e QGIS 1.5.0. 
Raccomandiamo l'uso di questa versione rispetto alle precedenti.

Questo rilascio include oltre 177 correzioni e molte nuove funzionalità e miglioramenti.  

\textbf{Miglioramenti generali}

\begin{itemize}[label=--]
\item Aggiunto supporto gpsd per il tracciamento gps in tempo reale. 
\item Incluso nuovo plugin che permette modifiche non in linea.
\item Il calcolatore di campi inserisce ora un valore NULL in caso di errori di calcolo, anziché fermarsi e ripetere i calcoli per tutte le geometrie. 
\item Permette percorsi di ricerca PROJ.4 specificamente definiti dall'utente e aggiorna srs.db per includere riferimenti di griglia. 
\item Aggiunta una implementazione nativa (C++) di calcolatore raster in grado di trattare in modo efficiente raster di grandi dimensioni. 
\item Migliore interazione con i widget nella barra di stato in modo che i contenuti di testo del widget possano essere copiati e incollati.
\item Molti miglioramenti e nuovi operatori nel calcolatore di campi per le tabelle di attributi vettoriali, inclusi concatenamento di campi, contatore di righe, ecc.
\item Aggiunta opzione --configpath che sovrascrive il percorso di default (~/.qgis) per le configurazioni utente e forza anche QSettings ad usare questo percorso. Questo permette al'utente, per esempio, di portare l'installazione di QGIS su un flash drive insieme a tutti i plugin ed impostazioni. 
\item Supporto sperimentale a WFS-T. In aggiunta, wfs portato al gestore di rete. 
\item Molti miglioramenti e ritocchi al georeferenziatore. 
\item Supporto ai numeri interi lunghi (long int) nella finestra attributi e nell'editor
\item Il progetto QGIS Mapserver è stato incorporato nel repository SVN principale e i pacchetti stanno per essere resi disponibili. QGIS Mapserver permette di rendere disponibile i file di progetto QGIS project attraverso il protocollo OGC WMS.
\item Sottomenu flyout per gli strumenti di selezione e misura. 
\item Aggiunto il supporto a tabelle non spaziali (al momento per providers OGR, testo delimitato e PostgreSQL). Queste tabelle possono essere usate per dare rapide occhiate ai campi oppure esplorate/modificate, a grandi linee, usando la visualizzazione tabella.
\item Aggiunto supporto di ricerca di stringhe per gli ID (\$id) delle geometrie e vari altri miglioramenti relativi alla ricerca.
\item Aggiunto metodo di ricaricamento per i layer di mappa e i provider di interfaccia. I provider di caching (attualmente WMS e WFS) possono sincronizzarsi con le modifiche della sorgente dei dati.
\end{itemize}

\textbf{Miglioramenti al pannello Layer (legenda)}

\begin{itemize}[label=--]
\item Aggiunta nuova opzione al menu della legenda raster che ridimensiona il layer corrente usando i valori min e max pixel dell'attuale estensione. 
\item Salvando shape file con il menu contestuale "Salva come", è ora possibile specificare le opzioni di creazione OGR. 
\item Ora è possibile selezionare e rimuovere più layer contemporaneamente.
\end{itemize}

\textbf{Etichette (solo per l'ambiente di nuova generazione)}

\begin{itemize}[label=--]
\item Posizionamento delle etichette definita dai dati. 
\item Wrapping di riga, caratteri definiti dai dati e impostazione del buffer per le etichette.
\end{itemize}

\textbf{Proprietà layer e simbologia}

\begin{itemize}[label=--]
\item Aggiunti tre nuovi modelli di classificazione alla resa graduata dei simboli (versione 2), che comprendono Natural Breaks (Algoritmo di Jenks), deviazione standard e Pretty Breaks (basato su pretty dell'ambiente statistico R).
\item Migliorata velocità di caricamento della finestra di proprietà dei simboli.
\item Rotazione e dimensione definiti dai dati per le rese graduate e categorizzate (simbologia di nuova generazione).
\item Uso della scala di dimensione anche per i simboli di linea, per modificarne la larghezza. 
\item Sostituita l'implementazione dell'istogramma raster con una basata su Qwt. Aggiunta opzione per salvare l'istogramma come file immagine. Mostra il valore attuale del pixel sull'asse x dell'istogramma raster. 
\item Aggiunta capacità di selezionare pixel in modo interattivo dalla vista per popolare la tabella della trasparenza nella finestra di dialogo delle proprietà raster.
\item Permette la creazione di scale di colore nella relativa casella combinata dei vettori.
\item Aggiunto pulsante "Gestore di stili" al selettore dei simboli per agevolare l'utente nella ricerca.
\end{itemize}

\textbf{Compositore di mappe}

\begin{itemize}[label=--]
\item Aggiunta capacità di mostrare e manipolare larghezza/altezza degli oggetti del compositore
nella scheda "Oggetto". 
\item Gli oggetti nel compositore si possono cancellare con il tasto backspace. 
\item Ordinamento per la tabella attributi del compositore (crescente/decrescente, per alcune colonne).
\end{itemize}

