%  !TeX  root  =  user_guide.tex  
\mainmatter
\pagestyle{scrheadings}
\addchap{Introduzione}\label{label_forward}
\pagenumbering{arabic}
\setcounter{page}{1}

% when the revision of a section has been finalized, 
% comment out the following line:
% \updatedisclaimer

Benvenuti nel meraviglioso mondo dei Sistemi Informativi Geografici (Geographical 
Information Systems, GIS). Quantum GIS (QGIS) è un Sistema Informativo Geografico 
a codice aperto (Open Source). Il progetto è nato nel maggio 2002 ed è stato 
ospitato su SourceForge nel giugno dello stesso anno. Abbiamo lavorato duramente 
per rendere il software GIS (che è tradizionalmente un software proprietario e costoso) 
una valida prospettiva per chiunque avesse disponibilità di un Personal Computer. 
QGIS gira attualmente su molte piattaforme Unix (incluso ovviamente Linux!), su 
Windows, e OS X. QGIS è sviluppato in Qt  (\url{http://qt.nokia.com}) e C++. 
Ciò rende QGIS reattivo e piacevole all’uso grazie all’interfaccia grafica 
(graphical user interface, GUI) semplice da usare.

QGIS si prefigge lo scopo di essere un GIS facile da usare, in grado di fornire 
funzioni e caratteristiche di uso comune. Inizialmente pensato come semplice
visualizzatore di dati GIS, attualmente QGIS ha raggiunto uno stato di maturità
tale da essere utilizzato da sempre più persone per il loro lavoro quotidiano in campo GIS. 
QGIS supporta nativamente un considerevole numero di formati raster e vettoriali: 
il supporto a nuovi formati dati è assicurato dall'uso di opportuni plugin.

QGIS è rilasciato con licenza GNU General Public License (GPL). Lo sviluppo di QGIS 
con questa licenza vi permette di esaminarne e modificarne il codice sorgente e vi
garantisce l'accesso ad un programma GIS esente da costi di licenza e  
liberamente modificabile. Dovreste aver ricevuto una copia completa della licenza 
con la vostra copia di QGIS, altrimenti potete trovarla nell’Appendice \ref{gpl_appendix}.  

\begin{Tip}\caption{\textsc{Documentazione aggiornata}}\index{documentation}
La versione più recente di questo documento è sempre reperibile all’indirizzo
\url{http://download.osgeo.org/qgis/doc/manual/}, o nell’area documentazione 
del sito di QGIS all’indirizzo \url{http://www.qgis.org/en/documentation/}
\end{Tip}

\addsec{Caratteristiche}\label{label_majfeat}

\qg offre molte funzionalità GIS di uso comune, sia nativamente che mediante 
plugin: è possibile offrirne una panoramica iniziale raggruppandole sinteticamente 
in sei categorie.

\minisec{Visualizzazione dei dati}

Si possono visualizzare e sovrapporre dati vettoriali e raster in diversi formati e proiezioni senza
necessità di conversioni verso un formato comune interno. Tra i formati supportati sono inclusi:

\begin{itemize}[label=--]
\item Tabelle con estensione spaziale usando PostGIS e SpatiaLite, formati vettoriali 
supportati dalla libreria OGR installata, inclusi gli Shapefile ESRI, i formati MapInfo, 
SDTS, GML e molti altri
\item Formati raster e immagine supportati dalla libreria GDAL (Geospatial Data 
Abstraction Library) installata, come GeoTiff, Erdas Img., ArcInfo Ascii Grid, 
JPEG, PNG e molti altri
\item Database SpatiaLite (Sezione \ref{label_spatialite}) 
\item Formati raster e vettoriali GRASS da database GRASS (location/mapset) (Sezione \ref{sec:grass})
\item Dati spaziali forniti da servizi di mappa online conformi agli standard OGC quali Web Map
Service (WMS) o Web Feature Service (WFS) (Sezione \ref{working_with_ogc})
\item Dati OpenStreetMap (Sezione \ref{plugins_osm}).
\end{itemize}

\minisec{Esplorazione dei dati e creazione di mappe} 

Si possono comporre mappe ed esplorare interattivamente dati spaziali tramite una gradevole 
interfaccia grafica. Tra i molti strumenti utili disponibili nell’interfaccia grafica sono inclusi:

\begin{itemize}[label=--]
\item riproiezione al volo
\item compositore di stampe
\item pannello vista panoramica
\item segnalibri geospaziali
\item identifica/seleziona elementi
\item modifica/visualizza/ricerca attributi
\item etichetta elementi
\item cambio della simbologia sia raster che vettoriale
\item aggiunta reticolo su un nuovo layer tramite il plugin fTools
\item decorazione della mappa con freccia del nord, barra di scala ed etichetta di copyright
\item salva e ricarica progetti
\end{itemize}

\minisec{Creazione, modifica, gestione ed esportazione di dati}

Possono essere creati, modificati, gestiti ed esportati dati vettoriali in molteplici formati. 
I dati raster devono essere importati in GRASS per poter essere modificati ed esportati in 
altri formati. QGIS offre le seguenti funzioni:

\begin{itemize}[label=--]
\item strumenti per digitalizzare formati supportati da OGR e layer vettoriali GRASS
\item creazione e modifica di shapefile e layer vettoriali GRASS
\item georeferenziazione di immagini con l’apposito plugin
\item strumenti GPS per l’importazione ed esportazione del formato GPX e conversione di altri formati
GPS al formato GPX o down/upload dei dati direttamente da unità GPS (su Linux, usb: è
stato aggiunto alla lista dei dispositivi GPS)
\item visualizzazione e modifica dei dati OpenStreetMap
\item creazione di layer PostGIS da shapefile grazie al plugin SPIT 
\item gestione migliorata delle tabelle PostGIS
\item gestione di tabelle degli attributi di dati vettoriali con il nuovo plugin Tabella attributi 
(Sezione \ref{sec:attribute table}) o il plugin Gestione Tabelle
\item salvare le schermate come immagini georeferenziate
\end{itemize}

\minisec{Analisi di dati} 

Possono essere eseguite analisi spaziali di dati PostgreSQL/PostGIS e di altri
formati supportati da OGR grazie al plugin python fTools. QGIS
offre attualmente strumenti per l'analisi, il campionamento, il geoprocessamento,
la gestione delle geometrie e del database di dati vettoriali. Possono
inoltre essere usati gli strumenti GRASS integrati, che includono
l'intera gamma delle funzioni di GRASS di oltre 300 moduli (Sezione \ref{sec:grass}).

\minisec{Pubblicazione di mappe su internet}

QGIS può essere usato per esportare dati in un mapfile che può essere
pubblicato su internet mediante un webserver sul quale sia installato
UMN MapServer. QGIS può anche essere impiegato come client WMS o
WFS e come server WMS. 

\minisec{Estensione delle funzioni di QGIS tramite plugin}

QGIS, grazie alla sua architettura estensibile, può essere adattato ad 
esigenze specifiche. QGIS fornisce librerie che possono essere usate per
creare i plugin o addirittura per creare nuovi programmi in C++ o Python.

\minisec{Plugin di base}

\begin{enumerate}
\item Aggiungi layer testo delimitato (carica e mostra file di testo delimitato
contenenti coordinate X e Y).
\item Cattura coordinate (cattura le coordinate del mouse usando un SR (Sistema di Riferimento) diverso).
\item Decorazioni (etichetta di copyright, freccia nord e barra di scala).
\item Diagramma sovrapposto (disegna diagrammi sopra layer vettoriali).
\item Plugin spostamento (gestisce automaticamente lo spostamento di punti che hanno la stessa posizione)
\item Convertitore Dxf2Shp (converte dal formato vettoriale dxf al formato vettoriale shp).
\item Strumenti GPS (carica ed importa dati GPS).
\item GRASS (integrazione del GIS GRASS).
\item GDALTools (integrazione di GDAL)
\item Georeferenziatore raster GDAL (aggiunge le informazioni sulla proiezione mediante GDAL).
\item Plugin interpolazione (interpolazione basata sui vertici di un layer vettoriale).
\item Carica raster PostGIS in QGIS
\item Esportazione verso MapServer (esporta un progetto QGIS in un mapfile di Mapserver).
\item OfflineEditing (modifica offline e sincronizzazione con il database)
\item Plugin OpenStreetMap (visualizza e modifica dati OpenStreetMap).
\item Oracle Spatial GeoRaster (accesso ai GeoRaster di Oracle Spatial).
\item Plugin Installer (scarica ed installa i plugin Python).
\item QSpatiaLite (interfaccia grafica per SpatiaLite)
\item Random HR - Animove (Animal Movements) (randomizzazione degli 'home ranges' all'interno di un'area di studio)
\item Raster terrain analysis (analisi geomorfologica raster)
\item Grafo strade (trova il percorso più breve)
\item SPIT (strumento per importare shapefile in PostGIS)
\item SQL Anywhere Plugin (salva vettori in un database SQL Anywhere)
\item Plugin interrogazione spaziale (effettua interrogazioni spaziali su dati vettoriali)
\item WFS Plugin (aggiunge un layer WFS alla mappa)
\item eVIS (uno strumento di visualizzazione di eventi)
\item fTools (strumento per la gestione e l'analisi di dati vettoriali)
\item Console Python (Accesso all’ambiente di gestione QGIS).
\end{enumerate}

\minisec{Plugin Python esterni}

La comunità di QGIS offre un numero crescente di plugin esterni scritti in python. Questi plugin sono 
ospitati sul repository ufficiale PyQGIS e si possono installare facilmente usando il 
Gestore di Plugin Python (Sezione \ref{sec:plugins}).

\subsubsection{Novità della versione \CURRENT} 

Questa versione rappresenta lo "stato dell'arte" della nostra serie di rilasci e come tale contiene nuove 
caratteristiche e funzionalità rispetto a QGIS 1.0.x e QGIS 1.6.0. 
Raccomandiamo l'uso di questa versione rispetto alle precedenti.

Questo rilascio include oltre 277 correzioni (bug fixes) e molte nuove funzionalità e miglioramenti.  

\minisec{Simbologia, etichette e diagrammi}

\begin{itemize}[label=--]
\item Nuova simbologia predefinita!
\item Sistema di grafici che utilizza lo stesso sistema di posizionamento intelligente delle etichette (nuova generazione).
\item Importazione ed esportazione della simbologia (nuova generazione).
\item Etichette per le regole degli stili tramite regole.
\item Il marcatore dei font può avere un offset  X e Y.
\item Simbologia linea:
\begin{itemize}[label=--]
\item Opzione per posizionare un marcatore sul punto centrale di una linea.
\item Opzione per posizionare un marcatore solo sul primo/ultimo vertice di una linea.
\item Il marcatore del layer simbolo linee può disegnare marcatori su ogni vertice.
\end{itemize}
\item Simbologia poligono:
\begin{itemize}[label=--]
\item Rotazione dei riempimenti svg.
\item Aggiunto il tipo layer del simbolo 'riempimento con centroide'.
\item Possibilità di usare i layer simbolo linea come profilo dei simboli poligono.
\end{itemize}
\item Etichette
\begin{itemize}[label=--]
\item Possibilità di impostare la distanza tra etichette in unità di mappa.
\item Strumento di modifica etichette Muovi/Ruota/Cambia per cambiare interattivamente 
le proprietà delle etichette basate su dati.
\end{itemize}
\item Nuovi strumenti
\begin{itemize}[label=--]
\item Aggiunta interfaccia grafica per gdaldem.
\item Aggiunto calcolatore campi con funzioni tipo \$x, \$y e \$perimetro.
\item Aggiunto 'Da linee a poligoni' nel menu vettore (fTools).
\item Aggiunto 'Poligoni di Voronoi' nel menu vettore (fTools). 
\end{itemize}
\end{itemize}

\minisec{Aggiornamenti interfaccia utente}

\begin{itemize}[label=--]
\item Gestione layer mancanti.
\item Zoom a gruppi di layer.
\item 'Suggerimento del giorno' all'avvio. I suggerimenti possono essere 
disabilitati/abilitati nel pannello delle opzioni.
\item Miglior organizzazione dei menu: aggiunto menu per i database.
\item Possibilità di mostrare il numero di elementi nelle classi di legenda.
Menu legenda accessibile mediante click tasto-destro.
\item Riordino generale e miglioramenti dell'usabilità.
\end{itemize}

\minisec{Gestione SR}

\begin{itemize}[label=--]
\item Mostra SR attivo nella barra di stato.
\item Possibilità di assegnare il SR di un layer al progetto (nel menu contestuale della legenda).
\item Selezione del SR predefinito per i nuovi progetti.
\item Possibilità di impostare il SR per più layer allo stesso tempo.
\item Ultima selezione predefinita come scelta del SR.
\end{itemize}

\minisec{Raster}

\begin{itemize}[label=--]
\item Aggiunti gli operatori AND e OR al calcolatore raster.
\item Aggiunta la proiezione al volo per i raster.

\item Migliore implementazione dei fornitori raster.
\item Aggiunta barra degli strumenti raster con la funzione di stiramento dell'istogramma.
\end{itemize}


\minisec{Fornitori e gestione dati}

\begin{itemize}[label=--]
\item Nuovo fornitore vettoriale SQLAnywhere.
\item Supporto per il join di tabelle.
\item Aggiornamenti modulo elemento.
\item Resa configurabile la rappresentazione delle stringhe con valore NULL.
\item Risolto aggiornamento modulo elemento da tabella attributi.
\item Aggiunto supporto per valori NULL nelle mappe valori.
\item Utilizzo del nome layer invece dell'id quando si caricano mappe valori da layer.
\item Supporto per campi espressione dei moduli elemento: le modifiche di linee nel modulo il cui nome inizia 
per 'expr\_' sono valutate. Il loro valore è interpretato come stringa del calcolatore di campi e rimpiazzato 
con il valore calcolato.
\item Supporto per la ricerca dei valori NULL nella tabella degli attributi.
\item Miglioramenti per la modifica degli attributi.
\item Migliorata in tabella la modifica interattiva degli attributi (aggiungi/modifica elementi, aggiorna attributo).
\item Possibilità di aggiungere elementi senza geometria.
\item Risolto annulla/ripristina per gli attributi.
\item Migliorata la gestione degi attributi.
\item Opzione per riusare l'ultimo valore attributo inserito per il successivo elemento digitalizzato.
\item Possibilità di unire/assegnare valori attributo ad un insieme di elementi.
\item Possibilità di usare il comando OGR 'salva come' senza attributi (es. DGN/DXF).
\end{itemize}

\minisec{API e Sviluppo}

\begin{itemize}[label=--]
\item Miglioramenti alla chiamata QgsFeatureAttribute.
\item Aggiunto QgsVectorLayer::featureAdded.
\item Aggiunta funzione menu layer.
\item Aggiunta opzione per caricare plugin c++ da cartelle specificate dall'utente. Richiede il riavvio dell'applicazione.
\item Nuovo strumento di controllo della geometria per fTools: più veloce e con messaggi di errore più rilevanti. 
Vedere la nuova funzione QgsGeometry.validateGeometry.
\end{itemize}

\minisec{QGIS Server}

\begin{itemize}[label=--]
\item Possibilità di specificare le capabilities nella sezione 'proprietà' del file di progetto.
\item Supporto per la stampa WMS con GetPrint-Request.
\end{itemize}

\minisec{Plugin}

\begin{itemize}[label=--]
\item Supporto per le icone dei plugin nel gestore di plugin.
\item Rimosso il plugin quickprint: utilizzare al suo posto il plugin easyprint.
\item Rimosso il plugin ogr convertor: utilizzare 'salva con nome...' dal menu contestuale.
\end{itemize}

\minisec{Stampa}

\begin{itemize}[label=--]
\item Supporto annulla/ripristina per il compositore di stampe.
\end{itemize}

\newpage