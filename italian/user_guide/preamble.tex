%  !TeX  root  =  user_guide.tex
\frontmatter
\pagestyle{scrplain}
\addchap{Premessa}
\vspace{1cm}

% when the revision of a section has been finalized, 
% comment out the following line:
%\updatedisclaimer

Questo documento costituisce la traduzione italiana dell’originale guida all’uso, installazione e
programmazione del programma Quantum GIS. Software e hardware citati in questo documento
sono per la maggior parte marchi registrati e quindi soggetti a restrizioni legali. Quantum GIS è 
soggetto alla GNU General Public License. Maggiori informazioni alla homepage di Quantum GIS
\url{http://www.qgis.org}.
\par\bigskip
Dettagli, dati, risultati ecc. presenti in questo documento sono stati scritti e verificati con la miglior
diligenza possibile da parte di autori ed editori. Non si escludono, tuttavia, errori inerenti il contenuto.
\par\bigskip
Di conseguenza nessun dato è da ritenere adatto ad alcuno scopo specifico né tanto meno viene
garantito. Gli autori e gli editori non si assumono alcuna responsabilità per eventuali danni e per le
loro conseguenze. Sono comunque ben accette le segnalazioni di possibili errori.
\par\bigskip
Questo documento è stato formattato con \LaTeX~ ed è disponibile sia come codice sorgente \LaTeX~scaricabile 
tramite \href{http://wiki.qgis.org/qgiswiki/DocumentationWritersCorner}{subversion} 
sia come documento PDF disponibile online all’indirizzo \url{http://qgis.osgeo.org/documentation/manuals.html}. 
Anche le versioni tradotte di questo documento possono essere scaricate dall’area 
documentazione del progetto QGIS. Per ulteriori informazioni sul come contribuire a questo 
documento e alla sua traduzione, si prega di visitare questo link: \url{http://www.qgis.org/wiki/} 

\vspace{1cm}
\noindent
\textbf{Collegamenti nel documento}
\par\bigskip
Questo documento contiene collegamenti interni ed esterni. Facendo click su un collegamento 
interno ci si muove all’interno del documento stesso, mentre facendo click su un collegamento 
esterno si aprirà un indirizzo internet. Nel formato PDF, i collegamenti interni sono mostrati 
in colore blu, mentre quelli esterni sono mostrati in colore rosso e gestiti dal browser di sistema. 
In formato HTML il browser mostra e gestisce in maniera identica entrambi i tipi di collegamento.

\newpage

\begin{flushleft}
\textbf{Autori ed editori della guida all’uso, installazione e programmazione:}
  \par\bigskip\noindent
\begin{tabular}{p{4cm} p{4cm} p{4cm}}
Tara Athan & Radim Blazek & Godofredo Contreras \\
Otto Dassau & Martin Dobias & Peter Ersts \\
Anne Ghisla & Stephan Holl & N. Horning \\
Magnus Homann & K. Koy & Lars Luthman \\ 
Werner Macho & Carson J.Q. Farmer & Tyler Mitchell \\
Claudia A. Engel & Brendan Morely & David Willis \\
Jürgen E. Fischer & Marco Hugentobler & Gavin Macaulay \\
Gary E. Sherman & Tim Sutton \\ \
\end{tabular}
\end{flushleft}

Si ringraziano Bertrand Masson per l’impaginazione, Tisham Dhar per aver preparato 
la documentazione iniziale dell’ambiente msys (MS Windows), Tom Elwertowski e William 
Kyngesburye per l’aiuto alla sezione di installazione su MAC OSX e Carlos Dàvila, 
Paolo Cavallini e Christian Gunning per le revisioni. Se avessimo dimenticato di 
menzionare qualche collaboratore, lo preghiamo di accettare le nostre scuse per la svista.
\par\bigskip\noindent
\textbf{Copyright \copyright~2004 - 2010 \QG Development Team}
\par\bigskip\noindent
\textbf{Internet :} \url{http://www.qgis.org}

\addsec{Licenza di questo documento}

È permessa la copia, distribuzione e/o modifica del presente documento sotto i termini della GNU
Free Documentation License, versione 1.3 o qualsiasi successiva versione pubblicata dalla Free 
Software Foundation; senza alcuna sezione invariante, senza testi di copertina e senza testi di retro copertina. 
Una copia della licenza è inclusa nella sezione \ref{label_fdl} entitled al titolo “GNU Free Documentation License”.

\newpage
