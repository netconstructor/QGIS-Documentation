%  !TeX  root  =  user_guide.tex

\chapter{QGIS Mapserver}\label{label_qgismapserver}
\index{WMS!QGISmapserver}

% when the revision of a section has been finalized,
% comment out the following line:
\updatedisclaimer

% Ripped off working_with_ogc.tex (SH: 20110330)
QGIS mapserver is an open source WMS 1.3 implementation which, in addition,
implements advanced cartographic features for thematic mapping. The QGIS
mapserver is a FastCGI/CGI (Common Gateway Interface) application written in
C++ that works together with a webserver (e.g. Apache, Lighttpd). 


It uses QGIS as backend for the GIS logic and for map rendering. Furthermore the 
Qt library is used for graphics and for platform independent 
C++ programming. In contrast to other WMS software, the QGIS mapserver uses 
cartographic rules in SLD/SE as a configuration language, both for the server 
configuration and for the user-defined cartographic rules. 

Moreover, the QGIS mapserver project provides the “Publish to Web” plugin, a 
plugin for QGIS desktop which exports the current layers and symbology as a 
web project for QGIS mapserver (containing cartographic visualisation rules 
expressed in SLD).

As QGIS desktop and QGIS mapserver use the same visualization libraries, the
maps that are published on the web look the same as in desktop GIS. The 
Publish to Web plugin currently supports basic symbolization, with more complex 
cartographic visualisation rules introduced manually. As the configuration is 
performed with the SLD standard and its documented extensions, there is only 
one standardised language to learn, which greatly simplifies the complexity 
of creating maps for the Web.

In one of the following manuals we will provide a sample configuration to 
set up a QGIS mapserver. But for now we recommend to read one of the following 
URLs to get more information:

\begin{itemize}
\item \url{http://karlinapp.ethz.ch/qgis\_wms/} \\
\item \url{http://www.qgis.org/wiki/QGIS\_mapserver\_tutorial} \\
\item \url{http://linfiniti.com/2010/08/qgis-mapserver-a-wms-server-for-the-masses/}
\end{itemize}


\FloatBarrier
