%  !TeX  root  =  user_guide.tex
\frontmatter
\pagestyle{scrplain}
\addchap{前言}
\vspace{1cm}

% when the revision of a section has been finalized, 
% comment out the following line:
%\updatedisclaimer

本文档是Quantum GIS软件的用户手册的最初版本。本文提及的大部分软件和硬件都注册了商标,因而使用需要法律授权。Quantum GIS软件依据GNU通用公共许可证发行。更多信息请访问Quantum GIS项目主页:
\url{http://www.qgis.org}.
\par\bigskip
尽管本文作者尽力地编写文档中的内容、数据和结果,编者也尽责地审核文档内容,但遗漏和错误还是可能在所难免。
\par\bigskip
因此,作者对文中的所有数据不承担任何责任,也不作任何保证。由于本文的疏误而导致的损失以及其他后果,本文作者、编者和出版者不承担任何责任。如果发现文中存在的错误,欢迎你向我们提出。
\par\bigskip
本文档使用 \LaTeX~进行排版。 \LaTeX~源文件可以通过 \href{http://wiki.qgis.org/qgiswiki/DocumentationWritersCorner}{subversion} 
获取,PDF文档可以在 \url{http://qgis.osgeo.org/documentation/manuals.html} 获取。 
本文档的各翻译版本可以在QGIS项目的文档区域下载。关于如何参与本文档的编写或翻译工作,更多内容请访问: \url{http://www.qgis.org/wiki/} 

\vspace{1cm}
\noindent
\textbf{文档中的链接}
\par\bigskip
文档中包含了大量内部链接和外部链接。点击内部链接可以在文档里跳转,而点击外部链接会打开一个互联网页面。在PDF格式的文档中,内部链接显示为蓝色,而外部链接显示为红色,且会调用系统默认浏览器打开。在HTML格式的文档中,浏览器会同样地处理和显示两种链接。

\newpage

\begin{flushleft}
\textbf{用户手册,安装与编程指南的作者及编者:}
  \par\bigskip\noindent
\begin{tabular}{p{4cm} p{4cm} p{4cm}}
Tara Athan & Radim Blazek & Godofredo Contreras \\
Otto Dassau & Martin Dobias & Peter Ersts \\
Anne Ghisla & Stephan Holl & N. Horning \\
Magnus Homann & K. Koy & Lars Luthman \\ 
Werner Macho & Carson J.Q. Farmer & Tyler Mitchell \\
Claudia A. Engel & Brendan Morely & David Willis \\
Jurgen E. Fischer & Marco Hugentobler & Gavin Macaulay \\
Gary E. Sherman & Tim Sutton \\ \
\end{tabular}
\end{flushleft}

感谢Bertrand Masson,他为排版做了很多工作;感谢Tisham Dhar,他收集准备了微软Windows环境的原始资料;感谢Tom Elwertowski和William Kyngesburye,他们帮助完成了MAC OSX系统上的安装说明部分;感谢Carlos Davila,Paolo
Cavallini和Christian Gunning,他们共同对本文进行了校正。如果此处对其他一些贡献者没有提及,我们愿致上最诚挚的歉意。
\par\bigskip\noindent
\textbf{\QG 开发团队 2004 - 2010 \copyright~版权所有}
\par\bigskip\noindent
\textbf{网站 :} \url{http://www.qgis.org}

\addsec{文档许可}

本文档的非固定部分、非封面文字和非封底文字可以在自由软件基金会发布的GNU开放文档许可证(1.3或以上版本)允许范围内自由复制、分发和/或修改。授权许可证的一个副本可参见 \ref{label_fdl} “GNU 开放文档许可证".

\newpage
