% vim: set textwidth=78 autoindent:
%  !TeX  root  =  user_guide.tex

\addchap{Gebrauch der Dokumentation}\label{label_conventions}

In diesem Abschnitt werden unterschiedliche Schreibstile vorgestellt, die
innerhalb der Dokumentation verwendet werden, um das Lesen intuitiver zu
machen.

\addsec{GUI Schreibstile}

Die Schreibstile der Grafischen Benutzeroberfl�che versuchen, das
Erscheinungsbild der GUI nachzuahmen. Allgemein soll der Benutzer dadurch
besser in der Lage sein, Elemente und Icons der GUI schneller mit den
Inhalten der Dokumentation zu verkn�pfen.

\begin{itemize}[label=--,itemsep=5pt]
\item Men� Optionen: \mainmenuopt{Layer} \arrow
\dropmenuopttwo{mActionAddRasterLayer}{Rasterlayer hinzuf�gen}

oder

\mainmenuopt{Einstellungen} \arrow
\dropmenuopt{Werkzeugkasten} \arrow \dropmenucheck{Digitalisierung}
\item Werkzeug: \toolbtntwo{mActionAddRasterLayer}{Rasterlayer hinzuf�gen}
\item Knopf : \button{Speicher als Standard}
\item Titel einer Dialogbox: \dialog{Layereigenschaften}
\item Reiter: \tab{Allgemein}

\item Werkzeugbox : \toolboxtwo{nviz}{nviz - �ffne 3D-Ansicht in NVIZ} 
\item Kontrollk�stchen: \checkbox{Darstellen}
\item Radioknopf:  \radiobuttonon{Postgis SRID} \radiobuttonoff{EPSG ID}
\item W�hle eine Zahl: \selectnumber{S�ttigung}{60}
\item W�hle ein Wort: \selectstring{Umrandungsoption}{--- durchg�ngig}
\item Suche nach einer Datei: \browsebutton 
\item W�hle eine Farbe: \selectcolor{Umrandungsfarbe}{yellow}
\item Schieberegler: \slider{Transparenz}
\item Eingabetext: \inputtext{Zeige Name}{lakes.shp}
\end{itemize}

Ein Schatten zeigt, dass dieses GUI Element mit der Maus anw�hlbar ist.

\addsec{Text oder Tastatur Schreibstile}

Die Dokumentation enth�lt Schreibstile, um eine bessere visuelle Verkn�pfung
mit bestimmten Textformen, Tastaturkommandos und Programmierelementen zu
erm�glichen.

\begin{itemize}[label=--]
%Use for all urls. Otherwise, it is not clickable in the document.
\item Querverweise: \url{http://qgis.org}
%\item Single Keystroke: press \keystroke{p}
\item Tastenkombination: Dr�cke \keystroke{Strg+B}, bedeutet, dr�cke und 
halte die Strg-Taste und dr�cke auf die B-Taste.
\item Name einer Datei: \filename{lakes.shp}
%\item Name of a Field: \fieldname{NAMES}
\item Name einer Klasse: \classname{NewLayer}
\item Name einer Methode: \method{classFactory}
\item Server: \server{myhost.de}
%\item SQL Table: \sqltable{example needed here}    
\item User Text: \usertext{qgis ---help}
\end{itemize}

Programmcode wird durch eine definierte Schrift und Schriftweite angezeigt:
\begin{verbatim}
PROJCS["NAD_1927_Albers",
  GEOGCS["GCS_North_American_1927",
\end{verbatim}

\addsec{Betriebssystemspezifische Anweisungen}

Einige Text- oder GUI-Anweisungen k�nnen sich f�r verschiedene
Betriebssysteme unterscheiden: Dr�cke \{\nix{}\win{Datei} \osx{QGIS}\} \arrow
\button{Beenden}, um QGIS zu schlie�en. Dieses Kommando bedeutet: QGIS wird unter
Linux, Unix und Windows beendet, indem man im Hauptmen� Datei auf Beenden
dr�ckt, w�hrend man unter Macintosh OSX im Hauptmen� QGIS auf Beenden dr�ckt.
L�ngere Texte k�nnen folgenderma�en formatiert sein:

\begin{itemize}[label=--]
\item \nix{mache dies;}
\item \win{klicke hier;}
\item \osx{dr�cke etwas anderes.}
\end{itemize}

oder als Paragraph.

\nix{} \osx{} Mache dies und das. Dann klicke dies und dies und dies und dies
und dies und dies und dies und dies und dies.

\win{} Mache dies. Dann dr�cke dies und dies und dies und dies und dies und
dies und dies und dies und dies und dies

Abbildungen innerhalb der Dokumentation k�nnen unter verschiedenen
Betriebssystemen erstellt worden sein. Das jeweilige Betriebssystem wird
dabei am Ende der Abbildungs�berschrift mit einem Icon angezeigt.

