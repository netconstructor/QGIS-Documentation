% vim: set textwidth=78 autoindent:

\subsection{Die Benutzung des Gitternetz Plugins}

Das Gitternetz Plugin erlaubt es, f�r unseren jeweils gew�hlten 
Kartenausschnitt ein 'Gitter' aus Punkten, Linien oder Polygonen zu erstellen. 
S�mtliche Einheiten m�ssen dabei in Dezimalgraden eingegben werden. Bei der 
Ausgabedatei handelt es sich um eine Shape-Datei, die direkt ('on the fly') 
und passend zu den anderen Daten projiziert werden kann.

\begin{figure}[ht]
\begin{center}
  \caption{Erstellung eines Gitternetz-Layers}\label{fig:graticule}\smallskip
  \includegraphics[scale=0.5]{graticule}
\end{center}
\end{figure}

Hier nun ein kurzes Beispiel f�r die Erstellung eines Gitternetzes:

\begin{enumerate}
\item Stellen Sie sicher, dass das Plugin geladen ist.
\item Klicken Sie auf das \texttt{Gitternetz} Werkzeug in der Plugin Funktionsleiste.
\item W�hlen Sie aus, ob das Gitter aus Punkten, Linien oder Polygonen bestehen soll.
\item Geben Sie die Breiten- und L�ngengrade f�r die obere linke und rechte Ecke des Gitters ein.
\item Geben Sie den Abstand an, der bei der Erstellung des Gitters verwendet werden soll. 
Sie k�nnen dabei unterschiedliche X und Y Werte verwenden (L�ngengrad, Breitengrad).
\item Geben Sie an, wo und unter welchem Namen die Shape-Datei erstellt werden soll.
\item Klicken Sie dann auf \texttt{OK} um das Gitternetz zu erstellen und f�gen Sie dieses abschlie�end dem Kartenausschnitt hinzu.
\end{enumerate} 

