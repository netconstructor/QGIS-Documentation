\section{Hilfe und Support}\label{label_helpsupport}

\subsection{Mailinglisten}
QGIS entwickelt sich st�ndig weiter, daher kann es vorkommen, dass es mal nicht
so funktioniert, wie erwartet. Die bevorzugte und effektivste Art, Hilfe zu
bekommen, besteht darin, sich in die qgis-users Mailingliste einzuschreiben. 

\minisec{qgis-users}
Ihre Fragen erreichen eine breite Basis von Anwendern und die Antworten auf Ihre
Fragen k�nnen auch anderen helfen. Sie k�nnen sich in die Mailingliste eintragen
unter der URL: \\
\url{http://lists.qgis.org/cgi-bin/mailman/listinfo/qgis-user}

\minisec{qgis-developer}
Wenn Sie ein Entwickler sind und technische Probleme haben, k�nnen Sie sich in
die qgis-developer Mailingliste eintragen unter der URL: \\
\url{http://lists.qgis.org/cgi-bin/mailman/listinfo/qgis-developer}

\minisec{qgis-commit}
Jedes Mal wenn ein Commit in den QGIS Quellcode erfolgt ist, wird eine Email an
folgende Liste geschickt. Wenn Sie alle Ver�nderungen im Quellcode verfolgen
m�chten, k�nnen Sie sich unter folgender URL eintragen:\\
\url{http://lists.qgis.org/cgi-bin/mailman/listinfo/qgis-commit}

\minisec{qgis-trac}
Diese Mailingliste stellt Nachrichten in Bezug auf das Projekt Management
bereit. Dazu geh�ren Fehlerberichte, Aufgaben und Anfragen f�r neue Funktionen.
Sie k�nnen sich f�r diese Mailingliste eintragen unter der URL: \\ 
\url{http://lists.qgis.org/cgi-bin/mailman/listinfo/qgis-trac}

\minisec{qgis-doc}
Diese Liste besch�ftigt sich mit Themen wie der Dokumentation, der inhaltlichen
Hilfe von QGIS, dem Benutzer- und Installationshandbuch sowie �bersetzungen in verschiedene Sprachen
Wenn Sie mit an dem n�chsten Benutzer- und Installationshandbuch arbeiten
m�chten oder es �bersetzen wollen, k�nnen Sie sich f�r diese Mailingliste eintragen unter der URL: \\
\url{http://lists.qgis.org/cgi-bin/mailman/listinfo/qgis-doc}

\minisec{qgis-psc}
Diese Liste befasst sich mit den Aufgaben des Lenkungsausschusses des
QGIS-Projektes. Sie k�nnen sich f�r diese Mailingliste eintragen unter der URL: \\
\url{http://mrcc.com/cgi-bin/mailman/listinfo/qgis-psc}

Wir heissen Sie herzlich Willkommen, sich auf jeder dieser Listen einzuschreiben
und den anderen QGIS Benutzern und Entwicklern mit ihrer Erfahrung zu helfen.
Beachten Sie bitte auch, dass die Mailinglisten qgis-commit und qgis-trac nur
dazu erstellt wurden, um Benachrichtigungen zu verteilen und nicht f�r
Anwenderfragen geeignet ist.

\subsection{IRC}
Wir sind au�erdem im IRC pr�sent - Sie k�nnen uns im  \#qgis Kanal unter
\url{irc.freenode.net} treffen. Bitte warten Sie ein wenig auf Antworten, da die
meisten nur zwischendurch mal vorbeischauen, was gerade so passiert.
Kommerzieller Support ist auch m�glich. Schauen Sie dazu auf die Internetseite
\url{http://qgis.org/content/view/90/91}.

Wenn Sie eine Disussion im IRC verpasst haben - kein Problem! Wir loggen alle
Diskussionen, damit Sie diese auch sp�ter lesen k�nnen. Lesen Sie die Logs unter
der URL:  \url{http://logs.qgis.org}.

\subsection{BugTracker}
W�hrend die qgis-users Mailingliste n�tzlich ist, wenn es um allgemeine Fragen
zu 'wie mache ich dies oder jenes in QGIS' geht, m�chten Sie uns vielleicht auch
auf richtige Fehler (Bugs) aufmerksam machen. Sie k�nnen dazu Fehlermeldungen mit
Hilfe des QGIS BugTracker unter \url{http://svn.qgis.org/trac} erstellen. Wenn
Sie ein neues Ticket f�r einen Fehler erstellen, geben Sie bitte auch eine
Emailadresse an, �ber die wir weitere Informationen von Ihnen erfragen k�nnen.

Denken Sie auch bitte daran, dass ein f�r Sie wichtiger Fehler nicht immer die
gleiche Priorit�t bei anderen Personen und besonders den Entwicklern hat. Einige
Fehler sind sehr aufwendig zu reparieren und daher kann es schon mal ein wenig
dauern, bis gen�gend Zeit vorhanden ist, ein Problem zu l�sen.

Anfragen f�r neue Funktionen k�nnen auch in demselben System gestellt werden.
Bitte geben Sie dann den Typ \textbf{enhancement} an.

Wenn Sie einen Fehler gefunden und selbst repariert haben, k�nnen Sie es auch
als Patch schicken. Der BugTracker bietet auch hierf�r einen Typ \textbf{patch}.
Die Entwickler schauen sich den Patch dann an und f�gen Ihn zum Quellcode hinzu.
Bitte haben Sie ein wenig Geduld, wenn dies ein wenig dauert.

% unused, since community.qgis.org seems to be lost. (SH)
% There is also a community site for QGIS where we encourage QGIS users to share
% their experiences and provide case studies about how they are using QGIS. The
% community site is available at: http://community.qgis.org 

\subsection{Blog}
Die QGIS-Gemeinschaft stellt auch einen Weblog (BLOG) unter
\url{http://blog.qgis.org} bereit, mit vielen interessanten Artikeln f�r
Anwender und Entwickler zum Thema QGIS. Sie sind herzlich eingeladen, selbst
einen Artikel zu verfassen, nachdem Sie sich auf der Seite registriert haben!

\subsection{Wiki}
Schlie�lich gibt es auch ein QGIS WIKI unter \url{http://wiki.qgis.org}, wo Sie
eine Vielzahl n�tzlicher Informationen �ber die QGIS-Entwicklung, Pl�ne f�r neue
Versionen, Links zum Herunterladen von Daten oder zu vorhandenen �bersetzungen
finden. Schauen Sie mal rein, da gibt es ein paar wirkliche Attraktionen!

