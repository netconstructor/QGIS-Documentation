% vim: set textwidth=78 autoindent:

\section{Vorwort}\label{label_forward}
\pagenumbering{arabic}
\setcounter{page}{1}

Willkommen in der wunderbaren Welt der Geographischen Informationssysteme (GIS)!
Quantum GIS ist ein Freies (Open Source) GIS. Die Idee zu dem Projekt wurde im
Mai 2002 geboren und bereits im Juni desselben Jahres bei SourceForge etabliert.  
Wir haben hart daran gearbeitet, traditionell sehr teure GIS Software kostenfrei 
f�r jeden, der Zugang zu einem PC hat, bereitzustellen. 
QGIS kann unter den meisten Unices, Windows und MacOSX betrieben werden. QGIS
wurde mit Hilfe des Qt toolkit (\url{http://www.trolltech.com}) und C++
entwickelt. Dadurch ist QGIS sehr benutzerfreundlich und besitzt eine einfach zu
bedienende und intuitive grafische Benutzeroberfl�che. 

QGIS soll ein einfach zu benutzendes GIS sein und grundlegende
GIS-Funktionalit�ten bieten. Das anf�ngliche Ziel bestand darin, einen 
einfachen Geo-Datenviewer zu entwickeln. Dieses Ziel wurde bereits mehr als 
erreicht, so dass QGIS mittlerweile von vielen Anwendern f�r ihre t�gliche 
Arbeit eingesetzt wird. 
QGIS unterst�tzt eine Vielzahl von Raster- und Vektorformaten. Mit Hilfe der
Plugin-Architektur k�nnen weitere Funktionalit�ten einfach erg�nzt werden (vgl. 
Appendix \ref{appdx_data_formats} f�r eine vollst�ndige Liste derzeit
unterst�tzter Datenformate).

QGIS wird unter der GNU Public License (GPL) herausgegeben. F�r die Entwicklung
des Programms bedeutet dies das Recht, den Quellcode einzusehen und
entsprechend der Lizens ver�ndern zu d�rfen. F�r die Anwendung der Software ist
damit garantiert, dass QGIS kostenfrei aus dem Internet heruntergeladen, genutzt
und weitergegeben werden kann. Eine vollst�ndige Kopie der Lizens ist dem
Programm beigef�gt und kann auch im Appendix \ref{gpl_appendix} eingesehen
werden.   

\begin{quote}
\begin{center}
\textbf{Bemerkung:} Die aktuellste englische Version dieser Dokumentation finden Sie hier: 
\newline http://qgis.org/docs/userguide.pdf 
\end{center}
\end{quote}

\subsection{Funktionalit�ten}\label{label_majfeat}

Quantum GIS bietet zahlreiche GIS Funktionalit�ten. Die wichtigsten sind hier
aufgelistet, unterteilt in Kernfunktionalit�ten und Plugins.\\

\textbf{Kernfunktionalit�ten}

\begin{itemize}
\item Raster- und Vektorsupport �ber die OGR-Bibliothek
\item Support f�r PostgreSQL/PostGIS Datenbanken
\item GRASS Integration zum Visualisieren, Editieren und Analysieren von GRASS Daten
\item Digitalisieren von GRASS und OGR/Shapefiles
\item Map Composer zum Erstellen von druckfertigen Kartenlayouts
\item OGC-Support
\item �bersichtsfenster
\item R�umliche Bookmarks
\item Identifizieren und Selektieren von Objekten
\item Editiren, Visualisieren und Suchen nach Attributdaten
\item Labeln von Objekten
\item 'On the fly' Projektion
\item Speichern und �ffnen von Projekten
\item Export als Mapserver Mapfile
\item �ndern der Vektor- und Rasterlayereigenschaften
\item Erweiterbare Plugin-Architektur
\end{itemize}

\textbf{Plugins}

\begin{itemize}
\item WFS Layer hinzuf�gen
\item ASCII Tabellen als Layer hinzuf�gen
\item Copyright Label, Nordpfeil und Ma�stabsleiste hinzuf�gen
\item Georeferenzierung
\item GPS Tools
\item GRASS-Anbindung
\item Erstellen eines Gradnetzes
\item PostgreSQL Geoprocessing Funktionalit�ten
\item SPIT (Shapefile nach PostgreSQL/PostGIS importieren)
\item Python-Konsole
\item openModeller
\end{itemize}

\subsection{Was is neu in 0.9}\label{label_whatsnew}

Wie immer gibt es eine Reihe neuer, interessanter Funktionen in Version 0.9.

\begin{itemize}
\item Python bindings - Dies war das Hauptaugenmerk f�r diese Ver�ffentlichung. Es ist nun m�glich, Plugins in Python zu erstellen. Ausserdem k�nnen nun GIS-unterst�tzende Applikation in Python geschrieben werden, die auf die QGIS-Bibliotheken zugreifen.
\item Das automake build system wurde entfernt - QGIS braucht nun CMake f�r das Kompilieren.
\item Viele neue GRASS Module in der GRASS Toolbox (dank an http://faunalia.it/)
\item Updates f�r den Map Composer
\item Crash fix f�r 2.5D Shapefiles
\item Optimierung des Georeferencer Plugins.
\item Mittlerweile werden 26 Sprachen unterst�tzt    
\end{itemize}

F�r QGIS Version \CURRENT stand Stabilisierung und Funktionsoptimierung im Vordergrund.

\begin{itemize}
\item 66 Bugfixes und Verbesserungen von Funktionen 
\item Automatische Anordnung der Kartenfenster beim Georeferencer Plugin
\item Einstellungsm�glichkeit der Sprache im Men� Optionen
\item Informationen �ber den Fortschritt des Datentransfers f�r WMS und WFS Daten
\item Integration weiterer GRASS Module in die GRASS Toolbox
\end{itemize}
