%\section{Getting Started}\label{label_getstarted}
%
%This chapter gives a quick overview of installing QGIS, some sample 
%data from the QGIS web page and running a first and simple session 
%visualizing raster and vector layers.

\section{Premiers Pas}\label{label_getstarted}

Ce chapitre donne un aper\c{c}u rapide de l'installation de QGIS, de quelques \'echantillons de donn\'ees provenant du site internet et du lancement d'une premi\`ere session d'affichage de couches raster et vecteur.

%\subsection{Installation}\label{label_installation}
%\index{installation}
%
%Installation of QGIS is very simple. Standard installer packages are
%available for MS Windows and Mac OS X. For many flavors of GNU/Linux binary
%packages (rpm and deb) or software repositories to add to your installation
%manager are provided. Get the latest information on binary packages at the
%QGIS website at \url{http://qgis.osgeo.org/download/}.
%
%If you need to build QGIS from source, this is documentated in Appendix
%\ref{sec:install_windows} for MS Windows \win, Appendix
%\ref{sec:install_macosx} for Mac OSX \osx and Appendix
%\ref{sec:install_linux} for GNU/Linux \nix. The Installation instructions are
%distributed with the QGIS source code and also available at
%\url{http://qgis.osgeo.org}.

\subsection{Installation}\label{label_installation} \index{installation}

L'installation de QGIS est tr\`es simple, des installateurs sont disponibles pour Windows et Mac OS X. Beaucoup de distributions Linux mettent \`a disposition des fichiers binaires (.rpm ou .deb) via leurs interfaces de gestion de logiciels. Obtenez les derni\`eres informations concernant les paquets binaires sur le site de QGIS sur \url{http://qgis.osgeo.org/download/}.

Si vous avez besoin de compiler QGIS depuis les sources, le processus est document\'e dans l'Annexe \ref{sec:install_windows} pour MS Windows \win, Annexe \ref{sec:install_macosx} pour Mac OSX \osx, Annexe \ref{sec:install_linux} pour GNU/Linux \nix. Les instructions d'installations sont distribu\'ees avec le code source, mais aussi sur \url{http://qgis.osgeo.org}.

%\subsection{Sample Data}\label{label_sampledata}
%\index{data!sample} 
%
%The user guide contains examples based on the QGIS sample dataset. 
%
%\win The Windows installer has an option to download the QGIS sample dataset.
%If checked, the data will be downloaded to your \filename{My Documents}
%folder and placed in a folder called \filename{GIS Database}. 
%You may use Windows Explorer to move this folder to any convenient location.
%If you did not select the checkbox to install the sample dataset
%during the initial QGIS installation, you can either
%\begin{itemize}
%\item use GIS data that you already have;
%\item download the sample data from the QGIS website
% \url{http://qgis.osgeo.org/download}; or
%\item uninstall QGIS and reinstall with the data download option checked.
%\end{itemize}
%
%\nix \osx For GNU/Linux and Mac OSX there are not yet dataset installation
%packages available as rpm, deb or dmg. To use the sample dataset download the
%file \filename{qgis\_sample\_data} as ZIP or TAR archive from
%\url{http://download.osgeo.org/qgis/data/} and unzip or untar the archive on
%your system. The Alaska dataset includes all GIS data that are used as
%examples and screenshots in the user guide, and also includes a small GRASS
%database. The projection for the QGIS sample dataset is Alaska Albers Equal
%Area with unit feet. The EPSG code is 2964.

\subsection{