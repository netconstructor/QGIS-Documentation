% \section{GRASS Toolbox modules}\label{appdx_grass_toolbox_modules}
\section{Modules de la bo\^ite \`a outils de GRASS }\label{appdx_grass_toolbox_modules}

% when the revision of a section has been finalized, comment out the following line:
% \updatedisclaimer

% The GRASS Shell inside the GRASS Toolbox provides access to almost all (more than 300) GRASS modules in command line modus. To offer a more user friendly working environment, about 200 of the available GRASS modules and functionalities are also provided by graphical dialogs.
La console GRASS dans la bo\^ite \`a outils GRASS permet d'acc\'eder \`a quasiment tous (plus de 300) les modules de GRASS au moyen de la ligne de commande. Pour offrir un environnement de travail plus ergonomique, \`a peu pr\`es 200 des modules et fonctionnalit\'es de GRASS disponibles sont aussi disponibles par des bo\^ites de dialogue graphique.

% \subsection{GRASS Toolbox data import and export modules}\index{GRASS!toolbox!modules}
\subsection{Modules d'import et d'export de GRASS de la bo\^ite \`a outils}\index{GRASS!toolbox!modules}

% This Section lists all graphical dialogs in the GRASS Toolbox to import and export data into a currently selected GRASS location and mapset.
Cette section liste toutes les bo\^ites de dialogue de la bo\^ite \`a outils de GRASS pour importer et exporter des donn\'ees dans une location et jeu de donn\'ees pr\'ealablement s\'electionn\'es dans GRASS.

\begin{table}[ht]
\centering
% \caption{GRASS Toolbox: Data import modules}\medskip
\caption{Bo\^ite \`a outils GRASS : modules d'import de donn\'ees}\medskip
 \begin{tabular}{|p{4cm}|p{12cm}|}
%   \hline \multicolumn{2}{|c|}{\textbf{Data import modules in the GRASS Toolbox}} \\ 
  \hline \multicolumn{2}{|c|}{\textbf{Modules d'import de donn\'ees dans la bo\^ite \`a outils de GRASS}} \\ 
%   \hline \textbf{Module name} & \textbf{Purpose} \\
  \hline \textbf{Nom du module} & \textbf{Objectif} \\
%   \hline r.in.arc & Convert an ESRI ARC/INFO ascii raster file (GRID) into a (binary) raster map layer\\
  \hline r.in.arc & Convertit un fichier raster ascii ARC/INFO d'ESRI (GRID) en une couche raster (binaire) \\
%   \hline r.in.ascii & Convert an ASCII raster text file into a (binary) raster map layer \\
  \hline r.in.ascii & Convertit un fichier raster texte ASCII en une couche raster (binaire)\\
%   \hline r.in.aster & Georeferencing, rectification, and import of Terra-ASTER imagery and relative DEM's using gdalwarp \\
  \hline r.in.aster & Georeferencement, rectification, et import d'image Terra-ASTER et des MNT relatif en utilisant gdalwarp \\
%   \hline r.in.gdal &  Import GDAL supported raster file into a GRASS binary raster map layer \\
  \hline r.in.gdal &  Importe un fichier raster g\'er\'e par GDAL dans une couche raster binaire de GRASS\\
%   \hline r.in.gdal.loc &  Import GDAL supported raster file into a GRASS binary raster map layer and create a fitted location \\
  \hline r.in.gdal.loc &  Importe un fichier raster g\'er\'e par GDAL dans une couche raster binaire de GRASS et cr\'eer une r\'egion lui correspondant\\
%   \hline r.in.gridatb & Imports GRIDATB.FOR map file (TOPMODEL) into GRASS raster map \\
  \hline r.in.gridatb & Importe un fichier GRIDATB.FOR (TOPMODEL)dans une couche raster de GRASS\\
%   \hline r.in.mat  & Import a binary MAT-File(v4) to a GRASS raster  \\
  \hline r.in.mat  & Importe un fichier binaire MAT-File(v4) dans une couche  raster GRASS \\
%   \hline r.in.poly  &  Create raster maps from ascii polygon/line data files in the current directory \\
  \hline r.in.poly  & Cr\'ee des couches raster \`a partir de fichiers ascii de donn\'ees polygonales/lin\'eairesdans le r\'epertoire s\'electionn\'e \\
%   \hline r.in.srtm  & Import SRTM HGT files into GRASS \\
  \hline r.in.srtm  & Importe des fichiers SRTM HGT dans GRASS \\
%   \hline i.in.spotvgt & Import of SPOT VGT NDVI file into a raster map \\
  \hline i.in.spotvgt & Importe de fichier NDVI VGT de SPOT dans une couche raster \\
%   \hline v.in.dxf & Import DXF vector layer \\
  \hline v.in.dxf & Importe une couche vecteur DXF \\
%   \hline v.in.e00 & Import ESRI E00 file in a vector map \\
  \hline v.in.e00 & Importe un fichier ESRI E00 dans une couche vecteur \\
%   \hline v.in.garmin & Import vector from gps using gpstrans \\
  \hline v.in.garmin & Importe un vecteur \`a partir d'un GPS en utilisant gpstrans \\
%   \hline v.in.gpsbabel & Import vector from gps using gpsbabel \\
  \hline v.in.gpsbabel & Importe un vecteur \`a partir d'un GPS en utilisant gpsbabel \\
%   \hline v.in.mapgen & Import MapGen or MatLab vectors in GRASS \\
  \hline v.in.mapgen & Importe des vecteurs MapGen ou MatLab dans GRASS \\
%   \hline v.in.ogr & Import OGR/PostGIS vector layers \\
  \hline v.in.ogr & Importe des couches vectorieles OGR/PostGIS \\
%   \hline v.in.ogr.loc & Import OGR/PostGIS vector layers and create a fitted location\\
  \hline v.in.ogr.loc & Importe des couches vectorielles OGR/PostGIS et cr\'eer une r\'egion leurs correspondants\\
%   \hline v.in.ogr.all & Import all the OGR/PostGIS vector layers in a given data source \\
  \hline v.in.ogr.all & Importe toutes les couches vectorielles OGR/PostGIS dans une source de donn\'ees d\'efinit \\
%   \hline v.in.ogr.all.loc & Import all the OGR/PostGIS vector layers in a given data source and create a fitted location \\
  \hline v.in.ogr.all.loc & Importe toutes les couches vectorielles OGR/PostGIS dans une source de donn\'ees d\'efinit et cr\'eer une r\'egion leurs correspondant\\
\hline
\end{tabular}
\end{table}

\begin{table}[ht]
\centering
% \caption{GRASS Toolbox: Data export modules}\medskip
\caption{Bo\^ite \`a outils GRASS : modules d'export de donn\'ees}\medskip
 \begin{tabular}{|p{4cm}|p{12cm}|}
%   \hline \multicolumn{2}{|c|}{\textbf{Data export modules in the GRASS Toolbox}} \\
  \hline \multicolumn{2}{|c|}{\textbf{Modules d'export des donn\'ees dans la bo\^ite \`a outils de GRASS}} \\
%   \hline \textbf{Module name} & \textbf{Purpose} \\
  \hline \textbf{Nom du module} & \textbf{Objectif} \\
%   \hline r.out.gdal.gtiff & Export raster layer to Geo TIFF \\
  \hline r.out.gdal.gtiff & Exporte une couche raster en Geo TIFF \\
%   \hline r.out.arc & Converts a raster map layer into an ESRI ARCGRID file \\
  \hline r.out.arc & Convertit une couche raster dans un fichier ARCGRID d'ESRI \\
%   \hline r.gridatb & Exports GRASS raster map to GRIDATB.FOR map file (TOPMODEL) \\
  \hline r.gridatb & Exporte une couche raster GRASS en fichier GRIDATB.FOR (TOPMODEL) \\
%   \hline r.out.mat & Exports a GRASS raster to a binary MAT-File \\
  \hline r.out.mat & Exporte un raster GRASS en un fichier binaire MAT-File \\
%   \hline r.out.bin & Exports a GRASS raster to a binary array \\
  \hline r.out.bin & Exporte un raster GRASS en tableau binaire \\
%   \hline r.out.png & Export GRASS raster as non-georeferenced PNG image format \\
  \hline r.out.png & Exporte un raster GRASS dans une image non g\'eor\'ef\'erenc\'e au format PNG \\
%   \hline r.out.ppm & Converts a GRASS raster map to a PPM image file at the pixel resolution of the CURRENTLY DEFINED REGION \\
  \hline r.out.ppm & Convertit une couche raster GRASS dans un fichier image PPM \`a la r\'esolution du pixel de CURRENTLY DEFINED REGION \\
%   \hline r.out.ppm3 & Converts 3 GRASS raster layers (R,G,B) to a PPM image file at the pixel resolution of the CURRENTLY DEFINED REGION \\
  \hline r.out.ppm3 & Convertit 3 couches raster GRASS (R,G,B) dans un fichier image PPM \`a la r\'esolution du pixel CURRENTLY DEFINED REGION \\
%   \hline r.out.pov & Converts a raster map layer into a height-field file for POVRAY\\
  \hline r.out.pov & Convertit une couche raster dans un fichier avec un champ poids pour POVRAY\\
%   \hline r.out.tiff & Exports a GRASS raster map to a 8/24bit TIFF image file at the pixel resolution of the currently defined region\\
  \hline r.out.tiff & Exporte une couche raster GRASS dans une image TIFF de 8/24bit \`a la r\'esolution du pixel de la r\'egion s\'electionn\'ee\\
%   \hline r.out.vrml &  Export a raster map to the Virtual Reality Modeling Language (VRML)\\
  \hline r.out.vrml & Exporte une couche raster dans le format Virtual Reality Modeling Language (VRML)\\
%   \hline v.out.ogr & Export vector layer to various formats (OGR library) \\
  \hline v.out.ogr & Exporte une couche vecteur dans diff\'erents formats (biblioth\`eque OGR) \\
%   \hline v.out.ogr.gml & Export vector layer to GML \\
  \hline v.out.ogr.gml & Exporte une couche vectorielle en GML \\
%   \hline v.out.ogr.postgis & Export vector layer to various formats (OGR library) \\
  \hline v.out.ogr.postgis & Exporte une couche vectorielle en diff\'erents formats (biblioth\`eque OGR) \\
%   \hline v.out.ogr.mapinfo & Mapinfo export of vector layer \\
  \hline v.out.ogr.mapinfo & Export au format Mapinfo d'une couche vectorielle\\
%   \hline v.out.ascii & Convert a GRASS binary vector map to a GRASS ASCII vector map  \\
  \hline v.out.ascii & Convertit une couche vecteur binaire de GRASS en une couche vectorielle ASCII de GRASS\\
%   \hline v.out.dxf & converts a GRASS vector map to DXF  \\
  \hline v.out.dxf & Convertit un vecteur de GRASS en DXF \\
\hline
\end{tabular}
\end{table}

% \subsection{GRASS Toolbox data type conversion modules}
\subsection{Modules de convertion de type de donn\'ees de la bo\^ite \`a outils de GRASS}

This Section lists all graphical dialogs in the GRASS Toolbox to convert raster to vector or vector to raster data in a currently selected GRASS location and mapset.

\begin{table}[ht]
\centering
% \caption{GRASS Toolbox: Data type conversion modules}\medskip
\caption{bo\^ite \`a outils de GRASS : modules de conversion detype de donn\'ees}\medskip
 \begin{tabular}{|p{4cm}|p{12cm}|}
%   \hline \multicolumn{2}{|c|}{\textbf{Data type conversion modules in the GRASS Toolbox}} \\
  \hline \multicolumn{2}{|c|}{\textbf{Modules de conversion de types de donn\'ees dans la bo\^ite \`a outils de GRASS}} \\
%   \hline \textbf{Module name} & \textbf{Purpose} \\
  \hline \textbf{Nom du module} & \textbf{Objectif} \\
%   \hline r.to.vect.point & Convert a raster to vector points \\
  \hline r.to.vect.point & Convertit un raster en points vectoriels \\
%   \hline r.to.vect.line & Convert a raster to vector lines \\
  \hline r.to.vect.line & Convertit un raster en lignes vectorielles \\
%   \hline r.to.vect.area & Convert a raster to vector areas \\
  \hline r.to.vect.area & Convertit un raster en polygones vectoriels \\
%   \hline v.to.rast.constant & Convert a vector to raster using constant \\
  \hline v.to.rast.constant & Convertit un vecteur en raster en utilisant une constante \\
%   \hline v.to.rast.attr & Convert a vector to raster using attribute values \\
  \hline v.to.rast.attr & Convertit un vecteur en raster en utilisant des valeurs attributaires\\
\hline
\end{tabular}
\end{table}

% \subsection{GRASS Toolbox region and projection configuration modules}
\subsection{Modules de configuration de la projections et de la r\'egion de la bo\^ite \`a outils de GRASS }

% This Section lists all graphical dialogs in the GRASS Toolbox to manage and change the current mapset region and to configure your projection.
Cette section liste tous les bo\^ites de dialogue dans la bo\^ite \`a outils de GRASS pour g\'erer et modifier la r\'egion du jeu de donn\'ees s\'electionn\'e et de configurer la projection.

\begin{table}[ht]
\centering
% \caption{GRASS Toolbox: Region and projection configuration modules}\medskip
\caption{Bo\^ite \`a outils de GRASS : modules de configuration de la projection et de la r\'egion}\medskip
 \begin{tabular}{|p{4cm}|p{12cm}|}
%   \hline \multicolumn{2}{|c|}{\textbf{Region and projection configuration modules in the GRASS Toolbox}} \\
  \hline \multicolumn{2}{|c|}{\textbf{Modules de configuration de la projection et de la r\'egion de la bo\^ite \`a outils de GRASS}} \\
%   \hline \textbf{Module name} & \textbf{Purpose} \\
  \hline \textbf{Nom du module} & \textbf{Objectif} \\
%   \hline g.region.save & Save the current region as a named region \\
\hline g.region.save & Sauve la r\'egion actuelle dans la r\'egion nomm\'ee \\
%   \hline g.region.zoom & Shrink the current region until it meets non-NULL data from a given raster map \\
  \hline g.region.zoom & R\'eduction de la r\'egion courante jusqu'\`a ce qu'il renvoie des donn\'ees non-NULL \`a partir d'une carte raster \\
  \hline g.region.multiple.raster & D\'efini la r\'egion pour correspondre \`a de multiples couches raster \\
%   \hline g.region.multiple.vector & Set the region to match multiple vector maps \\
  \hline g.region.multiple.vector & D\'efinis la r\'egion pour correspondre \`a de multiples couches vecteur \\
%   \hline g.proj.print & Print projection information of the current location\\
  \hline g.proj.print & Affiche des informations de la projection de la localisation actuelle \\
%   \hline g.proj.geo & Print projection information from a georeferenced file (raster, vector or image)\\
  \hline g.proj.geo & Affiche des informations de la projection \`a partir d'un fichier g\'eor\'ef\'erenc\'e (raster, vecteur ou image)\\
%   \hline g.proj.ascii.new & Print projection information from a georeferenced ASCII file containing a WKT projection description\\
  \hline g.proj.ascii.new & Affiche des informations de la projection \`a partir d'un fichier ASCII g\'eor\'ef\'erenc\'e contenant une description WKT de la projection\\
%   \hline g.proj.proj & Print projection information from a PROJ.4 projection description file\\
  \hline g.proj.proj & Affiche des informations de la projection \`a partir d'un fichier de description de la projection PROJ.4 \\
%   \hline g.proj.ascii.new & Print projection information from a georeferenced ASCII file containing a WKT projection description and create a new location based on it\\
  \hline g.proj.ascii.new & Affiche des informations de la projection \`a partir d'un fichier ASCII g\'eor\'ef\'erenc\'e contenant une description WKT de la projection et cr\'e\'e une nouvelle location bas\'ee sur celui-ci \\
%   \hline g.proj.geo.new & Print projection information from a georeferenced file (raster, vector or image) and create a new location based on it\\
  \hline g.proj.geo.new &  Affiche des informations de la projection \`a partir d'un fichier g\'eor\'ef\'erenc\'e (raster, vecteur ou image) et cr\'e\'e une nouvelle location bas\'e sur celui-ci \\
%   \hline g.proj.proj.new & Print projection information from a PROJ.4 projection description file and create a new location based on it \\
  \hline g.proj.proj.new & Affiche des informations de la projection \`a partir d'un fichier de description de la projection PROJ.4 et cr\'e\'e une nouvelle location bas\'ee sur celui-ci \\
%   \hline m.cogo & A simple utility for converting bearing and distance measurements to coordinates and vice versa. It assumes a cartesian coordinate system \\
  \hline m.cogo & Une commande simple pour convertir des mesures de distances et d'orientation en coordonn\'ees et vice-versa. Il suppose un syst\`eme de coordonn\'es cart\'esiennes \\
\hline
\end{tabular}
\end{table}

\clearpage

% \subsection{GRASS Toolbox raster data modules}
\subsection{Modules de donn\'ees raster de la bo\^ite \`a outils de GRASS}

% This Section lists all graphical dialogs in the GRASS Toolbox to work with and analyse raster data in a currently selected GRASS location and mapset.
Cette section liste toutes les bo\^ites de dialogue dans la bo\^ite \`a outils de GRASS pour utiliser et analyser des donn\'ees raster dans un jeu de donn\'ees et une r\'egion de GRASS s\'electionn\'es.

\begin{table}[ht]
\centering
% \caption{GRASS Toolbox: Develop raster map modules}\medskip
\caption{Bo\^ite \`a outils de GRASS : Modules de d\'eveloppements de couches raster}\medskip
 \begin{tabular}{|p{4cm}|p{12cm}|}
%   \hline \multicolumn{2}{|c|}{\textbf{Develop raster map modules in the GRASS Toolbox}} \\
    \hline \multicolumn{2}{|c|}{\textbf{Modules de d\'eveloppements de couches raster de la bo\^ite \`a outils de GRASS}} \\
%   \hline \textbf{Module name} & \textbf{Purpose} \\
  \hline \textbf{Nom du module} & \textbf{Objectif} \\
%   \hline r.compress & Compresses and decompresses raster maps \\
  \hline r.compress & Compresse et d\'ecompresse des couches raster \\
%   \hline r.region.region & Sets the boundary definitions to current or default region \\
  \hline r.region.region & D\'efinis la d\'efinition des fronti\`eres \`a la r\'egion par d\'efaut ou celle actuelle \\
%   \hline r.region.raster & Sets the boundary definitions from existent raster map\\
  \hline r.region.raster & D\'efinis la d\'efinition des fronti\`eres \`a partir d'une couche raster existante \\
%   \hline r.region.vector & Sets the boundary definitions from existent vector map \\
  \hline r.region.vector & D\'efinis la d\'efinition des fronti\`eres \`a partir d'une couche vecteur existante \\
%   \hline r.region.edge & Sets the boundary definitions by edge (n-s-e-w) \\
  \hline r.region.edge & D\'efinis la d\'efinition des fronti\`eres par le bord (n-s-e-o) \\
%   \hline r.region.alignTo & Sets region to align to a raster map\\
  \hline r.region.alignTo & D\'efinis la r\'egion sur laquelle aligner la couche raster \\
%   \hline r.null.val & Transform cells with value in null cells\\
  \hline r.null.val & Transforme les cellules avec des valeurs en cellules nulles\\
%   \hline r.null.to & Transform null cells in value cells\\
  \hline r.null.to & Transforme les cellules nulles en cellules avec une valeur\\
%   \hline r.quant & This routine produces the quantization file for a floating-point map \\
  \hline r.quant & Cette routine produit le fichier de quantification pour une carte en virgule flottante \\
%   \hline r.resamp.stats & Resamples raster map layers using aggregation \\
  \hline r.resamp.stats & Re\'echantillonne des couches raster en utilisant l'aggr\'egation \\
%   \hline r.resamp.interp & Resamples raster map layers using interpolation \\
  \hline r.resamp.interp & Re\'echantillonne des couches raster en utilisant l'interpolation \\
%   \hline r.resample & GRASS raster map layer data resampling capability. Before you must set new resolution\\
  \hline r.resample & Fonctionnalit\'e de re\'echantillonage de donn\'ees raster de GRASS. Vous devez auparavant d\'efinir une nouvelle r\'esolution \\
%   \hline r.resamp.rst & Reinterpolates and computes topographic analysis using regularized spline with tension and smoothing \\
  \hline r.resamp.rst & Reinterpole et calcul l'analyse topographique en utilisant des courbes r\'egularis\'ees avec une tension et un lissage \\
%   \hline r.support & Allows creation and/or modification of raster map layer support files\\
  \hline r.support & Permet la cr\'eation et/ou la modification des fichiers support de la couche raster\\
%   \hline r.support.stats & Update raster map statistics \\
  \hline r.support.stats & Met \`a jour les statistiques du raster \\
%   \hline r.proj & Re-project a raster map from one location to the current location \\
  \hline r.proj & Reprojette une couche raster d'une location \`a la localition actuelle \\
\hline
\end{tabular}
\end{table}

\begin{table}[ht]
\centering
% \caption{GRASS Toolbox: Raster color management modules}\medskip
\caption{Bo\^ite \`a outils de GRASS : Modules de gestion de la couleur des raster}\medskip
 \begin{tabular}{|p{4cm}|p{12cm}|}
%   \hline \multicolumn{2}{|c|}{\textbf{Raster color management modules in the GRASS Toolbox}} \\
  \hline \multicolumn{2}{|c|}{\textbf{Modules de gestion de la couleur des raster dans la bo\^ite \`a outils de GRASS}} \\
%   \hline \textbf{Module name} & \textbf{Purpose} \\
  \hline \textbf{Nom du module} & \textbf{Objectif} \\
%   \hline r.colors.table & Set raster color table from setted tables \\
  \hline r.colors.table & D\'efinit une table de couleur \`a partir de tables \'etablies \\
%   \hline r.colors.rules & Set raster color table from setted rules \\
  \hline r.colors.rules & D\'efinit une table de couleur \`a partir de r\`egles \'etablies \\
%   \hline r.colors.rast & Set raster color table from existing raster \\
  \hline r.colors.rast & D\'efinit une table de couleur d'un raster existant \\
%   \hline r.blend & Blend color components for two raster maps by given ratio \\
  \hline r.blend & M\'elange les composants de couleurs de deux rasters en fonction d'un ratio \\
%   \hline r.composite & Blend red, green, raster layers to obtain one color raster \\
  \hline r.composite & M\'elange le rouge, le vert et le bleu de couches raster pour obtenir un raster d'une couleur \\
%   \hline r.his & Generates red, green and blue raster map layers combining hue, intensity, and saturation (his) values from user-specified input raster map layers \\
  \hline r.his & G\'en\`ere des couches raster rouge, vert et bleu combinant les valeurs de la teinte, de l'intensit\'e et de la saturation (HIS) \`a partir de couches raster d\'efinit par l'utilisateur en entr\'ee \\
\hline
\end{tabular}
\end{table}

\begin{table}[ht]
\centering
% \caption{GRASS Toolbox: Spatial raster analysis modules}\medskip
\caption{Bo\^ite \`a outils de GRASS : Modules d'analyse spatiale de raster}\medskip
 \begin{tabular}{|p{4cm}|p{12cm}|}
%   \hline \multicolumn{2}{|c|}{\textbf{Spatial raster analysis modules in the GRASS Toolbox}} \\
  \hline \multicolumn{2}{|c|}{\textbf{Modules d'analyse spatiale de raster dans la bo\^ite \`a outils GRASS }} \\
%   \hline \textbf{Module name} & \textbf{Purpose} \\
  \hline \textbf{Nom du module} & \textbf{Objectif} \\
%   \hline r.buffer & Raster buffer \\
  \hline r.buffer & Buffer raster\\
%   \hline r.mask & Create a MASK for limiting raster operation \\
  \hline r.mask & Cr\'e\'e un MASK pour limiter les op\'erations raster\\
%   \hline r.mapcalc & Raster map calculator \\
  \hline r.mapcalc & Calculateur de couche raster \\
%   \hline r.mapcalculator & Simple map algebra \\
  \hline r.mapcalculator & Alg\`ebre cartographique simple \\
%   \hline r.neighbors & Raster neighbors analyses \\
  \hline r.neighbors & Analyse raster des voisins\\
%   \hline v.neighbors & Count of neighbouring points \\
  \hline v.neighbors & Compte les points voisins \\
%   \hline r.cross & Create a cross product of the category value from multiple raster map layers \\
  \hline r.cross & Cr\'ee un produit crois\'e de la valeur de la cat\'egorie \`a partir de plus couches raster \\
%   \hline r.series & Makes each output cell a function of the values assigned to the corresponding cells in the output raster map layers\\
  \hline r.series & Fait de chaque cellule en sortie une fonction de la valeur attribu\'ee aux cellules correspondantes \`a la sortie des couches raster\\
%   \hline r.patch & Create a new raster map by combining other raster maps \\
  \hline r.patch & Cr\'ee une nouvelle couche raster en combinanet d'autres couches raster \\
%   \hline r.statistics & Category or object oriented statistics \\
  \hline r.statistics & Statistiques orient\'e categories ou objet\\
%   \hline r.cost & Outputs a raster map layer showing the cumulative cost of moving between different geographic locations on an input raster map layer whose cell category values represent cost\\
  \hline r.cost & Renvoie une couche raster montrant le co\^ut cumulatif du d\'eplacement entre des endroits g\'eographiques diff\'erents sur une couche raster en entr\'ee dont les valeurs des cat\'egories repr\'esentent le co\^ut\\
%   \hline r.drain & Traces a flow through an elevation model on a raster map layer \\
  \hline r.drain & Trace un flux \`a travers un mod\`ele d'\'el\'evation sur une couche raster\\
%   \hline r.shaded.relief & Create shaded map \\
  \hline r.shaded.relief & Cr\'e\'e une carte d'ombrage \\
%   \hline r.slope.aspect.slope & Generate slope map from DEM (digital elevation model) \\
  \hline r.slope.aspect.slope & G\'en\`ere une carte de pente \`a partir d'un MNT (Mod\`ele Num\'erique de Terrain) \\
%   \hline r.slope.aspect.aspect & Generate aspect map from DEM (digital elevation model) \\
  \hline r.slope.aspect.aspect & G\'en\`ere une carte d'aspect \`a partir d'un MNT (Mod\`ele Num\'erique de Terrain) \\
%   \hline r.param.scale & Extracts terrain parameters from a DEM \\
  \hline r.param.scale & Extrait les param\`etres terrain \`a partir d'un MNT \\
%   \hline r.texture & Generate images with textural features from a raster map (first serie of indices)\\
  \hline r.texture & G\'en\`ere des images avec des objets de texture \`a partir d'une couche raster (premi\`ere s\'erie d'indices)\\
%   \hline r.texture.bis & Generate images with textural features from a raster map (second serie of indices)\\
  \hline r.texture.bis & G\'en\`ere des images avec des objets de texture \`a partir d'une couche raster (seconde s\'erie d'indices)\\
%   \hline r.los & Line-of-sigth raster analysis \\
  \hline r.los & Analyse raster de la ligne de vue\\
%   \hline r.clump & Recategorizes into unique categories contiguous cells \\
  \hline r.clump & Recat\'egorise des cellules contig"ue en une cat\'egorie unique \\
%   \hline r.grow & Generates a raster map layer with contiguous areas grown by one cell\\
  \hline r.grow & G\'en\`ere une couche raster avec des zones contig"ues augment\'ees par une cellule\\
%   \hline r.thin & Thin no-zero cells that denote line features \\
  \hline r.thin & Cellules non null minces qui d\'enote un objet lin\'eaire \\
\hline
\end{tabular}
\end{table}

\begin{table}[ht]
\centering
% \caption{GRASS Toolbox: Surface management modules}\medskip
\caption{Bo\^ite \`a outils de GRASS : Modules de gestion des surfaces}\medskip
 \begin{tabular}{|p{4cm}|p{12cm}|}
  \hline \multicolumn{2}{|c|}{\textbf{Surface management modules in the GRASS Toolbox}} \\
%   \hline \textbf{Module name} & \textbf{Purpose} \\
  \hline \textbf{Nom du module} & \textbf{Objectif} \\
%   \hline r.random & Creates a random vector point map contained in a raster \\
  \hline r.random & Cr\'e\'e une couche vecteur de point al\'eatoire dans un raster\\
%   \hline r.random.cells & Generates random cell values with spatial dependence \\
  \hline r.random.cells & G\'en\`ere des valeurs de cellule al\'eatoire avec une d\'ependance spatiale \\
%   \hline v.kernel & Gaussian kernel density \\
  \hline v.kernel & Densit\'e du noyau Gaussien \\
%   \hline r.contour & Produces a contours vector map with specified step from a raster map\\
  \hline r.contour & Produit une couche vectoriel de contours avec des \'etapes d\'efinits \`a partir d'une couche raster\\
%   \hline r.contour2 & Produces a contours vector map of specified contours from a raster map \\
  \hline r.contour2 & Produit un contour vectoriel \`a partir de contours d\'efinit par une couche raster\\
%   \hline r.surf.fractal & Creates a fractal surface of a given fractal dimension\\
  \hline r.surf.fractal & Cr\'e\'e une surface fractal avec une dimension fractale donn\'ee\\
%   \hline r.surf.gauss & GRASS module to produce a raster map layer of gaussian deviates whose mean and standard deviation can be expressed by the user \\
  \hline r.surf.gauss & Module GRASS pour produire une couche raster de d\'eviation gaussienne dont la moyenne et la d\'eviation standard peuvent \^etre exprim\'e par l'utilisateur\\
%   \hline r.surf.random & Produces a raster map layer of uniform random deviates whose range can be expressed by the user \\
  \hline r.surf.random & Produit une couche raster de divation al\'eatoire uniforme dont le domaine peut \^etre exprim\'e par l'utilisateur\\
%   \hline r.bilinear & Bilinear interpolation utility for raster map layers \\
  \hline r.bilinear & Commande d'interpolation bilin\'eaire pour les couches raster \\
%   \hline v.surf.bispline & Bicubic or bilinear spline interpolation with Tykhonov regularization\\
  \hline v.surf.bispline & Interpolation spline bicubique ou bilin\'eaire avec r\'egularisation de Tykhonov \\
%   \hline r.surf.idw & Surface interpolation utility for raster map layers\\
  \hline r.surf.idw & Commande d'interpolation de surface pour des couches raster\\
%   \hline r.surf.idw2 & Surface generation program\\
  \hline r.surf.idw2 & Programme de g\'en\'eration de surface\\
%   \hline r.surf.contour & Surface generation program from rasterized contours \\
  \hline r.surf.contour & Programme de g\'en\'eration de surface \`a partir de contours rasteris\'es \\
%   \hline v.surf.idw & Interpolate attribute values (IDW) \\
  \hline v.surf.idw & Interpole les valeurs attributaires (IDW) \\
%   \hline v.surf.rst & Interpolate attribute values (RST) \\
  \hline v.surf.rst & Interpole les valeurs attributaires (RST) \\
%   \hline r.fillnulls & Fills no-data areas in raster maps using v.surf.rst splines interpolation \\
  \hline r.fillnulls & Remplis les zones sans donn\'ees dans une couche raster en utilisant l'interpolation de splines de v.surf.rst \\
\hline
\end{tabular}
\end{table}

\begin{table}[ht]
\centering
% \caption{GRASS Toolbox: Change raster category values and labels modules}\medskip
\caption{Bo\^ite \`a outils de GRASS : Modules pour changer les valeurs des cat\'egories et des \'etiquettes des raster}\medskip
 \begin{tabular}{|p{4cm}|p{12cm}|}
%   \hline \multicolumn{2}{|c|}{\textbf{Raster category and label modules in the GRASS Toolbox}} \\
  \hline \multicolumn{2}{|c|}{\textbf{Modules pour changer les valeurs des cat\'egories et des \'etiquettes des raster dans la bo\^ite \`a outils de GRASS}}\\
%   \hline \textbf{Module name} & \textbf{Purpose} \\
  \hline \textbf{Nom du module} & \textbf{Objectif} \\
%   \hline r.reclass.area.greater & Reclasses a raster map greater than user specified area size (in hectares) \\
  \hline r.reclass.area.greater & Reclasse une couche raster d'une zone sup\'erieure \`a celle donn\'ee par l'utilisateur (en hectares) \\
%   \hline r.reclass.area.lesser & Reclasses a raster map less than user specified area size (in hectares) \\
  \hline r.reclass.area.lesser & Reclasse une couche raster d'une zone inf\'erieure \`a celle donn\'ee par l'utilisateur (en hectares) \\
%   \hline r.reclass & Reclass a raster using a reclassification rules file \\
  \hline r.reclass & Reclasse un raster en utilisant un fichier de r\`egles de reclassification \\
%   \hline r.recode & Recode raster maps\\
  \hline r.recode & Recode des couches raster \\
%   \hline r.rescale & Rescales the range of category values in a raster map layer \\
  \hline r.rescale & Re\'echantillonne le domaine des valeurs des cat\'egories d'une couche raster \\
\hline
\end{tabular}
\end{table}

\begin{table}[ht]
\centering
% \caption{GRASS Toolbox: Hydrologic modelling modules}\medskip
\caption{Bo\^ite \`a outils de GRASS : Modules de mod\'elisation hydrologique}\medskip
 \begin{tabular}{|p{4cm}|p{12cm}|}
%   \hline \multicolumn{2}{|c|}{\textbf{Hydrologic modelling modules in the GRASS Toolbox}} \\
  \hline \multicolumn{2}{|c|}{\textbf{Modules de mod\'elisation hydrologique dans la bo\^ite \`a outils de GRASS}}\\
%   \hline \textbf{Module name} & \textbf{Purpose} \\
  \hline \textbf{Nom du module} & \textbf{Objectif} \\
%   \hline r.carve & Takes vector stream data, transforms it to raster, and subtracts depth from the output DEM \\
  \hline r.carve & Utilise des donn\'ees vecteur de flux, les transforme en raster et extrait la profondeur \`a partir du MNT en sortie\\
%   \hline r.fill.dir & Filters and generates a depressionless elevation map and a flow direction map from a given elevation layer \\
  \hline r.fill.dir & Filtre et g\'en\`ere une couche d'\'el\'evation sans d\'epression et un couche de direction de flux \`a partir d'une couche d'\'el\'evation donn\'ee \\
%   \hline r.lake.xy & Fills lake from seed point at given level \\
  \hline r.lake.xy & Remplit le lac \`a partir de donn\'ees ponctuelles \`a un niveau d\'efini \\
%   \hline r.lake.seed & Fills lake from seed at given level \\
  \hline r.lake.seed & Remplit le lac \`a partir de donn\'ees \`a un niveau d\'efini \\
%   \hline r.topidx & Creates a 3D volume map based on 2D elevation and value raster maps \\
  \hline r.topidx & Cr\'ee une carte en 3D bas\'e sur des couches raster et des \'el\'evations 2D \\
%   \hline r.basins.fill & Generates a raster map layer showing watershed subbasins \\
  \hline r.basins.fill & G\'en\`ere une couche raster montrant les sous-bassins hydrographiques \\
%   \hline r.water.outlet & Watershed basin creation program \\
  \hline r.water.outlet & Programme de cr\'eation de bassin hydrographique \\
\hline
\end{tabular}
\end{table}

\begin{table}[ht]
\centering
% \caption{GRASS Toolbox: Reports and statistic analysis modules}\medskip
\caption{Bo\^ite \`a outils de GRASS : Modules d'analyses statistiques et rapports}\medskip
 \begin{tabular}{|p{4cm}|p{12cm}|}
%   \hline \multicolumn{2}{|c|}{\textbf{Reports and statistic analysis modules in the GRASS Toolbox}} \\
  \hline \multicolumn{2}{|c|}{\textbf{Modules d'analyses statistiques et rapports dans la bo\^ite \`a outils de GRASS}} \\
%   \hline \textbf{Module name} & \textbf{Purpose} \\
  \hline \textbf{Nom du module} & \textbf{Objectif} \\
%   \hline r.category & Prints category values and labels associated with user-specified raster map layers \\
  \hline r.category & Affiche les valeurs des categories et leus \'etiquettes avec une couche rster d\'efinit par l'utilisateur \\
%   \hline r.sum & Sums up the raster cell values \\
  \hline r.sum & R\'ealise la somme des valeurs des cellules d'un raster \\
%   \hline r.report & Reports statistics for raster map layers \\
  \hline r.report & Renvoi des statistiques pour des couches raster \\
%   \hline r.average & Finds the average of values in a cover map within areas assigned the same category value in a user-specified base map \\
  \hline r.average & Trouve la moyenne des valeurs dans une couche de couverture dans des zones assign\'ees de m\^eme valeur de cat\'egorie dans une couche d\'efinit par l'utilisateur \\
%   \hline r.median & Finds the median of values in a cover map within areas assigned the same category value in a user-specified base map \\
  \hline r.median & Trouve la mediane des valeurs dans une couche de couverture dans des zones assign\'ees de m\^eme valeur de cat\'egorie dans une couche d\'efinit par l'utilisateur \\
%   \hline r.mode & Finds the mode of values in a cover map within areas assigned the same category value in a user-specified base map.reproject raster image \\
  \hline r.mode & Trouve le mode des valeurs dans une couche de couverture dans des zones assign\'ees de m\^eme valeur de cat\'egorie  dans une couche d\'efinit par l'utilisateur \\
%   \hline r.volume & Calculates the volume of data clumps, and produces a GRASS vector points map containing the calculated centroids of these clumps \\
  \hline r.volume & Calcule le volume d'un amas de donn\'ees, et produit une couche vecteur poncutel GRAS contenant le centro"ide calcul\'e de ces amas \\
%   \hline r.surf.area & Surface area estimation for rasters \\
  \hline r.surf.area & Estimation de la surface d'une pour des rasters \\
%   \hline r.univar & Calculates univariate statistics from the non-null cells of a raster map \\
  \hline r.univar & Calcule des statistiques univari\'ees \`a partir de cellules non nulles d'une couche raster\\
%   \hline r.covar & Outputs a covariance/correlation matrix for user-specified raster map layer(s)\\
  \hline r.covar & Affiche une matrice de corr\'elation/covariance pour des couches raster d\'efinit par l'utilisateur\\
%   \hline r.regression.line & Calculates linear regression from two raster maps: y = a + b * x \\
  \hline r.regression.line & Calcule la r\'egression lin\'eaire \`a partir de deux cartes raster : y = a + b * x \\
%   \hline r.coin & Tabulates the mutual occurrence (coincidence) of categories for two raster map layers\\
  \hline r.coin & Tabule les occurences mutuelles (co"incidence) des cat\'egories pour deux couches raster\\
\hline
\end{tabular}
\end{table}

\clearpage

% \subsection{GRASS Toolbox vector data modules}
\subsection{Modules de donn\'ees vecteur de la bo\^ite \`a outils de GRASS}

% This Section lists all graphical dialogs in the GRASS Toolbox to work with and analyse vector data in a currently selected GRASS location and mapset.
Cette section liste toutes les bo\^ites de dialogue dans la bo\^ite \`a outils de GRASS pour utiliser et analyser des donn\'ees vecteur dans un jeu de donn\'ees et une r\'egion de GRASS s\'electionn\'es.

\begin{table}[ht]
\centering
% \caption{GRASS Toolbox: Develop vector map modules}\medskip
\caption{Bo\^ite \`a outils de GRASS : Modules de d\'eveloppement des couches vecteurs}\medskip
 \begin{tabular}{|p{4cm}|p{12cm}|}
%   \hline \multicolumn{2}{|c|}{\textbf{Develop vector map modules in the GRASS Toolbox}} \\
  \hline \multicolumn{2}{|c|}{\textbf{Modules de d\'eveloppement des couches vecteurs de la bo\^ite \`a outils de GRASS}} \\
%   \hline \textbf{Module name} & \textbf{Purpose} \\
  \hline \textbf{Nom du module} & \textbf{Objectif} \\
%   \hline v.build.all & Rebuild topology of all vectors in the mapset \\
  \hline v.build.all & Reconstruit la top\^ologie de tous les vecteurs dans le jeu de donn\'ees\\
%   \hline v.clean.break & Break lines at each intersection of vector map \\
  \hline v.clean.break & Coupe les lignes \`a chaque intersection de la couche vecteur\\
%   \hline v.clean.snap & Cleaning topology: snap lines to vertex in threshold \\
  \hline v.clean.snap & Nettoyage topologique : aimante les lignes vers les sommets en fonction d'un seuil\\
%   \hline v.clean.rmdangles & Cleaning topology: remove dangles \\
  \hline v.clean.rmdangles & Nettoyage topologique : supprime les noeuds pendants ([NdT] noeuds isol\'es qui ne ferme pas proprement l'objet) \\
%   \hline v.clean.chdangles & Cleaning topology: change the type of boundary dangle to line \\
  \hline v.clean.chdangles & Nettoyage topologique : change le type contour d'arc en ligne \\
%   \hline v.clean.rmbridge & Remove bridges connecting area and island or 2 islands \\
  \hline v.clean.rmbridge & Supprime les ponts connectant une surface et une ou deux \^iles\\
%   \hline v.clean.chbridge & Change the type of bridges connecting area and island or 2 islands \\
  \hline v.clean.chbridge & Change les ponts connectant les surfaces et une ou 2 \^iles \\
%   \hline v.clean.rmdupl & Remove duplicate lines (pay attention to categories!) \\
  \hline v.clean.rmdupl & Enl\`eve les lignes dupliqu\'ees  (fa\^ites attention aux cat\'egories !) \\
%   \hline v.clean.rmdac & Remove duplicate area centroids \\
  \hline v.clean.rmdac & Enl\`eve les centro"ides dupliqu\'es des surfaces\\
%   \hline v.clean.bpol & Break polygons. Boundaries are broken on each point shared between 2 and more polygons where angles of segments are different \\
  \hline v.clean.bpol & Casse les polygones. Les contours sont cass\'es sur chaque point partag\'e entre deux ou plus polygones o\`u les angles des segments sont diff\'erents\\
%   \hline v.clean.prune & Remove vertices in threshold from lines and boundaries \\
  \hline v.clean.prune & Enl\`eve les sommets dans un seuil des lignes et contours\\
%   \hline v.clean.rmarea & Remove small areas (removes longest boundary with adjacent area) \\
  \hline v.clean.rmarea & Enl\`eve les petites surfaces (supprime les contours les plus grand avec des zones adjacantes) \\
%   \hline v.clean.rmline & Remove all lines or boundaries of zero length \\
  \hline v.clean.rmline & Enl\`eve toutes les lignes ou contours de longueur nulle\\
%   \hline v.clean.rmsa & Remove small angles between lines at nodes \\
  \hline v.clean.rmsa & Enl\`eve les petits angles entre les lignes aux niveaux des noeuds\\
%   \hline v.type.lb & Convert lines to boundaries \\
  \hline v.type.lb & Convertit des lignes en limites\\
%   \hline v.type.bl & Convert boundaries to lines \\
  \hline v.type.bl & Convertit des limites en lignes\\
%   \hline v.type.pc & Convert points to centroids \\
  \hline v.type.pc & Convertit des points en centroides \\
%   \hline v.type.cp & Convert centroids to points \\
  \hline v.type.cp & Convertit des centroides en points \\
%   \hline v.centroids & Add missing centroids to closed boundaries  \\
  \hline v.centroids & Ajoute les centroides manquants aux limites ferm\'ees\\
%   \hline v.build.polylines & Build polylines from lines \\
  \hline v.build.polylines & Construit des polylignes \`a partir de lignes\\
%   \hline v.segment & Creates points/segments from input vector lines and positions \\
  \hline v.segment & Cr\'ee des points/segments \`a partir de positions et de lignes vectorielles en entr\'ee\\
%   \hline v.to.points & Create points along input lines \\
  \hline v.to.points & Cr\'ee des points le longs d'une ligne en entr\'ee\\
%   \hline v.parallel & Create parallel line to input lines \\
  \hline v.parallel & Cr\'ee une ligne parall\`ele \`a des lignes en entr\'ee\\
%   \hline v.dissolve & Dissolves boundaries between adjacent areas \\
  \hline v.dissolve & Dissous des limites dans des zones adjacentes\\
%   \hline v.drape & Convert 2D vector to 3D vector by sampling of elevation raster\\
  \hline v.drape & Convertie des vecteurs 2D en vecteur 3D par re\'echantillonage de raster d'\'el\'elvation\\
%   \hline v.transform & Performs an affine transformation on a vector map \\
  \hline v.transform & R\'ealise une transformation affine d'une couche vecteur\\
%   \hline v.proj & Allows projection conversion of vector files \\
  \hline v.proj & Permet une conversion de la projection de fichier vecteur\\
%   \hline v.support & Updates vector map metadata \\
  \hline v.support & Met \`a jour les m\'eta-donn\'ees des couches vecteurs\\
%   \hline generalize & Vector based generalization \\
  \hline generalize & G\'en\'eralisation vectorielle\\
\hline
\end{tabular}
\end{table}

\begin{table}[ht]
\centering
% \caption{GRASS Toolbox: Database connection modules}\medskip
\caption{Bo\^ite \`a outils de GRASS : Modules de connexion aux bases de donn\'ees}\medskip
 \begin{tabular}{|p{4cm}|p{12cm}|}
%   \hline \multicolumn{2}{|c|}{\textbf{Database connection modules in the GRASS Toolbox}} \\
  \hline \multicolumn{2}{|c|}{\textbf{Modules de connexion aux bases de donn\'ees de la bo\^ite \`a outils de GRASS}} \\
%   \hline \textbf{Module name} & \textbf{Purpose} \\
  \hline \textbf{Nom du module} & \textbf{Objectif} \\
%   \hline v.db.connect & Connect a vector to database \\
  \hline v.db.connect & Connecte un vecteur \`a une base de donn\'ees\\
%   \hline v.db.sconnect & Disconnect a vector from database \\
  \hline v.db.sconnect & D\'econnecte un vecteur d'une base de donn\'ees\\
%   \hline v.db.what.connect & Set/Show database connection for a vector \\
  \hline v.db.what.connect & D\'efinit/affiche une connexion \`a une base de donn\'ees pour un vecteur\\
\hline
\end{tabular}
\end{table}

\begin{table}[ht]
\centering
% \caption{GRASS Toolbox: Change vector field modules}\medskip
\caption{Bo\^ite \`a outils de GRASS : Modules de modification des champs vectoriels}\medskip
 \begin{tabular}{|p{4cm}|p{12cm}|}
%   \hline \multicolumn{2}{|c|}{\textbf{Change vector field modules in the GRASS Toolbox}} \\
  \hline \multicolumn{2}{|c|}{\textbf{Modules de modification des champs vectoriels de la bo\^ite \`a outils de GRASS}} \\
%   \hline \textbf{Module name} & \textbf{Purpose} \\
  \hline \textbf{Nom du module} & \textbf{Objectif} \\
%   \hline v.category.add & Add elements to layer (ALL elements of the selected layer type!)\\
  \hline v.category.add & Ajoute des \'el\'ements \`a la couche (tous les \'el\'ements du type de la couche s\'electionn\'ee !)\\
%   \hline v.category.del & Delete category values \\
  \hline v.category.del & Supprime les valeurs des cat\'egories\\
%   \hline v.category.sum & Add a value to the current category values \\
  \hline v.category.sum & Ajoute une valeur aux valeurs des cat\'egories en cours\\
%   \hline v.reclass.file & Reclass category values using a rules file \\
  \hline v.reclass.file & Reclasse les valeurs des cat\'egories en utilisant un fichier de r\`egles\\
%   \hline v.reclass.attr & Reclass category values using a column attribute (integer positive) \\
  \hline v.reclass.attr & Reclasse les valeurs des cat\'egories en utilisant une colonne attributaire (entier positif)\\
\hline
\end{tabular}
\end{table}

\begin{table}[ht]
\centering
% \caption{GRASS Toolbox: Working with vector points modules}\medskip
\caption{Bo\^ite \`a outils de GRASS : Travailler avec les modules des vecteurs ponctuels}\medskip
 \begin{tabular}{|p{4cm}|p{12cm}|}
%   \hline \multicolumn{2}{|c|}{\textbf{Working with vector points modules in the GRASS Toolbox}} \\
\hline \multicolumn{2}{|c|}{\textbf{Travailler avec les modules des vecteurs ponctuels de la bo\^ite \`a outils de GRASS}} \\
%   \hline \textbf{Module name} & \textbf{Purpose} \\
  \hline \textbf{Nom du module} & \textbf{Objectif} \\
%   \hline v.in.region & Create new vector area map with current region extent \\
  \hline v.in.region & Cr\'ee une nouvelle couche vecteur avec une \'etendue de la r\'egion actuelle\\
%   \hline v.mkgrid.region & Create grid in current region \\
  \hline v.mkgrid.region & Cr\'ee une grille dans la r\'egion actuelle\\
%   \hline v.in.db & Import vector points from a database table containing coordinates \\
  \hline v.in.db & Importe des points vectoriels d'une table d'une base de donn\'ees contenant des coordonn\'ees\\
%   \hline v.random & Randomly generate a 2D/3D GRASS vector point map \\
  \hline v.random & G\'en\`ere al\'eaoitement une couche de points vectorielle GRASS en 2D/3D\\
%   \hline v.kcv & Randomly partition points into test/train sets \\
  \hline v.kcv & place des points al\'eatoires dans un jeu test\\
%   \hline v.outlier & Remove outliers from vector point data \\
  \hline v.outlier & Supprime les valeurs atypiques des donn\'ees ponctuelles vectorielles\\
%   \hline v.hull & Create a convex hull \\
  \hline v.hull & Cr\'ee une enveloppe convexe\\
%   \hline v.delaunay.line & Delaunay triangulation (lines) \\
  \hline v.delaunay.area & Triangulation de Delaunay (lin\'eaire) \\
%   \hline v.delaunay.area & Delaunay triangulation (areas) \\
  \hline v.delaunay.area & Triangulation de Delaunay (surface) \\
%   \hline v.voronoi.line & Voronoi diagram (lines) \\
  \hline v.voronoi.area & Diagramme de Vorono"i (lin\'eaire) \\
%   \hline v.voronoi.area & Voronoi diagram (areas) \\
  \hline v.voronoi.area & Diagramme de Vorono"i (surface) \\
\hline
\end{tabular}
\end{table}

\begin{table}[ht]
\centering
% \caption{GRASS Toolbox: Spatial vector and network analysis modules}\medskip
\caption{Bo\^ite \`a outils de GRASS : Modules d'analyse spatiale de vecteur et de r\'eseau}\medskip
 \begin{tabular}{|p{4cm}|p{12cm}|}
%   \hline \multicolumn{2}{|c|}{\textbf{Spatial vector and network analysis modules in the GRASS Toolbox}} \\
  \hline \multicolumn{2}{|c|}{\textbf{Modules d'analyse spatiale de vecteur et de r\'eseau de la bo\^ite \`a outils de GRASS}} \\
%   \hline \textbf{Module name} & \textbf{Purpose} \\
  \hline \textbf{Nom du module} & \textbf{Objectif} \\
%   \hline v.extract.where & Select features by attributes \\
  \hline v.extract.where & S\'electionne les objets par attributs\\
%   \hline v.extract.list & Extract selected features \\
  \hline v.extract.list & Extrait les objets s\'electionn\'es\\
%   \hline v.select.overlap & Select features overlapped by features in another map\\
  \hline v.select.overlap & S\'electione les objets superpos\'es par des objets d'une autre couche\\
%   \hline v.buffer & Vector buffer \\
  \hline v.buffer & Buffer de vecteur\\
%   \hline v.distance & Find the nearest element in vector 'to' for elements in vector 'from'. \\
  \hline v.distance & Trouve l'\'el\'ement le plus proche dans un vecteur 'to' pour des \'el\'ements dans un vecteur 'from'\\
%   \hline v.net.nodes & Create nodes on network \\
  \hline v.net.nodes & Cr\'ee des noeuds sur un r\'eseau\\
%   \hline v.net.alloc & Allocate network\\
  \hline v.net.alloc & Alloue un r\'eseau\\
%   \hline v.net.iso & Cut network by cost isolines \\
  \hline v.net.iso & Coupe un r\'eseau par des isolignes de co\^ut\\
%   \hline v.net.salesman & Connect nodes by shortest route (traveling salesman) \\
  \hline v.net.salesman & Connecte des noeuds par la route la plus courte (probl\`eme du voyageur de commerce) \\
%   \hline v.net.steiner & Connect selected nodes by shortest tree (Steiner tree) \\
  \hline v.net.steiner & Connecte des noeuds s\'electionn\'es par l'arbre le plus court (arbre Steiner) \\
%   \hline v.patch & Create a new vector map by combining other vector maps \\
  \hline v.patch & Cr\'ee une nouvelle couche vecteur par combinaison de couches vecteurs\\
%   \hline v.overlay.or & Vector union \\
  \hline v.overlay.or & Union de vecteur\\
%   \hline v.overlay.and & Vector intersection \\
  \hline v.overlay.and & Intersection de vecteur\\
%   \hline v.overlay.not & Vector subtraction \\
  \hline v.overlay.not & Soustraction de vecteur \\
%   \hline v.overlay.xor & Vector non-intersection \\
  \hline v.overlay.xor & Non intersection de vecteur\\
\hline
\end{tabular}
\end{table}

\begin{table}[ht]
\centering
% \caption{GRASS Toolbox: Vector update by other maps modules}\medskip
% \caption{GRASS Toolbox: Vector update by other maps modules}\medskip
\caption{Bo\^ite \`a outils de GRASS : Mise \`a jour de vecteur \`a partir d'autres modules cartographiques}\medskip
 \begin{tabular}{|p{4cm}|p{12cm}|}
%   \hline \multicolumn{2}{|c|}{\textbf{Vector update by other maps modules in the GRASS Toolbox}} \\
  \hline \multicolumn{2}{|c|}{\textbf{Mise \`a jour de vecteur \`a partir d'autres modules cartographiques de la bo\^ite \`a outils de GRASS}} \\
%   \hline \textbf{Module name} & \textbf{Purpose} \\
  \hline \textbf{Nom du module} & \textbf{Objectif} \\
%   \hline v.rast.stats & Calculates univariate statistics from a GRASS raster map based on vector objects\\
  \hline v.rast.stats & Calcule des statistiques univari\'ees d'une couche raster GRASS bas\'ee sur des objets vecteurs\\
%   \hline v.what.vect & Uploads map for which to edit attribute table \\
  \hline v.what.vect & T\'el\'echarge des cartes pour \'editer la table d'attributs\\
  \hline v.what.rast & T\'el\'echarge des valeurs raster \`a la position des points vecteurs vers la table\\
%   \hline v.sample & Sample a raster file at site locations \\
  \hline v.sample & \'Echantillonne une fichier raster \`a l'endroit des sites\\
\hline
\end{tabular}
\end{table}

\begin{table}[ht]
\centering
% \caption{GRASS Toolbox: Vector report and statistic modules}\medskip
 \caption{Bo\^ite \`a outils de GRASS : modules de statistique et de rapport de vecteur}\medskip
 \begin{tabular}{|p{4cm}|p{12cm}|}
%   \hline \multicolumn{2}{|c|}{\textbf{Vector report and statistic modules in the GRASS Toolbox}} \\
  \hline \multicolumn{2}{|c|}{\textbf{Modules de statistique et de rapport de vecteur de la bo\^ite \`a outils de GRASS}} \\
%   \hline \textbf{Module name} & \textbf{Purpose} \\
  \hline \textbf{Nom du module} & \textbf{Objectif} \\
%   \hline v.to.db & Put geometry variables in database \\
  \hline v.to.db & Mais des variables de g\'eom\'etrie dans la base de donn\'ees\\
%   \hline v.report & Reports geometry statistics for vectors \\
  \hline v.report & Cr\'ee un rapport de statistisque des g\'eom\'etries pour les vecteurs\\
%   \hline v.univar & Calculates univariate statistics on selected table column for a GRASS vector map \\
  \hline v.univar & Calcule des statistiques univari\'ees sur la colonne de la table s\'electionn\'ee pour une couche vecteur de GRASS\\
%   \hline v.normal & Tests for normality for points\\
  \hline v.normal & Teste la normalit\'e des points\\
\hline
\end{tabular}
\end{table}

\clearpage

% \subsection{GRASS Toolbox imagery data modules}
\subsection{Modules de donn\'ees d'imagerie de la bo\^ite \`a outils de GRASS}

% This Section lists all graphical dialogs in the GRASS Toolbox to work with and analyse imagery data in a currently selected GRASS location and mapset.
Cette section liste toutes les bo\^ites de dialogue dans la bo\^ite \`a outils de GRASS pour utiliser et analyser les donn\'ees d'images dans une r\'egion et un jeu de donn\'ees GRASS s\'electionn\'es.

\begin{table}[ht]
\centering
% \caption{GRASS Toolbox: Imagery analysis modules}\medskip
\caption{Bo\^ite \`a outils de GRASS : modules analyse d'image}\medskip
 \begin{tabular}{|p{4cm}|p{12cm}|}
%   \hline \multicolumn{2}{|c|}{\textbf{Imagery analysis modules in the GRASS Toolbox}} \\
  \hline \multicolumn{2}{|c|}{\textbf{Module d'analyse d'image de la bo\^ite \`a outils de GRASS}} \\
%   \hline \textbf{Module name} & \textbf{Purpose} \\
  \hline \textbf{Nom du module} & \textbf{Objectif} \\
%   \hline i.image.mosaik & Mosaic up to 4 images \\
  \hline i.image.mosaik & Mosaique jusqu'\`a 4 images\\
%   \hline i.rgb.his & Red Green Blue (RGB) to Hue Intensity Saturation (HIS) raster map color transformation function \\
  \hline i.rgb.his & Fonction de transformation de la carte de couleur du raster de Rouge Vert Bleu (RVB) en Nuance Intensit\'e Saturation (HIS)\\
%   \hline i.his.rgb & Hue Intensity Saturation (HIS) to Red Green Blue (RGB) raster map color transform function \\
  \hline i.his.rgb & Fonction de transformation de la carte de couleur du raster de Nuance Intensit\'e Saturation (HIS) en Rouge Vert Bleu (RVB) \\
%   \hline i.landsat.rgb & Auto-balancing of colors for LANDSAT images \\
  \hline i.landsat.rgb & Balance automatique des couleurs des images LANDSAT \\
%   \hline i.fusion.brovey & Brovey transform to merge multispectral and high-res pancromatic channels \\
  \hline i.fusion.brovey & Transformation de Brovey pour fusionner des cannaux panchromatique multispectrale et de haute r\'esolution\\
%   \hline i.zc & Zero-crossing edge detection raster function for image processing \\
  \hline i.zc & Fonction raster de d\'etection de bord vide ([NdT] Zero-crossing edge detection) dans le traitement des images \\
%   \hline i.mfilter &  \\
  \hline i.mfilter &  \\
%   \hline i.tasscap4 & Tasseled Cap (Kauth Thomas) transformation for LANDSAT-TM 4 data \\
  \hline i.tasscap4 & Transformation de Tasseled Cap (Kauth Thomas) pour les donn\'ees LANDSAT-TM 4 \\
%   \hline i.tasscap5 & Tasseled Cap (Kauth Thomas) transformation for LANDSAT-TM 5 data \\
  \hline i.tasscap4 & Transformation de Tasseled Cap (Kauth Thomas) pour les donn\'ees LANDSAT-TM 5 \\
%   \hline i.tasscap7 & Tasseled Cap (Kauth Thomas) transformation for LANDSAT-TM 7 data \\
  \hline i.tasscap4 & Transformation de Tasseled Cap (Kauth Thomas) pour les donn\'ees LANDSAT-TM 7 \\
%   \hline i.fft & Fast fourier transform (FFT) for image processing \\
  \hline i.fft & Transformation rapide de Fourier (FFT) pour le traitement des images \\
%   \hline i.ifft & Inverse fast fourier transform for image processing \\
  \hline i.ifft & Transformation inverse rapide de Fourier pour le traitement des images \\
%   \hline r.describe & Prints terse list of category values found in a raster map layer \\
  \hline r.describe & Affiche une liste tierce de valeurs de cat\'egorie trouv\'e dans une couche raster\\
%   \hline r.bitpattern & Compares bit patterns with a raster map \\
  \hline r.bitpattern & Compare des motifs d'octets avec une couche raster\\
%   \hline r.kappa & Calculate error matrix and kappa parameter for accuracy assessment of classification result \\
  \hline r.kappa & Calcule une matrice d'erreur et de param\`etre kappa pour l'\'evaluation de la pr\'ecision des r\'esultats d'une classification \\
%   \hline i.oif & Calculates optimal index factor table for landsat tm bands \\
  \hline i.oif & Calcule une table de facteur d'index optimal pour les bandes tm landsat \\
\hline
\end{tabular}
\end{table}

\clearpage

% \subsection{GRASS Toolbox database modules}
\subsection{Modules de base de donn\'ees de la bo\^ite \`a outils de GRASS}

% This Section lists all graphical dialogs in the GRASS Toolbox to manage, connect and work with internal and external databases. Working with spatial external databases is enabled via OGR and not covered by these modules.
Cette section liste toutes les bo\^ites de dialogue dans la bo\^ite \`a outils de GRASS pour g\'erer, se connecter et travailler avec les bases de donn\'ees externes et internes. Travailler avec des bases de donn\'ees spatiales externes est possible via OGR n'est pas couvert par ces modules.

\begin{table}[ht]
\centering
% \caption{GRASS Toolbox: Database modules}\medskip
\caption{Bo\^ite \`a outils de GRASS : Modules base de donn\'ees}\medskip
 \begin{tabular}{|p{4cm}|p{12cm}|}
%   \hline \multicolumn{2}{|c|}{\textbf{Database management and analysis modules in the GRASS Toolbox}} \\
\hline \multicolumn{2}{|c|}{\textbf{Modules de gestion de base de donn\'ees et d'analyse de la bo\^ite \`a outils de GRASS}} \\
%   \hline \textbf{Module name} & \textbf{Purpose} \\
  \hline \textbf{Nom du module} & \textbf{Objectif} \\
%   \hline db.connect & Sets general DB connection mapset \\
  \hline db.connect & D\'efinie la connexion \`a la BdD g\'en\'erale du jeu de donn\'ees \\
%   \hline db.connect.schema & Sets general DB connection mapset with a schema \\
  \hline db.connect.schema & D\'efinie la connexion \`a la BdD g\'en\'erale avec un sch\'ema du jeu de donn\'ees\\
%   \hline v.db.reconnect.all & Reconnect vector to a new database \\
  \hline v.db.reconnect.all & Reconnecte un vecteur avec une nouvelle base de donn\'ees\\
%   \hline db.login & Set user/password for driver/database \\
  \hline db.login & D\'efinie un utilisateur/mot de passe pour un pilote/base de donn\'ees\\
%   \hline db.in.ogr & Imports attribute tables in various formats \\
  \hline db.in.ogr & Importe une table d'attribut dans diff\'erents formats\\
%   \hline v.db.addtable & Create and add a new table to a vector \\
  \hline v.db.addtable & Cr\'e\'e et ajoute une nouvelle table \`a un vecteur\\
%   \hline v.db.addcol & Adds one or more columns to the attribute table connected to a given vector map \\
  \hline v.db.addcol & Ajoute une ou plusieurs colonnes \`a une table attributaire connect\'ee \`a une couche vecteur donn\'ee\\
%   \hline v.db.dropcol & Drops a column from the attribute table connected to a given vector map\\
  \hline v.db.dropcol & Supprime une colonne de la table attributaire connect\'ee \`a une couche vecteur donn\'ee\\
%   \hline v.db.renamecol & Renames a column in a attribute table connected to a given vector map\\
  \hline v.db.renamecol & Renomme une colonne dans une table attributaire connect\'ee \`a une couche vecteur donn\'ee\\
%   \hline v.db.update\_const & Allows to assign a new constant value to a column \\
  \hline v.db.update\_const & Permet d'assigner une nouvelle valeur d'une constante \`a une colonne\\
%   \hline v.db.update\_query & Allows to assign a new constant value to a column only if the result of a query is TRUE \\
  \hline v.db.update\_query & Permet d'assigner une nouvelle valeur d'une constante \`a une colonne seulement si le r\'esultat de la requ\^ete est TRUE\\
%   \hline v.db.update\_op & Allows to assign a new value, result of operation on column(s), to a column in the attribute table connected to a given map\\
  \hline v.db.update\_op & Permet d'assigner une nouvelle valeur, r\'esultat d'une op\'eration sur une ou plusieurs colonne(s), \`a une colonne dans la table attributaire connect\'ee \`a une couche donn\'ee\\
%   \hline v.db.update\_op\_query & Allows to assign a new value to a column, result of operation on column(s), only if the result of a query is TRUE \\
  \hline v.db.update\_op\_query & Permet d'assigner une nouvelle valeur \`a une colonne, r\'esultat d'une op\'eration sur ou plusieurs colonne(s), seulement si le r\'esutlat de la requ\^ete est TRUE \\
%   \hline db.execute & Execute any SQL statement \\
  \hline db.execute & \'Execute une requ\^ete SQL\\
%   \hline db.select & Prints results of selection from database based on SQL \\
  \hline db.select & Affiche les r\'esultats d'une s\'election d'une base de donn\'ees bas\'e sur une requ\^ete SQL\\
%   \hline v.db.select & Prints vector map attributes \\
  \hline v.db.select & Affiche les attributs d'une couche vecteur\\
%   \hline v.db.select.where & Prints vector map attributes with SQL \\
  \hline v.db.select.where & Affiche les attributs d'une couche vecteur avec une requ\^ete SQL\\
%   \hline v.db.join & Allows to join a table to a vector map table \\
  \hline v.db.join & Permet de r\'ealiser une jointure de table avec une table d'une couche vecteur\\
%   \hline v.db.univar & Calculates univariate statistics on selected table column for a GRASS vector map \\
  \hline v.db.univar & Calcule des statistiques univari\'ees sur une colonne d'une table s\'electionn\'ee pour une couche vecteur de GRASS\\
\hline
\end{tabular}
\end{table}

\clearpage

% \subsection{GRASS Toolbox 3D modules}
\subsection{Modules 3D de la bo\^ite \`a outils de GRASS}

% This Section lists all graphical dialogs in the GRASS Toolbox to work with 3D data. GRASS provides more modules, but they are currently only available using the GRASS Shell.
Cette section liste toutes les bo\^ites de dialogues de la bo\^ite \`a outils de GRASS pour travailler avec les donn\'ees 3D. GRASS fournit plus de modules, mais ils sont actuellement seulement disponibles en utilisant la console de GRASS.

\begin{table}[ht]
\centering
% \caption{GRASS Toolbox: 3D Visualization}\medskip
\caption{Bo\^ite \`a outils de GRASS : visualisation 3D}\medskip
 \begin{tabular}{|p{4cm}|p{12cm}|}
%   \hline \multicolumn{2}{|c|}{\textbf{3D visualization and analysis modules in the GRASS Toolbox}} \\
  \hline \multicolumn{2}{|c|}{\textbf{Modules de visualisation 3D et d'analyses de la bo\^ite \`a outils de GRASS}} \\
%   \hline \textbf{Module name} & \textbf{Purpose} \\
  \hline \textbf{Nom du module} & \textbf{Objectif} \\
%   \hline nviz & Open 3D-View in nviz\\
  \hline nviz & Vue 3D dans nviz\\
\hline
\end{tabular}
\end{table}

% \subsection{GRASS Toolbox help modules}
\subsection{Modules d'aide de la bo\^ite \`a outils de GRASS}

% The GRASS GIS Reference Manual offers a complete overview of the available GRASS modules, not limited to the modules and their often reduced functionalities implemented in the GRASS Toolbox.
Le manuel de r\'ef\'erence du SIG GRASS offre un aper\c{c}u complet des modules de GRASS disponibles, non limit\'e aux modules et leurs fonctionnalit\'es souvent limit\'es impl\'ement\'e dans la bo\^ite \`a outils de GRASS.

\begin{table}[ht]
\centering
% \caption{GRASS Toolbox: Reference Manual}\medskip
\caption{Bo\^ite \`a outils de GRASS : manuel de r\'ef\'erence}\medskip
 \begin{tabular}{|p{4cm}|p{12cm}|}
%   \hline \multicolumn{2}{|c|}{\textbf{Reference Manual modules in the GRASS Toolbox}} \\
  \hline \multicolumn{2}{|c|}{\textbf{Modules de manuel de r\'ef\'erence de la bo\^ite \`a outils de GRASS}} \\
%   \hline \textbf{Module name} & \textbf{Purpose} \\
  \hline \textbf{Nom du module} & \textbf{Objectif} \\
%   \hline g.manual & Display the HTML manual pages of GRASS \\
  \hline g.manual & Affiche la page HTML du manuel de GRASS \\
\hline
\end{tabular}
\end{table}
