% vim: set textwidth=78 autoindent:

% \section{Help and Support}\label{label_helpsupport}
\section{Aide et support}\label{label_helpsupport}

% when the revision of a section has been finalized, 
% comment out the following line:
% \updatedisclaimer

\subsection{Mailinglists}
% QGIS is under active development and as such it won't always work like
% you expect it to. The preferred way to get help is by joining the qgis-users
% mailing list.
QGIS est en cours de d\'eveloppement par cons\'equent il ne fonctionne pas toujours
comme attendu. La mani\`ere pr\'ef\'er\'ee d'obtenir de l'aide est de rejoindre la
liste de diffusion qgis-users.

\minisec{qgis-users}
% Your questions will reach a broader audience and answers will
% benefit others. You can subscribe to the qgis-users mailing list by visiting
% the following URL: \\
% \url{http://lists.osgeo.org/mailman/listinfo/qgis-user}
Vos questions atteindront une audience plus large et les r\'eponses b\'en\'eficieront
\`a tous. Vous pouvez rejoindre la liste de diffusion qgis-users en allant sur la
page suivante : \\
\url{http://lists.osgeo.org/mailman/listinfo/qgis-user}

\minisec{qgis-developer}
% If you are a developer facing problems of a more technical nature, you may
% want to join the qgis-developer mailing list here:\\
% \url{http://lists.osgeo.org/mailman/listinfo/qgis-developer}
Si vous \^etes un d\'eveloppeur et que vous faites face \`a un probl\`eme plus
technique, vous pourrez pr\'ef\'erer rejoindre la liste de diffusion qgis-developer
ici :\\
\url{http://lists.osgeo.org/mailman/listinfo/qgis-developer}

\minisec{qgis-commit}
% Each time a commit is made to the QGIS code repository an email is posted to
% this list. If you want to be up to date with every change to the current code
% base, you can subscribe to this list at:\\
% \url{http://lists.osgeo.org/mailman/listinfo/qgis-commit}
\`A chaque fois qu'un commit est r\'ealis\'e sur le d\'ep\^ot du code de QGIS un email
est envoy\'e \`a cette liste. Si vous voulez \^etre \`a jour de chaque changement au
code en cours, vous pouvez vous inscrire \`a cette liste :\\
\url{http://lists.osgeo.org/mailman/listinfo/qgis-commit}

\minisec{qgis-trac}
% This list provides email notification related to project management,
% including bug reports, tasks, and feature requests. You can subscribe to this
% list at:\\
% \url{http://lists.osgeo.org/mailman/listinfo/qgis-trac}
Cette liste fournit une notification par mail li\'ee \`a la gestion du projet,
incluant les rapports de bugs, t\^aches, et demande de fonctionnalit\'es. Vous
pouvez vous inscrire \`a cette liste ici :\\
\url{http://lists.osgeo.org/mailman/listinfo/qgis-trac}

\minisec{qgis-community-team}
% This list deals with topics like documentation, context help, user-guide,
% online experience including web sites, blog, mailing lists, forums, and
% translation efforts. If you like to work on the user-guide as well, this list
% is a good starting point to ask your questions. You can subscribe to this
% list at:\\
% \url{http://lists.osgeo.org/mailman/listinfo/qgis-community-team}
Cette liste re\c{c}oit les mails des th\'ematiques li\'e \`a la documentation, au
contexte d'aide, au guide utilisateur, \`a ce qui est li\'e \`a Internet donc les
sites, listes de diffusion, forums et efforts de traduction. Si vous voulez
travailler sur le guide utilisateur, cette liste est un bon point de d\'epart
pour poser vos questions. Vous pouvez vous inscrire \`a cette liste ici :\\
\url{http://lists.osgeo.org/mailman/listinfo/qgis-community-team}

\minisec{qgis-release-team}
% This list deals with topics like the release process, packaging binaries for
% various OS and announcing new releases to the world at large. You can
% subscribe to this list at:\\
% \url{http://lists.osgeo.org/mailman/listinfo/qgis-release-team}
Cette liste re\c{c}oit les mails des th\'ematiques comme les proc\'edures de
publication de version, paquetage binaire pour diff\'erents syst\`emes et annonce
des nouvelles versions \`a un monde plus large. Vous pouvez vous inscrire \`a cette
liste ici :\\
\url{http://lists.osgeo.org/mailman/listinfo/qgis-release-team}

\minisec{qgis-psc}
% This list is used to discuss Steering Committee issues related to overall
% management and direction of Quantum GIS. You can subscribe to this list at:\\
% \url{http://lists.osgeo.org/mailman/listinfo/qgis-psc}
Cette liste est utilis\'ee pour discuter des probl\`emes du Comit\'e de Pilotage li\'e \`a
l'ensemble de la gestion et de la direction de Quantum GIS. Vous pouvez vous
inscrire \`a cette liste ici :\\
\url{http://lists.osgeo.org/mailman/listinfo/qgis-psc}

% You are welcome to subscribe to any of the lists. Please remember to
% contribute to the list by answering questions and sharing your experiences.
% Note that the qgis-commit and qgis-trac are designed for notification only
% and not meant for user postings.
Vous \^etes invit\'e \`a vous inscrire \`a ces listes. S'il vous plait, souvenez-vous de
contribuer \`a la liste en r\'epondant \`a des questions et en partageant vos
exp\'eriences. Remarquez que les listes qgis-commit et qgis-trac ont \'et\'e
configur\'ees pour notification seulement et n'acceptent pas de mail
d'utilisateurs.

\subsection{IRC}
% We also maintain a presence on IRC - visit us by joining the \#qgis channel on
% \url{irc.freenode.net}. Please wait around for a response to your question as
% many folks on the channel are doing other things and it may take a while for
% them to notice your question. Commercial support for QGIS is also available.
% Check the website \url{http://qgis.org/content/view/90/91} for more
% information.
Nous maintenons une pr\'esence sur IRC - rejoignez-nous sur le canal \#qgis sur
\url{irc.freenode.net}. S'il vous plait, patientez pour obtenir une r\'eponse
puisque la plupart des personnes font autre chose et cela peut leur prendre un
peu de temps pour remarquer votre question. Un support commercial pour QGIS est
disponible. Regardez la page du site \url{http://qgis.org/content/view/90/91}
pour plus d'informations.

% If you missed a discussion on IRC, not a problem! We log all discussion so you
% can easily catch up. Just go to \url{http://logs.qgis.org} and read the
% IRC-logs.
Si vous ratez une discussion sur IRC, pas de probl\`eme ! Nous loguons toutes les
discussions afin que vous puisiez facilement les suivre. Allez simplement sur
\url{http://logs.qgis.org} et lisez les logs IRC.

\subsection{BugTracker}
% While the qgis-users mailing list is useful for general 'how do I do xyz in
% QGIS' type questions, you may wish to notify us about bugs in QGIS. You can
% submit bug reports using the QGIS bug tracker at
% \url{https://trac.osgeo.org/qgis/}. When creating a new ticket for a bug,
% please provide an email address where we can request additional information.
Bien que la liste de diffusion utilisateur est utile pour des questions
g\'en\'erales du type 'Comment je r\'ealise xyz dans QGIS ?', vous pouvez vouloir
nous avertir de bugs dans QGIS. Vous pouvez soumettre un rapport de bug en
utilisant le tracker de bug sur \url{https://trac.osgeo.org/qgis/}. Lors de la
cr\'eation d'un ticket pour un bug, fournissez s'il vous plait une adresse mail
valide o\`u nous pouvons vous demander des informations suppl\'ementaires.

% Please bear in mind that your bug may not always enjoy the priority you might
% hope for (depending on its severity). Some bugs may require significant
% developer effort to remedy and the manpower is not always available for this.
Garder en m\'emoire que votre bug peut ne pas avoir la priorit\'e \`a laquelle vous
vous attendiez (cela d\'ependra de sa s\'ev\'erit\'e). Certains bugs peuvent n\'ecessiter du
travail suppl\'ementaire de la part des d\'eveloppeurs pour y rem\'edier et la personne
comp\'etente n'est pas forc\'ement disponible.

% Feature requests can be submitted as well using the same ticket system as for
% bugs. Please make sure to select the type \usertext{enhancement}.
Les demandes de fonctionnalit\'e peuvent \^etre soumises \'egalement en utilisant le
m\^eme syst\`eme de ticket que pour les bugs. Assurez-vous de s\'electionner le type
\usertext{enhancement}.

% If you have found a bug and fixed it yourself you can submit this patch also.
% Again, the lovely trac ticketsystem at \url{https://trac.osgeo.org/qgis/} has
% this type as well. Select \usertext{patch} from the type-menu. Someone of the 
% developers will review it and apply it to QGIS. \\
Si vous avez trouv\'e un bug et l'avez corrig\'e vous m\^eme, vous pouvez
aussi soumettre un patch. Encore, le superbe syst\`eme de ticket Trac sur
\url{https://trac.osgeo.org/qgis/} a \'egalement ce type. S\'electionnez
\usertext{patch} dans le menu type. Un des d\'eveloppeurs le v\'erifiera et
l'appliquera \`a QGIS.\\
% Please don't be alarmed if your patch is not applied straight away -
% developers may be tied up with other committments.
Ne vous alarmez pas si votre correctif n'est pas appliqu\'e directement - les
d\'eveloppeurs peuvent \^etre occup\'es sur d'autres commits.

% unused, since community.qgis.org seems to be lost. (SH)
% There is also a community site for QGIS where we encourage QGIS users to share
% their experiences and provide case studies about how they are using QGIS. The
% community site is available at: http://community.qgis.org 

\subsection{Blog}
% The QGIS-community also runs a weblog (BLOG) at \url{http://blog.qgis.org} 
% which has some interesting articles for users and developers as well. 
% You are invited to contribute to the blog after registering yourself!
La communaut\'e QGIS tient \'egalement un weblog (BLOG) sur
\url{http://blog.qgis.org} qui publie d'int\'eressants articles \`a la fois pour les
utilisateurs et les d\'eveloppeurs. Vous \^etes invit\'es \`a contribuer au blog apr\`es
vous \^etre enregistr\'es.

\subsection{Wiki}
% Lastly, we maintain a WIKI web site at \url{http://wiki.qgis.org} where you 
% can find a variety of useful information relating to QGIS development, 
% release plans, links to download sites, message translation-hints and so
% on. Check it out, there are some goodies inside!
Enfin, nous maintenons un site web wiki sur \url{http://wiki.qgis.org} o\`u vous
pouvez trouver diverses informations utiles li\'ees au d\'eveloppement de QGIS, plan
des versions, liens vers les sites de t\'el\'echargement, astuces de
traduction des messages, etc. Parcourez le, il y a des choses int\'eressantes.
