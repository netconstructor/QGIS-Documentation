\section{Avant-propos}\label{label_forward}
\pagenumbering{arabic}
\setcounter{page}{1}

% when the revision of a section has been finalized, 
% comment out the following line:
% \updatedisclaimer

%Welcome to the wonderful world of Geographical Information Systems (GIS)!
%Quantum GIS (QGIS) is an Open Source Geographic Information System. The project
%was born in May of 2002 and was established as a project on SourceForge in June
%of the same year. We've worked hard to make GIS software (which is traditionally
%expensive proprietary software) a viable prospect for anyone with basic access
%to a Personal Computer. QGIS currently runs on most Unix platforms, Windows, and
%OS X. QGIS is developed using the Qt toolkit (\url{http://www.trolltech.com})
%and C++. This means that QGIS feels snappy to use and has a pleasing, easy-to-
%use graphical user interface (GUI). 

Bienvenue dans le monde merveilleux des Syst\`emes d'Information g\'eographiques (SIG) ! Quantum GIS est un SIG libre qui a d\'ebut\'e en mai 2002 et s'est \'etabli en tant que projet en juin 2002 sur SourceForge. Nous avons travaill\'e dur pour faire de ce logiciel SIG (qui sont traditionnellement des logiciels propri\'etaires assez co\^uteux) un choix viable pour toute personne ayant un ordinateur. QGIS est utilisable sur la majorit\'e des Unix, Mac OS X et Windows. QGIS utilise la biblioth\`eque logicielle Qt 4 (\url{http://www.trolltech.com}) et le langage C++, ce qui ce traduit par une interface graphique simple et r\'eactive.

%QGIS aims to be an easy-to-use GIS, providing common functions and features.
%The initial goal was to provide a GIS data viewer. QGIS has reached the point
%in its evolution where it is being used by many for their daily GIS data viewing
%needs. QGIS supports a number of raster and vector data formats, with new
%format support easily added using the plugin architecture (see Appendix
%\ref{appdx_data_formats} for a full list of currently supported data formats).

QGIS se veut simple \`a utiliser, fournissant des fonctionnalit\'es courantes. Le but initial \'etait de fournir un visualisateur de donn\'ees SIG, QGIS a depuis atteint un stade dans son \'evolution o\`u beaucoup y recourent pour leurs besoins journaliers. QGIS supporte un grand nombre de formats raster et vecteur, avec un support de nouveaux formats facilit\'es par l'architecture des modules d'extension (lisez l'Annexe \ref{appdx_data_formats} pour une liste compl\`ete des formats actuellement support\'es)

%QGIS is released under the GNU General Public License (GPL). Developing QGIS 
%under this license means that you can inspect and modify the source code,
%and guarantees that you, our happy user, will always have access to a GIS
%program that is free of cost and can be freely modified. You should have
%received a full copy of the license with your copy of QGIS, and you also can
%find it in Appendix \ref{gpl_appendix}.  

QGIS est distribu\'e sous la licence GPL. Ceci vous permet de pouvoir regarder et modifier le code source, tout en vous garantissant un acc\`es \`a un programme SIG sans co\^ut et librement modifiable. Vous devez avoir re\c{c}u une copie compl\`ete de la licence avec votre exemplaire de QGIS, vous la trouverez \'egalement dans l'Annexe \ref{gpl_appendix}.

%\begin{Tip}\caption{\textsc{Up-to-date Documentation}}\index{documentation}
%\qgistip{The latest version of this document can always be found at 
%\url{http://download.osgeo.org/qgis/doc/manual/}, or in the documentation
%area of the QGIS website at \url{http://qgis.osgeo.org/documentation/}
%}
%\end{Tip}

\begin{Astuce}\caption{\textsc{Documentation \`a jour}}\index{documentation}
\qgistip{La derni\`ere version de ce document est disponible sur \url{http://download.osgeo.org/qgis/doc/manual/}, ou dans la section documentation du site de QGIS \url{http://qgis.osgeo.org/documentation/}
}
\end{Astuce}

%\subsection{Features}\label{label_majfeat}

%QGIS offers many common GIS functionalities provided by core features and
%plugins. As a short summary they are presented in six categories to gain a
%first insight.

\subsection{Fonctionnalit\'es}\label{label_majfeat}

QGIS offre beaucoup d'outils SIG standards par d\'efaut et via les extensions. Voici un bref r\'esum\'e en six cat\'egories qui vous donnera un premier aper\c{c}u.

%\minisec{View data}

%You can view and overlay vector and raster data in different formats and
%projections without conversion to an internal or common format. Supported
%formats include:

\minisec{Visualiser des donn\'ees}

Vous pouvez afficher et superposer des couches de donn\'ees rasters et vecteurs dans diff\'erents formats et projections sans avoir \`a faire de conversion dans un format commun. Les formats support\'es incluent :

%\begin{itemize}
%\item spatially-enabled PostgreSQL tables using PostGIS, vector formats
%\footnote{OGR-supported database formats such as Oracle or mySQL are not yet
%supported in QGIS.} supported by the installed OGR library, including ESRI
%shapefiles, MapInfo, SDTS and GML.
%\item Raster and imagery formats supported by the installed GDAL (Geospatial
%Data Abstraction Library) library, such 
%as GeoTiff, Erdas Img., ArcInfo Ascii Grid, JPEG, PNG,
%\item GRASS raster and vector data from GRASS databases (location/mapset), 
%\item Online spatial data served as OGC-compliant Web Map Service (WMS) or
%Web Feature Service (WFS).%
%\end{itemize}

\begin{itemize}
\item les tables spatiales de PostgreSQL/PostGIS, les formats vecteurs support\'es par la biblioth\`eque OGR install\'ee \footnote{les formats de base de donn\'ees Oracle et MySQL sont support\'es par OGR mais pas encore par QGIS.}, ce qui inclue les fichiers de forme ESRI (shapefiles), MapInfo, STDS et GML.
\item les formats raster support\'es par la biblioth\`eque GDAL (Geospatial Data Abstraction Library) tel que GeoTiff, Erads Img., ArcInfo Ascii Grid, JPEG, PNG...
\item les formats raster et vecteur provenant des bases donn\'ees GRASS. 
\item les donn\'ees spatiales provenant des services r\'eseaux compatibles OGC comme le Web Map Service (WMS) ou le Web Feature Service (WFS).
\end{itemize}

%\minisec{Explore data and compose maps} 

%You can compose maps and interactively explore spatial data with a friendly
%GUI. The many helpful tools available in the GUI include:

\minisec{Parcourir les donn\'ees et cr\'eer des cartes} 

Vous pouvez cr\'eer des cartes et les parcourir de mani\`ere interactive avec une interface abordable. Les outils disponibles dans l'interface sont :

%\begin{itemize}
%\item on the fly projection
%\item map composer
%\item overview panel
%\item spatial bookmarks
%\item identify/select features
%\item edit/view/search attributes
%\item feature labeling
%\item change vector and raster symbology
%\item add a graticule layer
%\item decorate your map with a north arrow scale bar and copyright label
%\item save and restore projects
%\end{itemize}

\begin{itemize}
\item projection \`a la vol\'ee
\item cr\'eateur de carte
\item panneau de navigation
\item marque-pages spatiaux
\item identifier et s\'electionner des entit\'es
\item voir, \'editer et rechercher des attributs
\item \'etiquetage des entit\'es
\item changer la symbologie des rasters et vecteurs
\item ajouter une couche de graticule
\item ajout d'une barre d'\'echelle, d'une fl\`eche indiquant le nord et d'une \'etiquette de droits d'auteur
\item sauvegarde et chargement de projets
\end{itemize}

%\minisec{Create, edit, manage and export data}

%You can create, edit, manage and export vector maps in several formats. Raster data
%have to be imported into GRASS to be able to edit and export them into other
%formats. QGIS offers the following: 

%\begin{itemize}
%\item digitizing tools for OGR supported formats and GRASS vector layer
%\item create and edit shapefiles and GRASS vector layer
%\item geocode images with the georeferencer plugin
%\item GPS tools to import and export GPX format, and convert other GPS
%formats to GPX or down/upload directly to a GPS unit
%\item create PostGIS layers from shapefiles with the SPIT plugin
%\item manage vector attribute tables with the table manager plugin  
%\end{itemize}

\minisec{Cr\'eer, \'editer, g\'erer et exporter des donn\'ees}

Vous pouvez cr\'eer, \'editer, g\'erer et exporter des donn\'ees vecteurs dans plusieurs formats. Les donn\'ees raster doivent \^etre import\'e dans GRASS pour pouvoir \^etre \'edit\'e et exporter dans d'autres formats. QGIS permet ce qui suit :  

\begin{itemize}
\item outils de num\'erisation pour les formats d'OGR et les couches vecteurs de GRASS
\item cr\'eer et \'editer des fichiers de forme (shapefiles) et les couches vecteurs de GRASS
\item g\'eocodifier des images avec l'extension de g\'eor\'ef\'erencement
\item outils d'import/export du format GPX pour les donn\'ees GPS, avec la conversion des autres formats GPS vers le GPX ou l'envoi/r\'eception directement vers une unit\'e GPS
\item cr\'eer des couches PostGIS \`a partir de fichiers de forme (shapefiles) avec l'extension SPIT
\item g\'erer les attributs de tables des couches vecteurs gr\^ace \`a l'extension de gestion des tables
\end{itemize}

\minisec{Analyser les donn\'ees} 

Vous pouvez op\'erer des analyses spatiales sur des donn\'ees PostgreSQL/PostGIS et autres formats OGR en utilisant l'extension ftools. QGIS permet actuellement l'analyse vectorielle, l'\'echantillonnage, la gestion de la g\'eom\'etrie et des bases de donn\'ees. Vous pouvez aussi utiliser les outils GRASS int\'egr\'es, qui comportent plus de 300 modules (voir la section \ref{sec:grass})


%\minisec{Publish maps on the internet}

%QGIS can be used to export data to a mapfile and to publish them on the
%internet using a webserver with UMN MapServer installed. QGIS can also
%be used as a WMS or WFS client, and as WMS server. 

\minisec{Publier une carte sur Internet}

QGIS peut \^etre employ\'e pour exporter des donn\'ees vers un mapfile et le publiqer sur internet via un serveur web employant l'UMN MapServer. QGIS peut aussi servir de client WMS/WFS ou comme un serveur WMS.

%\minisec{Extend QGIS functionality through plugins} 

%QGIS can be adapted to your special needs with the extensible
%plugin architecture. QGIS provides libraries that can be used to create
%plugins.  You can even create new applications with C++ or Python!

\minisec{Etendre les fonctionnalit\'es de QGIS gr\^a\c{c}e \`a des extensions} 

QGIS peut \^etre adapt\'ee \`a vos besoins particuliers du fait de son architecture d'extensions. QGIS fournit des biblioth\`eques qui peuvent \^etre employ\'ees pour cr\'eer des extensions, vous pouvez m\^eme cr\'eer de nouvelles applications en C++ ou python !

%\begin{itemize}
%\item \textbf{Core Plugins}
%\\ \\ Add WFS Layer
%\\ Add Delimited Text Layer
%\\ Coordinate Capture
%\\ Decorations (Copyright Label, North Arrow and Scale bar)
%\\ Georeferencer
%\\ Dxf2Shp Converter
%\\ GPS Tools
%\\ GRASS integration
%\\ Graticules Creator
%\\ Interpolation Plugin
%\\ OGR Layer Converter
%\\ Quick Print
%\\ SPIT Shapefile to PostgreSQL/PostGIS Import Tool
%\\ Mapserver Export
%\\ Python Console
%\\ Python Plugin Installer
%\\ \item \textbf{Python Plugins}
%\\ \\ QGIS offers a growing number of external python plugins that are 
%provided by the
%community. These plugins reside in the the official
%PyQGIS repository, and can be easily installed using the python plugin 
%installer (See Section \ref{sec:plugins}).
%\end{itemize}

\begin{itemize}
\item \textbf{Extensions principales}
\\ \\ Ajouter une couche WFS 
\\ Ajouter une couche de texte d\'elimit\'e
\\ Capture de coordonn\'ees
\\ D\'ecorations (\'Etiquette de droit d'auteur, fl\`eche indiquant le nord et barre d'\'echelle)
\\ Geor\'er\'erencement
\\ Convertisser Dxf2Shp 
\\ Outils GPS 
\\ Int\'egration de GRASS
\\ Cr\'eateur de graticules
\\ Exenstion d'interpolation
\\ Convertisseur de couche OGR
\\ impression rapide
\\ SPIT, outil d'importation de Shapefile vers PostgreSQL/PostGIS
\\ Exportation vers Mapserver
\\ Terminal Python
\\ Installateur d'extension Python
\\ \item \textbf{Extensions Python}
\\ \\ QGIS offre un nombre crossant d'extensions compl\'ementaires en python fourni par la communaut\'e. Ces extensions sont entrepos\'ees dan le r\'epertoire PyQGIS et peuvent \^etre facilement install\'e en utilisant l'extension d'installation python (Voir Section \ref{sec:extensions}).
\end{itemize}
