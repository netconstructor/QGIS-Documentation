% vim: set textwidth=78 autoindent:

\section{Prefacio}\label{label_forward}
\pagenumbering{arabic}
\setcounter{page}{1}

¡Bienvenido al maravilloso mundo de los Sistemas de Información Geográfica(SIG)!
Quantum GIS (QGIS) es un Sistema de Información Geográfica de Código Abierto. 
El proyecto nació en Mayo de 2002 y se estableció como un proyecto dentro de 
SourceForge en junio del mismo año. Hemos trabajado duro para hacer del 
software SIG (que tradicionalmente es software comercial caro) una posibilidad
viable para cualquiera con un acceso básico a un ordenador personal. 
Actualmente QGIS corre en la mayoría de plataformas Unix, Windows y OS X. 
QGIS está desarrollado utilizando el Qt toolkit (\url{http://www.trolltech.com})
y C++. Esto hace que QGIS sea rápido y tenga una interfaz de usuario agradable
y fácil de usar.

QGIS espera ser un SIG fácil de usar, proporcionando características y 
funciones comunes. El objetivo incial fue proporcionar un visor de datos SIG. 
QGIS ha alcanzado este punto en su evolución y está siendo utilizado por muchos 
para sus necesidades diarias de visualización de datos SIG. QGIS soporta un 
buen número de formatos ráster y vectoriales, con nuevos soportes fácilmente
añadidos utilizando su arquitectura de complementos (ver Apéndice \ref{appdx_data_formats} 
para consultar la lista completa de los formatos de datos soportados).
QGIS se ha publicado bajo Licencia Pública (GNU General Public License) (GPL). 
Desarrollar QGIS bajo esta licencia quiere decir que se puede inspeccionar y 
modificar el código fuente y las garantías que se tienen, nuestros felices 
usuarios siempre tienen acceso a un programa SIG gratis y que puede ser 
libremente modificado. Debe tener una copia de la licencia con su copia 
de QGIS, y tampién se puede encontra como Apéndice \ref{gpl_appendix}.  

\begin{quote}
\begin{center}
\textbf{Note:} La última versión de este documento siempre se encuentra en \newline
http://qgis.org/docs/userguide.pdf 
\end{center}
\end{quote}

\subsection{Características}\label{label_majfeat}

QGIS tiene muchas funciones y características comunes a todos los SIG. 
Las características princiaples se enumeran aqui debajo, divididas en
elementos del núcleo y complementos. \\

\textbf{Elementos del núcleo}

\begin{itemize}
\item Soporte ráster y vectorial mediante la librería OGR.
\item Soporte para PostgreSQL con tablas espaciales utilizando PostGIS.
\item Integración con GRASS, incluída visualización, edición y análisis.
\item Digitalización GRASS y OGR/Shapefile.
\item Diseño de Mapas.
\item Soporte OGC.
\item Panel de Vista General.
\item Marcadores espaciales.
\item Identificar/Seleccionar elementos.
\item Editar/Visualizar/Buscar atributos.
\item Etiquetado de elementos.
\item Proyecciones al vuelo.
\item Guardar y recuperar proyectos.
\item Exportar ficheros map a Mapserver.
\item Cambiar simbología vectorial y raster.
\item Arquitectura extensible con complementos.
\end{itemize}

\textbf{Complementos}

\begin{itemize}
\item Añadir capas WFS.
\item Añadir capas de texto delimitado.
\item Decoración (etiqueta de copyright, flecha de Norte y barra de escala)
\item Georreferencación.
\item Herramientas GPS.
\item GRASS.
\item Generador de mallas.
\item Funciones de geoprocesamiento PostgreSQL.
\item Herramienta de importación de archivos shape a PostgreSQL/PostGIS (SPIT - Shapefile to PostgreSQL/PostGIS Import Tool)
\item Consola de Python.
\item openModeller.
\end{itemize}

\subsection{Qué es nuevo en 0.9.0}\label{label_whatsnew}

Como es habitual, la nueva versión 0.9.0 te ofrece muchas características 
interesantes.

\begin{itemize}
\item El lenguaje Python posibilita escribir complementos en Python y crear 
aplicaciones SIG que utilicen librerías de QGIS.
\item Eliminado el sistema de compilación con automake build system - QGIS ahora necesita CMake para su 
compilación.
\item Algunos módulos nuevos de GRASS añadidos a la barra de herramientas.
\item Actualizaciones en el editor de Mapas.
\item Correcciones en los ficheros shape 2.5D.
\item Mejoras en la georeferenciación.
\item Soporte de localización extendido a 26 idiomas.
\end{itemize}

