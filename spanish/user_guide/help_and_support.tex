\section{Ayuda y soporte}\label{label_helpsupport}

\subsection{Listas de correo}
QGIS aún está bajo un desarrollo activo y como tal, no siempre funcionará como se espera. El modo preferible para obtener ayuda es apuntarse a la lista de correo de usuarios de qgis (qgis-users).

\minisec{qgis-users}
Sus preguntas llegarán a una audiencia mayor y las respuestas beneficiarán a otros. Puede suscribirse a la lista de correo de usuarios de qgis (qgis-users) visitando la siguiente URL: \\
\url{http://lists.qgis.org/cgi-bin/mailman/listinfo/qgis-user}

\minisec{qgis-developer}
Si es un desarrollador con problemas de carácter más técnico, quizá quiera unirse a la lista de correo de desarrolladores (qgis-developer) aquí:\\
\url{http://lists.qgis.org/cgi-bin/mailman/listinfo/qgis-developer}

\minisec{qgis-commit}
Cada vez que se hace un envío al repositorio del código de QGIS se envía un correo a esta lista. Si quiere estar al día con cada cambio en el actual código base, se puede suscribir a esta lista en:\\
\url{http://lists.qgis.org/cgi-bin/mailman/listinfo/qgis-commit}

\minisec{qgis-trac}
Esta lista proporciona notificaciones por correo electrónico relacionadas con la administración del proyecto, incluyendo informes de errores, tareas y solicitudes de funciones. Puede suscribirse a esta lista en:\\
\url{http://lists.qgis.org/cgi-bin/mailman/listinfo/qgis-trac}

\minisec{qgis-doc}
Esta lista trata asuntos como la documentación, ayuda contextual, guía de usuario y esfuerzos de traducción. Si también quiere trabajar en la guía de usuario, esta lista es un buen punto de inicio para hacer sus preguntas. Puede suscribirse a esta lista en:\\
\url{http://lists.qgis.org/cgi-bin/mailman/listinfo/qgis-doc}

\minisec{qgis-psc}
Esta lista se usa para debatir asuntos del Comité de Dirección (Steering Committee) relacionados con la administración general de Quantum GIS. Puede suscribirse a esta lista en:\\
\url{http://mrcc.com/cgi-bin/mailman/listinfo/qgis-psc}

Será bienvenido a cualquiera de las listas. Por favor, recuerde contribuir a la lista contestando preguntas y compartiendo sus experiencias. Tenga en cuenta que  las listas qgis-commit y qgis-trac están diseñadas para la notificación solamente y no para correos de los usuarios. 

\subsection{IRC}
También mantenemos una presencia en IRC – visítenos uniéndose al canal \#qgis en
\url{irc.freenode.net}. Por favor, espere un poco a las respuestas a sus preguntas, ya que muchos colegas en el canal están haciendo otras cosas y puede llevar un rato hasta que noten su pregunta. También hay disponible soporte comercial para QGIS. Compruebe la página web \url{http://qgis.org/content/view/90/91} para más información.

¡Si se has perdido un debate en IRC, no hay problema! Registramos todos los debates, así que puede recuperarlo fácilmente. Simplemente vaya a \url{http://logs.qgis.org} y lea los registros del IRC.

\subsection{Seguimiento de errores (BugTracker)}
Mientras que la lista de correo de usuarios de qgis es útil para preguntas del tipo ¿Cómo hago xyz en QGIS?, puede que desee notificarnos un error en QGIS. Puede enviar informes de errores usando el seguidor de errores de QGIS en \url{http://svn.qgis.org/trac}. Cuando cree un nuevo registro para un error, por favor proporcione una dirección de correo electrónico en la que podamos solicitar información adicional.

Por favor, tenga en mente que su error no siempre tendrá la prioridad que pueda esperar (dependiendo de la gravedad). Algunos errores pueden requerir un esfuerzo significativo para remediarlos y no siempre se dispone de los recursos humanos para ello.

Las solicitudes de funciones se pueden enviar usando el mismo sistema que para los errores. Por favor, asegúrese de seleccionar el tipo \textbf{enhancement} (mejora).

Si ha encontrado un error y lo soluciona usted mismo, puede enviar también el parche. El sistema de seguimiento en \url{http://svn.qgis.org/trac} tiene también este tipo. Seleccione \textbf{patch} (parche) en el menú tipo. Alguno de los desarrolladores lo revisará y lo aplicará a QGIS. No se alarme si su parche no se aplica inmediatamente, los desarrolladores pueden estar desbordados con otros envíos.

% unused, since community.qgis.org seems to be lost. (SH)
% There is also a community site for QGIS where we encourage QGIS users to share
% their experiences and provide case studies about how they are using QGIS. The
% community site is available at: http://community.qgis.org 

\subsection{Blog}
La comunidad de QGIS también lleva un blog web (BLOG) en \url{http://blog.qgis.org} el cual tiene algunos artículos interesantes para usuarios y desarrolladores. Le invitamos a contribuir al blog después de registrarse.

\subsection{Wiki}
Por último, mantenemos un WIKI en la web en \url{http://wiki.qgis.org} donde puede encontrar información útil variada relacionada con el desarrollo de QGIS, planes de lanzamientos, enlaces a sitios de descarga, apuntes sobre la traducción de mensajes, etc. ¡Compruébelo, hay buenas cosas dentro!

