%  !TeX  root  =  user_guide.tex
\mainmatter
\pagestyle{scrheadings}
\addchap{Предисловие}\label{label_forward}
\pagenumbering{arabic}
\setcounter{page}{1}


% when the revision of a section has been finalized,
% comment out the following line:
% \updatedisclaimer

Добро пожаловать в удивительный мир географических информационных систем
(ГИС)!

Quantum~GIS (QGIS) является ГИС с открытым исходным кодом. Работа
над QGIS была начата в мае 2002~года, а в июне того же года"--- создан
проект на площадке SourceForge. Мы много работали, чтобы сделать
программное обеспечение ГИС (которое традиционно является дорогим
проприетарным ПО) доступным любому, кто имеет
доступ к персональному компьютеру. В настоящее время QGIS работает на
большинстве платформ: Unix, Windows, и OS~X. QGIS разработан с
использованием инструментария Qt (\url{http://qt.nokia.com}) и языка
программирования C++.
Это означает, что QGIS легок в использовании, имеет приятный и простой
графический интерфейс.

QGIS стремится быть легкой в использовании ГИС, предоставляя общую
функциональность. Первоначальная цель заключалась в облегчении
просмотра геоданных и QGIS достиг той стадии в своем развитии, когда
многие используют ее в своих ежедневных задачах просмотра.
QGIS поддерживает множество растровых и векторных форматов данных, а
поддержка новых форматов реализуется с помощью модулей (полный список
поддерживаемых форматов данных см. в Приложении~\ref{appdx_data_formats}).

QGIS выпускается на условиях лицензии GNU General Public License (GPL).
Разработка QGIS под этой лицензией означает, что вы можете просмотреть и
изменить исходный код, и гарантирует, что вы, наш счастливый
пользователь, всегда будете иметь доступ к программному обуспечению ГИС,
которое является бесплатным и может свободно адаптироваться. Вы должны
были получить полную копию лицензии с вашей копией QGIS, лицензию также
можете найти в Приложении~\ref{gpl_appendix}.

\begin{Tip}\caption{\textsc{Актуальная версия документации}}\index{документация}
Актуальную версию данного документа всегда можно найти на странице
\url{http://download.osgeo.org/qgis/doc/manual/}, или в разделе
документации на веб-сайте QGIS \url{http://qgis.osgeo.org/documentation/}
\end{Tip}

%%% TODO: add link to the manual
Русскоязычную версию руководства, созданную в рамках коллективного
проекта GIS-Lab, можно найти по адресу:
\url{http://gis-lab.info/docs/qgis/manual15/qgis-1.5.0_user_guide_ru.pdf}.
На данный момент доступен перевод версии 1.5 руководства.

\addsec{Возможности}\label{label_majfeat}

\qg позволяет использовать большое количество распространенных ГИС функций,
обеспечиваемых встроенными инструментами и модулями. Первое представление
можно получить из краткого резюме ниже, где функции разбиты на шесть
категорий.

\minisec{Просмотр данных}

Можно просматривать и накладывать друг на друга векторные и растровые
данные в различных форматах и проекциях без преобразования во внутренний
или общий формат. Поддерживаются следующие основные форматы:

\begin{itemize}[label=--]
\item пространственные таблицы PostgreSQL с использованием PostGIS, векторные
форматы,
%\footnote{форматы баз данных, поддерживаемые библиотекой OGR, такие как Oracle или
%mySQL, в QGIS пока не поддерживаются.}
поддерживаемые установленной библиотекой OGR, включая shape-файлы ESRI,
MapInfo, SDTS (Spatial Data Transfer Standard) и GML (Geography Markup
Language) (полный список см. в Приложении~\ref{appdx_ogr}).
\item Форматы растров и графики, поддерживаемые
библиотекой GDAL (Geospatial Data Abstraction Library), такие, как
GeoTIFF, Erdas IMG, ArcInfo ASCII Grid, JPEG, PNG (полный список см. в
Приложении~\ref{appdx_gdal}).
\item базы данных SpatiaLite (см. Раздел~\ref{label_spatialite})
\item растровый и векторный форматы GRASS (область/набор данных),
см. Раздел~\ref{sec:grass}.
\item Пространственные данные, публикуемые в сети Интернет с помощью
OGC-совместимых (Open Geospatial Consortium) сервисов Web Map Service
(WMS) или Web Feature Service (WFS), см. Раздел~\ref{working_with_ogc},
\item данные OpenStreetMap (OSM) (см. Раздел~\ref{plugins_osm}).
\end{itemize}

\minisec{Исследование данных и компоновка карт}

С помощью удобного графического интерфейса можно создавать карты и
исследовать пространственные данные. Графический интерфейс включает в
себя множество полезных инструментов,например:

\begin{itemize}[label=--]
\item перепроецирование <<на лету>>
\item компоновщик карт
\item панель обзора
\item пространственные закладки
\item определение/выборка объектов
\item редактирование/просмотр/поиск атрибутов
\item подписывание объектов
\item изменение символики векторных и растровых слоев
\item добавление слоя координатной сетки"--- теперь средствами
  расширения fTools
\item добавление к макету карты стрелки на север, линейки масштаба
и знака авторского права
\item сохранение и загрузка проектов
\end{itemize}

\minisec{Управление данными: создание, редактирование и экспорт}

В QGIS можно создавать и редактировать векторные данные, а также
экспортировать их в разные форматы. Чтоб иметь возможность редактировать
и экпортировать в другие форматы растровые данные, необходимо
сначала импортировать их в GRASS. QGIS предоставляет следующие возможности
работы с данными, в частности:

\begin{itemize}[label=--]
\item инструменты оцифровки для форматов, поддерживаемых библиотекой OGR,
и векторных слоев GRASS
\item создание и редактирование shape-файлов и векторных слоев GRASS
\item геокодирование изображений с помощью модуля пространственной
привязки
\item инструменты GPS для импорта и экспорта данных в формате GPX,
преобразования прочих форматов GPS в формат GPX или скачивание/загрузка
непосредственно в прибор GPS (в Linux usb: был добавлен в список
устройств GPS)
\item визуализация и редактирование данных OpenStreetMap
\item создание слоёв PostGIS из shape-файлов с помощью плагина SPIT
\item обработка слоёв PostGIS
\item управление атрибутами векторных данных с помощью новой таблицы
атрибутов (см. Раздел~\ref{sec:attribute table}) или модуля Table Manager
\item сохранение снимков экрана как изображений с пространственной
привязкой
\end{itemize}

\minisec{Анализ данных}

Вы можете анализировать векторные пространственные данные в PostgreSQL/PostGIS
и других форматах, поддерживаемых OGR, используя модуль fTools, написанный на
языке программирования Python. В настоящее время QGIS предоставляет возможность
использовать инструменты анализа, выборки, геопроцессинга, управления
геометрией и базами данных. Также можно использовать интегрированные
инструменты GRASS, которые включают в себя функциональность более чем
300 модулей GRASS (см. Раздел~\ref{sec:grass}).

\minisec{Публикация карт в сети Интернет}

QGIS может использоваться для экспорта данных в map-файл и публикации
его в сети Интернет, используя установленный веб-сервер Mapserver.
QGIS может использоваться как клиент WMS/WFS и как сервер WMS.

\minisec{Расширение функциональности QGIS с помощью модулей расширения}

QGIS может быть адаптирован к особым потребностям с помощью расширяемой
архитектуры модулей. QGIS предоставляет библиотеки, которые могут
использоваться для создания модулей. Можно создавать отдельные
приложения, используя языки программирования C++ или Python.

\minisec{Основные модули}

\begin{enumerate}
\item Добавить слой из текста с разделителями (загружает и выводит
текстовые файлы, содержащие координаты x,y)
\item Захват координат (получает координаты мыши в различных системах
координат)
\item Оформление (знак авторского права, стрелка на север, масштабная
линейка)
\item Наложение диаграмм (наложение диаграмм на векторные слои)
\item Преобразователь Dxf2Shp (преобразование файлов DXF в shape-файлы)
\item Инструменты GPS (загрузка и импорт данных GPS)
\item GRASS (Поддержка ГИС GRASS)
\item Привязка растров GDAL (географическая привязка растров)
\item Модуль интерполяции (интерполяция векторных данных)
\item Экспорт в Mapserver (экспорт проекта QGIS в map-файл Mapserver)
\item Преобразователь слоев OGR (преобразование векторных данных в форматы,
поддерживаемые библиотекой OGR)
\item Модуль OpenStreetMap (просмотр и редактирование данных
OpenStreetMap)
\item Доступ к данным Oracle Spatial GeoRaster
\item Установщик модулей Python (загрузка и установка модулей QGIS)
\item Быстрая печать (печать карты с минимумом параметров)
\item Морфометрический анализ (морфометрический анализ растровых слоев)
\item SPIT (инструмент импорта shape-файлов в PostgreSQL/PostGIS)
\item Модуль WFS (загрузка слоёв WFS)
\item eVIS (инструмент визуализации событий"--- показ изображений, связанных
с векторными объектами)
\item fTools (инструменты для управления векторными данными и их анализа)
\item Консоль Python (доступ к среде разработки QGIS из самой программы)
\item Инструменты GDAL
\end{enumerate}

\minisec{Внешние модули Python}

QGIS предлагает постоянно растущее число модулей Python, которые
разрабатываются сообществом. Они находятся в официальном
репозитории PyQGIS, и могут быть легко установлены с помощью Установщика
модулей Python (см. Раздел~\ref{sec:plugins}).

\subsubsection{Что нового в версии \CURRENT}

Имейте ввиду, что этот выпуск является <<нестабильным>>. Это значит, что
помимо новых возможностей в нём, по сравнению с QGIS 1.0.x и QGIS 1.5.0,
расширен программный интерфейс. Мы рекомендуем использовать именно эту
версию вместо предыдущих.

Этот выпуск свыше 177 исправлений и множество новых возможностей и улучшений.

\textbf{Общие улучшения}

\begin{itemize}[label=--]
\item добавлена поддержка gpsd дляотслеживания GPS в режиме реального времени.
\item добавлен модуль оффлайнового редактирования.
\item калькулятор полей вставляет значение NULL, если привычислении выражения
возникла ошибка, а не завершает работу и отменяет все изменения, как раньше.
\item обновлённая база проекций srs.db.
\item встроенный растровый калькулятор (C++), позволяющий эффективно обрабытывать
большие изображения.
\item значения охвата в строке состояния можно копировать и вставлять.
\item множество улучшений и новые операторы в калькуляторе полей, включая
объединение полей и вставку счетчика записей.
\item добавлен параметр командной строки --configpath, который позволяет
перекрыть настройки по умолчанию для хранения данных пользователя (~/.qgis).
Это позволит пользователям создавать переносимую версию QGIS для USB-дисков.
\item экспериментальная поддержка WFS-T. Поддержка WFS переписана с
использованием Network Manager.
\item множество улучшений в модуле привязки растров.
\item Поддержка long int в таблице атрибутов и в редакторе полей.
\item QGIS Mapserver включен в состав QGIS и доступен в виде пакетов.
QGIS Mapserver позволяет публиковать проекты QGIS в Интернет с использованием
протокола OGC WMS.
\item расширены функции выбора и измерения.
\item добавлена поддержка непространственных таблиц (пока только в провайдерах
OGR, текст с разделителями и PostgreSQL). Такие таблицы могут использоваться
для поиска полей или же просто просматриваться и редактироваться.
\item поддержка поиска объектов по ID (\$id) и другие улучшения поиска.
\item в интерфейс слоёв и провайдеров добавлен метод reload. Это позволит
кэширующим провайдерам (например, WMS и WFS) синхронизироваться с источником
данных.
\end{itemize}

\textbf{Улучшения в Легенде}

\begin{itemize}[label=--]
\item растяжение гистограммы растровых слоёв по минимуму/максимуму
используя только текущее окно.
\item при сохранении векторных файлов из контекстного меню <<Save as>>
можно указать дополнительные параметры OGR.
\item возможность выделять и удалять несколько слоёв одновременно.
\end{itemize}

\textbf{Подписи (только новая символика)}

\begin{itemize}[label=--]
\item определяемое данными положение подписи.
\item перенос строк, определяемые данными шрифт и параметры буферизации.
\end{itemize}

\textbf{Свойства слоя и символика}

\begin{itemize}[label=--]
\item три новых режима классификации для градуированного условного знака
(новая символика), включая Естественные интервалы (Дженкс), Стандартные
отклонения и Наглядные интервалы (на основе алгоритма pretty пакета R).
\item улучшена скорость загрузки диалога свойств условного обозначения.
\item определяемые данными вращение и размер для градуированных и категорийных
условных знаков (новая символика).
\item масштабирование условного знака теперь влияет на ширину линии.
\item новая реализация растровой гистограммы на основе QWT. Добавлена
возможность сохранения гистограммы в файл. По оси X гистограммы выводятся
реальные значения пикселей.
\item возможность интерактивно выбирать пиксели с карты для заполнения
таблицы прозрачности в диалоге свойств растрового слоя.
\item возможность создавать градиенты при выборе градиента для векторного слоя.
\item в диалог выбора условных знаков добавлена кнопка <<Управление стилями>>.
\end{itemize}

\textbf{Компоновщик карт}

\begin{itemize}[label=--]
\item добавлена возможность изменять ширину и высоту элементов компоновки.
\item удаление элементов компоновки клавишей Backspace.
\item сортировка полей в таблице атрибутов компоновки (поддерживается
несколько колонок и сортировка по возрастанию / убыванию).
\end{itemize}

