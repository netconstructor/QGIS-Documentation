%  !TeX  root  =  user_guide.tex

\section{Модуль Table Manager}\label{sec:ftools}

% when the revision of a section has been finalized,
% comment out the following line:
%\updatedisclaimer

Модуль предоставляет инструменты управления атрибутивными таблицами. Он
позволяет управлять полями таблиц прямо из QGIS. Пользователю доступно
добавление, удаление, перемещение и клонирование полей, а также сохранение
результата в новый shape-файл. Модуль ничего не менят в исходной таблице.
Все изменения, по соображениям безопасности, сохраняются как новый файл.

\minisec{Установка модуля Table Manager}

Для использования вышеперечисленных функций в QGIS необходимо установить модуль
<<Table Manager>> при помощи \mainmenuopt{Загрузить модули\ldots} Установщика
(смотри Раздел~\ref{sec:load_external_plugin}). Затем выбрать и
активировать его в <<Менеджере модулей>>. Для этого выберите меню
\mainmenuopt{Модули} \arrow \mainmenuopt{Управление модулями}, выделите
\dropmenuopt{Table manager} и нажмите \button{OK}. На панели
инструментов появится новая кнопка \toolbtntwo{tableManagerIcon}{Table manager}.
