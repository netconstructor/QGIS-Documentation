\chapter{Модули GRASS}\label{appdx_grass_toolbox_modules}

% when the revision of a section has been finalized,
% comment out the following line:
% \updatedisclaimer

<<Оболочка GRASS>> в панели <<Инструменты GRASS>> предоставляет доступ почти ко всем
(более чем к 300) модулям GRASS в режиме командной строки. Чтобы предложить
более дружественную к пользователю рабочую среду, примерно 200 из числа
доступных модулей и их функций представлены также в графических диалогах
(меню).

\section{Модули импорта и экспорта данных}\index{GRASS!панель инструментов!модули}

В этом разделе перечислены все графические диалоги панели GRASS,
предназначенные для импорта и экспорта данных в текущие выбранные область
и набор GRASS.

{\renewcommand{\arraystretch}{0.7}
\begin{table}[H]
\centering
 \begin{tabular}{|p{2.5cm}|p{11.5cm}|}
  \hline \multicolumn{2}{|c|}{\textbf{Модули импорта растровых данных и изображений}} \\
  \hline \textbf{Модуль} & \textbf{Назначение} \\
  \hline r.in.arc & Конвертировать растровый ASCII-файл ESRI ARC/INFO (GRID)
  в (бинарный) растровый слой \\
  \hline r.in.ascii & Конвертирование растрового файла ASCII в бинарный
  растровый слой \\
  \hline r.in.aster & Привязка, трансформация и импорт данных съемки
  Terra-ASTER и связанных ЦМР с использованием gdalwarp \\
  \hline r.in.bin & Импорт бинарного растрового файла в растровый слой
  GRASS \\
  \hline r.in.gdal & Импорт растра, поддерживаемого GDAL, в бинарный
  растровый слой \\
  \hline r.in.gdal.loc & Импорт растра, поддерживаемого GDAL, в бинарный
  растровый слой с созданием соответствующей области \\
  \hline r.in.gdal.qgis & Импорт загруженного растра в бинарный
  растровый слой GRASS \\
  \hline r.in.gdal.qgis.loc & Импорт загруженного растра в бинарный
  растровый слой GRASS с созданием соответствующей области \\
  \hline r.in.gridatb & Импорт слоя GRIDATB.FOR (TOPMODEL) в растровый
  слой GRASS \\
  \hline r.in.mat & Импорт бинарного MAT-File(v4) в растр GRASS \\
  \hline r.in.poly & Создать растровый слой из полигональный/линейных
  или точечных файлов ASCII \\
  \hline r.in.srtm & Импорт файлов SRTM HGT в GRASS \\
  \hline i.in.spotvgt & Импорт файла SPOT VGT NDVI в растровый слой \\
  \hline r.in.wms & Загрузить и импортировать данные с серверов WMS \\
  \hline r.external & Загрузка GDAL-совместимого растра в виде бинарного
  растрового слоя GRASS \\
  \hline r.external.qgis & Загрузка GDAL-совместимого растра в виде бинарного
  растрового слоя GRASS \\
  \hline r.external & агрузка всех GDAL-совместимых растров в директории
  в виде бинарных растровых слоев GRASS \\
  \hline
\end{tabular}
\caption{Инструменты GRASS: Импорт > Импорт растровых данных}
\end{table}}

{\renewcommand{\arraystretch}{0.7}
\begin{table}[htb]
\centering
 \begin{tabular}{|p{2.5cm}|p{11.5cm}|}
  \hline \multicolumn{2}{|c|}{\textbf{Модули импорта векторных данных}} \\
  \hline \textbf{Модуль} & \textbf{Назначение} \\
  \hline v.in.db & Создать новый векторный (точечный) слой из таблицы
  базы данных, содержащей координаты \\
  \hline v.in.dxf & Конвертирование файлов формата AutoCAD DXF в векторные
  слои GRASS \\
  \hline v.in.e00 & Импорт файла формата ESRI E00 в векторный слой \\
  \hline v.in.garmin & Загрузить точки, маршруты и треки из приемника
  Garmin GPS в векторный слой (используется gpstrans) \\
  \hline v.in.gpsbabel & Импорт векторных данных из приемника GPS,
  используя gpsbabel \\
  \hline v.in.geonames & Импорт файлов названий geonames.org \\
  \hline v.in.gns & Импорт файлов названий US-NGA GEOnet Names Server (GNS) \\
  \hline v.in.mapgen & мпорт в GRASS векторных слоев формата Mapgen
  или Matlab \\
  \hline v.in.ogr & Импорт векторных слоев OGR/PostGIS \\
  \hline v.in.ogr.qgis & Импорт загруженных векторных слоев в векторные
  слои GRASS \\
  \hline v.in.ogr.loc & Импорт векторных слоев OGR/PostGIS с созданием
  соответствующей области \\
  \hline v.in.ogr.qgis.loc & Импорт загруженных векторных слоев в
  векторные слои GRASS с созданием соответствующей области \\
  \hline v.in.ogr.all & Импорт всех векторных слоев OGR/PostGIS из
  данного источника \\
  \hline v.in.ogr.all.loc & Импорт всех векторных слоев OGR/PostGIS из
  данного источника с созданием соответствующей области \\
\hline
\end{tabular}
\caption{Инструменты GRASS > Импорт > Импорт векторных данных}
\end{table}}

{\renewcommand{\arraystretch}{0.7}
\begin{table}[H]
\centering
 \begin{tabular}{|p{2.5cm}|p{11.5cm}|}
  \hline \multicolumn{2}{|c|}{\textbf{Модули импорта баз данных}} \\
  \hline \textbf{Модуль} & \textbf{Назначение} \\
  \hline db.in.ogr & Импорт атрибутивных таблиц в различных форматах \\
\hline
\end{tabular}
\caption{Инструменты GRASS: Импорт > Модули импорта баз данных}
\end{table}}

{\renewcommand{\arraystretch}{0.7}
\begin{table}[H]
\centering
 \begin{tabular}{|p{4cm}|p{10cm}|}
  \hline \multicolumn{2}{|c|}{\textbf{Модули экспорта растровых данных
  и изображений}} \\
  \hline \textbf{Модуль} & \textbf{Назначение} \\
  \hline r.out.gdal & Экспорт растрового слоя GRASS в форматы,
  поддерживаемые GDAL \\
  \hline r.out.gdal.gtiff & Экспорт растрового слоя GRASS в формат GeoTIFF \\
  \hline r.out.ascii & Экспорт растрового слоя в текстовый файл ASCII \\
  \hline r.out.arc & Конвертировать растровый слой в файл ESRI ARCGRID \\
  \hline r.out.xyz & Экспорт растрового слоя в текстовый файл как значения
  x, y, z центров ячеек \\
  \hline r.gridatb & Экспорт растрового слоя GRASS в файл карты GRIDATB.FOR
  (TOPMODEL) \\
  \hline r.out.mat & Экспорт растра GRASS в бинарный файл MAT \\
  \hline r.out.bin & Экспорт растра GRASS в бинарный массив \\
  \hline r.out.mpeg & Экспорт серии растров в MPEG видео \\
  \hline r.out.png & Экспорт растра GRASS в не привязанный графический
  файл PNG \\
  \hline r.out.ppm & Конвертировать растровый слой GRASS в PPM
  изображение с разрешением ТЕКУЩЕГО РЕГИОНА \\
  \hline r.out.ppm3 & Конвертировать 3 растровых слоя GRASS (R, G, B) в
  файл PPM с разрешением ТЕКУЩЕГО РЕГИОНА \\
  \hline r.out.pov & Конвертирует растровый слой в файл высот для POVRAY \\
  \hline r.out.tiff & Экспорт растра GRASS в 8 / 24-битное изображение в
  формате TIFF с разрешением текущего региона \\
  \hline r.out.vrml & Экспортировать растровый слой в Virtual Reality
  Modeling Language (VRML) \\
  \hline r.out.vtk & Конвертация растровых слоев в формат VTK-ASCII \\
\hline
\end{tabular}
\caption{Инструменты GRASS: Экспорт > Экспорт растровых данных и изображений}
\end{table}}


{\renewcommand{\arraystretch}{0.7}
\begin{table}[H]
\centering
 \begin{tabular}{|p{4cm}|p{10cm}|}
  \hline \multicolumn{2}{|c|}{\textbf{Модули экспорта векторных данных}} \\
  \hline \textbf{Модуль} & \textbf{Назначение} \\
  \hline v.out.ogr & Экспорт векторного слоя в форматы, поддерживаемые OGR \\
  \hline v.out.ogr.gml & Экспорт векторного слоя в формат GML \\
  \hline v.out.ogr.pg & кспорт векторного слоя в таблицу базы данных
  PostGIS \\
  \hline v.out.ogr.mapinfo & Экспорт векторного слоя в формат MapInfo \\
  \hline v.out.ascii & Конвертировать бинарный векторный слой GRASS в
  векторный ASCII слой GRASS \\
  \hline v.out.dxf & Экспорт векторного слоя в формат DXF \\
  \hline v.out.pov & Экспорт векторного слоя в формат POV-Ray \\
  \hline v.out.svg & Экспорт векторного слоя в формат SVG \\
  \hline v.out.vtk & Экспорт векторного слоя в формат VTK-ASCII \\
\hline
\end{tabular}
\caption{Инструменты GRASS: Экспорт > Экспорт векторных данных}
\end{table}}

{\renewcommand{\arraystretch}{0.7}
\begin{table}[H]
\centering
 \begin{tabular}{|p{4cm}|p{10cm}|}
  \hline \multicolumn{2}{|c|}{\textbf{Модули экспорта атрибутивных таблиц}} \\
  \hline \textbf{Модуль} & \textbf{Назначение} \\
  \hline db.out.ogr & Экспорт атрибутивных таблиц в различные форматы \\
\hline
\end{tabular}
\caption{Инструменты GRASS: Экспорт > Экспорт атрибутивных таблиц}
\end{table}}


\section{Модули конвертации типов данных}

В этом разделе перечислены все графические диалоги в панели инструментов
GRASS для конвертации растровых данных в векторные и векторных в растровые
для текущих области и набора GRASS.

{\renewcommand{\arraystretch}{0.7}
\begin{table}[H]
\centering
 \begin{tabular}{|p{4cm}|p{10cm}|}
  \hline \multicolumn{2}{|c|}{\textbf{Модули конвертации типов данных}} \\
  \hline \textbf{Модуль} & \textbf{Назначение} \\
  \hline r.to.vect.point & Преобразовать растровый слой в векторные точки \\
  \hline r.to.vect.line & Преобразовать растровый слой в векторные линии \\
  \hline r.to.vect.area & Преобразовать растровый слой в векторные полигоны \\
  \hline v.to.rast.constant & Преобразовать векторный слой в растровый
  с использованием константы \\
  \hline v.to.rast.attr & Преобразовать векторный слой в растровый на
  основе значений атрибутов \\
\hline
\end{tabular}
 \caption{Инструменты GRASS: модули конвертации типов данных}
\end{table}}


\section{Модули для работы с регионом и проекцией}

В этом разделе перечислены все графические диалоги для управления
текущим регионом и параметрами проекции.

{\renewcommand{\arraystretch}{0.7}
\begin{table}[H]
\centering
 \begin{tabular}{|p{4cm}|p{10cm}|}
  \hline \multicolumn{2}{|c|}{\textbf{Модули для работы с регионом и проекцией}} \\
  \hline \textbf{Модуль} & \textbf{Назначение} \\
  \hline g.region.save & Сохранить текущий регион с заданным именем \\
  \hline g.region.zoom & Сокращение текущего региона до ненулевых значений
  заданного растра \\
  \hline g.region.multiple.raster & Задать регион в соответствии с
  несколькими растровыми слоями \\
  \hline g.region.multiple.vector & Задать регион в соответствии с
  несколькими векторными слоями \\
  \hline g.proj.print & Вывести сведения о проекции для текущего региона \\
  \hline g.proj.geo & Вывести сведения о проекции из геопривязанного
  файла (растра, вектора или изображения) \\
  \hline g.proj.ascii.new & Вывести сведения о проекции из
  ASCII-файла, содержащего WKT описание проекции \\
  \hline g.proj.proj & Вывести сведения о проекции из файла описания
  проекции PROJ.4 \\
  \hline g.proj.ascii.new & Вывести сведения о проекции из ASCII-файла,
  содержащего WKT описание проекции и создать новый регион на его основе \\
  \hline g.proj.geo.new & Вывести сведения о проекции из геопривязанного
  файла (растра, вектора или изображения) и создать новый регион на его
  основе \\
  \hline g.proj.proj.new & Вывести сведения о проекции из файла описания
  PROJ.4 и создать новый регион на его основе \\
  \hline m.cogo & Простая утилита для преобразования азимута и расстояния
  в координаты и наоборот. Предполагает декартовы координаты \\
\hline
\end{tabular}
\caption{Инструменты GRASS: Регион}
\end{table}}


\section{Модули для работы с растровыми данными}

В этом разделе перечислены все графические диалоги для работы с
растровыми данными и их анализа в текущих области и наборе GRASS.

{\renewcommand{\arraystretch}{0.7}
\begin{table}[H]
\centering
 \begin{tabular}{|p{4cm}|p{10cm}|}
  \hline \multicolumn{2}{|c|}{\textbf{Обработка карт}} \\
  \hline \textbf{Модуль} & \textbf{Назначение} \\
  \hline r.compress & Сжать / распаковать растр \\
  \hline r.region.region & Задать границы на основе текущего региона
  или региона по умолчанию \\
  \hline r.region.raster & Задать регион на основе существующего растра \\
  \hline r.region.vector & Задать регион на основе существующего векторного
  слоя \\
  \hline r.region.edge & Задать регион вручную (С-Ю-В-З) \\
  \hline r.region.alignTo & Задать регион в соответствии с растром \\
  \hline r.null.val & Преобразовать ячейки со значением в ячейки без
  значения (<<null>>) \\
  \hline r.null.to & Преобразовать ячейки без значения (<<null>>) в ячейки
  со значением \\
  \hline r.quant & Создать файл квантования для растра с плавающей точкой \\
  \hline r.resamp.stats & Преобразование растра методом агрегации \\
  \hline r.resamp.interp & Преобразование растра с использованием
  интерполяции \\
  \hline r.resample & Преобразование растра. Перед использованием
  необходимо задать новое разрешение \\
  \hline r.resamp.rst & Переинтерполяция и топографический анализ с
  использованием метода <<регулируемого сплайна с натяжением>> и
  сглаживания \\
  \hline r.support &Создать и/или изменить файлы поддержки для
  растрового слоя \\
  \hline r.support.stats & Обновление статистики по растру \\
  \hline r.proj & Перепроецирование растра из какого-либо региона в
  текущий регион \\
\hline
\end{tabular}
\caption{Инструменты GRASS: Растр > Обработка карт}
\end{table}}

{\renewcommand{\arraystretch}{0.7}
\begin{table}[H]
\centering
 \begin{tabular}{|p{4cm}|p{10cm}|}
  \hline \multicolumn{2}{|c|}{\textbf{Управление цветами}} \\
  \hline \textbf{Модуль} & \textbf{Назначение} \\
  \hline r.colors.table & Задать растровую цветовую таблицу из числа
  предопределенных \\
  \hline r.colors.rules & Задать растровую цветовую таблицу на основе
  заданных правил \\
  \hline r.colors.rast & Задать растровую цветовую таблицу из
  существующего растра \\
  \hline r.colors.stddev & адать цветовые правила на основе
  стандартного отклонения от среднего значения карты \\
  \hline r.blend & Смешать цветовые компоненты двух растровых слоев в
  заданном соотношении \\
  \hline r.composite & Смешать красный, зеленый и синий растровые слои
  для получения композитного растра \\
  \hline r.his & Создать  красный, зеленый и синий растры, объединяющие
  значения тона, насыщенности и яркости из заданных растров \\
\hline
\end{tabular}
\caption{Инструменты GRASS: Растр > Управление цветами}
\end{table}}

{\renewcommand{\arraystretch}{0.7}
\begin{table}[H]
\centering
 \begin{tabular}{|p{4cm}|p{10cm}|}
  \hline \multicolumn{2}{|c|}{\textbf{Пространственный анализ}} \\
  \hline \textbf{Модуль} & \textbf{Назначение} \\
  \hline r.buffer & Буферизация растровых данных \\
  \hline r.mask & Создать маску (слой MASK) для ограничения растровых
  операций \\
  \hline r.mapcalc & Растровый калькулятор \\
  \hline r.mapcalculator & Простая растровая алгебра \\
  \hline r.neighbors & Растровый анализ близости \\
  \hline v.neighbors & Расчет количества соседних точек \\
  \hline r.cross & Создать растровый слой с результатом пересечения
  значений категорий из нескольких растровых слоев \\
  \hline r.series & Создать растр из функции на основе значений
  соответствующих ячеек исходных растров \\
  \hline r.patch & Создать новый растр из комбинации существующих растров \\
  \hline r.statistics & Рассчитать статистику по категориям или объектам \\
  \hline r.sunmask.position & Построить карты освещенности на основе
  точной позиции Солнца \\
  \hline r.sunmask.date.time & Построить карты освещенности на основе
  позиции Солнца, определенной по календарю \\
  \hline r.cost & Создать растровый слой, показывающий кумулятивную
  стоимость перемещения между точками, на основе растра, значения
  категорий которого отражают стоимость \\
  \hline r.drain & Трассировка потока по модели рельефа \\
  \hline r.shaded.relief & Создать слой светотеневой отмывки рельефа \\
  \hline r.slope.aspect.slope & Создать карту уклонов на основе
  цифровой модели рельефа (ЦМР) \\
  \hline r.slope.aspect.aspect & Создать карту экспозиции на основе
  цифровой модели рельефа (ЦМР) \\
  \hline r.param.scale & Извлечь параметры рельефа из ЦМР \\
  \hline r.texture & Создать растровые изображения с текстурными
  свойствами из растрового слоя (первая серия индексов) \\
  \hline r.texture.bis & Создать растровые изображения с текстурными
  свойствами из растрового слоя (вторая серия индексов) \\
  \hline r.los & Растровый анализ линии видимости \\
  \hline r.grow.distance & Создать растровый слой расстояний до объектов
  в исходном слое \\
  \hline r.clump & Преобразовать ячейки растра, образующие дискретные
  области, в уникальные категории \\
  \hline r.grow & Создать растровый слой с протяженными зонами,
  выращенными на одну ячейку \\
  \hline r.thin & Утонить ненулевые ячейки, представляющие линейные
  объекты \\
\hline
\end{tabular}
\caption{Инструменты GRASS: Растр > Управление цветами > Пространственный анализ}
\end{table}}

{\renewcommand{\arraystretch}{0.7}
\begin{table}[H]
\centering
 \begin{tabular}{|p{4cm}|p{10cm}|}
  \hline \multicolumn{2}{|c|}{\textbf{Гидрологическое моделирование}} \\
  \hline \textbf{Модуль} & \textbf{Назначение} \\
  \hline r.watershed & Анализ водосборов \\
  \hline r.carve & Выбрать векторные водотоки, преобразовать их в растр
  и вычесть глубину из ЦМР \\
  \hline r.fill.dir & Отфильтровать и создать растр без депрессий и слой
  направлений потоков на основе ЦМР \\
  \hline r.lake.xy & Заполнить озеро от начальной точки до заданного
  уровня \\
  \hline r.lake.seed & аполнить озеро до заданного уровня \\
  \hline r.topidx & Создать слой топографического индекса [ln(a/tan(beta))]
  на основе ЦМР \\
  \hline r.basins.fill & Создать растровый слой водосборных суббассейнов \\
  \hline r.water.outlet & Создать водосборный бассейн \\
\hline
\end{tabular}
\caption{Инструменты GRASS: Растр > Пространственные модели > Гидрологическое моделирование}
\end{table}}

{\renewcommand{\arraystretch}{0.7}
\begin{table}[H]
\centering
 \begin{tabular}{|p{4cm}|p{10cm}|}
  \hline \multicolumn{2}{|c|}{\textbf{Изменение значений категорий и подписей}} \\
  \hline \textbf{Модуль} & \textbf{Назначение} \\
  \hline r.reclass.area.greater & Переклассифицировать растр с областями,
  большими по площади, чем задано (в гектарах) \\
  \hline r.reclass.area.lesser & Переклассифицировать растр с областями,
  меньшими по площади, чем задано (в гектарах) \\
  \hline r.reclass & Переклассифицировать растр, используя правила
  переклассификации \\
  \hline r.recode & Перекодировать растр \\
  \hline r.rescale & Изменить масштаб значений категорий в растровом слое \\
\hline
\end{tabular}
\caption{Инструменты GRASS: Растр > Изменение значений категорий и подписей}
\end{table}}

{\renewcommand{\arraystretch}{0.7}
\begin{table}[H]
\centering
 \begin{tabular}{|p{4cm}|p{10cm}|}
  \hline \multicolumn{2}{|c|}{\textbf{Обработка поверхностей}} \\
  \hline \textbf{Модуль} & \textbf{Назначение} \\
  \hline r.circle & Создать карту, содержащую концентрические окружности \\
  \hline r.random & Создать случайные векторные точки со значениями данного
  растра \\
  \hline r.random.cells & Генерация случайных значений ячеек с
  пространственной зависимостью \\
  \hline v.kernel & Создать растровый слой плотности из векторных
  точек, используя функцию Гаусса \\
  \hline r.contour & Создать из растра векторный слой изолиний с
  заданным шагом \\
  \hline r.contour2 & Создать из растра векторный слой изолиний с
  заданными уровнями \\
  \hline r.surf.fractal & Создать фрактальную поверхность заданной
  фрактальной размерности \\
  \hline r.surf.gauss & Создать растровый слой гауссовых отклонений,
  среднее и стандартное отклонение которого определяются пользователем \\
  \hline r.surf.random & Создать растровый слой равномерных случайных
  отклонений, чей диапазон определяется пользователем \\
  \hline r.plane & Создать растровую плоскость \\
  \hline r.bilinear & Средство билинейной интерполяции для растров \\
  \hline v.surf.bispline & Бикубическая или билинейная интерполяция с
  регуляризацией Тихонова \\
  \hline r.surf.idw & Интерполяция методом квадратов взвешенных обратных
  расстояний \\
  \hline r.surf.idw2 & Интерполяция методом квадратов взвешенных обратных
  расстояний \\
  \hline r.surf.contour & Создать поверхность из растеризованных изолиний \\
  \hline v.surf.idw & Интерполяция методом квадратов взвешенных обратных
  расстояний (IDW) на основе атрибутов векторного слоя \\
  \hline v.surf.rst & Интерполяция векторных точек методом упорядоченного
  сплайна с натяжением (RST) \\
  \hline r.fillnulls & Заполнить null-значения в растровом слое, используя
  модуль v.surf.rst \\
\hline
\end{tabular}
\caption{Инструменты GRASS: Растр > Обработка поверхностей}
\end{table}}

{\renewcommand{\arraystretch}{0.7}
\begin{table}[H]
\centering
 \begin{tabular}{|p{4cm}|p{10cm}|}
  \hline \multicolumn{2}{|c|}{\textbf{Отчеты и статистика}} \\
  \hline \textbf{Модуль} & \textbf{Назначение} \\
  \hline r.category & Вывести значения категорий и метки, связанные с
  выбранным растром \\
  \hline r.sum & Сумма значений ячеек растра \\
  \hline r.report & Рассчитать статистику для растра \\
  \hline r.average & Вычислить усредненные значения растра внутри областей
  тех же категорий, что и слое, заданном пользователем \\
  \hline r.median & Вычислить медиану значений растра внутри областей тех
  же категорий, что и слое, заданном пользователем \\
  \hline r.mode & Вычислить моду значений растра внутри областей тех же
  категорий, что и слое, заданном пользователем \\
  \hline r.volume & Рассчитать объем групп растровых данных и создать
  векторный слой, содержащий вычисленные центроиды этих групп \\
  \hline r.surf.area & Расчет площади поверхности растра \\
  \hline r.univar & Рассчитать одномерную статистику для ненулевых ячеек
  растра \\
  \hline r.covar & Вывести матрицу ковариации / корреляции для
  определенного(ых) растра(ов) \\
  \hline r.regression.line & Рассчитать линейную регрессию для двух
  растров: y = a + b * x \\
  \hline r.coin & Создать таблицу взаимных совпадений категорий двух
  растровых слоев \\
\hline
\end{tabular}
\caption{Инструменты GRASS: Растр > Отчеты и статистика}
\end{table}}

\clearpage

\section{Модули для работы с векторными данными}

В этом разделе перечислены все графические диалоги для работы с
векторными данными и их анализа в текущих области и наборе GRASS.

{\renewcommand{\arraystretch}{0.7}
\begin{table}[H]
\centering
 \begin{tabular}{|p{3cm}|p{11cm}|}
  \hline \multicolumn{2}{|c|}{\textbf{Обработка карт}} \\
  \hline \textbf{Модуль} & \textbf{Назначение} \\
  \hline v.build.all & Перестроить топологию всех векторных данных в
  наборе \\
  \hline v.clean.break & Разбить линии в каждой точке пересечения
  векторных объектов \\
  \hline v.clean.snap & Прилепить линии к вершинам в пределах заданного
  порога \\
  \hline v.clean.rmdangles & Удалить висящие узлы \\
  \hline v.clean.chdangles & Изменить тип висящих границ на линии \\
  \hline v.clean.rmbridge & Удалить мосты, соединяющие полигоны и
  острова или пары островов \\
  \hline v.clean.chbridge & Изменить тип границ, соединяющих полигоны
  и острова или пары островов \\
  \hline v.clean.rmdupl & Удалить дублирующиеся линии (обратите внимание
  на категории!) \\
  \hline v.clean.rmdac & Удалить дублирующиеся центроиды полигонов \\
  \hline v.clean.bpol & Разбить полигоны. Границы разбиваются в каждой
  точке, общей для двух или более полигонов, где различаются углы сегментов \\
  \hline v.clean.prune & Удалить вершины из линий и границ в пределах
  заданного порога \\
  \hline v.clean.rmarea & Удалить незначительные полигоны (удаляется самая
  длинная граница с прилегающим полигоном) \\
  \hline v.clean.rmline & Удалить все линии и границы нулевой длины \\
  \hline v.clean.rmsa & Удалить незначительные углы в узлах между линиями \\
  \hline v.type.lb & Преобразовать линии в границы полигонов \\
  \hline v.type.bl & Преобразовать границы полигонов в линии \\
  \hline v.type.pc & Преобразовать точки в центроиды \\
  \hline v.type.cp & Преобразовать центроиды в точки \\
  \hline v.centroids & Добавить отсутствующие центроиды к замкнутым
  границам \\
  \hline v.build.polylines & Построить полилинии из линий \\
  \hline v.segment & Создать точки/сегменты из векторных линий и
  координат \\
  \hline v.to.points & Создать точки вдоль линий \\
  \hline v.parallel & Создать линии, параллельные исходным \\
  \hline v.dissolve & Убрать границы между смежными полигонами \\
  \hline v.drape & Преобразовать двумерный векторный слой в трехмерный
  с помощью выборки значений с цифровой модели рельефа \\
  \hline v.transform & Аффинная трансформация векторных данных \\
  \hline v.proj & Преобразование проекции векторных данных \\
  \hline v.support & Обновить метаданные векторного слоя \\
  \hline generalize & Генерализация векторных данных \\
\hline
\end{tabular}
\caption{Инструменты GRASS: Вектор > Обработка карт}
\end{table}}

{\renewcommand{\arraystretch}{0.7}
\begin{table}[H]
\centering
 \begin{tabular}{|p{4cm}|p{10cm}|}
  \hline \multicolumn{2}{|c|}{\textbf{Связь с базами данных}} \\
  \hline \textbf{Модуль} & \textbf{Назначение} \\
  \hline v.db.connect & Связать векторный слой с базой данных \\
  \hline v.db.sconnect & Отключить векторный слой от базы данных \\
  \hline v.db.what.connect & Задать/показать связь векторного слоя с
  базой данных \\
\hline
\end{tabular}
\caption{Инструменты GRASS: Вектор > Связь с базами данных}
\end{table}}

{\renewcommand{\arraystretch}{0.7}
\begin{table}[H]
\centering
 \begin{tabular}{|p{4cm}|p{10cm}|}
  \hline \multicolumn{2}{|c|}{\textbf{Пространственный анализ (в т.\,ч. сетевой)}} \\
  \hline \textbf{Модуль} & \textbf{Назначение} \\
  \hline v.extract.where & Выбрать объекты по атрибутам \\
  \hline v.extract.list & Извлечь выбранные объекты \\
  \hline v.select.overlap & Выбрать объекты, пересекаемые объектами из
  другого слоя \\
  \hline v.buffer & Буферизация векторных данных \\
  \hline v.distance & Найти ближайший элемент в целевом векторном слое
  для элементов из исходного векторного слоя \\
  \hline v.net & Обслуживание сети \\
  \hline v.net.nodes & Создать узлы в сети \\
  \hline v.net.visibility & Построение графа видимости \\
  \hline v.net.path & Найти кратчайший путь в векторной сети \\
  \hline v.net.alloc & Выделить подсети \\
  \hline v.net.iso & Разбить сеть по изолиниям стоимости \\
  \hline v.net.salesman & Связать узлы по кратчайшему пути (задача коммивояжера) \\
  \hline v.net.steiner & Связать узлы по кратчайшему дереву (дерево Штайнера) \\
  \hline v.patch & Создать новый векторный слой из комбинации других
  векторных слоев \\
  \hline v.overlay.or & Объединение векторных слоев \\
  \hline v.overlay.and & Пересечение векторных слоев \\
  \hline v.overlay.not & Разность векторных слоев \\
  \hline v.overlay.xor & Исключающее ИЛИ для векторных слоев \\
\hline
\end{tabular}
\caption{Инструменты GRASS: Вектор > Пространственный анализ}
\end{table}}

{\renewcommand{\arraystretch}{0.7}
\begin{table}[H]
\centering
 \begin{tabular}{|p{4cm}|p{10cm}|}
  \hline \multicolumn{2}{|c|}{\textbf{Изменение полей}} \\
  \hline \textbf{Модуль} & \textbf{Назначение} \\
  \hline v.category.add & Добавить значения категорий (для ВСЕХ элементов
  выбранного слоя)\\
  \hline v.category.del & Удалить значения категорий \\
  \hline v.category.sum & обавить значения к текущим значениям категорий \\
  \hline v.reclass.file & Переклассифицировать значения категорий,
  используя файл правил \\
  \hline v.reclass.attr & Переклассифицировать значения категорий,
  используя поле в таблице атрибутов (положительное целое) \\
\hline
\end{tabular}
\caption{Инструменты GRASS: Вектор > Изменение полей}
\end{table}}

{\renewcommand{\arraystretch}{0.7}
\begin{table}[H]
\centering
 \begin{tabular}{|p{4cm}|p{10cm}|}
  \hline \multicolumn{2}{|c|}{\textbf{Работа с векторными точками}} \\
  \hline \textbf{Модуль} & \textbf{Назначение} \\
  \hline v.in.region & Создать новый площадной векторный слой из охвата
  текущего региона \\
  \hline v.mkgrid.region & Создать векторную сетку в текущем регионе \\
  \hline v.in.db & Импорт векторных точек из таблицы базы данных, содержащей
  координаты \\
  \hline v.random & Генерация случайных двумерных / трехмерных векторных
  точек \\
  \hline v.perturb & Случайное смещение местоположений векторных точек \\
  \hline v.kcv & Случайным образом разделить точки на проверочные / тренировочные
  наборы \\
  \hline v.outlier & Удалить обособленные точки из набора точечных данных \\
  \hline v.hull & Создать выпуклую оболочку \\
  \hline v.delaunay.line & Триангуляция Делоне (линии) \\
  \hline v.delaunay.area & Триангуляция Делоне (площади) \\
  \hline v.voronoi.line & Диаграммы Вороного (линии) \\
  \hline v.voronoi.area & Диаграммы Вороного (площади) \\
\hline
\end{tabular}
\caption{Инструменты GRASS: Вектор > Работа с векторными точками}
\end{table}}

{\renewcommand{\arraystretch}{0.7}
\begin{table}[H]
\centering
 \begin{tabular}{|p{4cm}|p{10cm}|}
  \hline \multicolumn{2}{|c|}{\textbf{Обновление данных на основе других карт}} \\
  \hline \textbf{Модуль} & \textbf{Назначение} \\
  \hline v.rast.stats & Рассчитать одномерную статистику для растра на
  основе векторных объектов \\
  \hline v.what.vect & Загрузить в таблицу значения вектора в точках
  векторного слоя \\
  \hline v.what.rast & Загрузить в таблицу значения растра в точках
  векторного слоя \\
  \hline v.sample & Выборка значений растра в точках \\
\hline
\end{tabular}
\caption{Инструменты GRASS: Вектор > Обновление данных на основе других карт}
\end{table}}

{\renewcommand{\arraystretch}{0.7}
\begin{table}[H]
\centering
 \begin{tabular}{|p{4cm}|p{10cm}|}
  \hline \multicolumn{2}{|c|}{\textbf{Отчеты и статистика}} \\
  \hline \textbf{Модуль} & \textbf{Назначение} \\
  \hline v.to.db & Поместить значения геометрии в базу данных \\
  \hline v.report & Рассчитать статистику геометрии для векторных данных \\
  \hline v.univar & Рассчитать одномерную статистику для атрибутов
  векторного слоя \\
  \hline v.normal & Проверка векторных точек на нормальность \\
\hline
\end{tabular}
\caption{Инструменты GRASS: Вектор > Отчеты и статистика}
\end{table}}

\section{Модули для работы с изображениями}

В этом разделе перечислены все графические диалоги для работы с
изображениями и их анализа в текущих области и наборе GRASS.

{\renewcommand{\arraystretch}{0.7}
\begin{table}[H]
\centering
 \begin{tabular}{|p{4cm}|p{10cm}|}
  \hline \multicolumn{2}{|c|}{\textbf{Изображения}} \\
  \hline \textbf{Модуль} & \textbf{Назначение} \\
  \hline i.image.mosaic & Создать мозаику из 4-х изображений \\
  \hline i.rgb.his & Преобразование цветов растра из модели
  <<красный - зеленый - синий>> (RGB) в модель <<тон - яркость - насыщенность>> (HIS) \\
  \hline i.his.rgb & Преобразование цветов растра из модели
  <<тон - яркость - насыщенность>> (HIS) в модель <<красный - зеленый - синий>> (RGB) \\
  \hline i.landsat.rgb & Автоматическая балансировка цветов для
  изображений LANDSAT \\
  \hline i.fusion.brovey & Трансформация Бруви для объединения
  мультиспектральных и панхроматических каналов \\
  \hline i.zc & Растровая функция определения границ с пересечением
  нулевого значения \\
  \hline r.mfilter &  Матричный фильтр растровых данных \\
  \hline i.tasscap4 & Трансформация Tasseled Cap (Kauth Thomas) для
  данных LANDSAT-TM 4 \\
  \hline i.tasscap5 & Трансформация Tasseled Cap (Kauth Thomas) для
  данных LANDSAT-TM 5 \\
  \hline i.tasscap7 & Трансформация Tasseled Cap (Kauth Thomas) для
  данных LANDSAT-TM 7 \\
  \hline i.fft & Быстрое преобразование Фурье (FFT) для обработки
  изображений \\
  \hline i.ifft & Обратное быстрое преобразование Фурье (FFT) для обработки
  изображений \\
  \hline r.describe & Вывести краткий список значений категорий в
  выбранном растре \\
  \hline r.bitpattern & Сравнить битовые шаблоны с растровым слоем \\
  \hline r.kappa & Рассчитать матрицу ошибок и параметр <<kappa>> для
  оценки точности результатов классификации \\
  \hline i.oif & Рассчитать таблицу <<optimal index factor>> для
  изображений LANDSAT-TM \\
\hline
\end{tabular}
\caption{Инструменты GRASS: Изображения}
\end{table}}

\clearpage

\section{Модули для работы с базами данных}

В этом разделе перечислены все графические диалоги в <<Инструментах
GRASS>> для соединения и работы с внутренними и внешними
базами данных. Работа с пространственными внешними базами данных
осуществляется посредством OGR и не затрагивается перечисленными модулями.

{\renewcommand{\arraystretch}{0.7}
\begin{table}[H]
\centering
 \begin{tabular}{|p{4cm}|p{10cm}|}
  \hline \multicolumn{2}{|c|}{\textbf{Базы данных}} \\
  \hline \textbf{Модуль} & \textbf{Назначение} \\
  \hline db.connect & Установить общую связь набора с базой данных \\
  \hline db.connect.schema & Установить общую связь набора с базой
  данных со схемой \\
  \hline db.connect-login.pg & Установить связь с БД PostgreSQL \\
  \hline v.db.reconnect.all & Повторно связать векторный слой с новой базой
  данных \\
  \hline db.login & Задать имя пользователя/пароль для драйвера/базы
  данных \\
  \hline v.db.addtable & Создать новую таблицу и добавить ее к
  векторному слою \\
  \hline v.db.droptable & Удалить существующую таблицу \\
  \hline v.db.addcol & Добавить одно или более полей к атрибутивной
  таблице \\
  \hline v.db.dropcol & Удалить поле из атрибутивной таблицы,
  подключенной к выбранному векторному слою \\
  \hline v.db.renamecol & Переименовать поле из атрибутивной таблицы,
  подключенной к выбранному векторному слою \\
  \hline v.db.update\_const & Позволяет назначить новое постоянное
  значение поля \\
  \hline v.db.update\_query & Позволяет назначить новое постоянное
  значение поля, только если результат запроса "--- TRUE \\
  \hline v.db.update\_op & Позволяет назначить новое значение поля в
  результате операции над полями \\
  \hline v.db.update\_op\_query & Позволяет назначить новое значение поля в
  результате операции над полями, только если результат запроса "--- TRUE \\
  \hline db.execute & Выполнить произвольный SQL-запрос \\
  \hline db.select & Вывести результат SQL-выборки из базы данных \\
  \hline v.db.select & Вывести атрибуты выбранного векторного слоя \\
  \hline v.db.select.where & Вывести атрибуты выбранного векторного
  слоя, используя SQL \\
  \hline v.db.join & Позволяет добавить таблицу к существующей таблице
  векторного слоя \\
  \hline v.db.univar & Рассчитать одномерную статистику для выбранного
  поля в таблице атрибутов \\
\hline
\end{tabular}
\caption{Инструменты GRASS: Базы данных}
\end{table}}

\clearpage

\section{Модули для работы с трехмерными данными}

В этом разделе перечислены все графические диалоги для работы с
трехмерными данными. GRASS предоставляет большее число модулей, но в
настоящее время они доступны только через <<Оболочку GRASS>>.

{\renewcommand{\arraystretch}{0.7}
\begin{table}[H]
\centering
 \begin{tabular}{|p{4cm}|p{10cm}|}
  \hline \multicolumn{2}{|c|}{\textbf{3D-визуализация}} \\
  \hline \textbf{Модуль} & \textbf{Назначение} \\
  \hline nviz & 3D-визуализация (NVIZ) \\
\hline
\end{tabular}
\caption{Инструменты GRASS: 3D-визуализация}
\end{table}}

\section{Справка}

<<Справка GRASS GIS>> предоставляет полный обзор всех доступных модулей,
не ограниченный только лишь теми модулями, которые реализованы через
<<Инструменты GRASS>> и их неполной функциональностью.

{\renewcommand{\arraystretch}{0.7}
\begin{table}[H]
\centering
 \begin{tabular}{|p{4cm}|p{10cm}|}
  \hline \multicolumn{2}{|c|}{\textbf{Справка}} \\
  \hline \textbf{Модуль} & \textbf{Назначение} \\
  \hline g.manual & Открыть документацию GRASS (в виде HTML-страниц) \\
\hline
\end{tabular}
\caption{Инструменты GRASS: Справка}
\end{table}}

