% vim: set textwidth=78 autoindent:

%%%%%%%%%%%%%%%%%%%%%%%%%%%%%%%%%%%%%%%%%%%%%%%%%%%
\newcommand{\nix}[1]{Under GNU/Linix, #1}
\newcommand{\win}[1]{Under MS Windows, #1}
\newcommand{\osx}[1]{Under Mac OS X, #1}
\renewcommand{\button}[1]{\cornersize{0.1}\ovalbox{\textsf{\scriptsize#1}}}
\renewcommand{\classname}[1]{\textsf{\textbf{#1}}}
\renewcommand{\fieldname}[1]{\textsl{#1}}
\renewcommand{\filename}[1]{\texttt{#1}}
\renewcommand{\keystroke}[1]{\fbox{\textsf{\scriptsize#1}}}
\renewcommand{\menuopt}[1]{\textsf{#1}}
\renewcommand{\method}[1]{\textsf{\textit{#1}}}
\renewcommand{\server}[1]{\textit{#1}}
\renewcommand{\sqltable}[1]{\textsf{\textbf{#1}}}
\renewcommand{\toolbtn}[1]{\cornersize{7mm}\Ovalbox{\textsf{\scriptsize#1}}}
\renewcommand{\usertext}[1]{\texttt{#1}}
\newcommand{\toolbox}[2]{\includegraphics[width=0.7cm]{#1} #2}
\newcommand{\tab}[1]{\cornersize{0.1}\ovalbox{\textsf{\scriptsize#1}}}
\newcommand{\checkbox}[1]{#1}


\subsection{Conventions}\label{label_conventions}

The conventions used in this manual are as follows. 


\begin{itemize}
\item Button: \button{Save as Default}
\item Name of a Class: \classname{NewLayer}
\item Name of a Field: \fieldname{NAMES}
\item Name of a File: \filename{lakes.shp}
\item Single Keystroke: press \keystroke{p}
\item Keystroke Combinations: press \keystroke{Ctrl-B}, meaning press and hold the Ctrl key and then press the B key.
\item Menu Option: \menuopt{File} -> \menuopt{Save Project}
\item Method: \method{classFactory}
\item Server: \server{example needed here}
\item SQL Table: \sqltable{example needed here}    
\item Tool Button: \toolbtn{Add Vector Layer}
\item User Text: \usertext{qgis ---help}
\item Toolbox Item: \toolbox{add_grass_vector}{Add Grass Vector Layer}
\item Tab Item: \tab{General}
\item Checkbox: \checkbox{Render}
\item Hyperlink: \url{http://qgis.org}
\end{itemize}

Code is indicated by a fixed-width font:
\begin{verbatim}
PROJCS["NAD_1927_Albers",
  GEOGCS["GCS_North_American_1927",
\end{verbatim}

Platform-specific instructions are indicated as follow.
\begin{itemize}
\item \nix{do this.} 
\item \win{do that.} 
\item \osx{do something else.}
\end{itemize} 



http://wms.alaskamapped.org/bdl

