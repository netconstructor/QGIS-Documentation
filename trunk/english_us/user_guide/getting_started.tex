% vim:autoindent:set textwidth=78:

\section{Getting Started}\label{label_getstarted}

% when the revision of a section has been finalized, 
% comment out the following line:
% \updatedisclaimer

This chapter gives a quick overview of running QGIS with data available on the QGIS web page.

\subsection{Installation}\label{label_installation}
\index{installation}

Installation of QGIS is very simple. Standard installer packages are
available for MS Windows and Mac OS X. For many flavors of GNU/Linux binary
packages (rpm and deb) or software repositories to add to your installation
manager are provided. Get the latest information on binary packages at the
QGIS website at \url{http://qgis.osgeo.org/download/}.

If you need to build QGIS from source, this is documentated in Appendix
\ref{sec:install_windows} for MS Windows \win, Appendix
\ref{sec:install_macosx} for Mac OSX \osx and Appendix
\ref{sec:install_linux} for GNU/Linux \nix. The Installation instructions are
distributed with the QGIS source code and also available at
\url{http://qgis.osgeo.org}.

\subsection{Sample Data}\label{label_sampledata}
\index{data!sample} 

The user guide contains examples based on the QGIS sample dataset. 

\win The Windows installer has an option to download the QGIS sample dataset.
If checked, the data will be downloaded to your \filename{My Documents}
folder and placed in a folder called \filename{GIS Database}. 
You may use Windows Explorer to move this folder to any convenient location.
If you did not select the checkbox to install the sample dataset
during the initial QGIS installation, you can either
\begin{itemize}
\item use GIS data that you already have;
\item download the sample data from the QGIS website
 \url{http://qgis.osgeo.org/download}; or
\item uninstall QGIS and reinstall with the data download option checked.
\end{itemize}

\nix \osx For GNU/Linux and Mac OSX there are not yet dataset installation
packages available as rpm, deb or dmg. To use the sample dataset download the
file \filename{qgis\_sample\_data} as ZIP or TAR archive from
\url{http://download.osgeo.org/qgis/data/} and unzip or untar the archive on
your system. The Alaska dataset includes all GIS data that are used as
examples and screenshots in the user guide, and also includes a small GRASS
database. The projection for the QGIS sample dataset is Alaska Albers Equal
Area with unit feet. The EPSG code is 2964.

\begin{verbatim}
PROJCS["Albers Equal Area",
    GEOGCS["NAD27",
        DATUM["North_American_Datum_1927",
            SPHEROID["Clarke 1866",6378206.4,294.978698213898,
                AUTHORITY["EPSG","7008"]],
            TOWGS84[-3,142,183,0,0,0,0],
            AUTHORITY["EPSG","6267"]],
        PRIMEM["Greenwich",0,
            AUTHORITY["EPSG","8901"]],
        UNIT["degree",0.0174532925199433,
            AUTHORITY["EPSG","9108"]],
        AUTHORITY["EPSG","4267"]],
    PROJECTION["Albers_Conic_Equal_Area"],
    PARAMETER["standard_parallel_1",55],
    PARAMETER["standard_parallel_2",65],
    PARAMETER["latitude_of_center",50],
    PARAMETER["longitude_of_center",-154],
    PARAMETER["false_easting",0],
    PARAMETER["false_northing",0],
    UNIT["us_survey_feet",0.3048006096012192]]
\end{verbatim}

If you intend to use QGIS as graphical frontend for GRASS, you find a
selection of semple locations (e.g. Spearfish or South Dakota) at the
official GRASS GIS-website \url{http://grass.osgeo.org/download/data.php}. 

\subsection{Sample Session}\label{samplesession}

Now that we have QGIS installed and a sample dataset available, we would 
like to demonstrate a short and simple QGIS sample session and visualize 
a raster and a vector layer. Therefore we will use the landcover raster 
layer \filename{qgis\_sample\_data/raster/landcover.img} and the lakes 
vector layer \filename{qgis\_sample\_data/gml/lakes.gml}.

\minisec{start QGIS}

\begin{itemize}
\item \nix{assuming that QGIS is installed in the PATH, you can start QGIS 
by typing: \usertext{qgis} at a command prompt or by double clicking on the QGIS
application link (or shortcut) on the desktop, if available} 
\item \win{start QGIS using the Start menu or desktop shortcut, 
or double click on a QGIS project file.}
\item \osx{double click the icon in your Applications folder.}
\end{itemize} 

\minisec{Load a raster layer from the sample dataset}

Raster layers are loaded either by clicking on the 
\toolbtntwo{mActionAddRasterLayer}{Load Raster} icon or by selecting the \mainmenuopt{View}>\dropmenuopttwo{mActionAddRasterLayer}{Add Raster Layer} 
menu option. More than one layer can be loaded at the same time by holding 
down the \keystroke{Control} or \keystroke{Shift} key and clicking on 
multiple items in the dialog \dialog{Open a GDAL Supported Raster Data Source}.


\minisec{Load a vector layer from the sample dataset}

\begin{enumerate} 
\item Load the shapefile \filename{lakes.shp}.
\item Zoom in a bit to your favorite area with some lake.
\item Make the \filename{lakes} layer active.
\item Open the \dialog{Layer Properties} dialog.
\item Click on the \tab{Labels} tab.
\item Check the \checkbox{Display labels} checkbox to enable labeling.
\item Choose the field to label with. We'll use \selectstring{Field containing label}{NAMES}.
\item Enter a default for lakes that have no name. The default label will be
  used each time QGIS encounters a lake with no value in the \guilabel{NAMES} field.
\item Click \button{Apply}.
\end{enumerate} 



