% vim: set textwidth=78 autoindent:

% when the revision of a section has been finalized,
% comment out the following line:
% \updatedisclaimer

\hypertarget{toc1}{}
\section{QGIS Coding Standards}

The following chapters provide coding information for QGIS Version \CURRENT.
This document corresponds almost to a \LaTeX~ conversion of the CODING.t2t
file coming with the QGIS sources from July, 27th 2010. 

These standards should be followed by all QGIS developers.

\hypertarget{toc2}{}
\subsection{Classes}
\hypertarget{toc3}{}
\subsubsection{Names}
Class in QGIS begin with Qgs and are formed using mixed case. 

\begin{verbatim}
Examples:
  QgsPoint
  QgsMapCanvas
  QgsRasterLayer
\end{verbatim}

\hypertarget{toc4}{}
\subsubsection{Members}
Class member names begin with a lower case \textit{m} and are formed using mixed
case.

\begin{verbatim}
  mMapCanvas  
  mCurrentExtent
\end{verbatim}

All class members should be private.
\textbf{Public class members are STRONGLY discouraged}

\hypertarget{toc5}{}
\subsubsection{Accessor Functions}
Class member values should be obtained through accesssor functions. The
function should be named without a \textit{get} prefix. Accessor functions for the
two private members above would be: 

\begin{verbatim}
  mapCanvas()
  currentExtent()
\end{verbatim}

\hypertarget{toc6}{}
\subsubsection{Functions}
Function names begin with a lowercase letter and are formed using mixed case.
The function name should convey something about the purpose of the function.

\begin{verbatim}
  updateMapExtent()
  setUserOptions()
\end{verbatim}

\hypertarget{toc7}{}
\subsection{Qt Designer}
\hypertarget{toc8}{}
\subsubsection{Generated Classes}
QGIS classes that are generated from Qt Designer (ui) files should have a
\textit{Base} suffix. This identifies the class as a generated base class.

\begin{verbatim}
Examples:
  QgsPluginMangerBase
  QgsUserOptionsBase
\end{verbatim}
\hypertarget{toc9}{}
\subsubsection{Dialogs}
All dialogs should implement the following:
 * Tooltip help for all toolbar icons and other relevant widgets
 * WhatsThis help for \textbf{all} widgets on the dialog
 * An optional (though highly recommended) context sensitive \textit{Help} button
   that directs the user to the appropriate help page by launching their web
   browser

\hypertarget{toc10}{}
\subsection{C++ Files}
\hypertarget{toc11}{}
\subsubsection{Names}
C++ implementation and header files should be have a .cpp and .h extension
respectively.  Filename should be all lowercase and, in the case of classes,
match the class name.

\begin{verbatim}
Example:
  Class QgsFeatureAttribute source files are 
    qgsfeatureattribute.cpp and qgsfeatureattribute.h
\end{verbatim}

/!$\backslash$ \textbf{Note:} in case it is not clear from the statement above, for a filename 
to match a class name it implicitly means that each class should be declared 
and implemented in its own file. This makes it much easier for newcomers to 
identify where the code is relating to specific class.

\hypertarget{toc12}{}
\subsubsection{Standard Header and License}
Each source file should contain a header section patterned after the following
example:

\begin{verbatim}
/***************************************************************************
    qgsfield.cpp - Describes a field in a layer or table
     --------------------------------------
    Date                 : 01-Jan-2004
    Copyright            : (C) 2004 by Gary E.Sherman
    Email                : sherman at mrcc.com
/***************************************************************************
 *                                                                         *
 *   This program is free software; you can redistribute it and/or modify  *
 *   it under the terms of the GNU General Public License as published by  *
 *   the Free Software Foundation; either version 2 of the License, or     *
 *   (at your option) any later version.                                   *
 *                                                                         *
 ***************************************************************************/
\end{verbatim}

\hypertarget{toc13}{}
\subsubsection{SVN Keyword}
Each source file should contain the \$Id\$ keyword. This will be expanded by SVN
to contain useful information about the file, revision, last committer, and
date/time of last checkin.

Place the keyword right after the standard header/license that is found at the
top of each source file:

\begin{verbatim}
  /* $Id$ */
\end{verbatim}

You also need to set 

svn propset svn:keywords "Id"

for the new files.

\hypertarget{toc14}{}
\subsection{Variable Names}
Variable names begin with a lower case letter and are formed using mixed case.

\begin{verbatim}
Examples:
  mapCanvas
  currentExtent
\end{verbatim}

\hypertarget{toc15}{}
\subsection{Enumerated Types}
Enumerated types should be named in CamelCase with a leading capital e.g.:

\begin{verbatim}
    enum UnitType
    {
      Meters,
      Feet,
      Degrees,
      UnknownUnit
    } ;
\end{verbatim}

Do not use generic type names that will conflict with other types. e.g. use
"UnkownUnit" rather than "Unknown"

\hypertarget{toc16}{}
\subsection{Global Constants}
Global constants should be written in upper case underscore separated e.g.:

\begin{verbatim}
const long GEOCRS_ID = 3344;
\end{verbatim}

\hypertarget{toc17}{}
\subsection{Editing}
Any text editor/IDE can be used to edit QGIS code, providing the following
requirements are met.

\hypertarget{toc18}{}
\subsubsection{Tabs}
Set your editor to emulate tabs with spaces. Tab spacing should be set to 2
spaces.

\hypertarget{toc19}{}
\subsubsection{Indentation}
Source code should be indented to improve readability. There is a .indent.pro
file in the QGIS src directory that contains the switches to be used when
indenting code using the GNU indent program. If you don't use GNU indent, you
should emulate these settings.

\hypertarget{toc20}{}
\subsubsection{Braces}
Braces should start on the line following the expression:

\begin{verbatim}
  if(foo == 1)
  {
    // do stuff
    ...
   }else
  {
    // do something else
    ...
  }
\end{verbatim}

\hypertarget{toc21}{}
\subsection{API Compatibility}
From QGIS 1.0 we will provide a stable, backwards compatible API. This will
provide a stable basis for people to develop against, knowing their code will
work against any of the 1.x QGIS releases (although recompiling may be
required).Cleanups to the API should be done in a manner similar to the
Trolltech developers e.g.

\begin{verbatim}
class Foo 
{
  public:
    /** This method will be deprecated, you are encouraged to use 
      doSomethingBetter() rather.
      @see doSomethingBetter()
     */
    bool doSomething();

    /** Does something a better way.
      @note This method was introduced in QGIS version 1.1
     */
    bool doSomethingBetter();

}
\end{verbatim}

\hypertarget{toc22}{}
\subsection{Coding Style}
Here are described some programming hints and tips that will hopefully reduce
errors, development time, and maintenance.

\hypertarget{toc23}{}
\subsubsection{Where-ever Possible Generalize Code}
\begin{verbatim}
If you are cut-n-pasting code, or otherwise writing the same thing more than
once, consider consolidating the code into a single function.
\end{verbatim}

This will:
  * allow changes to be made in one location instead of in multiple places
  * help prevent code bloat
  * make it more difficult for multiple copies to evolve differences over time,
    thus making it harder to understand and maintain for others

\hypertarget{toc24}{}
\subsubsection{Prefer Having Constants First in Predicates}
Prefer to put constants first in predicates. 

\begin{verbatim}
"0 == value" instead of "value == 0"
\end{verbatim}

This will help prevent programmers from accidentally using "=" when they meant
to use "==", which can introduce very subtle logic bugs.  The compiler will
generate an error if you accidentally use "=" instead of "==" for comparisons
since constants inherently cannot be assigned values.

\hypertarget{toc25}{}
\subsubsection{Whitespace Can Be Your Friend}
Adding spaces between operators, statements, and functions makes it easier for
humans to parse code.

Which is easier to read, this:

\begin{verbatim}
if (!a&&b)
\end{verbatim}

or this:

\begin{verbatim}
if ( ! a && b )
\end{verbatim}

\hypertarget{toc26}{}
\subsubsection{Add Trailing Identifying Comments}
Adding comments at the end of function, struct and class implementations makes
it easier to find them later.

Consider that you're at the bottom of a source file and need to find a very
long function -- without these kinds of trailing comments you will have to page
up past the body of the function to find its name.  Of course this is ok if you
wanted to find the beginning of the function; but what if you were interested
at code near its end?  You'd have to page up and then back down again to the
desired part.

e.g.,

\begin{verbatim}
void foo::bar()
{ 
    // ... imagine a lot of code here 
} // foo::bar()
\end{verbatim}

\hypertarget{toc27}{}
\subsubsection{Use Braces Even for Single Line Statements}
Using braces for code in if/then blocks or similar code structures even for
single line statements means that adding another statement is less likely to
generate broken code.

Consider:

\begin{verbatim}
  if (foo)
    bar();
 else
    baz();
\end{verbatim}

Adding code after bar() or baz() without adding enclosing braces would create
broken code.  Though most programmers would naturally do that, some may forget
to do so in haste.

So, prefer this:

\begin{verbatim}
 if (foo)
 {
   bar();
 }
 else
 { 
    baz();
 } 
\end{verbatim}

\hypertarget{toc28}{}
\subsubsection{Book recommendations}
\begin{itemize}
\item \htmladdnormallink{Effective C++}{http://www.awprofessional.com/title/0321334876}, Scott Meyers
\item \htmladdnormallink{More Effective C++}{http://www.awprofessional.com/bookstore/product.asp?isbn=020163371X\&rl=1}, Scott Meyers
\item \htmladdnormallink{Effective STL}{http://www.awprofessional.com/title/0201749629}, Scott Meyers
\item \htmladdnormallink{Design Patterns}{http://www.awprofessional.com/title/0201634988}, GoF
\end{itemize}


You should also really read this article from Qt Quarterly on \\
\url{http://doc.trolltech.com/qq/qq13-apis.html}


\hypertarget{toc29}{}
\section{SVN Access}
This page describes how to get started using the QGIS Subversion repository

\hypertarget{toc30}{}
\subsection{Accessing the Repository}
To check out QGIS HEAD:

\begin{verbatim}
  svn --username [your user name] co https://svn.osgeo.org/qgis/trunk/qgis
\end{verbatim}

\hypertarget{toc31}{}
\subsection{Anonymous Access}
You can use the following commands to perform an anonymous checkout from the
QGIS Subversion repository.  Note we recommend checking out the trunk (unless
you are a developer or really HAVE to have the latest changes and don't mind
lots of crashing!).

You must have a subversion client installed prior to checking out the code. See
the Subversion website for more information. The Links page contains a good
selection of SVN clients for various platforms.

To check out a branch

\begin{verbatim}
  svn co https://svn.osgeo.org/qgis/branches/<branch name>
\end{verbatim}

To check out SVN stable trunk:

\begin{verbatim}
  svn co https://svn.osgeo.org/qgis/trunk/qgis qgis_trunk
\end{verbatim}

\textbf{Note:} If you are behind a proxy server, edit your \~{}/subversion/servers
file to specify your proxy settings first!

\textbf{Note:} In QGIS we keep our most stable code in the version 1\_0 branch.
Trunk contains code for the so called 'unstable' release series. Periodically
we will tag a release off trunk, and then continue stabilisation and selective
incorporation of new features into trunk.

See the INSTALL file in the source tree for specific instructions on building
development versions. 

\hypertarget{toc32}{}
\subsection{QGIS documentation sources}
If you're interested in checking out Quantum GIS documentation sources:

\begin{verbatim}
  svn co https://svn.osgeo.org/qgis/docs/trunk qgis_docs
\end{verbatim}

You can also take a look at DocumentationWritersCorner for more information.

\hypertarget{toc33}{}
\subsection{SVN Documentation}
The repository is organized as follows:

\htmladdnormallink{http://wiki.qgis.org/images/repo.png}{http://wiki.qgis.org/images/repo.png}

See the Subversion book 
\htmladdnormallink{http://svnbook.red-bean.com}{http://svnbook.red-bean.com} 
for information on becoming a SVN master.

\hypertarget{toc34}{}
\subsection{Development in branches}
\hypertarget{toc35}{}
\subsubsection{Purpose}
The complexity of the QGIS source code has increased considerably during the
last years. Therefore it is hard to anticipate the side effects that the
addition of a feature will have. In the past, the QGIS project had very long
release cycles because it was a lot of work to reetablish the stability of the
software system after new features were added. To overcome these problems, QGIS
switched to a development model where new features are coded in svn branches
first and merged to trunk (the main branch) when they are finished and stable.
This section describes the procedure for branching and merging in the QGIS
project.

\hypertarget{toc36}{}
\subsubsection{Procedure}
\begin{itemize}
\item \textbf{Initial announcement on mailing list:}
Before starting, make an announcement on the developer mailing list to see if
another developer is already working on the same feature. Also contact the
technical advisor of the project steering committee (PSC). If the new feature
requires any changes to the QGIS architecture, a request for comment (RFC) is
needed. 

\item \textbf{Create a branch:} 
Create a new svn branch for the development of the new feature (see
UsingSubversion for the svn syntax). Now you can start developing.

\item \textbf{Merge from trunk regularly:}
It is recommended to merge the changes in trunk to the branch on a regular
basis. This makes it easier to merge the branch back to trunk later.

\item \textbf{Documentation on wiki:} 
It is also recommended to document the intended changes and the current status
of the work on a wiki page.

\item \textbf{Testing before merging back to trunk:} 
When you are finished with the new feature and happy with the stability, make
an announcement on the developer list.  Before merging back, the changes will
be tested by developers and users. Binary packages (especially for OsX and
Windows) will be generated to also involve non-developers. In trac, a new
Component will be opened to file tickets against.  Once there are no remaining
issues left, the technical advisor of the PSC merges the changes into trunk.

\end{itemize}


\hypertarget{toc37}{}
\subsubsection{Creating a branch}
We prefer that new feature developments happen out of trunk so that trunk
remains in a stable state. To create a branch use the following command:

\begin{verbatim}
svn copy https://svn.osgeo.org/qgis/trunk/qgis \
https://svn.osgeo.org/qgis/branches/qgis_newfeature
svn commit -m "New feature branch"
\end{verbatim}

\hypertarget{toc38}{}
\subsubsection{Merge regularly from trunk to branch}
When working in a branch you should regularly merge trunk into it so that your
branch does not diverge more than necessary. In the top level dir of your
branch, first type \texttt{`svn info`} to determine the revision numbers of your
branch which will produce output something like this:

\begin{verbatim}
timlinux@timlinux-desktop:~/dev/cpp/qgis_raster_transparency_branch svn info
Caminho: .
URL: https://svn.osgeo.org/qgis/branches/raster_transparency_branch
Raiz do Repositorio: https://svn.osgeo.org/qgis
UUID do repositorio: c8812cc2-4d05-0410-92ff-de0c093fc19c
Revisao: 6546
Tipo de No: diretorio
Agendado: normal
Autor da Ultima Mudanca: timlinux
Revisao da Ultima Mudanca: 6495
Data da Ultima Mudanca: 2007-02-02 09:29:47 -0200 (Sex, 02 Fev 2007)
Propriedades da Ultima Mudanca: 2007-01-09 11:32:55 -0200 (Ter, 09 Jan 2007)
\end{verbatim}

The second revision number shows the revision number of the start revision of
your branch and the first the current revision. You can do a dry run of the
merge like this:

\begin{verbatim}
svn merge --dry-run -r 6495:6546 https://svn.osgeo.org/qgis/trunk/qgis
\end{verbatim}

After you are happy with the changes that will be made do the merge for real
like this:

\begin{verbatim}
svn merge -r 6495:6546 https://svn.osgeo.org/qgis/trunk/qgis
svn commit -m "Merged upstream changes from trunk to my branch"
\end{verbatim}

\hypertarget{toc39}{}
\subsection{Submitting Patches}
There are a few guidelines that will help you to get your patches into QGIS
easily, and help us deal with the patches that are sent to use easily.

\hypertarget{toc40}{}
\subsubsection{Patch file naming}
If the patch is a fix for a specific bug, please name the file with the bug
number in it e.g.  \textbf{bug777fix.diff}, and attach it to the original bug report
in trac (\htmladdnormallink{https://trac.osgeo.org/qgis/}{https://trac.osgeo.org/qgis/}).

If the bug is an enhancement or new feature, its usually a good idea to create
a ticket in trac (\htmladdnormallink{https://trac.osgeo.org/qgis/}{https://trac.osgeo.org/qgis/}) first and then attach you 

\hypertarget{toc41}{}
\subsubsection{Create your patch in the top level QGIS source dir}
This makes it easier for us to apply the patches since we don't need to
navigate to a specific place in the source tree to apply the patch. Also when I
receive patches I usually evaluate them using kompare, and having the patch
from the top level dir makes this much easier. Below is an example of you you
can include multiple changed files into your patch from the top level
directory:

\begin{verbatim}
cd qgis
svn diff src/ui/somefile.ui src/app/somefile2.cpp > bug872fix.diff
\end{verbatim}

\hypertarget{toc42}{}
\subsubsection{Including non version controlled files in your patch}
If your improvements include new files that don't yet exist in the repository,
you should indicate to svn that they need to be added before generating your
patch e.g.

\begin{verbatim}
cd qgis
svn add src/lib/somenewfile.cpp
svn diff > bug7887fix.diff
\end{verbatim}

\hypertarget{toc43}{}
\subsubsection{Getting your patch noticed}
QGIS developers are busy folk. We do scan the incoming patches on bug reports
but sometimes we miss things.  Don't be offended or alarmed. Try to identify a
developer to help you - using the ["Project Organigram"] and contact them
asking them if they can look at your patch. If you don't get any response, you
can escalate your query to one of the Project Steering Committee members
(contact details also available on the ["Project Organigram"]).

\hypertarget{toc44}{}
\subsubsection{Due Diligence}
QGIS is licensed under the GPL. You should make every effort to ensure you only
submit patches which are unencumbered by conflicting intellectual property
rights. Also do not submit code that you are not happy to have made available
under the GPL.

\hypertarget{toc45}{}
\subsection{Obtaining SVN Write Access}
Write access to QGIS source tree is by invitation. Typically when a person
submits several (there is no fixed number here) substantial patches that
demonstrate basic competance and understanding of C++ and QGIS coding
conventions, one of the PSC members or other existing developers can nominate
that person to the PSC for granting of write access. The nominator should give
a basic promotional paragraph of why they think that person should gain write
access. In some cases we will grant write access to non C++ developers e.g. for
translators and documentors.  In these cases, the person should still have
demonstrated ability to submit patches and should ideally have submtted several
substantial patches that demonstrate their understanding of modifying the code
base without breaking things, etc.

\hypertarget{toc46}{}
\subsubsection{Procedure once you have access}
Checkout the sources:

\begin{verbatim}
svn co https://svn.osgeo.org/qgis/trunk/qgis qgis
\end{verbatim}

Build the sources (see INSTALL document for proper detailed instructions)

\begin{verbatim}
cd qgis
mkdir build
ccmake ..    (set your preferred options)
make
make install  (maybe you need to do with sudo / root perms)
\end{verbatim}

Make your edits

\begin{verbatim}
cd ..
\end{verbatim}

Make your changes in sources. Always check that everything compiles before
making any commits.  Try to be aware of possible breakages your commits may
cause for people building on other platforms and with older / newer versions of
libraries.

Add files (if you added any new files). The svn status command can be used to
quickly see if you have added new files.

\begin{verbatim}
svn status src/pluguns/grass/modules
\end{verbatim}

Files listed with ? in front are not in SVN and possibly need to be added by
you:

\begin{verbatim}
svn add src/pluguns/grass/modules/foo.xml
\end{verbatim}

Commit your changes

\begin{verbatim}
svn commit src/pluguns/grass/modules/foo.xml
\end{verbatim}

Your editor (as defined in EDITOR environment variable) will appear and you
should make a comment at the top of the file (above the area that says 'don't
change this'. Put a descriptive comment and rather do several small commits if
the changes across a number of files are unrelated. Conversely we prefer you to
group related changes into a single commit.

Save and close in your editor. The first time you do this, you should be
prompted to put in your username and password. Just use the same ones as your
trac account.


\hypertarget{toc47}{}
\section{Unit Testing}
As of November 2007 we require all new features going into trunk to be
accompanied with a unit test. Initially we have limited this requirement to
\textbf{qgis\_core}, and we will extend this requirement to other parts of the code base
once people are familiar with the procedures for unit testing explained in the
sections that follow.

\hypertarget{toc48}{}
\subsection{The QGIS testing framework  - an overview}
Unit testing is carried out using a combination of QTestLib (the Qt testing
library) and CTest (a framework for compiling and running tests as part of the
CMake build process).  Lets take an overview of the process before I delve into
the details:

\begin{itemize}
\item \textbf{There is some code you want to test}, e.g. a class or function. Extreme
   programming advocates suggest that the code should not even be written yet 
   when you start building your tests, and then as you implement your code you can
   immediately validate each new functional part you add with your test. In
   practive you will probably need to write tests for pre-existing code in QGIS
   since we are starting with a testing framework well after much application
   logic has already been implemented.

\item \textbf{You create a unit test.} This happens under $<$QGIS Source Dir$>$/tests/src/core 
   in the case of the core lib. The test is basically a client that creates an
   instance of a class and calls some methods on that class. It will check the
   return from each method to make sure it matches the expected value. If any
   one of the calls fails, the unit will fail.

\item \textbf{You include QtTestLib macros in your test class.} This macro is processed by 
   the Qt meta object compiler (moc) and expands your test class into a
   runnable application. 

\item \textbf{You add a section to the CMakeLists.txt} in your tests directory that will
   build your test.

\item \textbf{You ensure you have ENABLE\_TESTING enabled in ccmake / cmakesetup.} This 
   will ensure your tests actually get compiled when you type make.

\item \textbf{You optionally add test data to $<$QGIS Source Dir$>$/tests/testdata} if your 
   test is data driven (e.g. needs to load a shapefile). These test data should
   be as small as possible and wherever possible you should use the existing
   datasets already there. Your tests should never modify this data in situ,
   but rather may a temporary copy somewhere if needed.

\item \textbf{You compile your sources and install.} Do this using normal make \&\& (sudo) 
   make install procedure.

\item \textbf{You run your tests.} This is normally done simply by doing \textbf{make test} 
 after the make install step, though I will explain other aproaches that offer
 more fine grained control over running tests.

\end{itemize}


Right with that overview in mind, I will delve into a bit of detail. I've
already done much of the configuration for you in CMake and other places in the
source tree so all you need to do are the easy bits - writing unit tests!

\hypertarget{toc49}{}
\subsection{Creating a unit test}
Creating a unit test is easy - typically you will do this by just creating a
single .cpp file (not .h file is used) and implement all your test methods as
public methods that return void. I'll use a simple test class for
QgsRasterLayer throughout the section that follows to illustrate. By convention
we will name our test with the same name as the class they are testing but
prefixed with 'Test'.  So our test implementation goes in a file called
testqgsrasterlayer.cpp and the class itself will be TestQgsRasterLayer. First
we add our standard copyright banner:

\begin{verbatim}
/***************************************************************************
     testqgsvectorfilewriter.cpp
     --------------------------------------
    Date                 : Frida  Nov 23  2007
    Copyright            : (C) 2007 by Tim Sutton
    Email                : tim@linfiniti.com
 ***************************************************************************
 *                                                                         *
 *   This program is free software; you can redistribute it and/or modify  *
 *   it under the terms of the GNU General Public License as published by  *
 *   the Free Software Foundation; either version 2 of the License, or     *
 *   (at your option) any later version.                                   *
 *                                                                         *
 ***************************************************************************/
\end{verbatim}

Next we use start our includes needed for the tests we plan to run. There is 
one special include all tests should have:

\begin{verbatim}
#include <QtTest>
\end{verbatim}

\textbf{Note} that we use the new style Qt4 includes - i.e. QtTest is included not
qttest.h

Beyond that you just continue implementing your class as per normal, pulling 
in whatever headers you may need:

\begin{verbatim}
//Qt includes...
#include <QObject>
#include <QString>
#include <QObject>
#include <QApplication>
#include <QFileInfo>
#include <QDir>

//qgis includes...
#include <qgsrasterlayer.h> 
#include <qgsrasterbandstats.h> 
#include <qgsapplication.h>
\end{verbatim}

Since we are combining both class declaration and implementation in a single
file the class declaration comes next. We start with our doxygen documentation.
Every test case should be properly documented. We use the doxygen \textbf{ingroup}
directive so that all the UnitTests appear as a module in the generated Doxygen
documentation. After that comes a short description of the unit test:

\begin{verbatim}
/** \ingroup UnitTests
 * This is a unit test for the QgsRasterLayer class.
 */
\end{verbatim}

The class \textbf{must} inherit from QObject and include the Q\_OBJECT macro.

\begin{verbatim}
class TestQgsRasterLayer: public QObject
{
  Q_OBJECT;
\end{verbatim}

All our test methods are implemented as \textbf{private slots}. The QtTest framework
will sequentially call each private slot method in the test class. There are
four 'special' methods which if implemented will be called at the start of the
unit test (\textbf{initTestCase}), at the end of the unit test
(\textbf{cleanupTestCase}).  Before each test method is called, the \textbf{init()}
method will be called and after each test method is called the \textbf{cleanup()}
method is called. These methods are handy in that they allow you to allocate
and cleanup resources prior to running each test, and the test unit as a whole.

\begin{verbatim}
private slots:
  // will be called before the first testfunction is executed.
  void initTestCase();
  // will be called after the last testfunction was executed.
  void cleanupTestCase(){};
  // will be called before each testfunction is executed.
  void init(){};
  // will be called after every testfunction.
  void cleanup();
\end{verbatim}

Then come your test methods, all of which should take \textbf{no parameters} and
should \textbf{return void}. The methods will be called in order of declaration.  I
am implementing two methods here which illustrates to types of testing. In the
first case I want to generally test the various parts of the class are working,
I can use a \textbf{functional testing} approach. Once again, extreme programmers
would advocate writing these tests \textbf{before} implementing the class. Then as
you work your way through your class implementation you iteratively run your
unit tests. More and more test functions should complete sucessfully as your
class implementation work progresses, and when the whole unit test passes, your
new class is done and is now complete with a repeatable way to validate it.

Typically your unit tests would only cover the \textbf{public} API of your class,
and normally you do not need to write tests for accessors and mutators.  If it
should happen that an acccessor or mutator is not working as expected you would
normally implement a \textbf{regression} test to check for this (see lower down).

\begin{verbatim}
  //
  // Functional Testing
  //
  
  /** Check if a raster is valid. */
  void isValid();

  // more functional tests here ...
\end{verbatim}

Next we implement our \textbf{regression tests}. Regression tests should be
implemented to replicate the conditions of a particular bug. For example I
recently received a report by email that the cell count by rasters was off by
1, throwing off all the statistics for the raster bands. I opened a bug (ticket
\#832) and then created a regression test that replicated the bug using a small
test dataset (a 10x10 raster). Then I ran the test and ran it, verifying that
it did indeed fail (the cell count was 99 instead of 100). Then I went to fix
the bug and reran the unit test and the regression test passed. I committed the
regression test along with the bug fix. Now if anybody breakes this in the
source code again in the future, we can immediatly identify that the code has
regressed. Better yet before committing any changes in the future, running our
tests will ensure our changes don't have unexpected side effects - like breaking
existing functionality.

There is one more benifit to regression tests - they can save you time.  If you
ever fixed a bug that involved making changes to the source, and then running
the application and performing a series of convoluted steps to replicate the
issue, it will be immediately apparent that simply implementing your regression
test \textbf{before} fixing the bug will let you automate the testing for bug
resolution in an efficient manner.

To implement your regression test, you should follow the naming convention of
regression$<$TicketID$>$ for your test functions. If no trac ticket exists for the
regression, you should create one first.  Using this approach allows the person
running a failed regression test easily go and find out more information.

\begin{verbatim}
  //
  // Regression Testing
  //
  
  /** This is our second test case...to check if a raster
   reports its dimensions properly. It is a regression test 
   for ticket #832 which was fixed with change r7650. 
   */
  void regression832(); 
  
  // more regression tests go here ...
\end{verbatim}

Finally in our test class declaration you can declare privately any data
members and helper methods your unit test may need. In our case I will declare
a QgsRasterLayer * which can be used by any of our test methods. The raster
layer will be created in the initTestCase() function which is run before any
other tests, and then destroyed using cleanupTestCase() which is run after all
tests. By declaring helper methods (which may be called by various test
functions) privately, you can ensure that they wont be automatically run by the
QTest executeable that is created when we compile our test.

\begin{verbatim}
  private:
    // Here we have any data structures that may need to 
    // be used in many test cases.
    QgsRasterLayer * mpLayer;
};

\end{verbatim}

That ends our class declaration. The implementation is simply inlined in the
same file lower down. First our init and cleanup functions:

\begin{verbatim}
void TestQgsRasterLayer::initTestCase()
{
  // init QGIS's paths - true means that all path will be inited from prefix
  QString qgisPath = QCoreApplication::applicationDirPath ();
  QgsApplication::setPrefixPath(qgisPath, TRUE);
#ifdef Q_OS_LINUX
  QgsApplication::setPkgDataPath(qgisPath + "/../share/qgis");
#endif
  //create some objects that will be used in all tests...

  std::cout << "Prefix  PATH: " << QgsApplication::prefixPath().toLocal8Bit().data() << std::endl;
  std::cout << "Plugin  PATH: " << QgsApplication::pluginPath().toLocal8Bit().data() << std::endl;
  std::cout << "PkgData PATH: " << QgsApplication::pkgDataPath().toLocal8Bit().data() << std::endl;
  std::cout << "User DB PATH: " << QgsApplication::qgisUserDbFilePath().toLocal8Bit().data() << std::endl;

  //create a raster layer that will be used in all tests...
  QString myFileName (TEST_DATA_DIR); //defined in CmakeLists.txt
  myFileName = myFileName + QDir::separator() + "tenbytenraster.asc";
  QFileInfo myRasterFileInfo ( myFileName );
  mpLayer = new QgsRasterLayer ( myRasterFileInfo.filePath(),
            myRasterFileInfo.completeBaseName() );
}

void TestQgsRasterLayer::cleanupTestCase()
{
  delete mpLayer;
}

\end{verbatim}

The above init function illustrates a couple of interesting things.

 1. I needed to manually set the QGIS application data path so that
   resources such as srs.db can be found properly.
 2. Secondly, this is a data driven test so we needed to provide a 
   way to generically locate the 'tenbytenraster.asc file. This was 
   achieved by using the compiler define \textbf{TEST\_DATA\_PATH}. The 
   define is created in the CMakeLists.txt configuration file under 
   $<$QGIS Source Root$>$/tests/CMakeLists.txt and is available to all 
   QGIS unit tests. If you need test data for your test, commit it 
   under $<$QGIS Source Root$>$/tests/testdata. You should only commit 
   very small datasets here. If your test needs to modify the test 
   data, it should make a copy of if first.

Qt also provides some other interesting mechanisms for data driven 
testing, so if you are interested to know more on the topic, consult 
the Qt documentation.

Next lets look at our functional test. The isValid() test simply checks the
raster layer was correctly loaded in the initTestCase.  QVERIFY is a Qt macro
that you can use to evaluate a test condition.  There are a few other use
macros Qt provide for use in your tests including:

\begin{verbatim}
QCOMPARE ( actual, expected )
QEXPECT_FAIL ( dataIndex, comment, mode )
QFAIL ( message )
QFETCH ( type, name )
QSKIP ( description, mode )
QTEST ( actual, testElement )
QTEST_APPLESS_MAIN ( TestClass )
QTEST_MAIN ( TestClass )
QTEST_NOOP_MAIN ()
QVERIFY2 ( condition, message )
QVERIFY ( condition )
QWARN ( message ) 
\end{verbatim}

Some of these macros are useful only when using the Qt framework for data
driven testing (see the Qt docs for more detail).

\begin{verbatim}
void TestQgsRasterLayer::isValid()
{
  QVERIFY ( mpLayer->isValid() );
}
\end{verbatim}

Normally your functional tests would cover all the range of functionality of
your classes public API where feasible. With our functional tests out the way,
we can look at our regression test example.

Since the issue in bug \#832 is a misreported cell count, writing our test if
simply a matter of using QVERIFY to check that the cell count meets the
expected value:

\begin{verbatim}
void TestQgsRasterLayer::regression832()
{
   QVERIFY ( mpLayer->getRasterXDim() == 10 );
   QVERIFY ( mpLayer->getRasterYDim() == 10 );
   // regression check for ticket #832
   // note getRasterBandStats call is base 1
   QVERIFY ( mpLayer->getRasterBandStats(1).elementCountInt == 100 );
}
\end{verbatim}

With all the unit test functions implemented, there one final thing we need to
add to our test class:

\begin{verbatim}
QTEST_MAIN(TestQgsRasterLayer)
#include "moc_testqgsrasterlayer.cxx"
\end{verbatim}

The purpose of these two lines is to signal to Qt's moc that his is a QtTest
(it will generate a main method that in turn calls each test funtion.  The last
line is the include for the MOC generated sources. You should replace
'testqgsrasterlayer' with the name of your class in lower case.

\hypertarget{toc50}{}
\subsection{Adding your unit test to CMakeLists.txt}
Adding your unit test to the build system is simply a matter of editing the
CMakeLists.txt in the test directory, cloning one of the existing test blocks,
and then replacing your test class name into it.  For example:

\begin{verbatim}
# QgsRasterLayer test
ADD_QGIS_TEST(rasterlayertest testqgsrasterlayer.cpp)
\end{verbatim}

\hypertarget{toc51}{}
\subsection{The ADD\_QGIS\_TEST macro explained}
I'll run through these lines briefly to explain what they do, but if you are
not interested, just do the step explained in the above section and section.

\begin{verbatim}
MACRO (ADD_QGIS_TEST testname testsrc)
  SET(qgis_${testname}_SRCS ${testsrc} ${util_SRCS})
  SET(qgis_${testname}_MOC_CPPS ${testsrc})
  QT4_WRAP_CPP(qgis_${testname}_MOC_SRCS ${qgis_${testname}_MOC_CPPS})
  ADD_CUSTOM_TARGET(qgis_${testname}moc ALL DEPENDS ${qgis_${testname}_MOC_SRCS})
  ADD_EXECUTABLE(qgis_${testname} ${qgis_${testname}_SRCS})
  ADD_DEPENDENCIES(qgis_${testname} qgis_${testname}moc)
  TARGET_LINK_LIBRARIES(qgis_${testname} ${QT_LIBRARIES} qgis_core)
  SET_TARGET_PROPERTIES(qgis_${testname}
    PROPERTIES
    # skip the full RPATH for the build tree
    SKIP_BUILD_RPATH  TRUE
    # when building, use the install RPATH already
    # (so it doesn't need to relink when installing)
    BUILD_WITH_INSTALL_RPATH TRUE
    # the RPATH to be used when installing
    INSTALL_RPATH ${QGIS_LIB_DIR}
    # add the automatically determined parts of the RPATH
    # which point to directories outside the build tree to the install RPATH
    INSTALL_RPATH_USE_LINK_PATH true)
  IF (APPLE)
    # For Mac OS X, the executable must be at the root of the bundle's executable folder
    INSTALL(TARGETS qgis_${testname} RUNTIME DESTINATION ${CMAKE_INSTALL_PREFIX})
    ADD_TEST(qgis_${testname} ${CMAKE_INSTALL_PREFIX}/qgis_${testname})
  ELSE (APPLE)
    INSTALL(TARGETS qgis_${testname} RUNTIME DESTINATION ${CMAKE_INSTALL_PREFIX}/bin)
    ADD_TEST(qgis_${testname} ${CMAKE_INSTALL_PREFIX}/bin/qgis_${testname})
  ENDIF (APPLE)
ENDMACRO (ADD_QGIS_TEST)
\end{verbatim}

Lets look a little more in detail at the individual lines. First we define the
list of sources for our test. Since we have only one source file (following the
methodology I described above where class declaration and definition are in the
same file) its a simple statement:

\begin{verbatim}
SET(qgis_${testname}_SRCS ${testsrc} ${util_SRCS})
\end{verbatim}

Since our test class needs to be run through the Qt meta object compiler (moc)
we need to provide a couple of lines to make that happen too:

\begin{verbatim}
SET(qgis_${testname}_MOC_CPPS ${testsrc})
QT4_WRAP_CPP(qgis_${testname}_MOC_SRCS ${qgis_${testname}_MOC_CPPS})
ADD_CUSTOM_TARGET(qgis_${testname}moc ALL DEPENDS ${qgis_${testname}_MOC_SRCS})
\end{verbatim}

Next we tell cmake that it must make an executeable from the test class.
Remember in the previous section on the last line of the class implementation I
included the moc outputs directly into our test class, so that will give it
(among other things) a main method so the class can be compiled as an
executeable:

\begin{verbatim}
ADD_EXECUTABLE(qgis_${testname} ${qgis_${testname}_SRCS})
ADD_DEPENDENCIES(qgis_${testname} qgis_${testname}moc)
\end{verbatim}

Next we need to specify any library dependencies. At the moment classes have
been implemented with a catch-all QT\_LIBRARIES dependency, but I will be
working to replace that with the specific Qt libraries that each class needs
only. Of course you also need to link to the relevant qgis libraries as
required by your unit test.

\begin{verbatim}
TARGET_LINK_LIBRARIES(qgis_${testname} ${QT_LIBRARIES} qgis_core)
\end{verbatim}

Next I tell cmake to install the tests to the same place as the qgis binaries
itself. This is something I plan to remove in the future so that the tests can
run directly from inside the source tree.

\begin{verbatim}
SET_TARGET_PROPERTIES(qgis_${testname}
  PROPERTIES
  # skip the full RPATH for the build tree
  SKIP_BUILD_RPATH  TRUE
  # when building, use the install RPATH already
  # (so it doesn't need to relink when installing)
  BUILD_WITH_INSTALL_RPATH TRUE
  # the RPATH to be used when installing
  INSTALL_RPATH ${QGIS_LIB_DIR}
  # add the automatically determined parts of the RPATH
  # which point to directories outside the build tree to the install RPATH
  INSTALL_RPATH_USE_LINK_PATH true)
IF (APPLE)
  # For Mac OS X, the executable must be at the root of the bundle's executable folder
  INSTALL(TARGETS qgis_${testname} RUNTIME DESTINATION ${CMAKE_INSTALL_PREFIX})
  ADD_TEST(qgis_${testname} ${CMAKE_INSTALL_PREFIX}/qgis_${testname})
ELSE (APPLE)
  INSTALL(TARGETS qgis_${testname} RUNTIME DESTINATION ${CMAKE_INSTALL_PREFIX}/bin)
  ADD_TEST(qgis_${testname} ${CMAKE_INSTALL_PREFIX}/bin/qgis_${testname})
ENDIF (APPLE)
\end{verbatim}

Finally the above uses ADD\_TEST to register the test with cmake / ctest . Here
is where the best magic happens - we register the class with ctest. If you
recall in the overview I gave in the beginning of this section we are using
both QtTest and CTest together. To recap, \textbf{QtTest} adds a main method to your
test unit and handles calling your test methods within the class. It also
provides some macros like QVERIFY that you can use as to test for failure of
the tests using conditions. The output from a QtTest unit test is an
executeable which you can run from the command line.  However when you have a
suite of tests and you want to run each executeable in turn, and better yet
integrate running tests into the build process, the \textbf{CTest} is what we use. 

\hypertarget{toc52}{}
\subsection{Building your unit test}
To build the unit test you need only to make sure that ENABLE\_TESTS=true in the
cmake configuration. There are two ways to do this:

 1. Run ccmake .. (cmakesetup .. under windows) and interactively set 
 the ENABLE\_TESTS flag to ON.
 1. Add a command line flag to cmake e.g. cmake -DENABLE\_TESTS=true ..

Other than that, just build QGIS as per normal and the tests should build too.

\hypertarget{toc53}{}
\subsection{Run your tests}
The simplest way to run the tests is as part of your normal build process:

\begin{verbatim}
make && make install && make test
\end{verbatim}

The make test command will invoke CTest which will run each test that was
registered using the ADD\_TEST CMake directive described above. Typical output
from make test will look like this:

\begin{verbatim}
Running tests...
Start processing tests
Test project /Users/tim/dev/cpp/qgis/build
1/  3 Testing qgis_applicationtest          ***Exception: Other
2/  3 Testing qgis_filewritertest           *** Passed
3/  3 Testing qgis_rasterlayertest          *** Passed

0% tests passed, 3 tests failed out of 3

  The following tests FAILED:
  1 - qgis_applicationtest (OTHER_FAULT)
  Errors while running CTest
  make: *** [test] Error 8
\end{verbatim}

If a test fails, you can use the ctest command to examine more closely why it
failed. User the -R option to specify a regex for which tests you want to run
and -V to get verbose output:

\begin{verbatim}
[build] ctest -R appl -V
Start processing tests
Test project /Users/tim/dev/cpp/qgis/build
Constructing a list of tests
Done constructing a list of tests
Changing directory into /Users/tim/dev/cpp/qgis/build/tests/src/core
1/  3 Testing qgis_applicationtest          
Test command: /Users/tim/dev/cpp/qgis/build/tests/src/core/qgis_applicationtest
********* Start testing of TestQgsApplication *********
  Config: Using QTest library 4.3.0, Qt 4.3.0
PASS   : TestQgsApplication::initTestCase()
  Prefix  PATH: /Users/tim/dev/cpp/qgis/build/tests/src/core/../
  Plugin  PATH: /Users/tim/dev/cpp/qgis/build/tests/src/core/..//lib/qgis
  PkgData PATH: /Users/tim/dev/cpp/qgis/build/tests/src/core/..//share/qgis
  User DB PATH: /Users/tim/.qgis/qgis.db
PASS   : TestQgsApplication::getPaths()
  Prefix  PATH: /Users/tim/dev/cpp/qgis/build/tests/src/core/../
  Plugin  PATH: /Users/tim/dev/cpp/qgis/build/tests/src/core/..//lib/qgis
  PkgData PATH: /Users/tim/dev/cpp/qgis/build/tests/src/core/..//share/qgis
  User DB PATH: /Users/tim/.qgis/qgis.db
  QDEBUG : TestQgsApplication::checkTheme() Checking if a theme icon exists:
  QDEBUG : TestQgsApplication::checkTheme() 
  /Users/tim/dev/cpp/qgis/build/tests/src/core/..//share/qgis/themes/default/
/mIconProjectionDisabled.png
  FAIL!  : TestQgsApplication::checkTheme() '!myPixmap.isNull()' returned FALSE. ()
  Loc: [/Users/tim/dev/cpp/qgis/tests/src/core/testqgsapplication.cpp(59)]
PASS   : TestQgsApplication::cleanupTestCase()
  Totals: 3 passed, 1 failed, 0 skipped
  ********* Finished testing of TestQgsApplication *********
  -- Process completed
  ***Failed

  0% tests passed, 1 tests failed out of 1

  The following tests FAILED:
1 - qgis_applicationtest (Failed)
  Errors while running CTest
\end{verbatim}

Well that concludes this section on writing unit tests in QGIS. We hope you
will get into the habit of writing test to test new functionality and to check
for regressions. Some aspects of the test system (in particular the
CMakeLists.txt parts) are still being worked on so that the testing framework
works in a truly platform way. I will update this document as things
progress.


\hypertarget{toc54}{}
\section{HIG (Human Interface Guidelines)}
In order for all graphical user interface elements to appear consistant and to
all the user to instinctively use dialogs, it is important that the following
guidelines are followed in layout and design of GUIs.

 \begin{enumerate}
 \item Group related elements using group boxes:
   Try to identify elements that can be grouped together and then use group
   boxes with a label to identify the topic of that group.  Avoid using group
   boxes with only a single widget / item inside.
 \item Capitalise first letter only in labels:
   Labels (and group box labels) should be written as a phrase with leading
   capital letter, and all remaing words written with lower case first letters 
 \item Do not end labels for widgets or group boxes with a colon:
   Adding a colon causes visual noise and does not impart additional meaning,
   so don't use them. An exception to this rule is when you have two labels next
   to each other e.g.: Label1 {Plugin}{Path:} Label2 [/path/to/plugins]
 \item Keep harmful actions away from harmless ones:
   If you have actions for 'delete', 'remove' etc, try to impose adequate space
   between the harmful action and innocuous actions so that the users is less
   likely to inadvertantly click on the harmful action.
 \item Always use a QButtonBox for 'OK', 'Cancel' etc buttons:
   Using a button box will ensure that the order of 'OK' and 'Cancel' etc,
   buttons is consistent with the operating system / locale / desktop
   environment that the user is using.
 \end{enumerate}


