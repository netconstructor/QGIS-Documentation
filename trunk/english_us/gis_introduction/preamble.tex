% vim: set textwidth=78 autoindent:

\thispagestyle{empty}
\addcontentsline{toc}{section}{A word from the editor}

%%%%%%%%%%% nothing to change above %%%%%%%%%%

\section*{A word from the editor}

% when the revision of a section has been finalized, 
% comment out the following line:
%\updatedisclaimer


This project was sponsored by the Chief Directorate: Spatial Planning \&
Information, Department of Land Affairs (DLA), Eastern Cape, in conjunction
with the Spatial Information Management Unit, Office of the Premier, Eastern
Cape, South Africa.

GIS is becoming an increasingly important tool in environmental management,
retail, military, police, tourism and many other spheres of our daily lives.
If you use a computer or a cell phone, you have probably already used a GIS
in some form without even realising it. Maybe it was a map on a web site,
Google Earth, an information booth or your cell phone telling you where you
are. Proprietary GIS software (software that cannot be freely shared or
modified) is available that will let you do everything we describe in these
worksheets and a lot more. However this software is usually very expensive or
otherwise limits your freedom to copy, share and modify the software. GIS
vendors sometimes make an exception for educational activities, providing
cheaper or free copies of their software. They do this knowing that if
teachers and learners get to know their software, they will be reluctant to
learn other packages. When learners leave school they will go into the
workplace and buy the commercial software, never knowing that there are free
alternatives that they could be using.

With Quantum GIS, we offer an alternative - software that is free of cost and
free in a social sense. You can make as many copies as you like. When learners
leave school one day they can use this software to build their skills, solve
problems at work and make the world a better place.

When you buy commercial software, you limit your options for the future. By
learning, using and sharing Free and Open Source Software, you are building
your own skills, freeing money to be spent on important things like food and
shelter and boosting our own economy.

By sponsoring the creation of this resource, the DLA has created a foundation
to which young minds can be exposed. Exciting possibilities lie ahead when
principles of free sharing of knowledge and data are embraced. For this we
give our heartfelt thanks to the DLA!

We hope you enjoy using and learning QGIS in the spirit of Ubuntu!

\includegraphics[clip=true, width=5cm]{tim_signature}

Tim Sutton, April 2009


