%  !TeX  root  =  user_guide.tex  

\chapter{Help and Support}\label{label_helpsupport}

% when the revision of a section has been finalized, 
% comment out the following line:
% \updatedisclaimer

\section{Mailinglists}
QGIS is under active development and as such it won't always work like
you expect it to. The preferred way to get help is by joining the qgis-users
mailing list. Your questions will reach a broader audience and answers will
benefit others. 

\minisec{qgis-users}
This mailing list is used for discussion of QGIS in general, as well as
specific questions regarding its installation and use. You can subscribe to
the qgis-users mailing list by visiting the following URL: \\
\url{http://lists.osgeo.org/mailman/listinfo/qgis-user}

\minisec{fossgis-talk-liste}
For the german speaking audience the german FOSSGIS e.V. provides the
fossgis-talk-liste mailing list. This mailing list is used for discussion of
open source GIS in general including QGIS. You can subscribe to the
fossgis-talk-liste mailing list by visiting the following URL: \\
\url{https://lists.fossgis.de/mailman/listinfo/fossgis-talk-liste}

\minisec{qgis-developer}
If you are a developer facing problems of a more technical nature, you may
want to join the qgis-developer mailing list here:\\
\url{http://lists.osgeo.org/mailman/listinfo/qgis-developer}

\minisec{qgis-commit}
Each time a commit is made to the QGIS code repository an email is posted to
this list. If you want to be up to date with every change to the current code
base, you can subscribe to this list at:\\
\url{http://lists.osgeo.org/mailman/listinfo/qgis-commit}

\minisec{qgis-trac}
This list provides email notification related to project management,
including bug reports, tasks, and feature requests. You can subscribe to this
list at:\\
\url{http://lists.osgeo.org/mailman/listinfo/qgis-trac}

\minisec{qgis-community-team}
This list deals with topics like documentation, context help, user-guide,
online experience including web sites, blog, mailing lists, forums, and
translation efforts. If you like to work on the user-guide as well, this list
is a good starting point to ask your questions. You can subscribe to this
list at:\\
\url{http://lists.osgeo.org/mailman/listinfo/qgis-community-team}

\minisec{qgis-release-team}
This list deals with topics like the release process, packaging binaries for
various OS and announcing new releases to the world at large. You can
subscribe to this list at:\\
\url{http://lists.osgeo.org/mailman/listinfo/qgis-release-team}

\minisec{qgis-tr}

This list deals with the translation efforts. If you like to work on the translation 
of the manuals or the graphical user interface (GUI), this list is a good starting 
point to ask your questions.

\minisec{qgis-edu}

This list deals with QGIS education efforts. If you like to work on qgis education 
materials, this list is a good starting point to ask your questions.

\url{http://lists.osgeo.org/mailman/listinfo/qgis-edu}

\minisec{qgis-psc}
This list is used to discuss Steering Committee issues related to overall
management and direction of Quantum GIS. You can subscribe to this list at:\\
\url{http://lists.osgeo.org/mailman/listinfo/qgis-psc}

You are welcome to subscribe to any of the lists. Please remember to
contribute to the list by answering questions and sharing your experiences.
Note that the qgis-commit and qgis-trac are designed for notification only
and not meant for user postings. 

\section{IRC}
We also maintain a presence on IRC - visit us by joining the \#qgis channel on
\url{irc.freenode.net}. Please wait around for a response to your question as many
folks on the channel are doing other things and it may take a while for them to
notice your question. Commercial support for QGIS is also available.
Check the website \url{http://qgis.org/en/commercial-support.html} for more information.

If you missed a discussion on IRC, not a problem! We log all discussion so you can 
easily catch up. Just go to \url{http://logs.qgis.org} and read the IRC-logs.

\section{BugTracker}
While the qgis-users mailing list is useful for general 'how do I do xyz in
QGIS' type questions, you may wish to notify us about bugs in QGIS. You can
submit bug reports using the QGIS bug tracker at \url{https://trac.osgeo.org/qgis/}. 
When creating a new ticket for a bug, please provide an email
address where we can request additional information.

Please bear in
mind that your bug may not always enjoy the priority you might hope for
(depending on its severity). Some bugs may require significant
developer effort to remedy and the manpower is not always available for this.

Feature requests can be submitted as well using the same ticket system as for bugs.
Please make sure to select the type \usertext{enhancement}.

If you have found a bug and fixed it yourself you can submit this patch also.
Again, the lovely trac ticketsystem at \url{https://trac.osgeo.org/qgis/} has this 
type as well. Select \usertext{patch} from the type-menu. Someone of the 
developers will review it and apply it to QGIS. \\
Please don't be alarmed if your patch is not applied straight away - developers
may be tied up with other committments.

% unused, since community.qgis.org seems to be lost. (SH)
% There is also a community site for QGIS where we encourage QGIS users to share
% their experiences and provide case studies about how they are using QGIS. The
% community site is available at: http://community.qgis.org 

\section{Blog}
The QGIS-community also runs a weblog (BLOG) at \url{http://blog.qgis.org} 
which has some interesting articles for users and developers as well. 
You are invited to contribute to the blog after registering yourself!

\section{Wiki}
Lastly, we maintain a WIKI web site at \url{http://www.qgis.org/wiki} where you 
can find a variety of useful information relating to QGIS development, 
release plans, links to download sites, message translation-hints and so on. 
Check it out, there are some goodies inside!

