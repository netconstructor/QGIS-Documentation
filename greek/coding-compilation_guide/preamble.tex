% vim: set textwidth=78 autoindent:

\thispagestyle{empty}
\addcontentsline{toc}{section}{Preamble}

%%%%%%%%%%% nothing to change above %%%%%%%%%%

\section*{Εισαγωγή}

% when the revision of a section has been finalized, 
% comment out the following line:
%\updatedisclaimer

\vspace{1cm}

Αυτό το έγγραφο αποτελεί τον αυθεντικό Οδηγό Κωδικοποιήσης και Σύνταξης του περιγραφόμενου λογισμού QuantumGIS. Τα προγράμματα και τα ηλεκτρομηχανολογικά μέρη που περιγράφονται σε αυτό το έγγραφο είναι στις περισσότερες περιπτώσεις καταχωρημένα εμπορικά σήματα οπότε και είναι αντικείμενο νομικών απαιτήσεων. Το Quantum GIS είναι αντικείμενο της GNU Γενικης Δημόσιας Άδειας (General Public License). Περισσότερες πληροφορίες μπορείτε να βρείτε στην αρχική σελίδα του Quantum GIS http://qgis.osgeo.org.
\url{http://qgis.osgeo.org}.

Οι λεπτομέρειες, τα δεδομένα, τα αποτελέσματα κλπ σε αυτό το έγγραφο έχουν γραφεί και επαλυθευτεί εις γνώσην και υπευθυνότητα των συγγραφέων και επιμελητών έκδοσης. Παρ’όλα αυτά, λάθη που αφορούν τοο περιεχόμενο είναι πιθανά. 

Έτσι όλα τα δεδομένα δεν είναι υπόλογα σε καμία εγγύηση. Οι συγγραφείς, επιμελητές έκδοσης και εκδότες δεν φέρουν καμία ευθύνη για σφάλματα και τις συνέπειές τους. Είστε πάντα ευπρόσδεκτοι να επισημάνετε πιθανά λάθη.

Το έγγραφο αυτό έχει δακτυλογραφηθέι με \LaTeX. Είναι διαθέσιμο ως \LaTeX~πηγαίος κώδικας
απο \href{http://www.qgis.org/wiki/index.php/Manual_Writing}{subversion} 
και online σαν PDF έγραφο απο \url{http://qgis.osgeo.org/documentation/manuals.html}. 
Οι μετεγρασμένες εκδόσεις αυτού του κειμένου μπορούν επίσης να κατεβαστούν και απο την περιοχή εγγραων του QGIS. Για περισσότερες πληροφορίες σχετικά με τη συνεισφορά σε αυτό το έγγραφο και τη μετάφραση του, παρακαλούμε επισκευθείτε: 
\url{http://www.qgis.org/wiki/index.php/Community_Ressources} 

\vspace{0.5cm}

\textbf{Σύνδεσμοι μέσα στο έγγραφο}

Αυτό το έγγραφο εμπεριέχει εσωτερικούς και εξωτερικούς συνδέσμους. Κάνοντας κλικ σε έναν εσωτερικό σύνδεσμο σας μετακινεί μέσα στο έγγραφο, ενώ κάνοντας κλικ σε έναν εξωτερικό σύνδεσμο ανοίγει μια διεύθυνση internet. Σε pdf μορφή, οι εσωτερικοί σύνδεσμοι δείχνονται με μπλέ, ενώ οι εξωτερικοί δείχνονται με κόκκινο και χειρίζονται απο τον browser του συστήματος. Σε HTML μορφή, o browser δείχνει και χειρίζεται και τα δύο πανομοιότυπα. 

\begin{flushleft}
\textbf{Οι συγγραφείς και επιμελητές έκδοσης του Οδηγού Κωδικοποίησης και Σύνταξης}
 
\begin{tabular}{p{5cm} p{5cm} p{5cm}}
Tim Sutton & Marco Hugentobler & Gary E. Sherman \\
Tara Athan & Godofredo Contreras & Werner Macho \\
Carson J.Q. Farmer & Otto Dassau & J\"urgen E. Fischer \\
Davis Wills & Magnus Homann & Martin Dobias \\ 

Μετάφραση στα Ελληνικά 
Στέφανος Θεοφ. 
\end{tabular}

Ευχαριστίες στο Tisham Dhar για την ετοιμασία του αρχικού έγγραφο msys (MS Windows) περιβάλλοντος, στον Tom Elwertowski και William Kyngesburye για για τη βοήθεια στην εγκάσταση σε MAC OSX και στον Carlos D\'{a}vila. Αν παραλείψαμε να αναφέρουμε κάποιο που συνεισέφερε, παρακαλούμε δεχθείτε τη συγγνώμη μας. 
\vspace{0.5cm}

\textbf{Copyright \copyright~2004 - 2010 Quantum GIS Development Team} \\
\textbf{Internet:} \url{http://qgis.osgeo.org}
\end{flushleft}

\newpage

\minisec{Άδεια αυτού του εγγράφου }

Άδεια δίνεται για αντιγραφή, διανομή και/ή μετατροπή αυτού του εγγράφου κάτω απο τους όρους της GNU Ελεύθερης Αδειας Εγγράφων (Free Documentation Lisence) έκδοση 1.3 ή και μετ’έπειτα έκδοση απο το Ίδρυμα Ελευθερου Λογισμικού (Free Software Foundation); Χωρίς αλλαγές στο περιεχόμενο, χωρίς κείμενα στο εμπροσθόφυλλο και στο οπισθόφυλλο. Ένα αντίγραφο αυτής της άδεια συμπεριλαμβάνεται στο κεφάλαιο \ref{label_fdl} με τίτλο "GNU Free Documentation License" (GNU Ελέυθερη Άδεια Εγγράφου).

