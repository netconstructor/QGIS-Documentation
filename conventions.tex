% vim: set textwidth=78 autoindent:

% testing svn commit
% options for boxes provided by fancybox package are:
% \fbox, \shadowbox, \doublebox, \ovalbox and \Ovalbox
%%%%%%%%%%%%%%%%8%%%%%%%%%%%%%%%%%%%%%%%%%%%%%%%%%%%
%
\renewcommand{\usertext}[1]{\texttt{#1}}
\renewcommand{\classname}[1]{\textsf{\textbf{#1}}}
\renewcommand{\fieldname}[1]{\textsl{#1}}
\renewcommand{\filename}[1]{\texttt{#1}}
\renewcommand{\keystroke}[1]{\cornersize{.6}\ovalbox{\textsf{#1}}}
\renewcommand{\method}[1]{\textsf{\textit{#1}}}
\renewcommand{\server}[1]{\textit{#1}}
\renewcommand{\sqltable}[1]{\textsf{\textbf{#1}}}
\newcommand{\guilabel}[1]{\textsf{#1}}
%
\setlength{\shadowsize}{2pt}%
\newcommand{\nix}[1]{Under GNU/Linix, #1}
\newcommand{\win}[1]{Under MS Windows, #1}
\newcommand{\osx}[1]{Under Mac OS X, #1}
% button
\renewcommand{\button}[1]{%
\raisebox{-6pt}{%
\shadowbox{\guilabel{#1}}
}}
% mainmenuopt
\newcommand{\mainmenuopt}[1]{%
\raisebox{-6pt}{%
\setlength{\fboxsep}{0pt}%
\shadowbox{\setlength{\fboxsep}{2pt}%
\fcolorbox[gray]{0.9}[gray]{0.9}%
{\guilabel{#1}}%
}}}
% dropmenuopt
\newcommand{\dropmenuopt}[1]{%
\raisebox{-6pt}{{%
\setlength{\fboxsep}{0pt}%
\shadowbox{\setlength{\fboxsep}{2pt}%
\fcolorbox[rgb]{.95,.95,0.8}[rgb]{.95,.95,0.8}%
{ \guilabel{#1}}}%
}}}
% dropmenucheck
\newcommand{\dropmenucheck}[1]{%
\raisebox{-6pt}{{%
\setlength{\fboxsep}{0pt}%
\shadowbox{\setlength{\fboxsep}{2pt}%
\fcolorbox[rgb]{.95,.95,0.8}[rgb]{.95,.95,0.8}%
{ $\boxtimes$ \guilabel{#1}}}%
}}}
% dropmenuopttwo
\newcommand{\dropmenuopttwo}[2]{%
\raisebox{-6pt}{%
{%
\setlength{\fboxsep}{0pt}%
\shadowbox{\setlength{\fboxsep}{2pt}%
\fcolorbox[rgb]{.95,.95,0.8}[rgb]{.95,.95,0.8}%
{\includegraphics[width=3mm]{#1} \guilabel{#2}}}%
}}}
% tooltip
\newcommand{\tooltip}[1]{%
\raisebox{-2pt}{%
\fcolorbox{black}[rgb]{1,1,0.8}{\guilabel{#1}}%
}}
% toolbtntwo
\newcommand{\toolbtntwo}[2]{%
\raisebox{-6pt}{%
\shadowbox{\includegraphics[width=7mm]{#1}}} %
\tooltip{#2}%
}
% toolboxtwo
\newcommand{\toolboxtwo}[2]{%
\raisebox{-6pt}{%
\shadowbox{%
\raisebox{-2pt}{\includegraphics[width=5mm]{#1}}%
\guilabel{ #2}%
}}}
% toolboxthree
\newcommand{\toolboxthree}[3]{%
\raisebox{-6pt}{%
\shadowbox{%
\includegraphics[width=5mm]{#1}%
\includegraphics[width=5mm]{#2}%
 \guilabel{#3}%
}}}
% toolboxfour
\newcommand{\toolboxfour}[4]{%
\raisebox{-6pt}{%
\shadowbox{%
\includegraphics[width=5mm]{#1}%
\includegraphics[width=5mm]{#2}%
\includegraphics[width=5mm]{#3}%
 \guilabel{#4}%
}}}
%for backward compatibility!
\newcommand{\toolbox}[2]{%
\toolboxtwo{#1}{#2}%
}
% tab
\newcommand{\tab}[1]{%
\raisebox{-6pt}{%
\shadowbox{\guilabel{#1}}
}}
% checkbox
\newcommand{\checkbox}[1]{%
\raisebox{2pt}{%
\fbox{%
\raisebox{-4pt}{%
\shadowbox{x}} %
\guilabel{#1}}%
}}
% radiobuttonoff
\newcommand{\radiobuttonoff}[1]{%
\raisebox{-4pt}{%
\setlength{\fboxsep}{1pt}%
\shadowbox{%
$\bigcirc$%
}} %
\guilabel{#1}%
}
% radiobuttonon
\newcommand{\radiobuttonon}[1]{%
$\odot$ %
\guilabel{#1}%
}
% selectnumber
\newcommand{\selectnumber}[2]{%
\fbox{%
{#1} \fbox{{#2} %
\raisebox{-6pt}{%
\setlength{\fboxsep}{1pt}%
\shadowbox{%
${\blacktriangle}\atop{\blacktriangledown}$%
}}}}}
%
% selectstring
\newcommand{\selectstring}[2]{%
\fbox{%
\guilabel{#1} \fbox{\guilabel{#2} %
\raisebox{-2pt}{%
\setlength{\fboxsep}{1pt}%
\shadowbox{%
$\blacktriangledown$%
}}}}}
%
% browsebutton
\newcommand{\browsebutton}{%
\raisebox{-6pt}{%
\shadowbox{\rule[-1mm]{0mm}{4mm}{$\ldots$}}
}}
%
% selectcolor
\newcommand{\selectcolor}[2]{%
\fbox{\guilabel{#1} %
\raisebox{-6pt}{%
\setlength{\fboxsep}{0pt}%
\shadowbox{\setlength{\fboxsep}{2pt}%
\fcolorbox{#2}{#2}{\rule{0mm}{5mm}\rule{35mm}{0mm}%
}}}}}
%
% slider
\newcommand{\slider}[3]{%
\fbox{%
\guilabel{#1} \guilabel{#2}\% %
\raisebox{-2pt}{%
\setlength{\fboxsep}{1pt}%
\shadowbox{%
$\triangledown$%
}}%
\negthinspace\rule[1mm]{20mm}{1mm}
}}
%
% input text
\newcommand{\inputtext}[2]{%
\fbox{%
\guilabel{#1} %
\raisebox{-6pt}{%
\shadowbox{\usertext{#2}}%
}}}
%
%
\subsection{Conventions}\label{label_conventions}

The conventions used in this manual are as follows. 

\minisec{GUI Conventions}
\begin{itemize}
%
%Use \mainmenuopt for main menu items that have no icon only text
%Main Menu includes: File, View, Layer, Settings, Plugins, Help
\item Menu Options: \mainmenuopt{Layer} > %
%
%Use \dropmenuopttwo for a drop-down menu item with an icon
%Use \dropmenuoptone for a drop-down menu item with no icon 
\dropmenuopttwo{addraster}{Add a Raster Layer} 

or

\mainmenuopt{View} > %
\dropmenuopt{Toolbar Visibility} > \dropmenucheck{Digitizing}
%
%Use \toolbtntwo for the toolbar items, including those that open dialogs
%These have an icon, and display a tooltip on hover
%Its really important to get the icon in there because that's what a user
%has to search for. The tooltip adds confirmation
\item Tool: \toolbtntwo{addraster}{Add a Raster Layer}
%
%Use \button for a clickable button which has no icon, just text
%Save As Default is a button that appears in the Layer Properties dialog.
\item Button: \button{Save as Default}
%
%Use \tab for clickable tabs which have no icons, just text
%General is a tab that appears in the Layer Properties dialog.
%At the moment, it looks just like \button, but that may change.
\item Tab: \tab{General}
%
%Use \toolboxtwo, \toolboxthree or \toolboxfour
% for GRASS toolbox (not toolbar) items
%These are what you see in the menu after you click Open GRASS Tools
%The one you need depends on how many icons are required
%The number part of the name is the total number of arguments (N)
%The number of icons is (N-1)
\item Toolbox Item: \toolboxtwo{nviz.1.eps}{nviz - Open 3D-View in NVIZ}
%
%Use \checkbox for a checkbox item in a dialog popup
\item Checkbox: \checkbox{Render}
%
%Use \radiobuttonoff for a radio button item in a dialog popup
\item Radio Button:  \radiobuttonon{Postgis SRID} \radiobuttonoff{EPSG ID}
%
% Use \selectnumber for a selection box with up and down arrows
% and a numerical value
\item Select a Number: \selectnumber{Hue}{60}
%
% Use \selectstring for a selection box with down arrows
% and a string value
\item Select a String: \selectstring{Outline style}{---Solid Line}
%
%
% Use \browsebutton for a button that opens a file browser popup
\item Browse for a File: \browsebutton 
%
% Use \selectcolor for a button which opens a color selector popup
\item Select a Color: \selectcolor{Outline color}{yellow}
%
\item Slider: \slider{Transparency}{0}{20mm}
%
% Use \inputtext for a labelled field for user input of text 
\item Input Text: \inputtext{Display Name}{lakes.shp}
\end{itemize}
A shadow indicates a clickable GUI component.

\minisec{Text or Keyboard Conventions}
\begin{itemize}
%
%Use for all urls. Otherwise, it is not clickable in the document.
\item Hyperlinks: \url{http://qgis.org}
%
\item Single Keystroke: press \keystroke{p}
\item Keystroke Combinations: press \keystroke{Ctrl-B}, meaning press and
hold the Ctrl key and then press the B key.
\item Name of a File: \filename{lakes.shp}
\item Name of a Field: \fieldname{NAMES}
\item Name of a Class: \classname{NewLayer}
\item Method: \method{classFactory}
\item Server: \server{example needed here}
\item SQL Table: \sqltable{example needed here}    
%
%Use usertext for all other text that the user must enter from the keyboard
%that is not covered by any of the above cases
\item User Text: \usertext{qgis ---help}
\end{itemize}

Code is indicated by a fixed-width font:
\begin{verbatim}
PROJCS["NAD_1927_Albers",
  GEOGCS["GCS_North_American_1927",
\end{verbatim}

Platform-specific instructions are indicated as follow.
\begin{itemize}
\item \nix{do this.} 
\item \win{do that.} 
\item \osx{do something else.}
\end{itemize} 





