% vim: set textwidth=78 autoindent:

% options for boxes provided by fancybox package are:
% \fbox, \shadowbox, \doublebox, \ovalbox and \Ovalbox
%%%%%%%%%%%%%%%%%%%%%%%%%%%%%%%%%%%%%%%%%%%%%%%%%%%
\setlength{\shadowsize}{2pt}%
\newcommand{\nix}[1]{Under GNU/Linix, #1}
\newcommand{\win}[1]{Under MS Windows, #1}
\newcommand{\osx}[1]{Under Mac OS X, #1}
% button
\renewcommand{\button}[1]{%
\lower4pt\hbox{%
\shadowbox{\textsf{#1}}
}}
% mainmenuopt
\newcommand{\mainmenuopt}[1]{%
\mbox{}%
\lower6pt\hbox{%
\setlength{\fboxsep}{0pt}%
\shadowbox{\setlength{\fboxsep}{2pt}%
\fcolorbox[gray]{0.9}[gray]{0.9}%
{\textsf{#1}}%
}}}
% dropmenuopt
\newcommand{\dropmenuopt}[1]{%
\mbox{}%
\lower6pt\hbox{%
{%
\setlength{\fboxsep}{0pt}%
\shadowbox{\setlength{\fboxsep}{2pt}%
\fcolorbox[rgb]{.95,.95,0.8}[rgb]{.95,.95,0.8}%
{ \textsf{#1}}}%
}}}
% dropmenuopttwo
\newcommand{\dropmenuopttwo}[2]{%
\mbox{}%
\lower6pt\hbox{%
{%
\setlength{\fboxsep}{0pt}%
\shadowbox{\setlength{\fboxsep}{2pt}%
\fcolorbox[rgb]{.95,.95,0.8}[rgb]{.95,.95,0.8}%
{\includegraphics[width=3mm]{#1} \textsf{#2}}}%
}}}
% tooltip
\newcommand{\tooltip}[1]{%
\mbox{}%
\raise2pt\hbox{%
\fcolorbox{black}[rgb]{1,1,0.8}{\textsf{#1}}%
}}
% toolbtntwo
\newcommand{\toolbtntwo}[2]{%
\mbox{}%
\lower6pt\hbox{%
\shadowbox{\includegraphics[width=7mm]{#1}}} %
\tooltip{#2}%
}
% toolboxtwo
\newcommand{\toolboxtwo}[2]{%
\mbox{}%
\lower6pt\hbox{%
\shadowbox{%
\includegraphics[width=5mm]{#1} #2%
}}%
}
% toolboxthree
\newcommand{\toolboxthree}[3]{%
\mbox{}%
\lower6pt\hbox{%
\shadowbox{%
\includegraphics[width=5mm]{#1}%
\includegraphics[width=5mm]{#2}%
 #3%
}}%
}
% toolboxfour
\newcommand{\toolboxfour}[4]{%
\mbox{}%
\lower6pt\hbox{%
\shadowbox{%
\includegraphics[width=5mm]{#1}%
\includegraphics[width=5mm]{#2}%
\includegraphics[width=5mm]{#3}%
 #4%
}}%
}
%for backward compatibility!
\newcommand{\toolbox}[2]{%
\toolboxtwo{#1}{#2}%
}
% tab
\newcommand{\tab}[1]{%
\lower4pt\hbox{%
\shadowbox{\textsf{#1}}
}}
% checkbox
\newcommand{\checkbox}[1]{%
%\mbox{}%
\raise2pt\hbox{%
\fbox{%
\mbox{}%
\lower4pt\hbox{%
\shadowbox{x}} %
\textsf{#1}}%
}}

\renewcommand{\usertext}[1]{\texttt{#1}}
\renewcommand{\classname}[1]{\textsf{\textbf{#1}}}
\renewcommand{\fieldname}[1]{\textsl{#1}}
\renewcommand{\filename}[1]{\texttt{#1}}
\renewcommand{\keystroke}[1]{\cornersize{.6}\ovalbox{\textsf{#1}}}
\renewcommand{\method}[1]{\textsf{\textit{#1}}}
\renewcommand{\server}[1]{\textit{#1}}
\renewcommand{\sqltable}[1]{\textsf{\textbf{#1}}}


\subsection{Conventions}\label{label_conventions}

The conventions used in this manual are as follows. 

\minisec{GUI Conventions}
\begin{itemize}
%
%Use \mainmenuopt for main menu items that have no icon only text
%Main Menu includes: File, View, Layer, Settings, Plugins, Help
\item Menu Option: \mainmenuopt{Layer} > %
%
%Use \dropmenuopttwo for a drop-down menu item with an icon
%Use \dropmenuoptone for a drop-down menu item with no icon 
\dropmenuopttwo{addraster}{Add a Raster Layer}
%
%Use \toolbtntwo for the toolbar items, including those that open dialogs
%These have an icon, and display a tooltip on hover
%Its really important to get the icon in there because that's what a user
%has to search for. The tooltip adds confirmation
\item Tool: \toolbtntwo{addraster}{Add a Raster Layer}
%
%Use \button for a clickable button which has no icon, just text
%Save As Default is a button that appears in the Layer Properties dialog.
\item Button: \button{Save as Default}
%
%Use \tab for clickable tabs which have no icons, just text
%General is a tab that appears in the Layer Properties dialog.
%At the moment, it looks just like \button, but that may change.
\item Tab: \tab{General}
%
%Use \toolboxtwo, \toolboxthree or \toolboxfour
% for GRASS toolbox (not toolbar) items
%These are what you see in the menu after you click Open GRASS Tools
%The one you need depends on how many icons are required
%The number part of the name is the total number of arguments (N)
%The number of icons is (N-1)
\item Toolbox Item: \toolboxtwo{add_grass_vector}{nviz - Open 3D-View in NVIZ}
%
%Use checkbox for a checkbox item in a dialog popup
\item Checkbox: \checkbox{Render}
\end{itemize}
A shadow indicates a clickable GUI component.

\minisec{Text or Keyboard Conventions}
\begin{itemize}
%
%Use for all urls. Otherwise, it is not clickable in the document.
\item Hyperlinks: \url{http://qgis.org}
%
\item Single Keystroke: press \keystroke{p}
\item Keystroke Combinations: press \keystroke{Ctrl-B}, meaning press and
hold the Ctrl key and then press the B key.
\item Name of a File: \filename{lakes.shp}
\item Name of a Field: \fieldname{NAMES}
\item Name of a Class: \classname{NewLayer}
\item Method: \method{classFactory}
\item Server: \server{example needed here}
\item SQL Table: \sqltable{example needed here}    
%
%Use usertext for all other text that the user must enter from the keyboard
%that is not covered by any of the above cases
\item User Text: \usertext{qgis ---help}
\end{itemize}

Code is indicated by a fixed-width font:
\begin{verbatim}
PROJCS["NAD_1927_Albers",
  GEOGCS["GCS_North_American_1927",
\end{verbatim}

Platform-specific instructions are indicated as follow.
\begin{itemize}
\item \nix{do this.} 
\item \win{do that.} 
\item \osx{do something else.}
\end{itemize} 





