\section{GRASS Toolbox modules}\label{appdx_grass_toolbox_modules}

% when the revision of a section has been finalized, 
% comment out the following line:
\updatedisclaimer

\subsubsection{Data import and export modules}\index{GRASS!toolbox!modules}

In the GRASS Toolbox you find many modules to import and export data into
the binary GRASS raster and vector format.   

\begin{table}[ht]
\centering
\caption{GRASS Toolbox data import modules}\medskip
 \begin{tabular}{|l|p{5.3in}|}
  \hline \multicolumn{2}{|c|}{\textbf{Data import modules in the GRASS
  Toolbox}} \\ 
  \hline \textbf{Module name} & \textbf{Purpose} \\
  \hline r.in.arc & Convert an ESRI ARC/INFO ascii raster file (GRID) into a
  (binary) raster map layer\\
  \hline r.in.ascii & Convert an ASCII raster text file into a (binary)
  raster map layer \\
  \hline r.in.aster & Georeferencing, rectification, and import of
  Terra-ASTER imagery and relative DEM's using gdalwarp \\
  \hline r.in.gdal &  Import GDAL supported raster file into a GRASS binary
  raster map layer \\
  \hline r.in.gdal.loc &  Import GDAL supported raster file into a GRASS
  binary raster map layer and create a fitted location \\
  \hline r.in.gridatb & Imports GRIDATB.FOR map file (TOPMODEL) into GRASS
  raster map \\
  \hline r.in.mat  & Import a binary MAT-File(v4) to a GRASS raster  \\
  \hline r.in.poly  &  Create raster maps from ascii polygon/line data files
  in the current directory \\
  \hline r.in.srtm  & Import SRTM HGT files into GRASS \\
  \hline v.in.db & Import vector points from a database table containing
  coordinates \\
  \hline v.in.dxf.multiple & Import only same layers of a DXF vector \\
  \hline v.in.dxf & Import DXF vector layer \\
  \hline v.in.e00 & Import ESRI E00 file in a vector map \\
  \hline v.in.garmin & Import vector from gps using gpstrans \\
  \hline v.in.gpsbabel & Import vector from gps using gpsbabel \\
  \hline v.in.mapgen & Import MapGen or MatLab vectors in GRASS \\
  \hline v.in.ogr.all.loc & Import all the OGR/PostGIS vector layers in a
  given data source and create a fitted location \\
  \hline v.in.ogr.all & Import all the OGR/PostGIS vector layers in a given
  data source \\
  \hline v.in.ogr.loc & Import OGR/PostGIS vector layers and create a fitted
  location\\
  \hline v.in.ogr & Import OGR/PostGIS vector layers \\
  \hline v.in.region & Create new vector area map with current region extent \\
\hline
\end{tabular}
\end{table}

\begin{table}[ht]
\centering
\caption{GRASS Toolbox data export modules}\medskip
 \begin{tabular}{|l|p{5.3in}|}
  \hline \multicolumn{2}{|c|}{\textbf{Data export modules in the GRASS
  Toolbox}} \\ 
  \hline \textbf{Module name} & \textbf{Purpose} \\
  \hline r.out.gdal.gtiff & Export raster layer to Geo TIFF \\
  \hline r.out.arc & Converts a raster map layer into an ESRI ARCGRID file \\
  \hline r.gridatb & Exports GRASS raster map to GRIDATB.FOR map file
  (TOPMODEL) \\
  \hline r.out.mat & Exports a GRASS raster to a binary MAT-File \\
  \hline r.out.bin & Exports a GRASS raster to a binary array \\
  \hline r.out.png & Export GRASS raster as non-georeferenced PNG image
  format \\
  \hline r.out.ppm & Converts a GRASS raster map to a PPM image file at the
  pixel resolution of the CURRENTLY DEFINED REGION \\
  \hline r.out.ppm3 & Converts 3 GRASS raster layers (R,G,B) to a PPM image
  file at the pixel resolution of the CURRENTLY DEFINED REGION \\
  \hline r.out.pov & Converts a raster map layer into a height-field file for
  POVRAY\\
  \hline r.out.tiff & Exports a GRASS raster map to a 8/24bit TIFF image file
  at the pixel resolution of the currently defined region\\
  \hline r.out.vrml &  Export a raster map to the Virtual Reality Modeling
  Language (VRML)\\
  \hline r.out.xyz & Export a raster map to a text file as x,y,z values based
  on cell centers\\
  \hline v.out.ogr & Export vector layer to various formats (OGR library) \\
  \hline v.out.ogr.shape & Export vector layer to Shape \\
  \hline v.out.ogr.gml & Export vector layer to GML \\
  \hline v.out.ogr.postgis & Export vector layer to various formats (OGR
  library) \\
  \hline v.out.ogr.mapinfo & Mapinfo export of vector layer \\
  \hline v.out.ascii & Convert a GRASS binary vector map to a GRASS ASCII
  vector map  \\
  \hline v.out.dxf & converts a GRASS vector map to DXF  \\
\hline
\end{tabular}
\end{table}

\subsubsection{Data type conversion modules}

\begin{table}[ht]
\centering
\caption{GRASS Toolbox data type conversion modules}\medskip
 \begin{tabular}{|l|p{5.3in}|}
  \hline \multicolumn{2}{|c|}{\textbf{Data type conversion modules in the GRASS
  Toolbox}} \\
  \hline \textbf{Module name} & \textbf{Purpose} \\
  \hline r.to.vect.point & Convert a raster to vector points \\
  \hline r.to.vect.line & Convert a raster to vector lines \\
  \hline r.to.vect.area & Convert a raster to vector areas \\
  \hline r.to.null & Transform cells with value in null cells \\
  \hline v.to.rast.constant & Convert a vector to raster using constant \\
  \hline v.to.rast.attr & Convert a vector to raster using attribute values \\
  \hline v.to.points & Create points along input lines \\
\hline
\end{tabular}
\end{table}

\subsubsection{Region and projection configuration modules}

GRASS GIS region and projection configuration modules.

\begin{table}[ht]
\centering
\caption{GRASS Toolbox region and projection configuration modules}\medskip
 \begin{tabular}{|l|p{5.3in}|}
  \hline \multicolumn{2}{|c|}{\textbf{Region and projection configuration modules in the GRASS
  Toolbox}} \\
  \hline \textbf{Module name} & \textbf{Purpose} \\
  \hline g.region.multiple.raster & Set the region to match multiple raster
  maps \\
  \hline g.region.multiple.vector & Set the region to match multiple vector
  maps \\
  \hline g.region.save & Save the current region as a named region \\
  \hline g.region.zoom & Shrink the current region until it meets non-NULL
  data from a given raster map \\
  \hline g.proj.print & Print projection information of the current location\\
  \hline g.proj.geo & Print projection information from a georeferenced file
  (raster, vector or image)\\
  \hline g.proj.proj & Print projection information from a PROJ.4 projection
  description file\\
  \hline g.proj.ascii.new & Print projection information from a georeferenced
  ASCII file containing a WKT projection description and create a new
  location based on it\\
  \hline g.proj.geo.new & Print projection information from a georeferenced
  file (raster, vector or image) and create a new location based on it\\
  \hline g.proj.proj.new & Print projection information from a PROJ.4
  projection description file and create a new location based on it \\
  \hline m.cogo & A simple utility for converting bearing and distance
  measurements to coordinates and vice versa. It assumes a cartesian
  coordinate system \\
\hline
\end{tabular}
\end{table}

\subsubsection{Raster data modules}

Raster data analysis in GRASS GIS

\begin{table}[ht]
\centering
\caption{GRASS Toolbox develop raster map modules}\medskip
 \begin{tabular}{|l|p{5.3in}|}
  \hline \multicolumn{2}{|c|}{\textbf{Develop raster map modules in the GRASS
  Toolbox}} \\
  \hline \textbf{Module name} & \textbf{Purpose} \\
  \hline r.compress & Compresses and decompresses raster maps \\
  \hline r.region.region & Sets the boundary definitions to current or
  default region \\
  \hline r.region.raster & Sets the boundary definitions from existent raster
  map\\
  \hline r.region.vector & Sets the boundary definitions from existent vector map \\
  \hline r.region.edge & Sets the boundary definitions by edge (n-s-e-w) \\
  \hline r.region.alignTo & Sets region to align to a raster map\\
  \hline r.null.null & Transform cells with value in null cells\\
  \hline r.null.to & Transform null cells in value cells\\
  \hline r.quant & This routine produces the quantization file for a
  floating-point map \\
  \hline r.resamp.stats & Resamples raster map layers using aggregation \\
  \hline r.resamp.interp & Resamples raster map layers using interpolation \\
  \hline r.resample & GRASS raster map layer data resampling capability.
  Before you must set new resolution\\
  \hline r.resamp.rst & Reinterpolates and computes topographic analysis
  using regularized spline with tension and smoothing \\
  \hline r.support & Allows creation and/or modification of raster map layer
  support files\\
  \hline r.support.stats & Update raster map statistics \\
  \hline r.proj & Re-project a raster map from one location to the current
  location \\
\hline
\end{tabular}
\end{table}

\begin{table}[ht]
\centering
\caption{GRASS Toolbox raster color management modules}\medskip
 \begin{tabular}{|l|p{5.3in}|}
  \hline \multicolumn{2}{|c|}{\textbf{Raster color management modules in the GRASS
  Toolbox}} \\
  \hline \textbf{Module name} & \textbf{Purpose} \\
  \hline r.colors.table & Set raster color table from setted tables \\
  \hline r.colors.rules & Set raster color table from setted rules \\
  \hline r.blend & Blend color components for two raster maps by given ratio \\
  \hline r.composite & Blend red, green, raster layers to obtain one color
  raster \\
  \hline r.his & Generates red, green and blue raster map layers combining
  hue, intensity, and saturation (his) values from user-specified input
  raster map layers \\
\hline
\end{tabular}
\end{table}

\begin{table}[ht]
\centering
\caption{GRASS Toolbox spatial raster analysis modules}\medskip
 \begin{tabular}{|l|p{5.3in}|}
  \hline \multicolumn{2}{|c|}{\textbf{Spatial raster analysis modules in the GRASS
  Toolbox}} \\
  \hline \textbf{Module name} & \textbf{Purpose} \\
  \hline r.buffer & Raster buffer \\
  \hline r.mask & Create a MASK for limiting raster operation \\
  \hline r.mapcalc & Raster map calculator \\
  \hline r.mapcalculator & Simple map algebra \\
  \hline r.neighbors & Raster neighbors analyses \\
  \hline v.neighbors & Count of neighbouring points \\
  \hline r.statistics & Category or object oriented statistics \\
  \hline r.cost & Outputs a raster map layer showing the cumulative cost of
  moving between different geographic locations on an input raster map layer
  whose cell category values represent cost\\
  \hline r.drain & Traces a flow through an elevation model on a raster map
  layer \\
  \hline r.shaded.relief & Create shaded map \\
  \hline r.slope.aspect.slope & Generate slope map from DEM (digital
  elevation model) \\
  \hline r.slope.aspect.aspect & Generate aspect map from DEM (digital
  elevation model) \\
  \hline r.param.scale & Extracts terrain parameters from a DEM \\
  \hline r.texture & Generate images with textural features from a raster map
  (first serie of indices)\\
  \hline r.texture.bis & Generate images with textural features from a raster
  map (second serie of indices)\\
  \hline r.los & Line-of-sigth raster analysis \\
  \hline r.clump & Recategorizes into unique categories contiguous cells \\
  \hline r.grow & Generates a raster map layer with contiguous areas grown by
  one cell\\
  \hline r.thin & Thin no-zero cells that denote line features \\
\hline
\end{tabular}
\end{table}

\begin{table}[ht]
\centering
\caption{GRASS Toolbox hydrologic modelling modules}\medskip
 \begin{tabular}{|l|p{5.3in}|}
  \hline \multicolumn{2}{|c|}{\textbf{Hydrologic modelling modules in the GRASS
  Toolbox}} \\
  \hline \textbf{Module name} & \textbf{Purpose} \\
  \hline r.carve & Takes vector stream data, transforms it to raster, and
  subtracts depth from the output DEM \\
  \hline r.fill.dir & Filters and generates a depressionless elevation map
  and a flow direction map from a given elevation layer \\
  \hline r.lake.xy & Fills lake from seed point at given level \\
  \hline r.lake.seed & Fills lake from seed at given level \\
  \hline r.topidx & Creates a 3D volume map based on 2D elevation and value
  raster maps \\
  \hline r.basins.fill & Generates a raster map layer showing watershed
  subbasins \\
  \hline r.water.outlet & Watershed basin creation program \\
\hline
\end{tabular}
\end{table}

\begin{table}[ht]
\centering
\caption{GRASS Toolbox change raster category values and labels modules}\medskip
 \begin{tabular}{|l|p{5.3in}|}
  \hline \multicolumn{2}{|c|}{\textbf{Raster category and label modules in the GRASS Toolbox}} \\
  \hline \textbf{Module name} & \textbf{Purpose} \\
  \hline r.reclass.area.greater & Reclasses a raster map greater than user
  specified area size (in hectares) \\
  \hline r.reclass.area.lesser & Reclasses a raster map less than user
  specified area size (in hectares) \\
  \hline r.reclass & Reclass a raster using a reclassification rules file \\
  \hline r.recode & Recode raster maps\\
  \hline r.rescale & Rescales the range of category values in a raster map
  layer \\
\hline
\end{tabular}
\end{table}

\begin{table}[ht]
\centering
\caption{GRASS Toolbox surface management modules}\medskip
 \begin{tabular}{|l|p{5.3in}|}
  \hline \multicolumn{2}{|c|}{\textbf{Surface management modules in the GRASS
  Toolbox}} \\
  \hline \textbf{Module name} & \textbf{Purpose} \\
  \hline r.random & Creates a random vector point map contained in a raster \\
  \hline r.random.cells & Generates random cell values with spatial
  dependence \\
  \hline v.kernel & Gaussian kernel density \\
  \hline r.contour & Produces a contours vector map with specified step from
  a raster map\\
  \hline r.contour2 & Produces a contours vector map of specified contours
  from a raster map \\
  \hline r.surf.fractal & Creates a fractal surface of a given fractal
  dimension\\
  \hline r.surf.gauss & GRASS module to produce a raster map layer of
  gaussian deviates whose mean and standard deviation can be expressed by the
  user \\
  \hline r.surf.random & Produces a raster map layer of uniform random
  deviates whose range can be expressed by the user \\
  \hline r.bilinear & Bilinear interpolation utility for raster map layers \\
  \hline v.surf.bispline & Bicubic or bilinear spline interpolation with
  Tykhonov regularization\\
  \hline r.surf.idw & Surface interpolation utility for raster map layers\\
  \hline r.surf.idw2 & Surface generation program\\
  \hline r.surf.contour & Surface generation program from rasterized contours \\
  \hline v.surf.idw & Interpolate attribute values (IDW) \\
  \hline v.surf.rst & Interpolate attribute values (RST) \\
  \hline r.fillnulls & Fills no-data areas in raster maps using v.surf.rst
  splines interpolation \\
\hline
\end{tabular}
\end{table}

\begin{table}[ht]
\centering
\caption{GRASS Toolbox statistic analysis modules}\medskip
 \begin{tabular}{|l|p{5.3in}|}
  \hline \multicolumn{2}{|c|}{\textbf{Statistic analysis modules in the GRASS
  Toolbox}} \\
  \hline \textbf{Module name} & \textbf{Purpose} \\
  \hline r.category & Prints category values and labels associated with
  user-specified raster map layers \\
  \hline r.sum & Sums up the raster cell values \\
  \hline r.report & Reports statistics for raster map layers \\
  \hline r.average & Finds the average of values in a cover map within areas
  assigned the same category value in a user-specified base map \\
  \hline r.median & Finds the median of values in a cover map within areas
  assigned the same category value in a user-specified base map \\
  \hline r.mode & Finds the mode of values in a cover map within areas
  assigned the same category value in a user-specified base map.reproject
  raster image \\
  \hline r.volume & Calculates the volume of data clumps, and produces a
  GRASS vector points map containing the calculated centroids of these clumps \\
  \hline r.surf.area & Surface area estimation for rasters \\
  \hline r.covar & Outputs a covariance/correlation matrix for user-specified
  raster map layer(s)\\
  \hline r.regression.line & Calculates linear regression from two raster
  maps: y = a + b * x \\
  \hline r.coin & Tabulates the mutual occurrence (coincidence) of categories
  for two raster map layers\\
\hline
\end{tabular}
\end{table}

\subsubsection{GRASS Toolbox vector data modules}

GRASS vector data analysis ...

\begin{table}[ht]
\centering
\caption{GRASS Toolbox develop vector map modules}\medskip
 \begin{tabular}{|l|p{5.3in}|}
  \hline \multicolumn{2}{|c|}{\textbf{Develop vector map modules in the GRASS
  Toolbox}} \\
  \hline \textbf{Module name} & \textbf{Purpose} \\
  \hline v.build.all & Rebuild topology of all vectors in the mapset \\
  \hline v.clean.break & Break lines at each intersection of vector map \\
  \hline v.clean.snap & Cleaning topology: snap lines to vertex in threshold \\
  \hline v.clean.rmdangles & Cleaning topology: remove dangles \\
  \hline v.clean.chdangles & Cleaning topology: change the type of boundary
  dangle to line \\
  \hline v.clean.rmbridge & Remove bridges connecting area and island or 2
  islands \\
  \hline v.clean.rmdupl & Remove duplicate lines (pay attention to
  categories!) \\
  \hline v.clean.rmdac & Remove duplicate area centroids \\
  \hline v.clean.bpol & Break (topologically clean) polygons (imported from
  non topological format, like ShapeFile). Boundaries are broken on each
  point shared between 2 and more polygons where angles of segments are
  different \\
  \hline v.clean.prune & Remove vertices in threshold from lines and
  boundaries, boundary is pruned only if topology is not damaged (new
  intersection, changed attachement of centroid), first and last segment of
  the boundary is never changed \\
  \hline v.clean.rmarea & Remove small areas, the longest boundary with
  adjacent area is removed \\
  \hline v.clean.rmline & Remove all lines or boundaries of zero length \\
  \hline v.clean.rmsa & Remove small angles between lines at nodes \\
  \hline v.type.lb & Convert lines to boundaries \\
  \hline v.type.bl & Convert boundaries to lines \\
  \hline v.type.pc & Convert points to centroids \\
  \hline v.type.cp & Convert centroids to points \\
  \hline v.centroids & Add missing centroids to closed boundaries  \\
  \hline v.build.polylines & Build polylines from lines \\
  \hline v.segment & Creates points/segments from input vector lines and
  positions \\
  \hline v.to.points & Create points along input lines \\
  \hline v.parallel & Create parallel line to input lines \\
  \hline v.dissolve & Dissolves boundaries between adjacent areas sharing a
  common category number or attribute \\
  \hline v.drape & Convert 2D vector to 3D vector by sampling of elevation
  raster. Default sampling by nearest neighbor \\
  \hline v.transform & Performs an affine transformation (shift, scale and
  rotate, or GPCs) on vector map \\
  \hline v.proj & Allows projection conversion of vector files \\
\hline
\end{tabular}
\end{table}

\begin{table}[ht]
\centering
\caption{GRASS Toolbox database connection modules}\medskip
 \begin{tabular}{|l|p{5.3in}|}
  \hline \multicolumn{2}{|c|}{\textbf{Database connection modules in the GRASS
  Toolbox}} \\
  \hline \textbf{Module name} & \textbf{Purpose} \\
  \hline v.db.connect & Connect a vector to database \\
  \hline v.db.sconnect & Disconnect a vector from database \\
  \hline v.db.what.connect & Set/Show database connection for a vector \\
\hline
\end{tabular}
\end{table}

\begin{table}[ht]
\centering
\caption{GRASS Toolbox spatial and network analysis modules}\medskip
 \begin{tabular}{|l|p{5.3in}|}
  \hline \multicolumn{2}{|c|}{\textbf{Spatial and network analysis modules in the GRASS
  Toolbox}} \\
  \hline \textbf{Module name} & \textbf{Purpose} \\
  \hline v.extract.where & Select features by attributes \\
  \hline v.extract.list & Extract selected features \\
  \hline v.select.overlap & Select features overlapped by features in another
  map\\
  \hline v.buffer & Vector buffer \\
  \hline v.distance & Find the nearest element in vector 'to' for elements in
  vector 'from'. Various information about this relation may be uploaded to
  the attribute table of input vector 'from'\\
  \hline v.net.nodes & Create nodes on network \\
  \hline v.net.alloc & Allocate network\\
  \hline v.net.iso & Cut network by cost isolines \\
  \hline v.net.salesman & Connect nodes by shortest route (traveling
  salesman) \\
  \hline v.net.steiner & Connect selected nodes by shortest tree (Steiner
  tree) \\
  \hline v.overlay.or & Vector union \\
  \hline v.overlay.and & Vector intersection \\
  \hline v.overlay.not & Vector subtraction \\
  \hline v.overlay.xor & Vector non-intersection \\
\hline
\end{tabular}
\end{table}

\begin{table}[ht]
\centering
\caption{GRASS Toolbox change field modules}\medskip
 \begin{tabular}{|l|p{5.3in}|}
  \hline \multicolumn{2}{|c|}{\textbf{Change vector field modules in the GRASS
  Toolbox}} \\
  \hline \textbf{Module name} & \textbf{Purpose} \\
  \hline v.category.change & Change layer number \\
  \hline v.category.add & Add elements to layer (ALL elements of the selected
  layer type!)\\
  \hline v.category.del & Delete category values \\
  \hline v.category.sum & Add a value to the current category values \\
  \hline v.reclass.file & Reclass category values using a rules file \\
  \hline v.reclass.attr & Reclass category values using a column attribute
  (integer positive) \\
\hline
\end{tabular}
\end{table}

\begin{table}[ht]
\centering
\caption{GRASS Toolbox working with vector points modules}\medskip
 \begin{tabular}{|l|p{5.3in}|}
  \hline \multicolumn{2}{|c|}{\textbf{Working with vector points modules in the GRASS Toolbox}} \\
  \hline \textbf{Module name} & \textbf{Purpose} \\
  \hline v.in.region & Create new vector area map with current region extent \\
  \hline v.mkgrid.region & Create grid in current region \\
  \hline v.in.db & Import vector points from a database table containing
  coordinates \\
  \hline v.hull & Create a convex hull \\
  \hline v.delaunay.line & Delaunay triangulation (lines) \\
  \hline v.delaunay.area & Delaunay triangulation (areas) \\
  \hline v.voronoi.line & Voronoi diagram (lines) \\
  \hline v.voronoi.area & Voronoi diagram (areas) \\
\hline
\end{tabular}
\end{table}

\begin{table}[ht]
\centering
\caption{GRASS Toolbox vector update by other maps modules}\medskip
 \begin{tabular}{|l|p{5.3in}|}
  \hline \multicolumn{2}{|c|}{\textbf{Vector update by other maps modules in the GRASS
  Toolbox}} \\
  \hline \textbf{Module name} & \textbf{Purpose} \\
  \hline v.rast.stats & Calculates univariate statistics from a GRASS raster
  map based on vector objects\\
  \hline v.what.vect & Uploads map for which to edit attribute table \\
  \hline v.what.rast & Uploads raster values at positions of vector points to
  the table \\
  \hline v.sample & Sample a raster file at site locations \\
\hline
\end{tabular}
\end{table}

\begin{table}[ht]
\centering
\caption{GRASS Toolbox report and statistic modules}\medskip
 \begin{tabular}{|l|p{5.3in}|}
  \hline \multicolumn{2}{|c|}{\textbf{Vector report and statistic modules in the GRASS
  Toolbox}} \\
  \hline \textbf{Module name} & \textbf{Purpose} \\
  \hline v.to.db & Put geometry variables in database \\
  \hline v.report & Reports geometry statistics for vectors \\
  \hline v.univar & Calculates univariate statistics on selected table column
  for a GRASS vector map \\
  \hline v.normal & Tests for normality for points\\
\hline
\end{tabular}
\end{table}

\subsubsection{GRASS Toolbox imagery data modules}

Imagery Modules in GRASS GIS

\begin{table}[ht]
\centering
\caption{GRASS Toolbox imagery analysis modules}\medskip
 \begin{tabular}{|l|p{5.3in}|}
  \hline \multicolumn{2}{|c|}{\textbf{Imagery analysis modules in the GRASS
  Toolbox}} \\
  \hline \textbf{Module name} & \textbf{Purpose} \\
  \hline i.image.mosaik & Mosaic up to 4 images \\
  \hline i.rgb.his & Red Green Blue (RGB) to Hue Intensity Saturation (HIS)
  raster map color transformation function \\
  \hline i.his.rgb & Hue Intensity Saturation (HIS) to Red Green Blue (RGB)
  raster map color transform function \\
  \hline i.landsat.rgb & Auto-balancing of colors for LANDSAT images \\
  \hline i.fusion.brovey & Brovey transform to merge multispectral and
  high-res pancromatic channels \\
  \hline i.zc & Zero-crossing edge detection raster function for image
  processing \\
  \hline i.mfilter &  \\
  \hline i.tasscap4 & Tasseled Cap (Kauth Thomas) transformation for
  LANDSAT-TM 4 data \\
  \hline i.tasscap5 & Tasseled Cap (Kauth Thomas) transformation for
  LANDSAT-TM 5 data \\
  \hline i.tasscap7 & Tasseled Cap (Kauth Thomas) transformation for
  LANDSAT-TM 7 data \\
  \hline i.fft & Fast fourier transform for image processing \\
  \hline i.ifft & Inverse fast fourier transform for image processing \\
  \hline r.describe & Prints terse list of category values found in a raster
  map layer \\
  \hline r.bitpattern & Compares bit patterns with a raster map \\
  \hline r.kappa & Calculate error matrix and kappa parameter for accuracy
  assessment of classification result \\
  \hline i.oif & Calculates optimal index factor table for landsat tm bands \\
\hline
\end{tabular}
\end{table}

\subsubsection{GRASS Toolbox database modules}

Database Modules

\begin{table}[ht]
\centering
\caption{GRASS Toolbox database modules}\medskip
 \begin{tabular}{|l|p{5.3in}|}
  \hline \multicolumn{2}{|c|}{\textbf{Database management and analysis modules in the GRASS
  Toolbox}} \\
  \hline \textbf{Module name} & \textbf{Purpose} \\
  \hline db.connect & Sets general DB connection mapset \\
  \hline db.connect.schema & Sets general DB connection mapset with a schema \\
  \hline db.login & Set user/password for driver/database \\
  \hline db.in.ogr & Imports attribute tables in various formats \\
  \hline v.db.addtable & Create and add a new table to a vector \\
  \hline v.db.addcol & Adds one or more columns to the attribute table
  connected to a given vector map \\
  \hline v.db.dropcol & Drops a column from the attribute table connected to
  a given vector map\\
  \hline v.db.renamecol & Renames a column in a attribute table connected to
  a given vector map\\
  \hline v.db.update\_const & Allows to assign a new constant value to a
  column \\
  \hline v.db.update\_query & Allows to assign a new constant value to a
  column only if the result of a query is TRUE \\
  \hline v.db.update\_op & Allows to assign a new value, result of operation
  on column(s), to a column in the attribute table connected to a given map\\
  \hline v.db.update\_op\_query & Allows to assign a new value to a column,
  result of operation on column(s), only if the result of a query is TRUE \\
  \hline db.execute & Execute any SQL statement \\
  \hline v.db.join & Allows to join a table to a vector map table \\
  \hline v.db.univar & Calculates univariate statistics on selected table
  column for a GRASS vector map \\
\hline
\end{tabular}
\end{table}

\subsubsection{GRASS Toolbox 3D modules}

3D Visualization and analysis

\begin{table}[ht]
\centering
\caption{GRASS Toolbox 3D Visualization}\medskip
 \begin{tabular}{|l|p{5.3in}|}
  \hline \multicolumn{2}{|c|}{\textbf{3D visualization and analysis modules in the GRASS
  Toolbox}} \\
  \hline \textbf{Module name} & \textbf{Purpose} \\
  \hline nviz & 3D visualization tool\\
\hline
\end{tabular}
\end{table}

\subsubsection{GRASS Toolbox help modules}

GRASS GIS Help 

\begin{table}[ht]
\centering
\caption{GRASS Toolbox Reference Manual}\medskip
 \begin{tabular}{|l|p{5.3in}|}
  \hline \multicolumn{2}{|c|}{\textbf{Reference Manual modules in the GRASS
  Toolbox}} \\
  \hline \textbf{Module name} & \textbf{Purpose} \\
  \hline g.manual & Open local reference manual for all GRASS modules \\
\hline
\end{tabular}
\end{table}

\clearpage
