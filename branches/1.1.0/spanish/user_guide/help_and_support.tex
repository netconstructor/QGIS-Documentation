% vim: set textwidth=78 autoindent:

\section{Ayuda y soporte}\label{label_helpsupport}

% when the revision of a section has been finalized, 
% comment out the following line:
% \updatedisclaimer

\subsection{Listas de correo}
QGIS está bajo un desarrollo activo y como tal no siempre funcionará como espera. La forma preferida para obtener ayuda
es uniéndose a la lista de correo de usuarios de QGIS.

\minisec{qgis-users}
Sus preguntas llegarán a una audiencia más amplia y las respuestas beneficiarán a otros. Puede 
suscribirse a la lista de usuarios de QGIS (qgis-users) visitando la siguiente URL: \\
\url{http://lists.osgeo.org/mailman/listinfo/qgis-user}

\minisec{qgis-developer}
Si es un desarrollador con problemas de naturaleza más técnica, puede unirse a la lista de desarrolladores de QGIS (qgis-developer) aquí:\\
\url{http://lists.osgeo.org/mailman/listinfo/qgis-developer}

\minisec{qgis-commit}
Cada vez que se hace un envío al repositorio del código de QGIS se envía un correo a esta lista. Si quiere 
estar al día con cada cambio en el código base, puede suscribirse a esta lista en:\\
\url{http://lists.osgeo.org/mailman/listinfo/qgis-commit}

\minisec{qgis-trac}
Esta lista proporciona notificación por correo electrónico relacionada con la administración del 
proyecto, incluyendo informes de errores, tareas y peticiones de características. Puede suscribirse a esta lista en:\\
\url{http://lists.osgeo.org/mailman/listinfo/qgis-trac}

\minisec{qgis-community-team}
Esta lista trata sobre temas como documentación, ayuda contextual, guía de usuario, experiencia online que 
incluye páginas web, blog, listas de correo, foros y trabajos de traducción. Si quiere trabajar en la guía de 
usuario, esta lista es un buen punto de inicio para hacer sus preguntas. Puede suscribirse a esta lista en:\\
\url{http://lists.osgeo.org/mailman/listinfo/qgis-community-team}

\minisec{qgis-release-team}
Esta lista trata sobre temas como el proceso de lanzamiento, empaquetado de binarios para varios sistemas 
operativos y el anuncio de lanzamiento de nuevas versiones a todo el mundo. Puede suscribirse a esta lista en:\\
\url{http://lists.osgeo.org/mailman/listinfo/qgis-release-team}

\minisec{qgis-psc}
Esta lista se usa para debatir asuntos del Comité de dirección relacionados sobre la administración global y la dirección de Quantum GIS. Puede suscribirse a esta lista en:\\
\url{http://lists.osgeo.org/mailman/listinfo/qgis-psc}

Sea bienvenido a suscribirse a cualquiera de las listas. Por favor, recuerde aportar a la lista respondiendo 
preguntas y compartiendo sus experiencias. Tenga en cuenta que qgis-commit y qgis-trac están diseñadas sólo 
para notificación y no para envíos de usuarios. 

\subsection{IRC}
También mantenemos presencia en IRC - visítenos uniéndose al canal \#qgis en
\url{irc.freenode.net}. 
Por favor espere un poco a la respuesta a sus preguntas, ya que muchos colegas del canal están haciendo otras 
cosas y puede llevar un tiempo hasta que ellos su pregunta. También hay disponible soporte comercial para QGIS.
Vea la página web \url{http://qgis.org/content/view/90/91} para más información.

Si se ha perdido un debate en IRC, no es problema. Registramos todos los debates de forma que puede 
recuperarlos fácilmente. Simplemente vaya a \url{http://logs.qgis.org} y lea los registros del IRC.

\subsection{Seguimiento de errores}
Mientras que la lista de usuarios de QGIS es útil para el tipo de preguntas «¿Cómo hago XYZ en QGIS?», puede 
desear avisarnos de errores en QGIS. Puede remitir informes de errores usando el seguidor de errores 
en \url{https://trac.osgeo.org/qgis/}. 
Cuando cree un nuevo ticket para un error, por favor proporcione una dirección de correo electrónico en la 
que podamos solicitar información adicional. 

Por favor, tenga en mente que su error puede que no siempre alcance la prioridad que espera (dependiendo de su 
gravedad). Algunos errores pueden necesitar un esfuerzo considerable de desarrollo para solucionarlos y los 
recursos humanos no siempre están disponible para ello.

Las peticiones de características también se pueden remitir usando el mismo sistema de tickets que para los 
errores. Asegúrese de seleccionar el tipo \usertext{enhancement}.

Si ha encontrado un error y lo ha corregido usted mismo puede remitir un parche. De nuevo, el maravilloso 
sistema de tickets de \url{https://trac.osgeo.org/qgis/} tiene este tipo. Selecione \usertext{patch} en 
el menú Tipo. Alguno de los desarrolladores lo revisará y lo aplicará a QGIS. \\
Por favor, no se 
alarme si su parche no se aplica inmediatamente - los desarrolladores pueden estar ocupados con otros envíos.

% unused, since community.qgis.org seems to be lost. (SH)
% There is also a community site for QGIS where we encourage QGIS users to share
% their experiences and provide case studies about how they are using QGIS. The
% community site is available at: http://community.qgis.org 

\subsection{Blog}
La comunidad de QGIS también mantiene un blog en \url{http://blog.qgis.org} 
que tiene algunos artículos interesantes para usuarios y desarrolladores. ¡Está invitado a contribuir al blog después de registrarse!

\subsection{Wiki}
Por último, mantenemos una WIKI en \url{http://www.qgis.org/wiki} donde puede encontrar variedad de información útil relacionada con el desarrollo de QGIS, planes de lanzamientos, enlaces a sitios de descarga, consejos sobre traducción de mensajes y más. 
¡Compruébelo, hay cosas buenas dentro!

