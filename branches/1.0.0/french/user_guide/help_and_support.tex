% vim: set textwidth=78 autoindent:

% \section{Help and Support}\label{label_helpsupport}
\section{Aide et support}\label{label_helpsupport}

% when the revision of a section has been finalized, 
% comment out the following line:
% \updatedisclaimer

\subsection{Mailinglists}
% QGIS is under active development and as such it won't always work like
% you expect it to. The preferred way to get help is by joining the qgis-users
% mailing list.
QGIS est en cours de développement par conséquent il ne fonctionne pas toujours
comme attendu. La manière préférée d'obtenir de l'aide est de rejoindre la
liste de diffusion qgis-users.

\minisec{qgis-users}
% Your questions will reach a broader audience and answers will
% benefit others. You can subscribe to the qgis-users mailing list by visiting
% the following URL: \\
% \url{http://lists.osgeo.org/mailman/listinfo/qgis-user}
Vos questions atteindront une audience plus large et les réponses bénéficieront
à tous. Vous pouvez rejoindre la liste de diffusion qgis-users en allant sur la
page suivante : \\
\url{http://lists.osgeo.org/mailman/listinfo/qgis-user}

\minisec{qgis-developer}
% If you are a developer facing problems of a more technical nature, you may
% want to join the qgis-developer mailing list here:\\
% \url{http://lists.osgeo.org/mailman/listinfo/qgis-developer}
Si vous êtes un développeur et que vous faites face à un problème plus
technique, il est préf\'rable de rejoindre la liste de diffusion qgis-developer :\\
\url{http://lists.osgeo.org/mailman/listinfo/qgis-developer}

\minisec{qgis-commit}
% Each time a commit is made to the QGIS code repository an email is posted to
% this list. If you want to be up to date with every change to the current code
% base, you can subscribe to this list at:\\
% \url{http://lists.osgeo.org/mailman/listinfo/qgis-commit}
À chaque fois qu'un commit est réalisé sur le dépôt du code de QGIS un email
est envoyé à cette liste. Si vous voulez être à jour de chaque changement au
code en cours, vous pouvez vous inscrire à cette liste :\\
\url{http://lists.osgeo.org/mailman/listinfo/qgis-commit}

\minisec{qgis-trac}
% This list provides email notification related to project management,
% including bug reports, tasks, and feature requests. You can subscribe to this
% list at:\\
% \url{http://lists.osgeo.org/mailman/listinfo/qgis-trac}
Cette liste fournit une notification par mail liée à la gestion du projet,
incluant les rapports de bugs, tâches, et demandes de fonctionnalités. Vous
pouvez vous inscrire à cette liste ici :\\
\url{http://lists.osgeo.org/mailman/listinfo/qgis-trac}

\minisec{qgis-community-team}
% This list deals with topics like documentation, context help, user-guide,
% online experience including web sites, blog, mailing lists, forums, and
% translation efforts. If you like to work on the user-guide as well, this list
% is a good starting point to ask your questions. You can subscribe to this
% list at:\\
% \url{http://lists.osgeo.org/mailman/listinfo/qgis-community-team}
Cette liste reçoit les mails des thématiques lié à la documentation, au
contexte d'aide, au guide utilisateur, à ce qui est lié à Internet donc les
sites, listes de diffusion, forums et efforts de traduction. Si vous voulez
travailler sur le guide utilisateur, cette liste est un bon point de départ
pour poser vos questions. Vous pouvez vous inscrire à cette liste ici :\\
\url{http://lists.osgeo.org/mailman/listinfo/qgis-community-team}

\minisec{qgis-release-team}
% This list deals with topics like the release process, packaging binaries for
% various OS and announcing new releases to the world at large. You can
% subscribe to this list at:\\
% \url{http://lists.osgeo.org/mailman/listinfo/qgis-release-team}
Cette liste reçoit les mails des thématiques comme les procédures de
publication de version, paquetage binaire pour différents systèmes et annonce
des nouvelles versions à un monde plus large. Vous pouvez vous inscrire à cette
liste ici :\\
\url{http://lists.osgeo.org/mailman/listinfo/qgis-release-team}

\minisec{qgis-psc}
% This list is used to discuss Steering Committee issues related to overall
% management and direction of Quantum GIS. You can subscribe to this list at:\\
% \url{http://lists.osgeo.org/mailman/listinfo/qgis-psc}
Cette liste est utilisée pour discuter des problèmes du Comité de Pilotage lié à
l'ensemble de la gestion et de la direction de Quantum GIS. Vous pouvez vous
inscrire à cette liste ici :\\
\url{http://lists.osgeo.org/mailman/listinfo/qgis-psc}

% You are welcome to subscribe to any of the lists. Please remember to
% contribute to the list by answering questions and sharing your experiences.
% Note that the qgis-commit and qgis-trac are designed for notification only
% and not meant for user postings.
Vous êtes invité à vous inscrire à ces listes. S'il vous plait, souvenez-vous de
contribuer à la liste en répondant à des questions et en partageant vos
expériences. Remarquez que les listes qgis-commit et qgis-trac ont été
configurées pour notification seulement et n'acceptent pas de mail
d'utilisateurs.

\subsection{IRC}
% We also maintain a presence on IRC - visit us by joining the \#qgis channel on
% \url{irc.freenode.net}. Please wait around for a response to your question as
% many folks on the channel are doing other things and it may take a while for
% them to notice your question. Commercial support for QGIS is also available.
% Check the website \url{http://qgis.org/content/view/90/91} for more
% information.
Nous maintenons une présence sur IRC - rejoignez-nous sur le canal \#qgis sur
\url{irc.freenode.net}. S'il vous plait, patientez pour obtenir une réponse
puisque la plupart des personnes font autre chose et cela peut leur prendre un
peu de temps pour remarquer votre question. Un support commercial pour QGIS est
disponible. Regardez la page du site \url{http://qgis.org/content/view/90/91}
pour plus d'informations.

% If you missed a discussion on IRC, not a problem! We log all discussion so you
% can easily catch up. Just go to \url{http://logs.qgis.org} and read the
% IRC-logs.
Si vous ratez une discussion sur IRC, pas de problème ! Nous loguons toutes les
discussions afin que vous puisiez facilement les suivre. Allez simplement sur
\url{http://logs.qgis.org} et lisez les logs IRC.

\subsection{BugTracker}
% While the qgis-users mailing list is useful for general 'how do I do xyz in
% QGIS' type questions, you may wish to notify us about bugs in QGIS. You can
% submit bug reports using the QGIS bug tracker at
% \url{https://trac.osgeo.org/qgis/}. When creating a new ticket for a bug,
% please provide an email address where we can request additional information.
Bien que la liste de diffusion utilisateur est utile pour des questions
générales du type 'Comment je réalise xyz dans QGIS ?', vous pouvez vouloir
nous avertir de bugs dans QGIS. Vous pouvez soumettre un rapport de bug en
utilisant le tracker de bug sur \url{https://trac.osgeo.org/qgis/}. Lors de la
création d'un ticket pour un bug, fournissez s'il vous plait une adresse mail
valide où nous pouvons vous demander des informations supplémentaires.

% Please bear in mind that your bug may not always enjoy the priority you might
% hope for (depending on its severity). Some bugs may require significant
% developer effort to remedy and the manpower is not always available for this.
Garder en mémoire que votre bug peut ne pas avoir la priorité à laquelle vous
vous attendiez (cela dépendra de sa sévérité). Certains bugs peuvent nécessiter du
travail supplémentaire de la part des développeurs pour y remédier et la personne
compétente n'est pas forcément disponible.

% Feature requests can be submitted as well using the same ticket system as for
% bugs. Please make sure to select the type \usertext{enhancement}.
Les demandes de fonctionnalité peuvent être soumises également en utilisant le
même système de ticket que pour les bugs. Assurez-vous de sélectionner le type
\usertext{enhancement}.

% If you have found a bug and fixed it yourself you can submit this patch also.
% Again, the lovely trac ticketsystem at \url{https://trac.osgeo.org/qgis/} has
% this type as well. Select \usertext{patch} from the type-menu. Someone of the 
% developers will review it and apply it to QGIS. \\
Si vous avez trouvé un bug et l'avez corrigé vous même, vous pouvez
aussi soumettre un patch. Encore, le superbe système de ticket Trac sur
\url{https://trac.osgeo.org/qgis/} a également ce type. Sélectionnez
\usertext{patch} dans le menu type. Un des développeurs le vérifiera et
l'appliquera à QGIS.\\
% Please don't be alarmed if your patch is not applied straight away -
% developers may be tied up with other committments.
Ne vous alarmez pas si votre correctif n'est pas appliqué directement - les
développeurs peuvent être occupés sur d'autres commits.

% unused, since community.qgis.org seems to be lost. (SH)
% There is also a community site for QGIS where we encourage QGIS users to share
% their experiences and provide case studies about how they are using QGIS. The
% community site is available at: http://community.qgis.org 

\subsection{Blog}
% The QGIS-community also runs a weblog (BLOG) at \url{http://blog.qgis.org} 
% which has some interesting articles for users and developers as well. 
% You are invited to contribute to the blog after registering yourself!
La communauté QGIS tient également un weblog (BLOG) sur
\url{http://blog.qgis.org} qui publie d'intéressants articles à la fois pour les
utilisateurs et les développeurs. Vous êtes invités à contribuer au blog après
vous être enregistrés.

\subsection{Wiki}
% Lastly, we maintain a WIKI web site at \url{http://wiki.qgis.org} where you 
% can find a variety of useful information relating to QGIS development, 
% release plans, links to download sites, message translation-hints and so
% on. Check it out, there are some goodies inside!
Enfin, nous maintenons un site web wiki sur \url{http://wiki.qgis.org} où vous
pouvez trouver diverses informations utiles liées au développement de QGIS, plan
des versions, liens vers les sites de téléchargement, astuces de
traduction des messages, etc. Parcourez le, il y a des choses intéressantes.
