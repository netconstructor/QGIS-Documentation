% vim: set textwidth=78 autoindent:

% \section{Writing a QGIS Plugin in C++}\label{cpp_plugin}
\section{\'Ecrire des extensions pour QGIS en C++}\label{cpp_plugin}

% when the revision of a section has been finalized, 
% comment out the following line:
% \updatedisclaimer

% In this section we provide a beginner's tutorial for writing a simple QGIS
% C++ plugin. It is based on a workshop held by Dr. Marco Hugentobler. 
Dans cette section nous fournissons un tutoriel pour d\'ebutant pour l'\'ecriture d'
extension simple en C++ pour QGIS. Il est bas\'e sur le workshop r\'ealis\'e par Dr. 
Marco Hugentobler.

% QGIS C++ plugins are dynamically linked libraries (.so or .dll). They are
% linked to QGIS at runtime when requested in the plugin manager and extend the
% functionality of QGIS. They have access to the QGIS GUI and can be devided
% into core and external plugins.
Les extensions C++ de QGIS sont des biblioth\`eques dynamiquement li\'ees (.so ou .dll). 
Elles sont li\'ees \`a QGIS pendant son fonctionnement \`a la demande dans le 
gestionnaire d'extension et \'etendent les fonctionnalit\'es de QGIS. Ils ont acc\`es \`a 
l'interface de QGIS et peuvent \^etre divis\'es en extensions principales et externes.

% Technically the QGIS plugin manager looks in the lib/qgis directory for all
% .so files and loads them when it is started. When it is closed they are
% unloaded again, except the ones with a checked box. For newly loaded plugins,
% the \method{classFactory} method creates an instance of the plugin class and
% the \method{initGui} method of the plugin is called to show the GUI elements
% in the plugin menu and toolbar. The \method{unload()} function of the plugin
% is used to remove the allocated GUI elements and the plugin class itself is
% removed using the class destructor. To list the plugins, each plugin must
% have a few external 'C' functions for description and of course the
% \method{classFactory} method.
Techniquement le gestionnaire d'extension de QGIS cherche tous les fichiers .so 
dans le r\'epertoire lib/qgis et les charge lors du  d\'emarrage. Lors de la 
fermeture, ils sont d\'echarg\'es, sauf ceux avec une case coch\'ee. Pour les plugins 
nouvellement charg\'es, la m\'ethode \method{classFactory} du plugin est appel\'ee 
pour afficher l'interface de l'extension dans le menu extension et la barre d'outils. 
La fonction \method{unload()} de l' est utilis\'ee pour enlever les \'el\'ements 
de l'interface allou\'ee et la classe d'extension est enlev\'ee en utilisant le 
destructeur de classe. Pour lister les extensions, chaque extension doit avoir 
quelques fonctions 'C' externes pour la description et bien sur la m\'ethode 
\method{classFactory}.

% \subsection{Why C++ and what about licensing}
\subsection{Pourquoi C++ et quelle licence est utilis\'ee}

% QGIS itself is written in C++, so it also makes sense to write plugins in C++
% as well. It is an object-oriented programming (OOP) language that is viewed
% by many developers as a prefered language for creating large-scale
% applications.
QGIS lui-m\^eme est \'ecrit en C++, il est donc logique d'\'ecrire des extensions en C++.
C'est un langage orient\'e-objet (OOP) qui est consid\'er\'e par beaucoup de 
d\'eveloppeurs comme langage \`a pr\'ef\'erer pour la cr\'eation d'applications de taille 
importantes.

% QGIS C++ plugins use functionalities of libqgis*.so libraries. As they are
% licensed under GNU GPL, QGIS C++ plugins must be licenced under the GPL, too.
% This means you may use your plugins for any purpose and you are not forced to
% publish them. If you do publish them however, they must be published under
% the conditions of the GPL license. 
Les extensions C++ de QGIS utilisent des fonctionnalit\'es des biblioth\`eques 
libqgis*.so. Comme elles sont sous licence GNU GPL, les extensions C++ de QGIS 
doivent \^etre aussi sous cette licence. Cela signifie que vous pouvez utiliser 
vos extensions comme vous le souhaitez et que vous n'\^etes pas oblig\'e de les 
publier. Cependant si vous les publiez, vous devez les publier sous les 
conditions de la licence GPL.

% \subsection{Programming a QGIS C++ Plugin in four steps}
\subsection{Programmer une extension en C++ pour QGIS en quatre \'etapes}

% The example plugin is a point converter plugin and intentionally kept simple. 
% The plugin searches the active vector layer in QGIS, converts all vertices of
% the layer features to point features keeping the attributes and finally
% writes the point features into a delimited text file. The new layer can then
% be loaded into QGIS using the delimited text plugin (see Section
% \ref{label_dltext}).
L'extension d'exemple est une extension de conversion de point et reste simple 
intentionnellement. L'extension recherche la couche vecteur active dans QGIS, 
convertit tous les sommets des g\'eom\'etries de la couche en objets ponctuels en 
gardant les attributs et enfin \'ecrit les objets ponctuels dans un fichier csv. 
La nouvelle couche peut alors \^etre charg\'ee dans QGIS en utilisant l'extension de 
d\'elimitation de texte (voir section~\ref{label_dltext}).

% \minisec{Step 1: Make the plugin manager recognise the plugin}
\minisec{\'Etape 1: Permettre au gestionnaire d'extensions de reconnaitre l'extension}

% As a first step we create the \filename{QgsPointConverter.h} and
% \filename{QgsPointConverter.cpp} files. Then we add virtual methods inherited
% from QgisPlugin (but leave them empty for now), create necessary external 'C'
% methods and a .pro file, which is a Qt mechanism to easily create Makefiles.
% Then we compile the sources, move the compiled library into the plugin folder
% and load it in the QGIS plugin manager.
Pour commencer nous cr\'eons les fichiers \filename{QgsPointConverter.h} et 
\filename{QgsPointConverter.cpp}. Puis nous ajoutons les m\'ethodes h\'erit\'ees de 
QgisPlugin (mais les laissons vides pour l'instant), cr\'eons les m\'ethodes 'C' 
externes n\'ecessaires et un fichier .pro, qui est un m\'ecanisme de Qt pour cr\'eer 
facilement un Makefiles. Puis nous compilons les sources, d\'epla\c{c}ons la 
biblioth\`eque compil\'ee dans le r\'epertoire des extensions et le chargeons dans le 
gestionnaire d'extensions de QGIS.

% \textbf{a) Create new pointconverter.pro file and add}:
\textbf{a) Cr\'eez un nouveau fichier pointconverter.pro et ajoutez}:

\begin{verbatim}
#base directory of the qgis installation
QGIS_DIR = /home/marco/src/qgis

TEMPLATE = lib
CONFIG = qt
QT += xml qt3support
unix:LIBS += -L/$$QGIS_DIR/lib -lqgis_core -lqgis_gui
INCLUDEPATH += $$QGIS_DIR/src/ui $$QGIS_DIR/src/plugins  $$QGIS_DIR/src/gui \
        $$QGIS_DIR/src/raster $$QGIS_DIR/src/core $$QGIS_DIR 
SOURCES = qgspointconverterplugin.cpp
HEADERS = qgspointconverterplugin.h
DEST = pointconverterplugin.so
DEFINES += GUI_EXPORT= CORE_EXPORT=
\end{verbatim}

% \textbf{b) Create new qgspointconverterplugin.h file and add}:
\textbf{b) Cr\'eez un nouveau fichier qgspointconverterplugin.h et ajoutez}:

\begin{verbatim}
#ifndef QGSPOINTCONVERTERPLUGIN_H
#define QGSPOINTCONVERTERPLUGIN_H

#include "qgisplugin.h"

/**A plugin that converts vector layers to delimited text point files.
 The vertices of polygon/line type layers are converted to point features*/
class QgsPointConverterPlugin: public QgisPlugin
{
  public:
  QgsPointConverterPlugin(QgisInterface* iface);
  ~QgsPointConverterPlugin();
  void initGui();
  void unload();
  
  private:
  QgisInterface* mIface;
};
#endif
\end{verbatim}

% \textbf{c) Create new qgspointconverterplugin.cpp file and add}:
\textbf{b) Cr\'eez un nouveau fichier qgspointconverterplugin.cpp et ajoutez}:

\begin{verbatim}
#include "qgspointconverterplugin.h"

#ifdef WIN32
#define QGISEXTERN extern "C" __declspec( dllexport )
#else
#define QGISEXTERN extern "C"
#endif

QgsPointConverterPlugin::QgsPointConverterPlugin(QgisInterface* iface): mIface(iface)
{
}

QgsPointConverterPlugin::~QgsPointConverterPlugin()
{
}

void QgsPointConverterPlugin::initGui()
{
}

void QgsPointConverterPlugin::unload()
{
}

QGISEXTERN QgisPlugin* classFactory(QgisInterface* iface)
{
  return new QgsPointConverterPlugin(iface);
}

QGISEXTERN QString name()
{
  return "point converter plugin";
}

QGISEXTERN QString description()
{
  return "A plugin that converts vector layers to delimited text point files";
}

QGISEXTERN QString version()
{
  return "0.00001";
}

// Return the type (either UI or MapLayer plugin)
QGISEXTERN int type()
{
  return QgisPlugin::UI;
}

// Delete ourself
QGISEXTERN void unload(QgisPlugin* theQgsPointConverterPluginPointer)
{
  delete theQgsPointConverterPluginPointer;
}
\end{verbatim}

% \minisec{Step 2: Create an icon, a button and a menu for the plugin}
\minisec{\'Etape 2 : Cr\'eer un ic\^one et un menu pour le plugin}

% This step includes adding a pointer to the QgisInterface object in the plugin
% class. Then we create a QAction and a callback function (slot), add it to the
% QGIS GUI using QgisIface::addToolBarIcon() and QgisIface::addPluginToMenu()
% and finally remove the QAction in the \method{unload()} method.
Cette \'etape inclut l'ajout d'un pointeur \`a l'objet QgisInterface dans la classe 
d'extension. Puis nous cr\'eons une fonction QAction et un callback (slot), 
l'ajoutons \`a l'interface de QGIS en utilisant  QgisIface::addToolBarIcon() et
 QgisIface::addPluginToMenu() et enfin nous enlevons QAction dans la m\'ethode 
 \method{unload()}.

% \textbf{d) Open qgspointconverterplugin.h again and extend existing content to}:
\textbf{d) Ouvrez de nouveau le fichier qgspointconverterplugin.h et ajoutez au 
contenu existant} :

\begin{verbatim}
#ifndef QGSPOINTCONVERTERPLUGIN_H
#define QGSPOINTCONVERTERPLUGIN_H

#include "qgisplugin.h"
#include <QObject>

class QAction;

/**A plugin that converts vector layers to delimited text point files.
 The vertices of polygon/line type layers are converted to point features*/
class QgsPointConverterPlugin: public QObject, public QgisPlugin
{
  Q_OBJECT

 public:
  QgsPointConverterPlugin(QgisInterface* iface);
  ~QgsPointConverterPlugin();
  void initGui();
  void unload();
  
 private:
  QgisInterface* mIface;
  QAction* mAction;
  
   private slots:
   void convertToPoint();
};

#endif
\end{verbatim}

% \textbf{e) Open qgspointconverterplugin.cpp again and extend existing content to}:
\textbf{e) Ouvrez de nouveau le fichier qgspointconverterplugin.cpp et ajoutez 
au contenu existant} :

\begin{verbatim}
#include "qgspointconverterplugin.h"
#include "qgisinterface.h"
#include <QAction>

#ifdef WIN32
#define QGISEXTERN extern "C" __declspec( dllexport )
#else
#define QGISEXTERN extern "C"
#endif

QgsPointConverterPlugin::QgsPointConverterPlugin(QgisInterface* iface): \
    mIface(iface), mAction(0)
{

}

QgsPointConverterPlugin::~QgsPointConverterPlugin()
{

}

void QgsPointConverterPlugin::initGui()
{
  mAction = new QAction(tr("&Convert to point"), this);
  connect(mAction, SIGNAL(activated()), this, SLOT(convertToPoint()));
  mIface->addToolBarIcon(mAction);
  mIface->addPluginToMenu(tr("&Convert to point"), mAction);
}

void QgsPointConverterPlugin::unload()
{
  mIface->removeToolBarIcon(mAction);
  mIface->removePluginMenu(tr("&Convert to point"), mAction);
  delete mAction;
}

void QgsPointConverterPlugin::convertToPoint()
{
  qWarning("in method convertToPoint");
}

QGISEXTERN QgisPlugin* classFactory(QgisInterface* iface)
{
  return new QgsPointConverterPlugin(iface);
}

QGISEXTERN QString name()
{
  return "point converter plugin";
}

QGISEXTERN QString description()
{
  return "A plugin that converts vector layers to delimited text point files";
}

QGISEXTERN QString version()
{
  return "0.00001";
}

// Return the type (either UI or MapLayer plugin)
QGISEXTERN int type()
{
  return QgisPlugin::UI;
}

// Delete ourself
QGISEXTERN void unload(QgisPlugin* theQgsPointConverterPluginPointer)
{
  delete theQgsPointConverterPluginPointer;
}
\end{verbatim}


% \minisec{Step 3: Read point features from the active layer and write to text file}
\minisec{\'Etape 3: Lire des g\'eom\'etries ponctuelles \`a partir de la couche active et \'ecrire dans un fichier texte}

% To read the point features from the active layer we need to query the current
% layer and the location for the new text file. Then we iterate through all
% features of the current layer, convert the geometries (vertices) to points,
% open a new file and use QTextStream to write the x- and y-coordinates
% into it.
Pour lire la g\'eom\'etrie ponctuelle \`a partir de la couche active nous devons 
demander la couche en court et la localisation du fichier texte. Puis nous 
it\'erons sur toutes les g\'eom\'etries de la couche actuelle, convertissons les 
g\'eom\'etries (sommets) en points, ouvrons un nouveau fichier et utilisons 
QTextStream pour \'ecrire les coordonn\'ees x et y \`a l'int\'erieur.

% \textbf{f) Open qgspointconverterplugin.h again and extend existing content to}
\textbf{f) Ouvrez de nouveau qgspointconverterplugin.h et ajoutez ceci au contenu existant}

\begin{verbatim}
class QgsGeometry;
class QTextStream;

private:

void convertPoint(QgsGeometry* geom, const QString& attributeString, \
    QTextStream& stream) const;
void convertMultiPoint(QgsGeometry* geom, const QString& attributeString, \
    QTextStream& stream) const;
void convertLineString(QgsGeometry* geom, const QString& attributeString, \
    QTextStream& stream) const;
void convertMultiLineString(QgsGeometry* geom, const QString& attributeString, \
    QTextStream& stream) const;
void convertPolygon(QgsGeometry* geom, const QString& attributeString, \
    QTextStream& stream) const;
void convertMultiPolygon(QgsGeometry* geom, const QString& attributeString, \
    QTextStream& stream) const;
\end{verbatim}

% \textbf{g) Open qgspointconverterplugin.cpp again and extend existing content to}:
\textbf{g) Ouvrez qgspointconverterplugin.cpp et rajoutez ceci au contenu existant} :

\begin{verbatim}
#include "qgsgeometry.h"
#include "qgsvectordataprovider.h"
#include "qgsvectorlayer.h"
#include <QFileDialog>
#include <QMessageBox>
#include <QTextStream>

void QgsPointConverterPlugin::convertToPoint()
{
  qWarning("in method convertToPoint");
  QgsMapLayer* theMapLayer = mIface->activeLayer();
  if(!theMapLayer)
    {
      QMessageBox::information(0, tr("no active layer"), \
      tr("this plugin needs an active point vector layer to make conversions \ 
          to points"), QMessageBox::Ok);
      return;
    }
  QgsVectorLayer* theVectorLayer = dynamic_cast<QgsVectorLayer*>(theMapLayer);
  if(!theVectorLayer)
    {
      QMessageBox::information(0, tr("no vector layer"), \
      tr("this plugin needs an active point vector layer to make conversions \
          to points"), QMessageBox::Ok);
      return;
    }
  
  QString fileName = QFileDialog::getSaveFileName();
  if(!fileName.isNull())
    {
      qWarning("The selected filename is: " + fileName);
      QFile f(fileName);
      if(!f.open(QIODevice::WriteOnly))
      {
 QMessageBox::information(0, "error", "Could not open file", QMessageBox::Ok);
 return;
      }
      QTextStream theTextStream(&f);
      theTextStream.setRealNumberNotation(QTextStream::FixedNotation);

      QgsFeature currentFeature;
      QgsGeometry* currentGeometry = 0;

      QgsVectorDataProvider* provider = theVectorLayer->dataProvider();
      if(!provider)
      {
          return;
      }

      theVectorLayer->select(provider->attributeIndexes(), \
      theVectorLayer->extent(), true, false);

      //write header
      theTextStream << "x,y";
      theTextStream << endl;

      while(theVectorLayer->nextFeature(currentFeature))
      {
  QString featureAttributesString;
      
        currentGeometry = currentFeature.geometry();
        if(!currentGeometry)
        {
            continue;
        }

        switch(currentGeometry->wkbType())
        {
            case QGis::WKBPoint:
            case QGis::WKBPoint25D:
                convertPoint(currentGeometry, featureAttributesString, \
  theTextStream);
                break;

            case QGis::WKBMultiPoint:
            case QGis::WKBMultiPoint25D:
                convertMultiPoint(currentGeometry, featureAttributesString, \
  theTextStream);
                break;

            case QGis::WKBLineString:
            case QGis::WKBLineString25D:
                convertLineString(currentGeometry, featureAttributesString, \
  theTextStream);
                break;

            case QGis::WKBMultiLineString:
            case QGis::WKBMultiLineString25D:
                convertMultiLineString(currentGeometry, featureAttributesString \
  theTextStream);
                break;

            case QGis::WKBPolygon:
            case QGis::WKBPolygon25D:
                convertPolygon(currentGeometry, featureAttributesString, \
  theTextStream);
                break;

            case QGis::WKBMultiPolygon:
            case QGis::WKBMultiPolygon25D:
                convertMultiPolygon(currentGeometry, featureAttributesString, \
  theTextStream);
                break;
        }
      }
    }
}

//geometry converter functions
void QgsPointConverterPlugin::convertPoint(QgsGeometry* geom, const QString& \
attributeString, QTextStream& stream) const
{
    QgsPoint p = geom->asPoint();
    stream << p.x() << "," << p.y();
    stream << endl;
}

void QgsPointConverterPlugin::convertMultiPoint(QgsGeometry* geom, const QString& \
attributeString, QTextStream& stream) const
{
    QgsMultiPoint mp = geom->asMultiPoint();
    QgsMultiPoint::const_iterator it = mp.constBegin();
    for(; it != mp.constEnd(); ++it)
    {
        stream << (*it).x() << "," << (*it).y();
        stream << endl;
    }
}

void QgsPointConverterPlugin::convertLineString(QgsGeometry* geom, const QString& \
attributeString, QTextStream& stream) const
{
    QgsPolyline line = geom->asPolyline();
    QgsPolyline::const_iterator it = line.constBegin();
    for(; it != line.constEnd(); ++it)
    {
        stream << (*it).x() << "," << (*it).y();
        stream << endl;
    }
}

void QgsPointConverterPlugin::convertMultiLineString(QgsGeometry* geom, const QString& \
attributeString, QTextStream& stream) const
{
    QgsMultiPolyline ml = geom->asMultiPolyline();
    QgsMultiPolyline::const_iterator lineIt = ml.constBegin();
    for(; lineIt != ml.constEnd(); ++lineIt)
    {
        QgsPolyline currentPolyline = *lineIt;
        QgsPolyline::const_iterator vertexIt = currentPolyline.constBegin();
        for(; vertexIt != currentPolyline.constEnd(); ++vertexIt)
        {
            stream << (*vertexIt).x() << "," << (*vertexIt).y();
            stream << endl;
        }
    }
}

void QgsPointConverterPlugin::convertPolygon(QgsGeometry* geom, const QString& \
attributeString, QTextStream& stream) const
{
    QgsPolygon polygon = geom->asPolygon();
    QgsPolygon::const_iterator it = polygon.constBegin();
    for(; it != polygon.constEnd(); ++it)
    {
        QgsPolyline currentRing = *it;
        QgsPolyline::const_iterator vertexIt = currentRing.constBegin();
        for(; vertexIt != currentRing.constEnd(); ++vertexIt)
        {
            stream << (*vertexIt).x() << "," << (*vertexIt).y();
            stream << endl;
        }
    }
}

void QgsPointConverterPlugin::convertMultiPolygon(QgsGeometry* geom, const QString& \
attributeString, QTextStream& stream) const
{
    QgsMultiPolygon mp = geom->asMultiPolygon();
    QgsMultiPolygon::const_iterator polyIt = mp.constBegin();
    for(; polyIt != mp.constEnd(); ++polyIt)
    {
        QgsPolygon currentPolygon = *polyIt;
        QgsPolygon::const_iterator ringIt = currentPolygon.constBegin();
        for(; ringIt != currentPolygon.constEnd(); ++ringIt)
        {
            QgsPolyline currentPolyline = *ringIt;
            QgsPolyline::const_iterator vertexIt = currentPolyline.constBegin();
            for(; vertexIt != currentPolyline.constEnd(); ++vertexIt)
            {
                stream << (*vertexIt).x() << "," << (*vertexIt).y();
                stream << endl;
            }
        }
    }
}
\end{verbatim}

% \minisec{Step 4: Copy the feature attributes to the text file}
\minisec{\'Etape 4 : Copier les attributs des g\'eom\'etries dans le fichier texte}

% At the end we extract the attributes from the active layer using
% QgsVectorDataProvider::fieldNameMap(). For each feature we extract the field
% values using QgsFeature::attributeMap() and add the contents comma separated
% behind the x- and y-coordinates for each new point feature. For this step
% there is no need for any furter change in \filename{qgspointconverterplugin.h} 
Finalement, nous r\'ecuperons les attributs de la couche active en utilisant 
QgsVectorDataProvider::fieldNameMap(). Pour chaque g\'eom\'etrie nous r\'ecup\'erons les 
valeurs des champs en utilisant QgsFeature::attributeMap() et ajoutons le 
contenu s\'epar\'e d'une virgule apr\`es les coordonn\'ees x et y pour chaque nouvel 
objet ponctuel. Pour cette \'etape il n'y a pas besoin de changement suppl\'ementaire
 dans le fichier \filename{qgspointconverterplugin.h}.

% \textbf{h) Open qgspointconverterplugin.cpp again and extend existing content
% to}:
\textbf{h) Ouvrez de nouveau le fichier qgspointconverterplugin.cpp et rajoutez 
ceci au contenu}:

\begin{verbatim} 
#include "qgspointconverterplugin.h"
#include "qgisinterface.h"
#include "qgsgeometry.h"
#include "qgsvectordataprovider.h"
#include "qgsvectorlayer.h"
#include <QAction>
#include <QFileDialog>
#include <QMessageBox>
#include <QTextStream>

#ifdef WIN32
#define QGISEXTERN extern "C" __declspec( dllexport )
#else
#define QGISEXTERN extern "C"
#endif

QgsPointConverterPlugin::QgsPointConverterPlugin(QgisInterface* iface): \
mIface(iface), mAction(0)
{

}

QgsPointConverterPlugin::~QgsPointConverterPlugin()
{

}

void QgsPointConverterPlugin::initGui()
{
  mAction = new QAction(tr("&Convert to point"), this);
  connect(mAction, SIGNAL(activated()), this, SLOT(convertToPoint()));
  mIface->addToolBarIcon(mAction);
  mIface->addPluginToMenu(tr("&Convert to point"), mAction);
}

void QgsPointConverterPlugin::unload()
{
  mIface->removeToolBarIcon(mAction);
  mIface->removePluginMenu(tr("&Convert to point"), mAction);
  delete mAction;
}

void QgsPointConverterPlugin::convertToPoint()
{
  qWarning("in method convertToPoint");
  QgsMapLayer* theMapLayer = mIface->activeLayer();
  if(!theMapLayer)
    {
      QMessageBox::information(0, tr("no active layer"), \
      tr("this plugin needs an active point vector layer to make conversions \
          to points"), QMessageBox::Ok);
      return;
    }
  QgsVectorLayer* theVectorLayer = dynamic_cast<QgsVectorLayer*>(theMapLayer);
  if(!theVectorLayer)
    {
      QMessageBox::information(0, tr("no vector layer"), \
      tr("this plugin needs an active point vector layer to make conversions \
          to points"), QMessageBox::Ok);
      return;
    }
  
  QString fileName = QFileDialog::getSaveFileName();
  if(!fileName.isNull())
    {
      qWarning("The selected filename is: " + fileName);
      QFile f(fileName);
      if(!f.open(QIODevice::WriteOnly))
      {
	QMessageBox::information(0, "error", "Could not open file", QMessageBox::Ok);
	return;
      }
      QTextStream theTextStream(&f);
      theTextStream.setRealNumberNotation(QTextStream::FixedNotation);

      QgsFeature currentFeature;
      QgsGeometry* currentGeometry = 0;

      QgsVectorDataProvider* provider = theVectorLayer->dataProvider();
      if(!provider)
      {
          return;
      }

      theVectorLayer->select(provider->attributeIndexes(), \
      theVectorLayer->extent(), true, false);

      //write header
      theTextStream << "x,y";
      QMap<QString, int> fieldMap = provider->fieldNameMap();
      //We need the attributes sorted by index.
      //Therefore we insert them in a second map where key / values are exchanged
      QMap<int, QString> sortedFieldMap;
      QMap<QString, int>::const_iterator fieldIt = fieldMap.constBegin();
      for(; fieldIt != fieldMap.constEnd(); ++fieldIt)
      {
        sortedFieldMap.insert(fieldIt.value(), fieldIt.key());
      }

      QMap<int, QString>::const_iterator sortedFieldIt = sortedFieldMap.constBegin();
      for(; sortedFieldIt != sortedFieldMap.constEnd(); ++sortedFieldIt)
      {
          theTextStream << "," << sortedFieldIt.value();
      }

      theTextStream << endl;

      while(theVectorLayer->nextFeature(currentFeature))
      {
        QString featureAttributesString;
         const QgsAttributeMap& map = currentFeature.attributeMap();
         QgsAttributeMap::const_iterator attributeIt = map.constBegin();
         for(; attributeIt != map.constEnd(); ++attributeIt)
         {
            featureAttributesString.append(",");
            featureAttributesString.append(attributeIt.value().toString());
         }


        currentGeometry = currentFeature.geometry();
        if(!currentGeometry)
        {
            continue;
        }

        switch(currentGeometry->wkbType())
        {
            case QGis::WKBPoint:
            case QGis::WKBPoint25D:
                convertPoint(currentGeometry, featureAttributesString, \
		theTextStream);
                break;

            case QGis::WKBMultiPoint:
            case QGis::WKBMultiPoint25D:
                convertMultiPoint(currentGeometry, featureAttributesString, \
		theTextStream);
                break;

            case QGis::WKBLineString:
            case QGis::WKBLineString25D:
                convertLineString(currentGeometry, featureAttributesString, \
		theTextStream);
                break;

            case QGis::WKBMultiLineString:
            case QGis::WKBMultiLineString25D:
                convertMultiLineString(currentGeometry, featureAttributesString \
		theTextStream);
                break;

            case QGis::WKBPolygon:
            case QGis::WKBPolygon25D:
                convertPolygon(currentGeometry, featureAttributesString, \
		theTextStream);
                break;

            case QGis::WKBMultiPolygon:
            case QGis::WKBMultiPolygon25D:
                convertMultiPolygon(currentGeometry, featureAttributesString, \
		theTextStream);
                break;
        }
      }
    }
}

//geometry converter functions
void QgsPointConverterPlugin::convertPoint(QgsGeometry* geom, const QString& \
attributeString, QTextStream& stream) const
{
    QgsPoint p = geom->asPoint();
    stream << p.x() << "," << p.y();
    stream << attributeString;
    stream << endl;
}

void QgsPointConverterPlugin::convertMultiPoint(QgsGeometry* geom, const QString& \
attributeString, QTextStream& stream) const
{
    QgsMultiPoint mp = geom->asMultiPoint();
    QgsMultiPoint::const_iterator it = mp.constBegin();
    for(; it != mp.constEnd(); ++it)
    {
        stream << (*it).x() << "," << (*it).y();
        stream << attributeString;
        stream << endl;
    }
}

void QgsPointConverterPlugin::convertLineString(QgsGeometry* geom, const QString& \
attributeString, QTextStream& stream) const
{
    QgsPolyline line = geom->asPolyline();
    QgsPolyline::const_iterator it = line.constBegin();
    for(; it != line.constEnd(); ++it)
    {
        stream << (*it).x() << "," << (*it).y();
        stream << attributeString;
        stream << endl;
    }
}

void QgsPointConverterPlugin::convertMultiLineString(QgsGeometry* geom, const QString& \
attributeString, QTextStream& stream) const
{
    QgsMultiPolyline ml = geom->asMultiPolyline();
    QgsMultiPolyline::const_iterator lineIt = ml.constBegin();
    for(; lineIt != ml.constEnd(); ++lineIt)
    {
        QgsPolyline currentPolyline = *lineIt;
        QgsPolyline::const_iterator vertexIt = currentPolyline.constBegin();
        for(; vertexIt != currentPolyline.constEnd(); ++vertexIt)
        {
            stream << (*vertexIt).x() << "," << (*vertexIt).y();
            stream << attributeString;
            stream << endl;
        }
    }
}

void QgsPointConverterPlugin::convertPolygon(QgsGeometry* geom, const QString& \
attributeString, QTextStream& stream) const
{
    QgsPolygon polygon = geom->asPolygon();
    QgsPolygon::const_iterator it = polygon.constBegin();
    for(; it != polygon.constEnd(); ++it)
    {
        QgsPolyline currentRing = *it;
        QgsPolyline::const_iterator vertexIt = currentRing.constBegin();
        for(; vertexIt != currentRing.constEnd(); ++vertexIt)
        {
            stream << (*vertexIt).x() << "," << (*vertexIt).y();
            stream << attributeString;
            stream << endl;
        }
    }
}

void QgsPointConverterPlugin::convertMultiPolygon(QgsGeometry* geom, const QString& \
attributeString, QTextStream& stream) const
{
    QgsMultiPolygon mp = geom->asMultiPolygon();
    QgsMultiPolygon::const_iterator polyIt = mp.constBegin();
    for(; polyIt != mp.constEnd(); ++polyIt)
    {
        QgsPolygon currentPolygon = *polyIt;
        QgsPolygon::const_iterator ringIt = currentPolygon.constBegin();
        for(; ringIt != currentPolygon.constEnd(); ++ringIt)
        {
            QgsPolyline currentPolyline = *ringIt;
            QgsPolyline::const_iterator vertexIt = currentPolyline.constBegin();
            for(; vertexIt != currentPolyline.constEnd(); ++vertexIt)
            {
                stream << (*vertexIt).x() << "," << (*vertexIt).y();
                stream << attributeString;
                stream << endl;
            }
        }
    }
}

QGISEXTERN QgisPlugin* classFactory(QgisInterface* iface)
{
  return new QgsPointConverterPlugin(iface);
}

QGISEXTERN QString name()
{
  return "point converter plugin";
}

QGISEXTERN QString description()
{
  return "A plugin that converts vector layers to delimited text point files";
}

QGISEXTERN QString version()
{
  return "0.00001";
}

// Return the type (either UI or MapLayer plugin)
QGISEXTERN int type()
{
  return QgisPlugin::UI;
}

// Delete ourself
QGISEXTERN void unload(QgisPlugin* theQgsPointConverterPluginPointer)
{
  delete theQgsPointConverterPluginPointer;
}

\end{verbatim}

% \subsection{Further information}
\subsection{Plus d'informations}

% As you can see, you need information from different sources to write QGIS C++
% plugins. Plugin writers need to know C++, the QGIS plugin interface as
% well as Qt4 classes and tools. At the beginning it is best to learn from
% examples and copy the mechanism of existing plugins. 
Comme vous pouvez le voir, vous avez besoin d'informations \`a partir de diff\'erentes
sources pour \'ecrire des extensions C++ pour QGIS. Les d\'eveloppeurs d'extensions doivent 
conna\^itre le C++, l'interface des extensions de QGIS ainsi que les classes et 
outils Qt4. Au d\'ebut il est plus enrichissant d'apprendre \`a partir des exemples 
et de copier les m\'ecanismes d'extensions existants.

% There is a a collection of online documentation that may be usefull for
% QGIS C++ programers:
Il y a un certain nombre de documentations qui peut \^etre utile pour les programmeurs
C++ pour QGIS :

\begin{itemize}
% \item QGIS Plugin Debugging: \url{http://wiki.qgis.org/qgiswiki/DebuggingPlugins}
\item Debuguer des extensions de QGIS : \url{http://wiki.qgis.org/qgiswiki/DebuggingPlugins}
% \item QGIS API Documentation: \url{http://svn.qgis.org/api_doc/html/}
\item Documentation de l'API de QGIS : \url{http://svn.qgis.org/api_doc/html/}
% \item Qt documentation: \url{http://doc.trolltech.com/4.3/index.html}
\item Documentation de Qt : \url{http://doc.trolltech.com/4.3/index.html}
\end{itemize}
