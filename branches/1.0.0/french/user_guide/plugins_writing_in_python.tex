% vim: set textwidth=78 autoindent:

% \section{Writing a QGIS Plugin in Python}
\section{\'Ecrire une extension en Python pour QGIS}

% when the revision of a section has been finalized,
% comment out the following line:
% \updatedisclaimer

% In this section you find a beginner's tutorial for writing a QGIS Python
% plugins. It is based on the workshop "Extending the Functionality of QGIS
% with Python Plugins" held at FOSS4G 2008 by Dr. Marco Hugentobler, Dr. Horst
% D\"uster and Tim Sutton.
Dans cette section vous trouverez un cours pour d\'ebutant pour \'ecrire des
extensions Python Pour QGIS. Il est bas\'e sur le workshop "\'Etendre les
fonctionnalit\'es de QGIS avec des extensions en Python" r\'ealis\'e lors du FOSS4G 2008
par  Dr. Marco Hugentobler, Dr. Horst D\"uster et Tim Sutton.

% Apart from writing a QGIS Python plugin, it is also possible to use PyQGIS
% from a python command line console which is mainly interesting for debugging
% or to write standalone applications in Python with their own user interfaces
% using the functionality of the QGIS core library.
En plus d'\'ecrire des extensions en python pour QGIS, il est \'egalement possible
d'utiliser PyQGIS dans une console de ligne de commande python qui est
principalement utilis\'ee pour d\'eboguer ou pour \'ecrire des applications
ind\'ependantes en Python avec leurs propres interfaces en utilisant la
biblioth\`eque principale de QGIS.

% \subsection{Why Python and what about licensing}
\subsection{Pourquoi Python et \`a propos de la licence}

% Python is a scripting language which was designed with the goal of being easy
% to program. It has a mechanism that automatically releases memory that is no
% longer used (garbagge collector). A further advantage is that many programs
% that are written in C++ or Java offer the possibility to write extensions in
% Python, e.g. OpenOffice or Gimp. Therefore it is a good investment of time to
% learn the Python language.
Python est un langage de script qui a \'et\'e con\c{c}u afin d'\^etre facile \`a \'ecrire. Il
a un m\'ecanisme qui nettoie automatiquement la m\'emoire qui n'est plus utilis\'ee
(collecteur de d\'echet). Un avantage suppl\'ementaire est que plusieurs programmes
\'ecrits en C++ ou Java offrent la possibilit\'e d'\'ecrire des extensions en Python,
comme OpenOffice.org ou GIMP. C'est donc un bon investissement d'apprendre le
langage Python.

% PyQGIS plugins use functionality of libqgis\_core.so and libqgis\_gui.so. As
% both are licensed under GNU GPL, QGIS Python plugins must be licenced under
% the GPL, too. This means you may use your plugins for any purpose and you are
% not forced to publish them. If you do publish them however, they must be
% published under the conditions of the GPL license. 
Les extensions PyGQIS utilisent les fonctionnalit\'es de libqgis\_core.so et
libqgis\_gui.so. Comme les deux sont publi\'es sous licence GPL, les extensions
Python pour QGIS doivent \^etre publi\'es sous licence GPL \'egalement. Cela signifie
que vous pouvez utiliser votre extension dans n'importe quel but et vous n'\^etes pas
oblig\'e de les publier. Toutefois, si vous voulez les publier, ils doivent l'\^etre
dans les conditions de la licence GPL.

% \subsection{What needs to be installed to get started}
\subsection{ce que vous avez besoin d'installer pour d\'emarrer}

% On the lab computers, everything for the workshop is already installed. If
% you program Python plugins at home, you will need the following libraries and
% programs:
Sur les ordinateurs du labs, tout ce qui est n\'ecessaire est d\'ej\`a install\'e. Si
vous programmez chez vous, vous aurez besoin des biblioth\`eques et programmes
suivants :

\begin{itemize}
\item QGIS ;
\item Python ;
\item Qt ;
\item PyQT ;
% \item PyQt development tools
\item Outils de d\'eveloppement PyQt.
\end{itemize}

% If you use Linux, there are binary packages for all major distributions. For
% Windows, the PyQt installer already contains Qt, PyQt and the PyQt
% development tools.
Si vous utilisez un syst\`eme Linux ou \'equivalent, il existe des binaires pour
toutes les distributions majeures. Pour les utilisateurs de Windows,
l'installateur PyQT contient d\'ej\`a Qt, PyQT et les outils de d\'eveloppement de
PyQT.

%\subsection{Programming a simple PyQGIS Plugin in four steps}\label{subsec:pyfoursteps}
\subsection{Programmer une extension PyQGIS en quatre \'etapes}\label{subsec:pyfoursteps}

% The example plugin is intentionally kept simple. It adds a button to the menu
% bar of QGIS. If the button is clicked, a file dialog appears where the user
% may load a shape file.
Notre exemple d'extension restera intentionnellement simple. Il ajoute un bouton \`a
la barre de menu de QGIS. Si le bouton est cliqu\'e, une bo\^ite de dialogue
apparait dans laquelle un utilisateur peut charger un fichier shape.

% For each python plugin, a dedicated folder that contains the plugin files
% needs to be created. By default, QGIS looks for plugins in
% two locations: \$QGIS\_DIR/share/qgis/python/plugins and
% \$HOME/.qgis/python/plugins. Note that plugins installed in the latter
% location are only visible for one user.
Pour chaque extension Python, un r\'epertoire d\'edi\'e qui contient les fichiers du
extensions est n\'ecessaire. Par d\'efaut, QGIS cherche des extensions dans
\$QGIS\_DIR/share/qgis/python/plugins et \$HOME/.qgis/python/plugins.
Remarquez que les extensions install\'es dans ce dernier  sont seulement visible
par l'utilisateur.

% \minisec{Step 1: Make the plugin manager recognise the plugin}
\minisec{\'Etape 1 : reconnaissance d'une extension par le gestionnaire d'extension}

% Each Python plugin is contained in its own directory. When QGIS starts up it
% will scan each OS specific subdirectory and initialize any plugins it finds.
Chaque extension Python est contenu dans son propre r\'epertoire. Lors de d\'emarrage
de QGIS celui-ci parcourra chaque sous-r\'epertoire sp\'ecifique au syst\`eme et
initialisera toutes les extensions qu'il trouvera.

\begin{itemize}
% \item \nix{Linux and other unices}:\\
\item \nix{Linux et autre UNIX} : \\
./share/qgis/python/plugins \\
/home/\$USERNAME/.qgis/python/plugins
\item \osx{Mac OS X}:\\
./Contents/MacOS/share/qgis/python/plugins \\
/Users/\$USERNAME/.qgis/python/plugins
\item \win{Windows}:\\
C:\textbackslash Program Files\textbackslash QGIS\textbackslash python\textbackslash plugins \\
C:\textbackslash Documents and Settings\textbackslash\$USERNAME\textbackslash .qgis\textbackslash python\textbackslash plugins

\end{itemize}

% Once that's done, the plugin will show up in the
% \dropmenuopttwo{mActionShowPluginManager}{Plugin Manager...}
Une fois r\'ealis\'e, le plugin s'affichera dans
\dropmenuopttwo{mActionShowPluginManager}{gestionnaire de plugin...}

% \begin{Tip}\caption{\textsc{Two QGIS Python Plugin folders}}
\begin{Astuce}\caption{\textsc{Deux r\'epertoires de plugins Python}}
% \qgistip{There are two directories containing the python plugins.
% \$QGIS\_DIR/share/qgis/python/plugins
% is designed mainly for the core plugins while \$HOME/.qgis/python/plugins for
% easy installation of the external plugins. Plugins in the home location are
% only visible for one user but also mask the core plugins with the same name,
% what can be used to provide main plugin updates
\qgistip{Il y a deux r\'epertoires contenant les extensions en python.
\$QGIS\_DIR/share/qgis/python/plugins a \'et\'e con\c{c}u principalement pour les
extensions principales tandis que \$HOME/.qgis/python/plugins pour les extensions
seulement visibles par l'utilisateur, mais aussi masque les extensions principales de
m\^eme nom, ce qui peut \^etre pratique pour les mettre \`a jour.
}
\end{Astuce}

% To provide the neccessary information for QGIS, the plugin needs to implement
% the methods \method{name()}, \method{description()}, \method{version()},
% \method{qgisMinimumVersion()} and \method{authorName()} which return
% descriptive strings. The \method{qgisMinimumVersion()} should return a simple
% form, for example ``1.0``. A plugin also needs a method
% \method{classFactory(QgisInterface)} which is called by the plugin manager to
% create an instance of the plugin. The argument of type QGisInterface is used
% by the plugin to access functions of the QGIS instance. We are going to work
% with this object in step 2.
Pour fournir les informations n\'ecessaires pour QGIS, le plugin n\'ecessite
d'impl\'ementer les m\'ethodes \method{name()}, \method{description()} et
\method{version()} qui renvoient les cha\^ines descriptives.
\method{qgisMinimumVersion()} doit renvoyer une forme simple, par exemple
``1.0``. Une extension n\'ecessite \'egalement une m\'ethode
\method{classFactory(QgisInterface)} qui est appel\'ee par le gestionnaire d'extension
 pour cr\'eer une instance de l'extension. L'argument de type QGisInterface est
utilis\'e par l'extension pour acc\'eder aux fonctions de l'instance QGIS. Nous allons
travailler avec cet objet \`a l'\'etape 2.

% Note that, in contrast to other programing languages, indention is very
% important. The Python interpreter throws an error if it is not correct.
Notez que, contrairement aux autres langages de programmation, l'indentation est
tr\`es importante. L'interpr\'eteur Python renvoie une erreur si elle n'est pas
correcte.

% For our plugin we create the plugin folder 'foss4g\_plugin' in
% \filename{\$HOME/.qgis/python/plugins}. Then we add two new textfiles into
% this folder, \filename{foss4gplugin.py} and \filename{\_\_init\_\_.py}.
Pour nos plugins nous  cr\'eons un r\'epertoire de plugin 'foss4g\_plugin' dans
\filename{\$HOME/.qgis/python/plugins}. Puis nous ajoutons deux nouveaux
fichiers textes dans ce r\'epertoire \filename{foss4gplugin.py} et
\filename{\_\_init\_\_.py}.

% The file \filename{foss4gplugin.py} contains the plugin class:
Le fichier \filename{foss4gplugin.py} contient la classe de l'extension :

\begin{verbatim}
# -*- coding: utf-8 -*-
# Import des biblioth\`eques PyQt et QGIS
from PyQt4.QtCore import *
from PyQt4.QtGui import *
from qgis.core import *
# Initialisation des ressources Qt \`a partir du fichier resources.py
import resources

class FOSS4GPlugin:

def __init__(self, iface):
# Sauve la r\'ef\'erence \`a l'interface QGIS
  self.iface = iface

def initGui(self):
  print 'Initialising GUI'

def unload(self):
  print 'Unloading plugin'
\end{verbatim}

% The file \filename{\_\_init\_\_.py} contains the methods \method{name()},
% \method{description()}, \method{version()}, \method{qgisMinimumVersion()}
% and \method{authorName()} and \method{classFactory}. As
% we are creating a new instance of the plugin class, we need to import the
% code of this class:
Le fichier \filename{\_\_init\_\_.py} contient les m\'ethodes \method{name()},
\method{description()}, \method{version()}, \method{qgisMinimumVersion()}
et \method{authorName()} \'evoqu\'es plus haut. Comme nous somme en train de cr\'eer
une nouvelle instance de la classe plugin (extension)s, nous devons importer le code de cette
classe :

\begin{verbatim}
# -*- coding: utf-8 -*-
from foss4gplugin import FOSS4GPlugin
def name():
  return "FOSS4G example"
def description():
  return "A simple example plugin to load shapefiles"
def version():
  return "0.1"
def qgisMinimumVersion():
  return "1.0"
def authorName():
  return "John Developer"
def classFactory(iface):
  return FOSS4GPlugin(iface)
\end{verbatim}

% At this point the plugin already the neccessary infrastructure to appear in
% the QGIS \dropmenuopttwo{mActionShowPluginManager}{Plugin Manager...} to be
% loaded or unloaded.
Maintenant l'extension poss\`ede l'infrastructure n\'ecessaire pour appara\^itre dans le
\dropmenuopttwo{mActionShowPluginManager}{gestionnaire d'extension} QGIS et \^etre
charg\'e/d\'echarg\'e.

% \minisec{Step 2: Create an Icon for the plugin}
\minisec{\'Etape 2 : Cr\'eer un ic\^one pour le plugin}

% To make the icon graphic available for our program, we need a so-called
% resource file. In the resource file, the graphic is contained in hexadecimal
% notation. Fortunately, we don't need to care about its representation because
% we use the pyrcc compiler, a tool that reads the file
% \filename{resources.qrc} and creates a resource file. 
Pour que votre ic\^one graphique soit disponible dans votre programme, nous avons
besoin d'un fichier ressource. Dans ce fichier ressource, le graphique est
contenu sous forme hexad\'ecimale. Heureusement, nous n'avons pas \`a nous occuper de
sa repr\'esentation parce que nous utilisons le compilateur pyrcc, un outil qui
lit le fichier \filename{resources.qrc} et cr\'e\'e un fichier ressource.

% The file \filename{foss4g.png} and the \filename{resources.qrc} we use in
% this little workshop can be downloaded from
% \url{http://karlinapp.ethz.ch/python\_foss4g}. Move these 2 files into the
% directory of the example plugin
% \filename{\$HOME/.qgis/python/plugins/foss4g\_plugin} and enter there: pyrcc4
% -o resources.py resources.qrc.
Le fichier \filename{foss4g.png} et le ficher \filename{resources.qrc} peuvent
\^etre t\'el\'echarg\'e \`a partir de \url{http://karlinapp.ethz.ch/python\_foss4g}.
D\'eplacez ces fichiers dans le r\'epertoire de l'extension exemple
\filename{\$HOME/.qgis/python/plugins/foss4g\_plugin} et entrez : pyrcc4 -o
ressources.py ressources.qrc.

% \minisec{Step 3: Add a button and a menu}
\minisec{\'Etape 3 : ajouter un bouton au menu}

% In this section, we implement the content of the methods \method{initGui()}
% and \method{unload()}. We need an instance of the class \classname{QAction}
% that executes the \method{run()} method of the plugin. With the action object,
% we are then able to generate the menu entry and the button:
Dans cette partie, nous allons impl\'ementer le contenu des m\'ethodes
\method{initGui()} et \method{unload()}. Nous avons besoins d'une instance de la
classe \classname{QAction} qui ex\'ecute la m\'ethode \method{run()} de l'extension. Avec
l'objet action, nous somme alors capable de g\'en\'erer l'entr\'ee du menu et le
bouton :

\begin{verbatim}
import resources

  def initGui(self):
    # Cr\'eer une action qui d\'emmarera la configuration du plugin
    self.action = QAction(QIcon(":/plugins/foss4g_plugin/foss4g.png"), "FOSS4G
plugin",self.iface.getMainWindow())
    # Connecter l'action \`a la m\'ethode run
    QObject.connect(self.action, SIGNAL("activated()"), self.run)

    # Ajoutez le bouton de la barre d'outil et l'entr\'ee du menu
    self.iface.addToolBarIcon(self.action)
    self.iface.addPluginMenu("FOSS-GIS plugin...", self.action)

    def unload(self):
    # Supprime les entr\'ees des menu et de l'ic\^one
    self.iface.removePluginMenu("FOSSGIS Plugin...", self.action)
    self.iface.removeToolBarIcon(self.action)
\end{verbatim}

% \minisec{Step 4: Load a layer from a shape file}
\minisec{\'Etape 4 : charger une couche \`a partir d'un shapefile}

% In this step we implement the real functionality of the plugin in the
% \method{run()} method. The Qt4 method \method{QFileDialog::getOpenFileName}
% opens a file dialog and returns the path to the chosen file. If the user
% cancels the dialog, the path is a null object, which we test for. We then
% call the method \method{addVectorLayer} of the interface object which loads
% the layer. The method only needs three arguments: the file path, the name of
% the layer that will be shown in the legend and the data provider name. For
% shapefiles, this is 'ogr' because QGIS internally uses the OGR library to
% access shapefiles:
Dans cette \'etape nous allons impl\'ementer les fonctionnalit\'es r\'eelles de l'extension
dans la m\'ethode method{run()} La m\'ethode Qt4
\method{QFileDialog::getOpenFileName} ouvre une bo\^ite de dialogue et renvoie le
chemin du fichier choisit. Si l'utilisateur annule la bo\^ite de dialogue, le
chemin est un objet null, que nous allons tester. Puis nous appelons la m\'ethode
\method{addVectorLayer} de l'objet interface qui charge la couche. La m\'ethode
poss\`ede seulement trois arguments : le chemin du fichier, le nom de la couche
qui sera affich\'ee dans la l\'egende et le nom du fournisseur de donn\'ees. Pour les
Shapefiles, c'est 'ogr' car QGIS utilise en interne la biblioth\`eque OGR pour
acc\'eder aux shapefiles :

\begin{verbatim}
    def run(self):
    fileName = QFileDialog.getOpenFileName(None,QString.fromLocal8Bit("Select a file:"),
 "", "*.shp *.gml")
    if fileName.isNull():
      QMessageBox.information(None, "Cancel", "File selection canceled")
      else:
      vlayer = self.iface.addVectorLayer(fileName, "myLayer", "ogr")
\end{verbatim}


% \subsection{Committing the plugin to repository}
\subsection{Comiter l'extension dans un d\'ep\^ot}

% If you have written a plugin you consider to be useful and you want to share
% with other users you're welcome to upload it to the QGIS User-Contributed
% Repository.
Si vous avez \'ecrit une extension vous pouvez trouver utile et vouloir le partager
avec d'autres utilisateurs, vous \^etes invit\'e \`a le t\'el\'echarger sur le d\'ep\^ot de
contribution des utilisateurs de QGIS.
\begin{itemize}
% \item Prepare a plugin directory containing only necessary files (ensure that
% there is no compiled .pyc files, Subversion .svn directories etc).
\item Pr\'eparer un r\'epertoire d'extensions contenant les fichiers n\'ecessaires
(assurez-vous qu'il n'y ait pas de fichiers .pyc compil\'es, de r\'epertoires .svn
de subversion, etc.)
% \item Make a zip archive of it, including the directory. Be sure the zip file
% name is exactly the same as the directory inside (except the .zip extension of
% course). In other case the Plugin Installer won't be able to relate the
% available plugin with its locally installed instance.
\item Fa\^ites une archive zip, incluant le r\'epertoire. Assurez-vous que le nom du
fichier zip est exactement le m\^eme que le r\'epertoire \`a l'int\'erieur (sauf bien
sur avec l'extension .zip). Sinon l'installateur d'extension ne sera pas capable
de relier l'extension disponible avec celui install\'e localement.
% \item Upload it to the repository: \url{http://pyqgis.org/admin/contributed}
% (you will need to register at first time). Please pay attention when filling
% the form. Especially the Version Number field is often filled wrongly what
% confuses the Plugin Installer and causes false notifications of available
% updates.
\item T\'el\'echargez le dans le d\'ep\^ot : \url{http://pyqgis.org/admin/contributed}
(vous devez vous enregistrer la premi\`ere fois). S'il vous plait, ayez une
attention particuli\`ere lors du remplissage du formulaire. Sp\'ecialement le champ
du num\'ero de version qui est souvent remplie incorrectement ce qui pose
probl\`eme \`a l'installateur d'extension et cause de fausse notification de mise \`a
jour disponible.
\end{itemize}

% \subsection{Further information}
\subsection{Plus d'informations}

% As you can see, you need information from different sources to write PyQGIS
% plugins. Plugin writers need to know Python and the QGIS plugin interface as
% well as the Qt4 classes and tools. At the beginning it is best to learn from
% examples and copy the mechanism of existing plugins. Using the QGIS plugin
% installer, which itself is a Python plugin, it is possible to download a lot
% of existing Python plugins and to study their behaviour.
Comme vous pouvez voir, vous avez besoin d'informations de diff\'erentes sources
pour \'ecrire des extensions PyQGIS. Les d\'eveloppeurs de plugins doivent conna\^itre
Python et l'interface d'extension de QGIS ainsi que les classes et outils de Qt4.
Au d\'ebut il est important d'apprendre \`a partir des exemples et de copier les
m\'ecanismes d'extensions existants. En utilisant l'installateur d'extensionns de QGIS,
qui est lui m\^eme une extension Python, il est possible de t\'el\'echarger plusieurs
extensions et de les \'etudier.

% There is a a collection of online documentation that may be usefull for
% PyQGIS programers:
Il y a de nombreuses documentations qui peuvent \^etre utile pour les programmeurs de
PyQGIS :

\begin{itemize}
\item wiki de QGIS : \url{http://wiki.qgis.org/qgiswiki/PythonBindings}
\item Documentation de l'API QGIS :
\url{http://doc.qgis.org/index.html}
\item Documentation Qt : \url{http://doc.trolltech.com/4.3/index.html}
\item PyQt : \url{http://www.riverbankcomputing.co.uk/pyqt/}
\item Cours sur Python : \url{http://docs.python.org/}
\item Un livre sur les SIG bureautiques et QGIS. Il contient un chapitre sur la
programmation d'extension PyQGIS :
\url{http://www.pragprog.com/titles/gsdgis/desktop-gis} 
\end{itemize}

% You can also write plugins for QGIS in C++. See Section \ref{cpp_plugin} for
% more information about that.
Vous pouvez \'egalement \'ecrire des extensions pour QGIS en C++. Lisez la section
\ref{cpp_plugin} pour plus d'information l\`a dessus.

