% vim: set textwidth=78 autoindent:

\section{Premessa}\label{label_forward}
\pagenumbering{arabic}
\setcounter{page}{1}

% when the revision of a section has been finalized, 
% comment out the following line:
% \updatedisclaimer

Benvenuti nel meraviglioso mondo dei Sistemi Informativi Geografici (Geographical Information Systems, GIS)! Quantum GIS (QGIS) è un Sistema
Informativo Geografico a codice aperto (Open Source). Il progetto
è nato nel maggio 2002 ed è stato ospitato su SourceForge
nel giugno dello stesso anno. Abbiamo lavorato duramente per rendere
il software GIS (che è tradizionalmente un software proprietario costoso)
una valida prospettiva per chiunqe avesse disponibilit�  di un Personal
Computer. QGIS gira attualmente su molte piattaforme Unix (incluso ovviamente Linux!), su Windows,
e OS X. QGIS è sviluppato in Qt (\url{http://www.trolltech.com})
e C++. Ciò fa sì che QGis appaia responsivo nell'uso e piacevole e
semplice da usare nell'interfaccia grafica (graphical user interface,
GUI).

QGIS si prefigge lo scopo di essere un GIS facile da usare, in grado
di fornire funzioni e caratteristiche di uso comune. L’obiettivo iniziale
era di fornire un visore di dati GIS, ma attualmente QGIS ha oltrepassato questo
punto nel suo sviluppo, ed è usato da molti per il loro lavoro quotidiano nel campo GIS.
QGIS supporta nativamente un considerevole numero di formati raster e vettoriali,
il supporto a nuovi formati può essere facilmente inserito mediante
plugin (si veda l'Appendice \ref{appdx_data_formats} per una lista
completa dei formati attualmente supportati).

QGIS è rilasciato con licenza GNU General Public License (GPL). Lo
sviluppo di QGIS con questa licenza implica che si possa ispezionare
e modificare il codice sorgente in modo da garantirvi di avere sempre accesso
ad un programma GIS esente da costi di licenza e modificabile liberamente secondo le
vostre esigenze. Dovresti aver ricevuto una copia completa della licenza
con la tua copia di QGIS, puoi comunque trovarla nell'Appendice \ref{gpl_appendix}.



\begin{Tip}\caption{\textsc{Documentazione aggiornata}}\index{documentazione}
\qgistip{La versione più recente du questo documento è sempre reperibile 
all'indirizzo \url{http://download.osgeo.org/qgis/doc/manual/}, o nell'area documentazione
del sito di QGIS all'indirizzo \url{http://qgis.osgeo.org/documentation/}
}
\end{Tip}

\subsection{Caratteristiche}\label{label_majfeat}

QGIS offre nativamente e mediante plugin molte funzioni GIS di uso
comune. È possibile offrirne una panoramica inziale raggruppandole sinteticamente
in sei categorie.

\minisec{Visualizzazione di dati}

Possono essere visualizzati e sovrapposti dati vettoriali e raster
in diversi formati e proiezioni senza necessit�  di conversioni verso un formato
comune interno. Tra i formati supportati sono inclusi:

\begin{itemize}
\item tabelle PostgreSQL con estensione spaziale usando PostGIS, formati vettoriali
\footnote{i formati di database supportati da OGR come Oracle o mySQL non sono 
ancora supportati in QGIS.} supportati dalla libreria OGR installata, inclusi gli
shapefile ESRI e i formati MapInfo, SDTS e GML.
\item formati raster e immagine supportati dalla libreria GDAL (Geospatial
Data Abstraction Library) installata, come GeoTiff, Erdas Img., ArcInfo
Ascii Grid, JPEG, PNG,
\item formati raster GRASS e dati vettoriali da database GRASS (location/mapset), 
\item dati spaziali forniti da servizi di mappa online conformi agli standard
OGC quali Web Map Service (WMS) o Web Feature Service (WFS).
\end{itemize}

\minisec{Esplorazione dati e creazione di mappa} 

Possono essere composte mappe e esplorati interattivamente dati spaziali
con un'amichevole interfaccia grafica. I molti strumenti utili disponibili
nell'interfaccia grafico includono:
\begin{itemize}
\item proiezione al volo
\item compositore di mappa
\item pannello vista panoramica 
\item segnalibri geospaziali 
\item identifica/seleziona elementi
\item modifica/vedi/cerca attributi
\item etichetta elementi
\item cambio simbologia raster/vettoriale
\item aggiunta reticolato su un nuovo layer 
\item decorazione della la mappa con una freccia del nord, una barra di scala e un'etichetta di copyright
\item salva e ricarica progetti
\end{itemize}

\minisec{Creazione, modifica, gestione ed esportazione di dati}

Possono essere creati, modificati, gestiti ed esportati dati vettoriali
in molteplici formati. I dati raster devono essere importati in GRASS
per poter essere editati ed esportati in altri formati. QGIS offre
le seguenti funzioni:
\begin{itemize}
\item strumenti per digitalizzare formati supportati da OGR e layer vettoriali GRASS
\item creazione e modifica di shapefiles e layer vettoriali GRASS 
\item georeferenziazione di immagini con l'apposito plugin 
\item strumenti GPS per l'importazione ed esportazione del formato GPX, e conversione di altri formati GPS al formato GPX o down/upload dei dati direttamente ad unit�  GPS
\item creazione di layers PostGIS da shapefiles con il plugin SPIT 
\item gestione di tabelle degli attributi di dati vettoriali con il plugin gestione tabelle
\end{itemize}

\minisec{Analisi di dati} 

Possono essere eseguite analisi spaziali di dati PostgreSQL/PostGIS e di altri
formati supportati da OGR per mezzo del plugin python ftools. QGIS
offre attualmente strumenti per l'analisi, il campionamento, il geoprocessamento,
la gestione delle geometrie e del database di dati vettoriali. Possono
inoltre essere usati gli strumenti GRASS integrati, che includono
l'intera gamma delle funzioni di GRASS di oltre 300 moduli (Si veda la Sezione \ref{sec:grass}).

\minisec{Pubblicazione di mappe su internet}

QGIS può essere usato per esportare dati in un mapfile che può essere
pubblicato su internet mediante un webserver sul quale sia installato
UMN MapServer. QGIS può anche essere impiegato come client WMS or
WFS, e come server WMS. 

\minisec{Estensione delle funzioni di QGIS per mezzo di plugins}

QGIS può essere adattato a particolari esigenze grazie all'architettura
estensibile per mezzo di plugin. QGIS fornisce le librerie che possono
essere usate per creare i plugins. Possono addirittura essere creati
nuovi programmi in C++ or Python!

\begin{itemize}
\item \textbf{Plugin inclusi nel software}
\\ \\ Aggiunta layer WFS
\\ Aggiunta layer a partire da testo delimitato
\\ Cattura coordinate
\\ Decorazioni (Etichetta Copyright, Freccia Nord e Barra di Scala)
\\ Georeferenziatore
\\ Convertitore Dxf2Shp
\\ Strumenti GPS
\\ Integrazione con GRASS
\\ Generatore di reticolo
\\ Plugin per interpolazione
\\ Conversione tra layer supportati da OGR
\\ Stampa rapida
\\ Interfaccia di importazione di shapefile verso PostgreSQL/PostGIS (SPIT)
\\ Esportazione verso Mapserver
\\ Console Python
\\ Installatore plugins Python
\\ \item \textbf{Plugins Python}
\\ \\ QGIS offre un numero crescente di plugin esterni in Python forniti
dalla comunit� . Questi plugin sono ospitati sul repository ufficiale
PyQGIS, e possono essere facilmente installati usando il gestore di
plugins python (Si veda la Sezione \ref{sec:plugins}).
\end{itemize}

