\section{Moduli degli strumenti GRASS}\label{appdx_grass_toolbox_modules}

% when the revision of a section has been finalized, 
% comment out the following line:
% \updatedisclaimer

La shell di GRASS avviabile dagli strumenti GRASS fornisce accesso a praticamente tutti (oltre 300) moduli GRASS in modalità riga di comando. Per offrire un ambiente di lavoro più confortevole per l'utente, all'incirca 200 di questi moduli e funzionalità sono disponibili con interfaccia grafica.

\subsection{Moduli negli strumenti GRASS per l'importazione e l'esportazione di dati}\index{GRASS!strumenti!moduli}

Questa sezione elenca tutti i moduli con interfaccia grafica presenti tra gli strumenti GRASS per importare ed esportare dati nella location e mapset impostato.

\begin{table}[ht]
\centering
\caption{Strumenti GRASS: moduli per l'importazione di dati}\medskip
 \begin{tabular}{|p{4cm}|p{12cm}|}
  \hline \multicolumn{2}{|c|}{\textbf{Moduli per l'importazione di dati tra gli strumenti GRASS}} \\ 
  \hline \textbf{Nome del modulo} & \textbf{Scopo} \\
  \hline r.in.arc & Converte un file raster in formato ESRI ARC/INFO ascii (GRID) in un layer raster (binario)\\
  \hline r.in.ascii & Converte un file raster in formato testo ASCII in un layer raster (binario) \\
  \hline r.in.aster & Georeferenzia, rettifica, e importa immagini Terra-ASTER imagery e relativi DEM usando il comango gdalwarp \\
  \hline r.in.gdal &  Importa raster supportati da GDAL in formato raster binario GRASS \\
  \hline r.in.gdal.loc &  Importa raster supportati da GDAL in formato raster binario GRASS e crea una location su misura per la visualizzazione del dato \\
  \hline r.in.gridatb & Importa file GRIDATB.FOR (TOPMODEL) in formato raster GRASS \\
  \hline r.in.mat  & Importa file binari MAT-File(v4) in formato raster GRASS  \\
  \hline r.in.poly  &  Crea mappe raster da file ascii descriventi poligoni/linee presenti nella directory attuale \\
  \hline r.in.srtm  & Importa file SRTM HGT in GRASS \\
  \hline i.in.spotvgt & Importa file SPOT VGT NDVI in mappe raster \\
  \hline v.in.dxf & Importa layer vettoriali in formato DXF \\
  \hline v.in.e00 & Importa layer vettoriali ESRI E00 \\
  \hline v.in.garmin & Importa vettori da unità/formati gps usando il comando gpstrans \\
  \hline v.in.gpsbabel & Importa vettori da unità/formati gps usando il comando gpsbabel \\
  \hline v.in.mapgen & Importa vettoriali in formato MapGen or MatLab in GRASS \\
  \hline v.in.ogr & Importa layer vettoriali OGR/PostGIS \\
  \hline v.in.ogr.loc & Importa layer vettoriali OGR/PostGIS e crea una location su misura per la visualizzazione dei dati \\
  \hline v.in.ogr.all & Importa tutti i layer vettoriali OGR/PostGIS presenti in una certa sorgente dati \\
  \hline v.in.ogr.all.loc & Importa tutti i layer vettoriali OGR/PostGIS presenti in una certa sorgente dati e crea una location su misura per la visualizzazione del dato \\
\hline
\end{tabular}
\end{table}

\begin{table}[ht]
\centering
\caption{Strumenti GRASS: moduli per per l'esportazione di dati}\medskip
 \begin{tabular}{|p{4cm}|p{12cm}|}
  \hline \multicolumn{2}{|c|}{\textbf{Moduli per l'esportazione di dati tra gli strumenti GRASS}} \\ 
  \hline \textbf{Nome modulo} & \textbf{Scopo} \\
  \hline r.out.gdal.gtiff & Esporta layer raster in formato Geo TIFF \\
  \hline r.out.arc & Converte un raster in un file ESRI ARCGRID \\
  \hline r.gridatb & Esporta raster GRASS in file GRIDATB.FOR (TOPMODEL) \\
  \hline r.out.mat & Esporta un raster GRASS verso il formato binario MAT-File \\
  \hline r.out.bin & Esporta un raster GRASS in una matrice binaria \\
  \hline r.out.png & Esporta raster GRASS come immagini non georeferenziate in formato immagine PNG \\
  \hline r.out.ppm & Converte una mappa raster GRASS in formato immagine PPM alla risoluzione impostata per la region in GRASS \\
  \hline r.out.ppm3 & Converte 3 layer raster (RGB) GRASS in formato immagine PPM alla risoluzione impostata per la region in GRASS \\
  \hline r.out.pov & Converte un raster in un campo di altezze per POVRAY\\
  \hline r.out.tiff & Esporta una mappa raster GRASS verso un'immagine in formato TIFF a 8/24bit alla risoluzione impostata per la region in GRASS \\
  \hline r.out.vrml &  Esporta una mappa raster in formato Virtual Reality Modeling Language (VRML)\\
  \hline v.out.ogr & Esporta layer vettoriali in vari formati supportati dalla libreria OGR \\
  \hline v.out.ogr.gml & Esporta layer vettoriali verso il formato GML \\
  \hline v.out.ogr.postgis & Esporta layer vettoriali verso vari formati (tramite la libreria OGR) \\
  \hline v.out.ogr.mapinfo & Esporta layer vettoriali verso il formato Mapinfo \\
  \hline v.out.ascii & Converte un raster binario di GRASS verso il formato vettoriale GRASS ASCII  \\
  \hline v.out.dxf & Converte un vettoriale GRASS verso il formato DXF  \\
\hline
\end{tabular}
\end{table}

\subsection{Moduli negli strumenti GRASS per la conversione tra tipi di dato}

Questa Sezione elenca tutti i moduli con interfaccia grafica negli strumenti GRASS per convertire dati raster a vettoriali e viceversa nella location e mapset GRASS selezionati.

\begin{table}[ht]
\centering
\caption{Strumenti GRASS: moduli per la conversione tra tipi di dato}\medskip
 \begin{tabular}{|p{4cm}|p{12cm}|}
  \hline \multicolumn{2}{|c|}{\textbf{Moduli per la conversione tra tipi di dato tra gli strumenti GRASS}} \\
  \hline \textbf{Nome modulo} & \textbf{Scopo} \\
  \hline r.to.vect.point & Converte un raster in un vettoriale di punti \\
  \hline r.to.vect.line & Converte un raster in un vettoriale di linee \\
  \hline r.to.vect.area & Converte un raster in un vettoriale di aree \\
  \hline v.to.rast.constant & Converte un vettore ad un raster usando un valore costante per le categorie\\
  \hline v.to.rast.attr & Converte un vettore in un raster  usando gli attributi per le categorie \\
\hline
\end{tabular}
\end{table}

\subsection{Moduli negli strumenti GRASS per la definizione della regione e l'impostazione della proiezione}

Questa Sezione elenca tutti i moduli con interfaccia grafica negli strumenti GRASS per la gestione e la modifica della regione del mapset attuale e per configurare la proiezione.

\begin{table}[ht]
\centering
\caption{Strumenti GRASS: moduli per la gestione della regione e della proiezione}\medskip
 \begin{tabular}{|p{4cm}|p{12cm}|}
  \hline \multicolumn{2}{|c|}{\textbf{Moduli per la gestione della regione e della proiezione tra gli strumenti GRASS}} \\
  \hline \textbf{Nome modulo} & \textbf{Scopo} \\
  \hline g.region.save & Salva la regione corrente con nome \\
  \hline g.region.zoom & Restringi la regione corrente fino a quando non incontra celle non vuote di una data mappa raster \\
  \hline g.region.multiple.raster & Imposta la regione in modo che combaci con l'estensione di più raster \\
  \hline g.region.multiple.vector & Imposta la regione in modo che combaci con l'estensione di più vettoriali \\
  \hline g.proj.print & Stampa le informazioni di proiezione per la location corrente \\
  \hline g.proj.geo & Stampa le informazioni di proiezione di un file georeferenziato (raster, vettore o immagine)\\
  \hline g.proj.ascii.new & Stampa le informazioni di proiezione di un file ASCII georeferenziato contenente la descrizione della proiezione in formato WKT \\
  \hline g.proj.proj & Stampa le informazioni di proiezione da un file descrittivo di proiezione in formato PROJ.4 \\
  \hline g.proj.ascii.new & Stampa informazioni di proiezione da un file ASCII georeferenziato contenente la descrizione della proiezione in formato WKT e crea una nuova location basata su queste informazioni e sull'estensione di questo file \\
  \hline g.proj.geo.new & Stampa le informazioni di proiezione di un file georeferenziato (raster, vettore o immagine) e crea una nuova location basata su queste informazioni e sull'estensione di questo file \\
  \hline g.proj.proj.new & Stampa le informazioni di proiezione da un file descrittivo di proiezione in formato PROJ.4 e crea una nuova location basata su queste informazioni e sull'estensione di questo file \\
  \hline m.cogo & Una semplice utility per convertire misure di direzione e distanze in coordinate e viceversa. Viene assunto un sistema di coordinate cartesiano \\
\hline
\end{tabular}
\end{table}

\clearpage

\subsection{Moduli negli strumenti GRASS per le operazioni su dati raster}

Questa Sezione elenca tutti i moduli con interfaccia grafica negli strumenti GRASS per l'effettuazione di operazioni su dati raster salvati nella location e nel mapset selezionato.

\begin{table}[ht]
\centering
\caption{Strumenti GRASS: moduli per le operazioni su dati raster}\medskip
 \begin{tabular}{|p{4cm}|p{12cm}|}
  \hline \multicolumn{2}{|c|}{\textbf{Moduli per le operazioni su dati raster tra gli strumenti GRASS}} \\
  \hline \textbf{Nome modulo} & \textbf{Scopo} \\
  \hline r.compress & Comprime e decomprime mappe raster \\
  \hline r.region.region & Imposta i limiti della regione di un raster a quella di default \\
  \hline r.region.raster & Imposta i limiti della regione di un raster a quelli dell'estensione di una mappa raster esistente \\
  \hline r.region.vector & Imposta i limiti della regione di un raster a quelli di una mappa vettoriale esistente \\
  \hline r.region.edge & Imposta i limiti della regione di un raster ai valori impostati (n-s-e-w) \\
  \hline r.region.alignTo & Imposta la regione del raster in modo che abbia la risoluzione spaziale e i limiti di una mappa di riferimento indicata \\
  \hline r.null.val & Trasforma il valore di cella specificato in valore nullo \\
  \hline r.null.to & Assegna il valore specificato alle celle nulle \\
  \hline r.quant & Algoritmo per produrre il file dei quantili di una mappa con valori a virgola mobile \\
  \hline r.resamp.stats & Ricampiona una mappa raster usando il metodo dell'aggregazione statistica \\
  \hline r.resamp.interp & Ricampiona una mappa raster per mezzo dell'interpolazione dei valori \\
  \hline r.resample & Ricampiona una mappa raster verso una nuova risoluzione spaziale precedentemente impostata \\
  \hline r.resamp.rst & Reinterpola e esegue l'analisi topografica per mezzo del metodo "regularized spline" con i parametri "tension" e "smoothing" \\
  \hline r.support & Crea, rigenera o modifica i file di supporto di una mappa raster \\
  \hline r.support.stats & Aggiorna le statistiche della mappa raster \\
  \hline r.proj & Riproietta una mappa raster dalla location originale indicata a quella corrente \\
\hline
\end{tabular}
\end{table}

\begin{table}[ht]
\centering
\caption{Strumenti GRASS: moduli per la gestione del colore dei dati raster}\medskip
 \begin{tabular}{|p{4cm}|p{12cm}|}
  \hline \multicolumn{2}{|c|}{\textbf{Moduli per la gestione del colore dei dati raster tra gli strumenti GRASS}} \\
  \hline \textbf{Nome modulo} & \textbf{Scopo} \\
  \hline r.colors.table & Imposta la tabella colori del raster a quella selezionata tra quelle predefinite \\
  \hline r.colors.rules & Imposta la tabella colori del raster in base alle regole impostate \\
  \hline r.colors.rast & Imposta la tabella colori del raster usando le impostazioni di un raster esistente \\
  \hline r.blend & Miscela le componenti di colore di due mappe raster del valore impostato \\
  \hline r.composite & Miscela le componenti di colore rosso, verde e blu per ottenere un singolo file raster a colori \\
  \hline r.his & Genera mappe raster separate per le componenti rosso, verde e blu combinando i valori di tonalità (hue), intensità (intensity), e saturazione (saturation) (metodo his) di una mappa raster in input specificata dall'utente \\
\hline
\end{tabular}
\end{table}

\begin{table}[ht]
\centering
\caption{Strumenti GRASS: moduli per le operazioni di processamento spaziale di dati raster}\medskip
 \begin{tabular}{|p{4cm}|p{12cm}|}
  \hline \multicolumn{2}{|c|}{\textbf{Moduli per l'analisi geospaziale di dati raster tra gli stumenti di GRASS}} \\
  \hline \textbf{Nome modulo} & \textbf{Scopo} \\
  \hline r.buffer & Crea un buffer di dimensione impostata sui valori del raster \\
  \hline r.mask & Crea una maschera (MASK) per limitare l'estensione spaziale sulla quale eseguire le operazioni sul raster \\
  \hline r.mapcalc & Esegue operazioni geospaziali sul raster in modo algebrico \\
  \hline r.mapcalculator & Procedura guidata e semplificata per l'esecuzione di operazioni algebriche geospaziali su mappe raster \\
  \hline r.neighbors & Analisi raster di tipo "neighbors", con creazione di una nuova mappa raster i cui valori in una certa posizione sono funzione (media, mediana, moda, minimo, massimo...) di un certo numero di valori (size) scelti dall'utente nelle vicinanze di quella posizione \\
  \hline v.neighbors & Creazione di una mappa raster in cui i valori di una certa cella sono funzione dei valori degli attributi assegnati a punti o centroidi presenti nelle vicinanze della medesima entro un certo raggio di ricerca impostato dall'utente \\
  \hline r.cross & Creazione di una nuova mappa raster che contenga un numero di categorie pari alle singole combinazioni di categorie dei raster in input \\
  \hline r.series & Creazione di una mappa raster in cui ogni cella assume il valore stabilito dalla funzione impostata che opera sui valori di cella delle mappe scelte dall'utente \\
  \hline r.patch & Creazione di una singola mappa raster mediante "collage" di altre mappe raster singole \\
  \hline r.statistics & Fornisce le statistichedi un layer "cover" sulla base delle sue relazioni con un layer "base" stabilite dalla funzione specificata \\
  \hline r.cost & Creazione di una mappa raster contenente il costo cumulativo conseguente lo spostamento tra diverse posizioni geografiche su una mappa raster i cui valori di cella rappresentino costi \\
  \hline r.drain & Crea una mappa raster contenente una linea di flusso sulla base dei valori contenuti in una mappa di elevazione \\
  \hline r.shaded.relief & Crea mappe ombreggiate \\
  \hline r.slope.aspect.slope & Genera mappe di acclività da modelli digitali di elevazione \\
  \hline r.slope.aspect.aspect & Genera mappe di orientazione dei versanti da modelli digitali di elevazione \\
  \hline r.param.scale & Estrae i parametri morfometrici di un'area da un modello digitale di elevazione \\
  \hline r.texture & Genera una mappa raster con le caratteristiche tessiturali di una mappa raster (prima serie di indici) \\
  \hline r.texture.bis & Genera una mappa raster con le caratteristiche tessiturali di una mappa raster (seconda serie di indici) \\
  \hline r.los & Analisi di visibilità su mappe raster \\
  \hline r.clump & Raggruppa celle contigue in un'unica categoria \\
  \hline r.grow & Genera una mappa raster in cui le aree contigue vengono accresciute di una cella rispetto alla mappa di partenza \\
  \hline r.thin & Assottiglia gli elementi lineari di una mappa raster \\
\hline
\end{tabular}
\end{table}

\begin{table}[ht]
\centering
\caption{Strumenti GRASS: moduli per la gestione di superfici}\medskip
 \begin{tabular}{|p{4cm}|p{12cm}|}
  \hline \multicolumn{2}{|c|}{\textbf{Moduli per la gestione di superfici tra gli strumenti di GRASS}} \\
  \hline \textbf{Nome modulo} & \textbf{Scopo} \\
  \hline r.random & Crea una mappa di punti vettoriali campionando una mappa raster in modo casuale \\
  \hline r.random.cells & Genera celle casuali legate tra loro da relazioni spaziali \\
  \hline v.kernel & Genera una mappa raster di densità sulla base di un vettoriale di punti usando un Gaussian kernel \\
  \hline r.contour & Produce una mappa vettoriale delle isoipse a partire da una mappa raster con l'intervallo specificato \\
  \hline r.contour2 & Produce una mappa vettoriale delle isoipse elencate dall'utente a partire da una mappa raster \\
  \hline r.surf.fractal & Crea una superificie frattale con dimensione del frattale specificata \\
  \hline r.surf.gauss & Crea una mappa raster della deviazione gaussiana la cui media e deviazione standard è impostata dall'utente \\
  \hline r.surf.random & Genera una mappa raster delle deviazioni random il cui intervallo è impostato dall'utente \\
  \hline r.bilinear & Interpolazione bilineare di mappe raster \\
  \hline v.surf.bispline & Interpolazione bicubica o "bilinear spline" con regolarizzazione di Tykhonov \\
  \hline r.surf.idw & Interpolazione di superfici 3d con metodo idw (inverse distance weighted) \\
  \hline r.surf.idw2 & Altra opzione di generazione di superfici 3d con metodo idw \\
  \hline r.surf.contour & Generazione di superfici 3d da isoipse raster \\
  \hline v.surf.idw & Generazione di superfici 3d per interpolazione di valori di un vettoriale (metodo idw) \\
  \hline v.surf.rst & Generazione di superici 3d per interpolazione di valori di un vettoriale (metodo Regularized Spline Tension, RST) \\
  \hline r.fillnulls & Riempie le aree prive di dati in un raster usando l'interpolazione con il modulo v.surf.rst \\
\hline
\end{tabular}
\end{table}

\begin{table}[ht]
\centering
\caption{Strumenti GRASS: moduli per la modifica delle categorie raster ed etichette}\medskip
 \begin{tabular}{|p{4cm}|p{12cm}|}
  \hline \multicolumn{2}{|c|}{\textbf{Moduli per le categorie e le etichette tra gli strumenti GRASS}} \\
  \hline \textbf{Nome modulo} & \textbf{Scopo} \\
  \hline r.reclass.area.greater & Riclassifica una mappa raster con parcelle di estensioni in ettari superiori al valore stabilito dall'utente \\
  \hline r.reclass.area.lesser &  Riclassifica una mappa raster con parcelle di estensioni in ettari inferiori al valore stabilito dall'utente \\
  \hline r.reclass & Riclassifica un raster usando un file di regole preformattato dall'utente \\
  \hline r.recode & Ricodifica una mappa raster \\
  \hline r.rescale & Riscala gli intervalli dei valori di categoria di una mappa raster \\
\hline
\end{tabular}
\end{table}

\begin{table}[ht]
\centering
\caption{Strumenti GRASS: moduli per la modellazione idrologica}\medskip
 \begin{tabular}{|p{4cm}|p{12cm}|}
  \hline \multicolumn{2}{|c|}{\textbf{Moduli per la modellazione idrologica tra gli strumenti GRASS}} \\
  \hline \textbf{Nome modulo} & \textbf{Scopo} \\
  \hline r.carve & Trasforma un file vettoriale dei percorsi dei corsi d'acqua in un raster di determinata larghezza e altezza e lo sottrae al DEM di base per dare un DEM in uscita \\
  \hline r.fill.dir & Filtra e genera una mappa di elevazione priva di depressioni e una mappa delle direzioni di flusso a partire da un certa mappa di elevazione \\
  \hline r.lake.xy & Riempie i laghi a partire da un punto di alimentazione di coordinate specificate xy fino ad un livello determinato \\
  \hline r.lake.seed & Riempie i laghi a partire da una mappa raster contenente il punto di alimentazione fino ad un livello determinato \\
  \hline r.topidx & Crea una mappa di volume 3D sulla base di una mappa di elevazione 2D e di una mappa raster di valori \\
  \hline r.basins.fill & Genera una mappa raster dei bacini imbriferi \\
  \hline r.water.outlet & Programma per l'individuazione di bacini imbriferi \\
\hline
\end{tabular}
\end{table}

\begin{table}[ht]
\centering
\caption{Strumenti GRASS: moduli per la generazione di report e l'analisi statistica}\medskip
 \begin{tabular}{|p{4cm}|p{12cm}|}
  \hline \multicolumn{2}{|c|}{\textbf{Moduli per la generazione di report e l'analisi statistica tra gli strumenti di GRASS}} \\
  \hline \textbf{Nome modulo} & \textbf{Scopo} \\
  \hline r.category & Stampa le categorie e le etichette descrittive delle stesse di una mappa raster specificata dall'utente \\
  \hline r.sum & Somma i valori delle celle raster \\
  \hline r.report & Fornisce le statistiche di una mappa raster \\
  \hline r.average & Individua la media dei valori in una mappa di copertura entro le aree alle quali sia stata assegnata la stessa categoria di una mappa di base specificata dall'utente \\
  \hline r.median & Individua la mediana dei valori in una mappa di copertura entro le aree alle quali sia stata assegnata la stessa categoria di una mappa di base specificata dall'utente \\
  \hline r.mode & Individua la moda dei valori in una mappa di copertura entro le aree alle quali sia stata assegnata la stessa categoria di una mappa di base specificata dall'utente  \\
  \hline r.volume & Calcola il volume di gruppi di dati e produce una mappa di punti vettoriale contenente i centroidi di questi raggruppamenti \\
  \hline r.surf.area & Calcola l'area di una rete di punti triangolare 3D regolarmente distribuiti rappresentanti i centri di celle raster \\
  \hline r.univar & Calcola le statistiche univariate di celle non nulle in un raster \\
  \hline r.covar & Fornisce una matrice di covarianza/correlazione a partire da mappe raster specificate dall'utente \\
  \hline r.regression.line & Calcola la regressione lineare tra due mappe raster secondo la formula: y = a + b * x \\
  \hline r.coin & Tabella la ricorrenza reciproca (coincidenza) di categorie per due mappe raster \\
\hline
\end{tabular}
\end{table}

\clearpage

\subsection{Moduli negli strumenti GRASS per le operazioni su dati vettoriali}

Questa Sezione elenca tutti i moduli con interfaccia grafica negli strumenti GRASS per l'effettuazione di operazioni su dati vettoriali salvati nella location e nel mapset selezionato.

\begin{table}[ht]
\centering
\caption{Strumenti GRASS: moduli per la modifica di dati vettoriali}\medskip
 \begin{tabular}{|p{4cm}|p{12cm}|}
  \hline \multicolumn{2}{|c|}{\textbf{Moduli per la modifica di dati vettoriali tra gli strumenti di GRASS}} \\
  \hline \textbf{Nome modulo} & \textbf{Scopo} \\
  \hline v.build.all & Ricostruisce la topologia di tutti i vettori del mapset \\
  \hline v.clean.break & Spezza le linee ad ogni intersezione \\
  \hline v.clean.snap & Aggancia le linee ai vertici entro una distanza impostata \\
  \hline v.clean.rmdangles & Rimuove i dangles \\
  \hline v.clean.chdangles & Cambia il dangle dal tipo boundary al tipo linea \\
  \hline v.clean.rmbridge & Rimuove i bridges che collegano un'area e un'isola o due isole \\
  \hline v.clean.chbridge & Cambia il bridge che collega un'area e un'isola o due isole da un tipo ad un altro \\
  \hline v.clean.rmdupl & Rimuove le linee doppie (fare attenzione alle categorie!) \\
  \hline v.clean.rmdac & Rimuove i centroidi doppi in un'area \\
  \hline v.clean.bpol & Spezza il perimetro (boundary) in ogni punto condiviso tra due o più poligoni sul quale convergono segmenti con angolo diverso \\
  \hline v.clean.prune & Rimuove i vertici da linee e perimetri (boundaries) entro un valore impostato \\
  \hline v.clean.rmarea & Rimuove aree piccole (rimuove il più lungo perimetro con area adiacente) \\
  \hline v.clean.rmline & Rimuove tutte le linee o perimetri di lunghezza zero \\
  \hline v.clean.rmsa & Rimuove piccoli angoli tra linee e nodi \\
  \hline v.type.lb & Converte linee in perimetri \\
  \hline v.type.bl & Converte perimetri in linee \\
  \hline v.type.pc & Converte punti in centroidi \\
  \hline v.type.cp & Converte centroidi in punti \\
  \hline v.centroids & Aggiunge centroidi mancanti a perimetri chiusi \\
  \hline v.build.polylines & Costruisce polilinee da segmenti di linee \\
  \hline v.segment & Crea punti/segmenti da in layer vettoriali di linee e posizioni in input \\
  \hline v.to.points & Crea punti lungo un vettoriale di linee in input \\
  \hline v.parallel & Crea linee parallele a quelle in input \\
  \hline v.dissolve & Dissolve limiti tra aree adiacenti \\
  \hline v.drape & Converte vettori 2D in vettori 3D campionando un raster di elevazione \\
  \hline v.transform & Esegue una trasformazione affine su una mappa vettoriale \\
  \hline v.proj & Consente di riproiettare layer vettoriale \\
  \hline v.support & Aggiorna i metadati di una mappa vettoriale \\
  \hline v.generalize & Generalizzazione di mappe vettoriali \\
\hline
\end{tabular}
\end{table}

\begin{table}[ht]
\centering
\caption{Strumenti GRASS: moduli per la gestione del collegamento ai database}\medskip
 \begin{tabular}{|p{4cm}|p{12cm}|}
  \hline \multicolumn{2}{|c|}{\textbf{Moduli per la gestione del collegamento ai database tra gli stumenti di GRASS}} \\
  \hline \textbf{Nome modulo} & \textbf{Scopo} \\
  \hline v.db.connect & Connette un vettoriale ad un database \\
  \hline v.db.sconnect & Disconnette un vettoriale da un database \\
  \hline v.db.what.connect & Imposta/Mostra la connesione al database per un certo vettoriale \\
\hline
\end{tabular}
\end{table}

\begin{table}[ht]
\centering
\caption{Strumenti GRASS: moduli per la modifica delle categorie di un vettoriale}\medskip
 \begin{tabular}{|p{4cm}|p{12cm}|}
  \hline \multicolumn{2}{|c|}{\textbf{Moduli per la modifica delle categorie di un vettoriale tra gli strumenti di GRASS}} \\
  \hline \textbf{Nome modulo} & \textbf{Scopo} \\
  \hline v.category.add & Aggiunge elementi al layer (TUTTI gli elementi del layer selezionato!) \\
  \hline v.category.del & Cancella valori di categoria \\
  \hline v.category.sum & Aggiunge un valore alle categorie esistenti \\
  \hline v.reclass.file & Riclassifica i valori di categoria usando un file di regole \\
  \hline v.reclass.attr & Riclassifica i valori di categoria usando una colonna attributi (di tipo intero positivo) \\
\hline
\end{tabular}
\end{table}

\begin{table}[ht]
\centering
\caption{Strumenti GRASS: moduli per lavorare con vettoriali di punti}\medskip
 \begin{tabular}{|p{4cm}|p{12cm}|}
  \hline \multicolumn{2}{|c|}{\textbf{Moduli per lavorae con vettoriali di punti tra gli strumenti di GRASS}} \\
  \hline \textbf{Nome modulo} & \textbf{Scopo} \\
  \hline v.in.region & Crea una nuova mappa vettoriale con l'estensione della regione corrente \\
  \hline v.mkgrid.region & Crea una griglia nella regione corrente \\
  \hline v.in.db & Importa punti da una tabella database che ne contiene le coordinate \\
  \hline v.random & Genera una mappa vettoriale GRASS di punti casuali 2D/3D \\
  \hline v.kcv & Raggruppa casualmente punti in set di test \\
  \hline v.outlier & Rimuove estremi da dati vettoriali di punti \\
  \hline v.hull & Crea un poligono convesso \\
  \hline v.delaunay.line & Triangolazione Delaunay per punti (mediante linee) \\
  \hline v.delaunay.area & Triangolazione Delaunay per punti (mediante areae) \\
  \hline v.voronoi.line & Diagrammi Voronoi (mediante linee) \\
  \hline v.voronoi.area & Diagrammi Voronoi (mediante aree) \\
\hline
\end{tabular}
\end{table}

\begin{table}[ht]
\centering
\caption{Strumenti GRASS: moduli per l'analisi spaziale di vettoriali e per l'analisi di reti}\medskip
 \begin{tabular}{|p{4cm}|p{12cm}|}
  \hline \multicolumn{2}{|c|}{\textbf{Moduli per l'analisi spaziale di vettoriali e per l'analisi di reti tra gli strumenti di GRASS}} \\
  \hline \textbf{Nome modulo} & \textbf{Scopo} \\
  \hline v.extract.where & Seleziona elementi sulla base di attributi \\
  \hline v.extract.list & Estrae gli elementi selezionati in base alla lista \\
  \hline v.select.overlap & Seleziona gli elementi sovrapposti da quelli di un'altra mappa \\
  \hline v.buffer & Creazione di buffer su vettoriali \\
  \hline v.distance & Trova nel vettore 'to' l'elemento più vicino al vettore 'from' \\
  \hline v.net.nodes & Crea nodi su una rete \\
  \hline v.net.alloc & Alloca la rete \\
  \hline v.net.iso & Taglia la rete secondo isolinee di costo \\
  \hline v.net.salesman & Connette nodi lungo la strada più corta (commesso viaggiatore) \\
  \hline v.net.steiner & Connette i nodi selezionati lungo la diramazione più corta (Steiner
  tree) \\
  \hline v.patch & Crea una nuova mappa vettoriale combinando altre mappe \\
  \hline v.overlay.or & Unione logica di vettori \\
  \hline v.overlay.and & Intersezione logica di vettori \\
  \hline v.overlay.not & Sottrazione logica di vettori \\
  \hline v.overlay.xor & Non-intersezione logica di vettori \\
\hline
\end{tabular}
\end{table}

\begin{table}[ht]
\centering
\caption{Strumenti GRASS: moduli per l'aggiornamento di vettoriali sulla base di altre mappe}\medskip
 \begin{tabular}{|p{4cm}|p{12cm}|}
  \hline \multicolumn{2}{|c|}{\textbf{Moduli per l'aggiornamento di vettoriali sulla base di altre mappe tra gli strumenti di GRASS}} \\
  \hline \textbf{Nome modulo} & \textbf{Scopo} \\
  \hline v.rast.stats & Calcola le statistiche univariate da una mappa raster GRASS sulla base di oggetti vettoriali \\
  \hline v.what.vect & Carica mappe delle quali editare la tabella attributi \\
  \hline v.what.rast & Carica i valori del raster nella posizione di punti vettoriali nella tabella attributi di questi ultimi \\
  \hline v.sample & Campiona un file raster nella posizione dei punti vettoriali \\
\hline
\end{tabular}
\end{table}

\begin{table}[ht]
\centering
\caption{Strumenti GRASS: moduli per i report e le statistiche su un vettoriale}\medskip
 \begin{tabular}{|p{4cm}|p{12cm}|}
  \hline \multicolumn{2}{|c|}{\textbf{Moduli per i report e le statistiche su un vettoriale tra gli strumenti di GRASS}} \\
  \hline \textbf{Nome modulo} & \textbf{Scopo} \\
  \hline v.to.db & Inserisce le variabili geometriche nel database del vettoriale \\
  \hline v.report & Fornisce un report delle statistiche sulla geometria per i vettoriali \\
  \hline v.univar & Calcola le statistiche univariate di una colonna scelta nella tabella attributi di un vettoriale \\
  \hline v.normal & Test di normalità per punti vettoriali \\
\hline
\end{tabular}
\end{table}

\clearpage

\subsection{Moduli negli strumenti GRASS per le operazioni su immagini}

Questa Sezione elenca tutti i moduli con interfaccia grafica negli strumenti GRASS per modificare ed analizzare immagini salvate nella location e nel mapset selezionato.

\begin{table}[ht]
\centering
\caption{Strumenti GRASS: moduli per l'analisi di immagini}\medskip
 \begin{tabular}{|p{4cm}|p{12cm}|}
  \hline \multicolumn{2}{|c|}{\textbf{Moduli per l'analisi di immmagini tra gli strumenti di GRASS}} \\
  \hline \textbf{Nome modulo} & \textbf{Scopo} \\
  \hline i.image.mosaik & Mosaica fino a 4 immagini \\
  \hline i.rgb.his & Funzione per la trasformazione del colore di un'immagine dallo spazio Red Green Blue (RGB) a quello Hue Intensity Saturation (HIS) \\
  \hline i.his.rgb & Funzione per la trasformazione del colore di un'immagine dallo spazio Hue Intensity Saturation (HIS) a quello Red Green Blue (RGB) \\
  \hline i.landsat.rgb & Auto-bilanciamento del colore per immagini LANDSAT \\
  \hline i.fusion.brovey & Trasformazione di Brovey per fondere canali multispettrali e canali pancromatici ad alta risoluzione \\
  \hline i.zc & Individuazione degli zero-crossing edge per il processamento di immagini \\
  \hline i.mfilter &  \\
  \hline i.tasscap4 & Trasformazione Tasseled Cap (Kauth Thomas) per dati LANDSAT-TM 4 \\
  \hline i.tasscap5 & Trasformazione Tasseled Cap (Kauth Thomas) per dati LANDSAT-TM 5 \\
  \hline i.tasscap7 & Trasformazione Tasseled Cap (Kauth Thomas) per dati LANDSAT-TM 7 \\
  \hline i.fft & Trasformazione fast fourier (FFT) per il processamento di immagini \\
  \hline i.ifft & Trasformazioni inversa fast fourier per il processamento di immagini \\
  \hline r.describe & Stampa una lista chiara di valori di categoria trovati nella mappa raster indicata \\
  \hline r.bitpattern & Confronta bit patterns con una mappa raster \\
  \hline r.kappa & Calcola l'errore matriciale e il parametro kappa per stabilire l'accuratezza dei risultati di una classificazione \\
  \hline i.oif & Calcola il fattore indice ottimale per bande landsat tm \\
\hline
\end{tabular}
\end{table}

\clearpage

\subsection{Moduli negli strumenti GRASS per la gestione dei database}

Questa Sezione elenca tutti i moduli con interfaccia grafica negli strumenti GRASS per gestire, collegare e lavorare con database interni ed esterni. Le operazioni su database esterni sono eseguite tramite OGR e non sono descritte in questa sezione.

\begin{table}[ht]
\centering
\caption{Strumenti GRASS: moduli per i database}\medskip
 \begin{tabular}{|p{4cm}|p{12cm}|}
  \hline \multicolumn{2}{|c|}{\textbf{Moduli per la gestione ed analisi di database tra gli strumenti di GRASS}} \\
  \hline \textbf{Nome modulo} & \textbf{Scopo} \\
  \hline db.connect & Imposta la connessione al db generale per il mapset \\
  \hline db.connect.schema & Imposta la connessione generale al db per il mapset con uno schema \\
  \hline v.db.reconnect.all & Ricollega il vettoriale ad un nuovo database \\
  \hline db.login & Imposta user/password per il driver/database \\
  \hline db.in.ogr & Importa la tabella attributi in vari formati \\
  \hline v.db.addtable & Crea e aggiunge una nuova tabella ad un vettoriale \\
  \hline v.db.addcol & Aggiunge una o più colonne alla tabella attributi associata ad una certa mappa vettoriale \\
  \hline v.db.dropcol & Elimina una colonna dalla tabella attributi collegata ad una certa mappa vettoriale \\
  \hline v.db.renamecol & Rinomina una colonna della tabella attributi associata ad un vettoriale \\
  \hline v.db.update\_const & Assegna un nuovo valore costante ad una colonna della tabella attributi di un vettoriale \\
  \hline v.db.update\_query & Assegna un nuovo valore costante ad una colonna della tabella attributi di un vettoriale solo per gli elementi per i quali il risultato della query impostata è TRUE \\
  \hline v.db.update\_op & Assegna un nuovo valore, risultante da operazioni su una o più colonne esistenti, ad una colonna della tabella attributi di un vettoriale \\
  \hline v.db.update\_op\_query & Assegna un nuovo valore, risultante da operazioni su una o più colonne esistenti, ad una colonna della tabella attributi di un vettoriale, solo per gli elementi per i quali il risultato della query impostata è TRUE \\
  \hline db.execute & Esegue un'espressione SQL \\
  \hline db.select & Stampa i risultati della selezione dal database secondo la richiesta SQL \\
  \hline v.db.select & Stampa gli attributi di una mappa vettoriale \\
  \hline v.db.select.where & Stampa gli attributi di una mappa vettoriale in base ad una richiesta SQL \\
  \hline v.db.join & Consente di unire una tabella a quella esistente di una mappa vettoriale \\
  \hline v.db.univar & Calcola la statistica univariata sulla colonna selezionata della tabella di un vettoriale \\
\hline
\end{tabular}
\end{table}

\clearpage

\subsection{Moduli negli strumenti GRASS per il 3D}

Questa Sezione elenca tutti i moduli con interfaccia grafica negli strumenti GRASS per lavorare con dati 3D. GRASS fornisce ulteriori moduli, attualmente disponibili solo tramite la shell.

\begin{table}[ht]
\centering
\caption{Strumenti GRASS: Visualizzazione 3D}\medskip
 \begin{tabular}{|p{4cm}|p{12cm}|}
  \hline \multicolumn{2}{|c|}{\textbf{Moduli per la visualizzazione e l'analisi 3D tra gli strumenti di GRASS}} \\
  \hline \textbf{Nome modulo} & \textbf{Scopo} \\
  \hline nviz & Apre la vista tridimensionale in nviz\\
\hline
\end{tabular}
\end{table}

\subsection{Moduli negli strumenti GRASS per l'aiuto in linea}

Il manuale di riferimento GRASS GIS Reference Manual offre una panoramica completa di tutti i moduli GRASS disponibili, senza limitarsi a quelli implementati, a volte con funzioni ridotte, negli strumenti di GRASS in QGIS. 

\begin{table}[ht]
\centering
\caption{Strumenti GRASS: manuale di riferimento}\medskip
 \begin{tabular}{|p{4cm}|p{12cm}|}
  \hline \multicolumn{2}{|c|}{\textbf{Moduli per consultare il Reference Manual tra gli strumenti di GRASS}} \\
  \hline \textbf{Nome modulo} & \textbf{Scopo} \\
  \hline g.manual & Mostra le pagine del manuale GRASS in formato HTML \\
\hline
\end{tabular}
\end{table}




