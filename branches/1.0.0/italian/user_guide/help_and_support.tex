% vim: set textwidth=78 autoindent:

\section{Aiuto e Supporto}\label{label_helpsupport}

% when the revision of a section has been finalized, 
% comment out the following line:
% \updatedisclaimer

\subsection{Mailinglist}
QGIS è attivamente in sviluppo e come tale non sempre funziona come ci si aspetterebbe. La maniera migliore di avere aiuto è registrarsi in una mailing list di utenti QGIS: qgis-users. 

\minisec{qgis-users}
Le tue domande raggiungerà un pubblico più ampio e della risposta beneficeranno anche altri. Ci si può registrare alla mailing list qgis-users vistitando il sito: \\
\url{http://lists.osgeo.org/mailman/listinfo/qgis-user}

\minisec{qgis-developer}
Se si è sviluppatori alle prese con problemi di natura più tecnica, si può registrarsi alla mailing list qgis-developer:\\
\url{http://lists.osgeo.org/mailman/listinfo/qgis-developer}

\minisec{qgis-commit}
Ogni volta che viene fatto un commit all'archivio codici QGIS, viene postata una e-mail in questa lista. Se si vuole essere informati di ogni cambiamento alla base codici attuale, si può registrarsi:\\
\url{http://lists.osgeo.org/mailman/listinfo/qgis-commit}

\minisec{qgis-trac}
Questa lista fornisce motifiche e-mail in relazione alla gestione del progetto, inclusi rapporti di malfunzionamenti, obiettivi, e richieste di funzioni e caratteristiche. Per registrarsi a questa lista:\\
\url{http://lists.osgeo.org/mailman/listinfo/qgis-trac}

\minisec{qgis-community-team}
Questa lista si occupa di argomenti come la documentazione, aiuto contestuale, guida utente, esperienza online incluso siti web, blog, mailing list, forum, e impegno di traduzione. Se si vuole anche lavorare su una guida utente, questa lista è un buon punto di partenza per far domande. si può registrarsi a:\\
\url{http://lists.osgeo.org/mailman/listinfo/qgis-community-team}

\minisec{qgis-release-team}
Questa lista si occupa di argomenti come il processo di rilascio, la creazione di pacchetti dei codici binari per i diversi sistemi operativi e l'annuncio al mondo del rilascio di nuove versioni. Si può sottoscrivere a:\\
\url{http://lists.osgeo.org/mailman/listinfo/qgis-release-team}

\minisec{qgis-psc}
Questa lista è usata per discutere questioni di pertinenza della Commissione Direttiva, cioè inerenti la generale gestione e direzione di Quantum GIS. Si può sottoscrivere a:\\
\url{http://lists.osgeo.org/mailman/listinfo/qgis-psc}

Siete i benvenuti a registrarvi in queste liste. Ricordate di contribuire alle liste fornendo risposte e condividendo le vostre esperienze. Tenete presente che qgis-commit e qgis-trac sono progettate come mezzo di notifica e non per accogliere post degli utenti. 

\subsection{IRC}
Siamo anche presenti su IRC - visitateci registrandovi al canale \#qgis su
\url{irc.freenode.net}. Si prega di attendere un po' per le risposte, dato che molte persone sul canler fanno anche altre cose e quindi ci può volere un po' di tempo per notare la vostra domanda. È disponibile anche un supporto commerciale per QGIS. Vedere questo sito \url{http://qgis.org/content/view/90/91} per ulteriori informazioni.

Se vi siete persi una discussione su IRC, non è un problema! Viene mantenuto registro di tutte le discussioni, così è più facile stare al passo. Basta andare su \url{http://logs.qgis.org} e leggere i log IRC.

\subsection{Tracciatore di malfunzionamenti (Bug Tracker)}
Mentre la mailing list qgis-users è utile per generiche domande "come fare xyz in
QGIS", potreste voler notificare noi per eventuali malfunzionamenti di QGIS. Potete farlo usando il tracciatore di malfunzionamenti a \url{https://trac.osgeo.org/qgis/}. Quando si crea un nuovo ticket per un malfunzionamento, fornire un indirizzo e-mail al quale noi possiamo richiedere informazioni aggiuntive.

Per favore ricordate che i vostri malfunzionamenti non sempre hanno la priorità che sperate (a seconda della gravità). Alcuni possono richiedere un significativo sforzo di sviluppo per rimediare e la forza lavoro non è sempre disponibile per questo.

Anche la richiesta di funzionalità può essere inoltrata usando lo stesso sistema di ticket come per i malfunzionamenti. Si prega di assicurarsi di selezionare il tipo \usertext{enhancement}.

Se avete trovato un malfunzionamento e l'avete risolto da soli, potete inoltrare anche questa riparazione. Di nuovo, il simpatico sistema di ticket a \url{https://trac.osgeo.org/qgis/} ha anche questa tipologia. Selezionare \usertext{patch} dal menu dei tipi. Qualcuno degli sviluppatori ne farà una revisione e lo applicherà in QGIS. \\
Per favore, non allarmatevi se la vostra riparazione non viene applicata subito - gli sviluppatori possono essere impegnati in altri lavori.
% unused, since community.qgis.org seems to be lost. (SH)
% There is also a community site for QGIS where we encourage QGIS users to share
% their experiences and provide case studies about how they are using QGIS. The
% community site is available at: http://community.qgis.org 

\subsection{Blog}
La comunità QGIS mantiene anche un weblog (BLOG) a \url{http://blog.qgis.org} che ha alcuni articoli interessanti per gli utenti come per gli sviluppatori. Siete invitati a contribuire al blog dopo esservi registrati!

\subsection{Wiki}
Infine, manteniamo anche un sito web WIKI a \url{http://wiki.qgis.org} dove potete trovare una varietà di utili informazioni correlate allo sviluppo di QGIS, ai piani di rilascio di nuove versioni, collegamenti ai siti da cui scaricare, consiglio di traduzione del messaggio e così via. Esplorateli, ci sono cose molto interessanti dentro.

