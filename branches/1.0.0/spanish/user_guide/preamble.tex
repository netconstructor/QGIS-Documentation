% vim: set textwidth=78 autoindent:

% QGIS Tips
% define tip float
% doesn't work if written in qgis_style.sty
% please keep the style definitions here and 
% and load float package in qgis_style.sty
\floatstyle{ruled}
\newfloat{Tip}{ht}{lox}
\floatname{Tip}{Tip}
\newcommand\qgistip[1]{\raggedright\small{#1}}
\renewcommand{\topfraction}{0.85}
\renewcommand{\textfraction}{0.1}
\renewcommand{\floatpagefraction}{0.75}

\thispagestyle{empty}
\addcontentsline{toc}{section}{Preámbulo}


%%%%%%%%%%% nothing to change above %%%%%%%%%%

\section*{Preámbulo}

% when the revision of a section has been finalized, 
% comment out the following line:
%\updatedisclaimer
\vspace{1cm}


Este documento es la guía original de usuario, instalación y codificación del 
software Quantum GIS. El software y hardware descrito en este  
documento son en la mayoría de los casos marcas registradas y están por tanto sujetas  
a requisitos legales. Quantum GIS está sujeto a la Licencia Pública General GNU (GNU General Public 
License). Puede encontrar más información en la web de Quantum GIS
\url{http://qgis.osgeo.org}.

Los detalles, datos, resultados, etc. que se ofrecen en este documento han sido 
escritos y verificados con el mejor conocimiento y resposabilidad de los autores y
editores. Sin embargo, son posibles errores respecto al contenido. 
Por eso, los datos no están sujetos a ningún derecho o garantía. Los autores y editores 
no se hacen responsables de fallos y 
sus consecuencias. Siempre se es bienvenido para indicar posibles errores.

Este documento ha sido preparado con \LaTeX~. Está disponible como código fuente \LaTeX~
desde \href{http://wiki.qgis.org/qgiswiki/DocumentationWritersCorner}{subversion} 
y en línea como documento PDF en \url{http://qgis.osgeo.org/documentation/manuals.html}. 
También se pueden descargar versiones traducidas de este documento desde el área de documentación 
del proyecto QGIS. Más información sobre cómo contribuir a este documento y sobre 
su traducción está disponible en: \url{http://wiki.qgis.org/qgiswiki/DocumentationWritersCorner} 

\textbf{Enlaces en este documento}

Este documento contiene enlaces internos y extermos. Pulsando en un
enlace interno se desplaza dentro del documento, mientras que pulsando en un enlace externo
se abre una dirección de internet. En formato PDF, los enlaces internos se muestran en azul y
los enlaces externos en rojo y se abren con el explorador del 
sistema. En formato HTML, el explorador muestra y maneja los dos de la misma forma. 

\begin{flushleft}
\textbf{Autores y editores de la Guía de Usuario, Instalación y Codificación:}
 
\begin{tabular}{p{5cm} p{5cm} p{5cm}}
Tara Athan & Radim Blazek & Godofredo Contreras \\
Otto Dassau & Martin Dobias & J\"urgen E. Fischer \\ 
Stephan Holl & Marco Hugentobler & Magnus Homann \\ 
Lars Luthman & Gavin Macaulay & Werner Macho \\
Tyler Mitchell & Brendan Morely & Gary E. Sherman \\ 
Tim Sutton & David Willis &  \\
\end{tabular}

%\vspace{3cm}

Con agradecimientos a Tisham Dhar por preparar la documentación del entorno inicial msys (MS Windows), a Tom 
Elwertowski y William Kyngesburye por ayudar en la Sección de Instalación en MAC OSX y a Carlos Dávila, Paolo 
Cavallini y Christian Gunning por las revisiones. Si se nos olvida mencionar a algún 
colaborador, por favor acepte nuestras disculpas por el descuido.

\textbf{Copyright \copyright~2004 - 2009 Quantum GIS Development Team} \\
\textbf{Internet:} \url{http://qgis.osgeo.org}
\end{flushleft}

