% vim: set textwidth=78 autoindent:

\section{Prefacio}\label{label_forward}
\pagenumbering{arabic}
\setcounter{page}{1}

% when the revision of a section has been finalized, 
% comment out the following line:
% \updatedisclaimer

¡Bienvenido al maravilloso mundo de los Sistemas de Información Geográfica (SIG)!
Quantum GIS (QGIS) es un Sistema de Información Geográfica de Código Abierto. 
El proyecto nació en Mayo de 2002 y se estableció como un proyecto dentro de 
SourceForge en junio del mismo año. Hemos trabajado duro para hacer del 
software SIG (que tradicionalmente es software propietario caro) una posibilidad
viable para cualquiera con un acceso básico a un ordenador personal. 
Actualmente QGIS funciona en la mayoría de plataformas Unix, Windows y OS X. 
QGIS está desarrollado utilizando el Qt toolkit (\url{http://www.trolltech.com})
y C++. Esto hace que QGIS sea rápido y tenga una interfaz de usuario agradable
y fácil de usar.

QGIS espera ser un SIG fácil de usar, proporcionando características y 
funciones comunes. El objetivo incial fue proporcionar un visor de datos SIG. 
QGIS ha alcanzado un punto en su evolución en el que está siendo utilizado por muchos 
para sus necesidades diarias de visualización de datos SIG. QGIS admite un 
buen número de formatos ráster y vectoriales, con posibilidad de añadir nuevos formatos fácilmente
utilizando su arquitectura de complementos (ver Apéndice \ref{appdx_data_formats} 
para consultar la lista completa de los formatos de datos admitidos).

QGIS se publica bajo Licencia Pública General (GNU General Public License) (GPL). 
Desarrollar QGIS bajo esta licencia quiere decir que se puede inspeccionar y 
modificar el código fuente y garantiza que nuestros felices 
usuarios siempre tendrán acceso a un programa SIG gratuito y que puede ser 
modificado libremente. Debe haber recibido una copia de la licencia con su copia 
de QGIS y tampién puede encontrarla en el Apéndice \ref{gpl_appendix}.

\begin{Tip}\caption{\textsc{Documentación actualizada}}\index{documentation}
\qgistip{La última versión de este documento siempre se puede encontrar en 
\url{http://download.osgeo.org/qgis/doc/manual/}, o en el área de documentación
de la web de QGIS en \url{http://qgis.osgeo.org/documentation/}
}
\end{Tip}

\subsection{Características}\label{label_majfeat}

QGIS ofrece muchas características SIG comunes proporcionadas por las funciones de núcleo y
los complementos. Como breve resumen se presentan en seis categorías para tener una 
primera idea.

\minisec{Ver datos}

Puede ver y superponer datos vectoriales y ráster en diferentes formatos y
proyecciones sin conversión a un formato interno o común. Los formatos
admitidos incluyen:

\begin{itemize}
\item Tablas de PostgreSQL con capacidad espacial usando PostGIS, formatos vectoriales
\footnote{Formatos de base de datos admitidos por OGR como Oracle o mySQL todavía no son
admitidos en QGIS.} admitidos por la biblioteca OGR instalada, incluyendo archivos shape
de ESRI, MapInfo, SDTS y GML.
\item Formatos ráster e imágenes admitidas por la bibliteca GDAL (Geospatial
Data Abstraction Library) instalada, tales como 
GeoTiff, Erdas Img., ArcInfo Ascii Grid, JPEG, PNG,
\item Datos ráster y vectoriales de GRASS de bases de datos de GRASS (localización/directorio de mapas), 
\item Datos espaciales en línea suministrados como Servicios de Mapas Web (WMS) o
Servicios de Elementos Web (WFS) que cumplan el estándar OGC.
\end{itemize}

\minisec{Explorar datos y diseñar mapas} 

Puede diseñar mapas y explorar datos espaciales de forma interactiva con una interfaz
amigable. Entre las muchas herramientas útiles disponibles en la interfaz están:

\begin{itemize}
\item Proyecciones al vuelo.
\item Diseñador de mapas.
\item Panel de vista general.
\item Marcadores espaciales.
\item Identificar/Seleccionar elementos.
\item Editar/Visualizar/Buscar atributos.
\item Etiquetado de objetos espaciales.
\item Cambiar simbología vectorial y ráster.
\item Añadir una capa de cuadrícula.
\item Decorar el mapa con una flecha de Norte, barra de escala y etiqueta de copyrigh.
\item Guardar y recuperar proyectos.
\end{itemize}

\minisec{Crear, editar, administrar y exportar datos}

Puede crear, editar, administrar y exportar mapas vectoriales en varios formatos. Los datos ráster
hay que importarlos a GRASS para poder editarlos y exportarlos a otros
formatos. QGIS ofrece lo siguiente: 

\begin{itemize}
\item Herramientas de digitalización para formatos admitidos por OGR y capas vectoriales de GRASS.
\item Crear y editar archivos shape y capas vectoriales de GRASS.
\item Geocodificar imágenes con el complemento Georreferenciador.
\item Herramientas GPS para importar y exportar formato GPX y convertir otros formatos GPS
a GPX o descargar/subir directamente a una unidad GPS.
\item Crear capas PostGIS a partir de archivos shape con el complemento SPIT.
\item Administrar tablas de atributos vectoriales con el complemento Table manager.
\end{itemize}

\minisec{Analizar datos}

Puede realizar análisis de datos espaciales de PostgreSQL/PostGIS y de otros formatos
admitidos por OGR usando el complemento de python fTools. QGIS actualmente ofrece
herramientas de análisis vectorial, muestreo, geoprocesamiento, geometría y administración de bases de datos. También puede usar las herramientas de GRASS integradas, que 
incluyen la funcionalidad completa de GRASS de más de 300 módulos (Ver
Sección \ref{sec:grass}).

\minisec{Publicar mapas en internet}

QGIS se puede usar para exportar datos a un archivo mapfile y para publicarlos en
internet usando un servidor web con UMN MapServer instalado. QGIS también se
puede usar como cliente WMS o WFS y como servidor WMS.

\minisec{Ampliar la funcionalidad de QGIS mediante complementos} 

QGIS se puede adaptar a sus necesidades especiales con la arquitectura extensible
de complementos. QGIS proporciona bibliotecas que se pueden usar para crear
complementos. ¡Puede incluso crear nuevas aplicaciones con C++ o Python!

\begin{itemize}
\item \textbf{Complementos del núcleo}
\\ Añadir capas WFS.
\\ Añadir capas de texto delimitado.
\\ Decoración (etiqueta de copyright, flecha de Norte y barra de escala)
\\ Georreferencación.
\\ Conversor DxfaShp.
\\ Herramientas GPS.
\\ Integración de GRASS.
\\ Generador de mallas.
\\ Complemento de interpolación.
\\ Conversor de capas OGR.
\\ Impresión rápida.
\\ Herramienta de importación de archivos shape a PostgreSQL/PostGIS (SPIT - Shapefile to PostgreSQL/PostGIS Import Tool).
\\ Exportación a Mapserver.
\\ Consola de Python.
\\ Instalador de complementos de Python.
\\ \item \textbf{Complementos de Python}
\\ \\ QGIS ofrece un creciente número de complementos externos de python que son aportados 
por la comunidad. Estos complementos se encuentran en el repositorio oficial
de complementos de PyQGIS y se pueden instalar fácilmente usando el instalador de 
complementos de python (Ver Sección \ref{sec:plugins}).
\end{itemize}

