%  !TeX  root  =  user_guide.tex  
\mainmatter
\pagestyle{scrheadings}
\addchap{Foreword}\label{label_forward}


% when the revision of a section has been finalized, 
% comment out the following line:
% \updatedisclaimer

Welcome to the wonderful world of Geographical Information Systems (GIS)!
Quantum GIS (QGIS) is an Open Source Geographic Information System. The project
was born in May of 2002 and was established as a project on SourceForge in June
of the same year. We've worked hard to make GIS software (which is traditionally
expensive proprietary software) a viable prospect for anyone with basic access
to a Personal Computer. QGIS currently runs on most Unix platforms, Windows, and
OS X. QGIS is developed using the Qt toolkit (\url{http://qt.nokia.com})
and C++. This means that QGIS feels snappy to use and has a pleasing, 
easy-to-use graphical user interface (GUI). 

QGIS aims to be an easy-to-use GIS, providing common functions and features.
The initial goal was to provide a GIS data viewer. QGIS has reached the point
in its evolution where it is being used by many for their daily GIS data viewing
needs. QGIS supports a number of raster and vector data formats, with new
format support easily added using the plugin architecture (see Appendix
\ref{appdx_data_formats} for a full list of currently supported data formats).

QGIS is released under the GNU General Public License (GPL). Developing QGIS 
under this license means that you can inspect and modify the source code,
and guarantees that you, our happy user, will always have access to a GIS
program that is free of cost and can be freely modified. You should have
received a full copy of the license with your copy of QGIS, and you also can
find it in Appendix \ref{gpl_appendix}.  

\begin{Tip}\caption{\textsc{Up-to-date Documentation}}\index{documentation}
The latest version of this document can always be found at 
\url{http://download.osgeo.org/qgis/doc/manual/}, or in the documentation
area of the QGIS website at \url{http://qgis.osgeo.org/documentation/}
\end{Tip}

\addsec{Features}\label{label_majfeat}

\qg offers many common GIS functionalities provided by core features and
plugins. As a short summary they are presented in six categories to gain a
first insight.

\minisec{View data}

You can view and overlay vector and raster data in different formats and
projections without conversion to an internal or common format. Supported
formats include:

\begin{itemize}[label=--]
\item spatially-enabled PostgreSQL tables using PostGIS, vector 
formats
%\footnote{OGR-supported database formats such as Oracle or 
%mySQL are not yet supported in QGIS.}
 supported by the installed OGR library, including ESRI shapefiles, MapInfo, 
SDTS and GML (see Appendix \ref{appdx_ogr} for the complete list) .
\item Raster and imagery formats supported by the installed GDAL (Geospatial
Data Abstraction Library) library, such as GeoTiff, Erdas Img., ArcInfo Ascii 
Grid, JPEG, PNG (see Appendix \ref{appdx_gdal} for the complete list).
\item SpatiaLite databases (see Section \ref{label_spatialite}) 
\item GRASS raster and vector data from GRASS databases (location/mapset),
see Section \ref{sec:grass}, 
\item Online spatial data served as OGC-compliant Web Map Service (WMS) or
Web Feature Service (WFS), see Section \ref{working_with_ogc},
\item OpenStreetMap data (see Section \ref{plugins_osm}).
\end{itemize}

\minisec{Explore data and compose maps} 

You can compose maps and interactively explore spatial data with a friendly
GUI. The many helpful tools available in the GUI include:

\begin{itemize}[label=--]
\item on the fly projection
\item map composers
\item overview panel
\item spatial bookmarks
\item identify/select features
\item edit/view/search attributes
\item feature labeling
\item change vector and raster symbology
\item add a graticule layer - now via fTools plugin
\item decorate your map with a north arrow scale bar and copyright label
\item save and restore projects
\end{itemize}

\minisec{Create, edit, manage and export data}

You can create, edit, manage and export vector maps in several formats. Raster data
have to be imported into GRASS to be able to edit and export them into other
formats. QGIS offers the following: 

\begin{itemize}[label=--]
\item digitizing tools for OGR supported formats and GRASS vector layer
\item create and edit shapefiles and GRASS vector layers
\item geocode images with the Georeferencer plugin
\item GPS tools to import and export GPX format, and convert other GPS
formats to GPX or down/upload directly to a GPS unit (on Linux, usb: has been added
to list of GPS devices)
\item visualize and edit OpenStreetMap data
\item create PostGIS layers from shapefiles with the SPIT plugin 
\item improved handling of PostGIS tables
\item manage vector attribute tables with the new attribute table (see Section 
\ref{sec:attribute table}) or Table Manager plugin
\item save screenshots as georeferenced images
\end{itemize}

\minisec{Analyse data} 

You can perform spatial data analysis on PostgreSQL/PostGIS and other OGR
supported formats using the fTools Python plugin. QGIS currently offers
vector analysis, sampling, geoprocessing, geometry and database management
tools. You can also use the integrated GRASS tools, which 
include the complete GRASS functionality of more than 300 modules (See
Section \ref{sec:grass}).

\minisec{Publish maps on the Internet}

QGIS can be used to export data to a mapfile and to publish them on the
Internet using a webserver with UMN MapServer installed. QGIS can also
be used as a WMS or WFS client, and as WMS server. 

\minisec{Extend QGIS functionality through plugins} 

QGIS can be adapted to your special needs with the extensible
plugin architecture. QGIS provides libraries that can be used to create
plugins.  You can even create new applications with C++ or Python!

\minisec{Core Plugins}

\begin{enumerate}
\item Add Delimited Text Layer (Loads and displays delimited text files
containing x,y coordinates)
\item Coordinate Capture (Capture mouse coordinates in different CRS)
\item Decorations (Copyright Label, North Arrow and Scale bar)
\item Diagram Overlay (Placing diagrams on vector layer)
\item Dxf2Shp Converter (Convert DXF to Shape)
\item GPS Tools (Loading and importing GPS data)
\item GRASS (GRASS GIS integration)
\item Georeferencer GDAL (Adding projection information to raster using GDAL)
\item Interpolation plugin (interpolate based on vertices of a vector layer)
\item Mapserver Export (Export QGIS project file to a MapServer map file)
\item OGR Layer Converter (Translate vector layer between formats)
\item OpenStreetMap plugin (Viewer and editor for openstreetmap data)
\item Oracle Spatial GeoRaster support
\item Python Plugin Installer (Download and install QGIS python plugins)
\item Quick Print (Print a map with minimal effort)
\item Raster terrain analysis (Raster based terrain analysis)
\item SPIT (Import Shapefile to PostgreSQL/PostGIS)
\item WFS Plugin (Add WFS layers to QGIS canvas)
\item eVIS (Event Visualization Tool)
\item fTools (Tools for vector data analysis and management)
\item Python Console (Access QGIS environment)
\item GDAL Tools
\end{enumerate}

\minisec{External Python Plugins}

QGIS offers a growing number of external python plugins that are provided by
the community. These plugins reside in the official PyQGIS repository, and
can be easily installed using the Python Plugin Installer (See Section
\ref{sec:plugins}).

\subsubsection{What's new in version \CURRENT} 

These are the most relevant additions and improvements for QGIS users:
\begin{itemize}[label=--]
%  \item TODO(anne): rearrange the list on ManualTasks wikipage!
 \item Added context help to dialogs
 \item Measure angle tool
 \item GPS tracker widget
 \item Several new external plugins "Spatial Query, GDAL Tools, Google Earth \dots"
 \item Add attribute table to the map composer
 \item Vector layer saving in editing mode 
 \item Moved new labeling plugin to core and show labels from new labeling in map composer (though not scaled correctly)
 \item Added "Drag and drop support" for composer legend model
 \item Extended georeferencer plugin with new user interface
 \item BBOX option for WFS 
 \item Join attributes tool now supports both dbf and csv files % table join rulez!
 \item Show feature count in attribute table title
 \item Python console runnable also outside QGIS
 \item New text annotation tool
 \item New WMS scale slider
 \item Projection search improvements (show/hide deprecated projections)
 \item Add GDAL compatible CRSes (without +towgs84 parameters) for Polish epsg: 2172-2180, 3120, 3328-3335 and 4179 as epsg+40000
 \item Change map unit in project dialog automatically when user selects a new CRS
 \item New Oracle raster connection dialog 
 \item Add New Spatialite Layer/Database and allow creation of multiple spatialite layers

\end{itemize}
