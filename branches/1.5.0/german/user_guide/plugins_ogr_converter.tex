% vim: set textwidth=78 autoindent:

\section{OGR-Layer-Konverter Plugin}
\index{Plugins!OGR-Layer-Konverter}

% when the revision of a chapter has been finalized, 
% comment out the following line:
%\updatedisclaimer

Das Plugin \toolbtntwo{ogr_converter}{OGR-Layer-Konverter} erm�glicht es,
Vektorlayer von einem OGR-unterst�tzten Vektorformat in ein anderes durch OGR
unterst�tztes Vektorformat zu konvertieren. Es ist sehr einfach zu
verwenden und bietet Funktionalit�ten wie in Abbildung
\ref{fig:ogrconverter_dialog} dargestellt. Die unterst�tzten Vektorformate
k�nnen je nach GDAL Installation unterschiedlich sein.

\begin{itemize}[label=--]
\item \textbf{Quelle Format/Datensatz/Layer}: Geben Sie das OGR-Format und
den Datensatz der konvertiert werden soll an.
\item \textbf{Ziel Format/Datensatz/Layer}: Geben Sie das OGR-Format und die
Vektor Ausgabedatei an.
\end{itemize}

\begin{figure}[ht]
   \begin{center}
   \caption{OGR-Layer-Konverter Plugin \nixcaption}
   \label{fig:ogrconverter_dialog}\smallskip
   \includegraphics[clip=true, width=7cm]{ogr_converter_dialog}
\end{center}  
\end{figure}

\minisec{Das Plugin anwenden}

\begin{enumerate}
\item Starten Sie QGIS, laden Sie das OGR-Konverter Plugin mit dem Plugin
Manager (siehe Kapitel~\ref{sec:load_core_plugin}) und klicken Sie auf das
Icon \toolbtntwo{ogr_converter}{OGR Layer Converter} in der Werkzeugleiste.
\item W�hlen Sie im Bereich Quelle als Vektorformat 
\selectstring{ESRI Shapefile}{} und als Datensatz \filename{alaska.shp}.
\item W�hlen Sie im Bereich Ziel als Vektorformat \selectstring{GML}{} 
und als Ausgabe \filename{alaska.gml}.
\item Klicken Sie auf \button{Ok}.
\end{enumerate}

