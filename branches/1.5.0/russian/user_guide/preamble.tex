%  !TeX  root  =  user_guide.tex
\frontmatter
\pagestyle{scrplain}
\addchap{Преамбула}
\vspace{1cm}

% when the revision of a section has been finalized,
% comment out the following line:
%\updatedisclaimer

Данный документ представляет собой перевод оригинального руководства
пользователя Quantum~GIS на русский язык. Программное обеспечение и
аппаратные средства, описанные в этом документе, в большинстве случаев
являются зарегистрированными торговыми марками, и, следовательно, являются
субъектами правового регулирования. Исходный код Quantum~GIS подлежит
лицензированию в соответствии с GNU General Public License. Подробную
информацию можно получить на домашней странице Quantum~GIS
\url{http://www.qgis.org}.
\par\bigskip
Подробная информация, данные, результаты и прочее в данном документе
были написаны и проверены в меру знаний и ответственности авторов и
редакторов. Тем не менее, в содержании документа возможны ошибки.
\par\bigskip
Таким образом, каких-либо гарантий или обязательств
относительно всей представленной здесь информации не предоставляется.
Авторы, редакторы и издатели не несут какой-либо ответственности за ошибки
и их последствия. Тем не менее, вы всегда можете указать на возможные ошибки.
\par\bigskip
Этот документ был создан с помощью системы компьютерной верстки \LaTeX.
Документ доступен в виде исходных кодов \LaTeX посредством
\href{http://wiki.qgis.org/qgiswiki/DocumentationWritersCorner}{subversion}
и как PDF документ на странице
\url{http://qgis.osgeo.org/documentation/manuals.html}.
Локализованные версии данного документа также можно загрузить со
страницы документации проекта QGIS.

%%% С русскоязычной версией руководства в форматах PDF и HTML можно
Русскоязычную версию руководства в формате PDF можно получить по адресу:\\
\url{http://gis-lab.info/docs/qgis/manual15/qgis-1.5.0_user_guide_ru.pdf}

\textbf{Версия руководства, которую вы держите сейчас в руках, является
предварительной. Для получения окончательной версии, пожалуйста,
воспользуйтесь одной из ссылок выше.}

Для получения подробной информации о сотрудничестве и локализации
посетите: \url{http://www.qgis.org/wiki/}

\vspace{1cm}
\noindent
\textbf{Ссылки в этом документе}
\par\bigskip
Этот документ содержит внутренние и внешние ссылки. При нажатии на
внутреннюю ссылку перемещение происходит внутри документа, в то время
как при нажатии на внешнюю ссылку"--- открывается адрес в сети Интернет.
В документе, представленном в формате PDF, внутренние ссылки показаны
синим цветом, тогда как внешние ссылки показаны красным цветом и
обрабатываются интернет-браузером, назначенным в системе по умолчанию. В
документе, представленном в формате HTML, интернет-браузер отображает и
обрабатывает внутренние и внешние ссылки одинаково.

\newpage

\begin{flushleft}
\textbf{Руководство пользователя, Руководство по установке и Руководство
по программированию"--- авторы и редакторы:}
\par\bigskip\noindent
\begin{tabular}{p{4cm} p{4cm} p{4cm}}
Tara Athan & Radim Blazek & Godofredo Contreras \\
Otto Dassau & Martin Dobias & Peter Ersts \\
Anne Ghisla & Stephan Holl & N. Horning \\
Magnus Homann & K. Koy & Lars Luthman \\
Werner Macho & Carson J.Q. Farmer & Tyler Mitchell \\
Claudia A. Engel & Brendan Morely & David Willis \\
Jurgen E. Fischer & Marco Hugentobler & Gavin Macaulay \\
Gary E. Sherman & Tim Sutton \\ \
\end{tabular}
\end{flushleft}

С благодарностями Bertrand Masson за макет, Tisham Dhar за подготовку
документации по MSYS (MS Windows), Tom Elwertowski и William Kyngesburye
за помощь в разделе <<Установка на MAC OSX>>, Carlos Davila, Paolo
Cavallini и Christian Gunning за проверку и исправления. Если мы
забыли упомянуть кого-либо из участников, пожалуйста, примите наши
извинения за это упущение.
\par\bigskip\noindent
\textbf{Copyright \copyright~2004--2010 \QG Development Team}
\par\bigskip\noindent
\textbf{Адрес в сети Интернет :} \url{http://www.qgis.org}
\par\bigskip\noindent

Перевод на русский язык выполнен в рамках
\href{http://gis-lab.info/docs/qgis/manual15.html}{коллективного проекта}
на ГИС-Лаб. Участники:
\begin{itemize}[label=--]
\item \href{http://gis-lab.info/forum/memberlist.php?mode=viewprofile&u=5325}{voltron}
"--- разделы 8, 12-3, общая координация
\item \href{http://gis-lab.info/forum/memberlist.php?mode=viewprofile&u=7967}{wickedshark}
"--- Преамбула, Предисловие, Элементы, разделы 1-2, локализованные скриншоты
\item \href{http://gis-lab.info/forum/memberlist.php?mode=viewprofile&u=7619}{Рябов Ю.\,В.}
"--- разделы 3.1-3.3, 3.6-3.7
\item \href{http://gis-lab.info/forum/memberlist.php?mode=viewprofile&u=9954}{Виктор Колесник}
"--- раздел 3.4
\item \href{http://gis-lab.info/forum/memberlist.php?mode=viewprofile&u=7392}{Евгения Селезнева}
"--- раздел 3.5
\item \href{http://gis-lab.info/forum/memberlist.php?mode=viewprofile&u=8193}{Ткаченко Павел}
"--- разделы 4 и 9
\item \href{http://gis-lab.info/forum/memberlist.php?mode=viewprofile&u=6901}{Денис Рыков}
"--- разделы 5, 6, 10.5
\item \href{http://gis-lab.info/forum/memberlist.php?mode=viewprofile&u=8430}{Александр Мурый} (amuriy)
"--- раздел 7, приложение B, вычитка и общая редакция (все разделы)
\item \href{http://gis-lab.info/forum/memberlist.php?mode=viewprofile&u=9129}{Сергей Гордин} (oxch)
"--- разделы 10.1-10.4
\item \href{http://gis-lab.info/forum/memberlist.php?mode=viewprofile&u=1394}{Alexander Manisha}
"--- разделы 10.6-10.7
\item \href{http://gis-lab.info/forum/memberlist.php?mode=viewprofile&u=9850}{h1-tek\_deamon}
"--- разделы 10.8-10.13
\item \href{http://gis-lab.info/forum/memberlist.php?mode=viewprofile&u=2}{Максим Дубинин}
"--- раздел 10.14
\item \href{http://gis-lab.info/forum/memberlist.php?mode=viewprofile&u=9719}{Хмелевский Андрей}
"--- разделы 10.15-10.16 и 11
\item \href{http://gis-lab.info/forum/memberlist.php?mode=viewprofile&u=9876}{Mike E. Semenov}
"--- приложение А
\item \href{http://gis-lab.info/forum/memberlist.php?mode=viewprofile&u=7246}{Артём Попов}
"--- вычитка и редакция (Введение, разделы 1.2-1.5, 2, 4, частично 3)
\end{itemize}

\newpage

\addsec{Лицензия этого документа}

Разрешается копировать, распространять и/или изменять этот документ в
соответствии с условиями GNU Free Documentation License, версии 1.3 или
более поздней, опубликованной Free Software Foundation; без каких-либо
неизменяемых разделов, текста, помещаемого на первой странице обложки, и
без текста, помещаемого на последней странице обложки. Копия текста
лицензии представлена в Разделе~\ref{label_fdl}, озаглавленном <<GNU Free
Documentation License>>.

\newpage
