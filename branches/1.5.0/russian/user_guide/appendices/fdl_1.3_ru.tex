\chapter{Лицензия GNU на Свободную Документацию}\label{label_fdl_ru}
\index{license!FDL}

Настоящий перевод GFDL версии 1.2 приводится исключительно для
ознакомления. Руководство пользователя QGIS распространяется по условиям
GNU Free Documentation License 1.3.

\bigskip

This is an unofficial translation of the GNU Free Documentation License
(GFDL) into Russian. It was not published by the Free Software
Foundation, and does not legally state the distribution terms for works
that uses the GFDL"--- only the original English text of the GFDL does
that. However, we hope that this translation will help Russian speakers
understand the GFDL better.

Настоящий перевод Лицензии GNU на Свободную Документацию (GFDL) на
русский язык не является официальным. Он не публикуется Free Software
Foundation и не устанавливает имеющих юридическую силу условий для
распространения произведений, которые распространяются на условиях GFDL.
Условия, имеющие юридическую силу, закреплены исключительно в аутентичном
тексте GFDL на английском языке. Я надеюсь, что настоящий перевод поможет
русскоязычным пользователям лучше понять содержание GFDL.

Автор перевода GFDL версии 1.1 Елена Тяпкина (tiapkina@hotmail.com),
09 августа 2001
Автор перевода GFDL версии 1.2 (с использованием перевода версии 1.1)"---
Владимир Медейко (\url{http://ru.wikipedia.org/wiki/User_talk:Drbug}),
7 августа 2003

 \begin{center}

       Версия 1.2, ноябрь 2002 г.


 Copyright \copyright{} 2000, 2001, 2002 Free Software Foundation, Inc.\\
 59 Temple Place - Suite 330, Boston, MA 02111-1307, USA

 \bigskip

     <http://fsf.org/>

 \bigskip

 Каждый вправе копировать и распространять экземпляры настоящей
 Лицензии без внесения изменений в ее текст.
\end{center}


\begin{center}
{\bf\large 0. Преамбула}
\end{center}

Цель настоящей Лицензии"--- сделать свободными справочники, руководства
пользователя или иные функциональные и полезные документы в письменной
форме, то есть обеспечить каждому право свободно копировать и
распространять как с изменениями, так и без изменений, за вознаграждение
или бесплатно указанные документы. Настоящая Лицензия также позволяет
авторам или издателям документа сохранить свою репутацию, не принимая на
себя ответственность за изменения, сделанные третьими лицами.

Настоящая Лицензия относится к категории <<copyleft>>. Это означает, что
все произведения, производные от документа, должны быть свободными в
соответствии с концепцией <<copyleft>>. Настоящая Лицензия дополняет
General Public License GNU, которая является лицензией <<copyleft>>,
разработанной для свободного программного обеспечения.

Настоящая Лицензия разработана для применения ее к документации на
свободное программное обеспечение, поскольку свободное программное
обеспечение должно сопровождаться свободной документацией. Пользователь
должен обладать теми же правами в отношении руководства пользователя,
какими он обладает в отношении свободного программного обеспечения. При
этом действие настоящей Лицензии не распространяется только на
руководство пользователя. Настоящая Лицензия может применяться к любому
текстовому произведению независимо от его темы или от того, издано ли
данное произведение в виде печатной книги или нет. Настоящую Лицензию
рекомендуется применять для произведений справочного или обучающего
характера.


\begin{center}
{\Large\bf 1. СФЕРА ДЕЙСТВИЯ, ТЕРМИНЫ И ИХ ОПРЕДЕЛЕНИЯ\par}
\end{center}

Условия настоящей Лицензии применяются к любому руководству пользователя
или иному произведению на любом носителе, которое в соответствии с
уведомлением, помещенным правообладателем, может распространяться на
условиях настоящей Лицензии. Таковое уведомление предоставляет всемирную,
свободную от выплат и неограниченную по сроку действия лицензию на
использование такового произведения на определённых в данном соглашении
условиях. Далее под термином <<Документ>> понимается любое подобное
руководство пользователя или произведение. Лицо, которому передаются
права по настоящей Лицензии, в дальнейшем именуется <<Лицензиат>>.
Лицензиат принимает условия этой лицензии если он копирует, модифицирует
или распространяет произведение способом, требующим разрешения в
соответствии с законодательством об авторском праве.

<<Модифицированная версия Документа>>"--- любое произведение, содержащее
Документ или его часть, скопированные как с изменениями, так и без них
и/или переведенные на другой язык.

<<Второстепенный раздел>>"--- имеющее название приложение или предисловие
к Документу, в котором отражено исключительно отношение издателей или
авторов Документа к его содержанию в целом, либо к вопросам, связанным с
содержанием Документа. Второстепенный раздел не может включать в себя то,
что относится непосредственно к содержанию Документа. (То есть, если
Документ является частью учебника по математике, во Второстепенном
разделе не может содержаться что-либо имеющее отношение непосредственно
к математике). Во Второстепенных разделах могут быть затронуты вопросы
истории того, что составляет содержание или что связано с содержанием
Документа, а также правовые, коммерческие, философские, этические или
политические взгляды относительно содержания Документа.

<<Неизменяемые разделы>>"--- определенные Второстепенные разделы,
названия которых перечислены как Неизменяемые разделы в уведомлении
Документа, определяющем лицензионные условия. Если раздел не
удовлетворяет приведённому выше определению Второстепенного раздела, то
он не может быть назван Неизменяемым. Документ может не содержать
Неизменяемых разделов. В случае, если в Документе не перечисляются какие
бы то ни было неизменяемые разделы, то такие разделы отсутствуют.

<<Текст, помещаемый на обложке>>"--- определенные краткие строки текста,
которые перечислены в уведомлении Документа, определяющем лицензионные
условия, как текст, помещаемый на первой и последней страницах обложки.
Текст, помещаемый на первой странице обложки, не может быть длиннее 5
слов, а текст, помещаемый на последней странице обложки, не может
содержать более 25 слов.

<<Прозрачный>> экземпляр Документа"--- экземпляр Документа в
машиночитаемой форме, представленный в формате с общедоступной
спецификацией, подходящим для просмотра и исправлений, при условии, что
документ может просматриваться и редактироваться непосредственно с
помощью общедоступных текстовых редакторов или общедоступных программ
для векторной или растровой графики (в случае, если в документе
содержатся изображения векторной или растровой графики). Указанный
формат должен обеспечить ввод текста Документа в программы форматирования
текста или автоматический перевод Документа в различные форматы,
подходящие для ввода текста Документа в программы форматирования текста.
Экземпляр Документа, представленный в ином формате, разметка или
отсутствие разметки которого затрудняет или препятствует внесению в
Документ последующих изменений пользователями, не является Прозрачным.
Графический формат не является Прозрачным, если он применён для
сколько-нибудь значительного количества текста. Экземпляр документа, не
являющийся Прозрачным, называется <<Непрозрачным>>.

Форматы, в которых может быть представлен Прозрачный экземпляр Документа,
включают простой формат ASCII без разметки, формат ввода Texinfo, формат
ввода LaTeX, SGML или XML с использованием общедоступного DTD, а также
соответствующий стандартам простой формат HTML, PostScript и PDF,
предназначений для внесения модификаций человеком. В число графических
форматов, являющихся Прозрачными, входят PNG, XCF и JPG. <<Непрозрачные>>
форматы включают в себя форматы, которые можно прочитать и редактировать
только с помощью текстовых редакторов, права на использование которых
свободно не передаются, форматы SGML или XML, для которых DTD или
инструменты для обработки не являются общедоступными, а также
генерируемый компьютером HTML, Postscript или PDF, который вырабатывается
некоторыми текстовыми редакторами исключительно в целях отображения.

<<Титульный лист>>"--- для печатной книги собственно титульный лист, а
также следующие за ним страницы, которые должны содержать сведения,
помещаемые на титульном листе в соответствии с условиями настоящей
Лицензии. Для произведений, формат которых не предполагает наличие
титульного листа, под Титульным листом понимается текст, который помещен
перед началом основного текста произведения, после его названия,
напечатанного наиболее заметным шрифтом.

Раздел, <<Озаглавленный XYZ>> означает подраздел Документа, который
озаглавлен либо точно XYZ, либо содержит XYZ в скобках, которые
сопровождают текст-перевод XYZ на другой язык. (Здесь XYZ означает
конкретное название подраздела, упомянутое ниже, такое как
<<Благодарности>>, <<Посвящения>>, <<Одобрения>> или <<История>>.)
<<Сохранять название>> такого раздела при модицировании Документа,
означает, что он остаётся разделом, <<Озаглавленным XYZ>> в соответствии
с этим определением.

Документ может включать Отказ от ответственности после уведомления о
том, что данная Лицензия применяется к Документу. Эти Отказы от
ответственности как включённые в данную Лицензию посредством ссылки, но
только в качестве отказов от ответственности"--- любые другие значения,
которые эти Отказы от ответственности могут иметь"--- ничтожны и не
оказывают влияния на значение данной Лицензии.


\begin{center}
{\Large\bf 2. КОПИРОВАНИЕ БЕЗ ВНЕСЕНИЯ ИЗМЕНЕНИЙ\par}
\end{center}

Лицензиат вправе воспроизводить и распространять экземпляры Документа на
любом носителе за вознаграждение или безвозмездно при условии, что
каждый экземпляр содержит текст настоящей Лицензии, знаки охраны
авторских прав, а также уведомление, что экземпляр распространяется в
соответствии с настоящей Лицензией, при этом Лицензиат не вправе
предусматривать иные лицензионные условия дополнительно к тем, которые
закреплены в настоящей Лицензии. Лицензиат не вправе использовать
технические средства для воспрепятствования или контроля за чтением или
последующим изготовлением копий с экземпляров, распространяемых
Лицензиатом. Лицензиат вправе получать вознаграждение за изготовление и
распространение экземпляров Документа. При распространении большого
количества экземпляров Документа Лицензиат обязан соблюдать условия
пункта 3 настоящей Лицензии.

Лицензиат вправе сдавать экземпляры Документа в прокат на условиях,
определенных в предыдущем абзаце, или осуществлять публичный показ
экземпляров Документа.


\begin{center}
{\Large\bf 3. ТИРАЖИРОВАНИЕ\par}
\end{center}


Если Лицензиат издает печатные экземпляры (или экземпляры на носителе,
обычно имеющем печатные обложки) Документа в количестве свыше 100, и в
соответствии с уведомлением Документа, определяющем лицензионные условия,
Документ должен содержать Текст, помещаемый на обложке, Лицензиат обязан
издавать экземпляры Документа в обложке с напечатанными на ней ясно и
разборчиво соответствующими Текстами, помещаемыми на обложке: Тексты,
помещаемые на первой странице обложки"--- на первой странице, Тексты,
помещаемые на последней странице"--- соответственно на последней. Также
на первой и последней странице обложки экземпляра Документа должно быть
ясно и разборчиво указано, что Лицензиат является издателем данных
экземпляров. На первой странице обложки должно быть указано полное
название Документа без пропусков и сокращений, все слова в названии
должны быть набраны шрифтом одинакового размера. Лицензиат вправе
поместить прочие сведения на обложке экземпляра. Если при издании
экземпляров Документа изменяются только сведения, помещенные на обложке
экземпляра, за исключением названия Документа, и при этом соблюдаются
требования настоящего пункта, такие действия приравниваются к
копированию без внесения изменений.

Если объем текста, который должен быть помещен на обложке экземпляра, не
позволяет напечатать его разборчиво полностью, Лицензиат обязан
поместить разумную часть текста с его начала непосредственно на обложке,
а остальной текст на страницах Документа, следующих сразу за обложкой.

Если Лицензиат издает или распространяет Непрозрачные экземпляры
Документа в количестве свыше 100, Лицензиат обязан к каждому такому
экземпляру приложить Прозрачный экземпляр этого Документа в
машиночитаемой форме или указать на каждом Непрозрачном экземпляре
Документа адрес в компьютерной сети общего пользования, где содержится
полный Прозрачный экземпляр без каких-либо добавленных материалов,
полный текст которого каждый пользователь компьютерной сети общего
пользования вправе записать в память компьютера с использованием
общедоступных сетевых протоколов. Во втором случае Лицензиат обязан
предпринять разумные шаги с тем, чтобы доступ к Прозрачному экземпляру
Документа по указанному адресу сохранялся по крайней мере в течение
одного года после последнего распространения Непрозрачного экземпляра
Документа данного тиража, независимо от того, было ли распространение
осуществлено Лицензиатом непосредственно или через агентов или розничных
продавцов.

Прежде, чем начать распространение большого количества экземпляров
Документа, Лицензиату заблаговременно следует связаться с авторами
Документа, чтобы они имели возможность предоставить Лицензиату
обновленную версию Документа. Лицензиат не обязан выполнять данное
условие.


\begin{center}
{\Large\bf 4. ВНЕСЕНИЕ ИЗМЕНЕНИЙ\par}
\end{center}


Лицензиат вправе воспроизводить и распространять Модифицированные версии
Документа в соответствии с условиями пунктов 2 и 3 настоящей Лицензии
при условии, что Модифицированная версия Документа публикуется в
соответствии с настоящей Лицензией. В частности, Лицензиат обязан
передать каждому обладателю экземпляра Модифицированной версии Документа
права на распространение и внесение изменений в данную Модифицированную
версию Документа, аналогично правам на распространение и внесение
изменений, которые передаются обладателю экземпляра Документа. При
распространении Модифицированных версий Документа Лицензиат обязан:

\begin{itemize}
\item[A.]
   Поместить на Титульном листе и на обложке при ее наличии название
   модифицированной версии, отличающееся от названия Документа и
   названий предыдущих версий. Названия предыдущих версий при их наличии
   должны быть указаны в Документе в разделе <<История>>. Лицензиат
   вправе использовать название предыдущей версии Документа с согласия
   издателя предыдущей версии;

\item[B.]
   Указать на Титульном листе в качестве авторов одно или более лиц,
   ответственных за изменения в Модифицированной версии, а также не
   менее пяти основных авторов Документа либо всех авторов, если их
   менее пяти, если только они не освободили Лицензиата от этого
   требования;

\item[C.]
   Указать на Титульном листе наименование издателя Модифицированной
   версии, с указанием, что он является издателем данной Версии;

\item[D.]
   Сохранить все знаки охраны авторского права Документа;

\item[E.]
   Поместить соответствующий знак охраны авторского права на внесенные
   Лицензиатом изменения рядом с прочими знаками охраны авторского права;

\item[F.]
   Поместить непосредственно после знаков охраны авторского права
   уведомление, в соответствии с которым каждому предоставляется право
   использовать Модифицированную Версию в соответствии с условиями
   настоящей Лицензии. Текст уведомления приводится в Приложении к
   настоящей Лицензии;

\item[G.]
   Сохранить в уведомлении, указанном в подпункте F, полный список
   Неизменяемых разделов и Текста, помещаемого на обложке, перечисленных
   в уведомлении Документа;

\item[H.]
   включить в Модифицированную версию текст настоящий Лицензии без
   каких-либо изменений.

\item[I.]
   Сохранить в Модифицированной версии раздел, Озаглавленный <<История>>,
   включая его Название, и дополнить его пунктом, в котором указать так
   же, как данные сведения указаны на Титульном листе, название, год
   публикации, наименования новых авторов и издателя Модифицированной
   версии. Если в Документе отсутствует раздел, Озаглавленный <<История>>,
   Лицензиат обязан создать в Модифицированной версии такой раздел,
   указать в нем название, год публикации, авторов и издателя Документа
   так же, как данные сведения указаны на Титульном листе Документа и
   дополнить этот раздел пунктом, содержание которого описано в
   предыдущем предложении;

\item[J.]
   Cохранить в Модифицированной версии адрес в компьютерной сети,
   указанный в Документе, по которому каждый вправе осуществить доступ к
   Прозрачному экземпляру Документа, а также адрес в компьютерной сети,
   указанный в Документе, по которому можно получить доступ к предыдущим
   версиям Документа. Адреса, по которым находятся предыдущие версии
   Документа, можно поместить в раздел <<История>>. Лицензиат вправе не
   указывать адрес произведения в компьютерной сети, которое было
   опубликовано не менее чем за четыре года до публикации самого
   Документа. Лицензиат вправе не указывать адрес определенной версии в
   компьютерной сети с разрешения первоначального издателя данной версии;

\item[K.]
   Сохранить без изменений названия разделов, озаглавленных
   <<Благодарности>> или <<Посвящения>>, а также содержание и стиль
   каждой благодарности и/или посвящения;

\item[L.]
   Cохранить без изменений названия и содержание всех Неизменяемых
   разделов Документа. Нумерация данных разделов или иной способ их
   перечисления не включается в состав названий разделов;

\item[M.]
   Удалить существующий раздел Документа, озаглавленный <<Одобрения>>.
   Такой раздел не может быть включен в Модифицированную версию;

\item[N.]
   Не присваивать существующим разделам Модифицированной версии название
   <<Одобрения>> или такие названия, которые повторяют название любого
   из Неизменяемых разделов;

\item[O.]
   сохранить без изменений любые Отказы от ответственности.
\end{itemize}

Если в Модифицированную версию включены новые предисловия или приложения,
которые могут быть определены как Второстепенные разделы и которые не
содержат текст, скопированный из Документа, Лицензиат вправе по своему
выбору определить все или некоторые из этих разделов как Неизменяемые.
Для этого следует добавить их названия в список Неизменяемых разделов в
уведомлении в Модифицированной версии, определяющем лицензионные условия.
Названия данных разделов должны отличаться от названий всех остальных
разделов.

Лицензиат вправе дополнить Модифицированную версию новым разделом,
озаглавленным <<Одобрения>> при условии, что в него включены
исключительно одобрения Модифицированной версии Лицензиата третьими
сторонами, например оценки экспертов или указания, что текст
Модифицированной версии был одобрен организацией в качестве официального
определения стандарта.

Лицензиат вправе дополнительно поместить на обложке Модифицированной
версии Текст, помещаемый на обложке, не превышающий пяти слов для первой
страницы обложки и 25 слов для последней страницы обложки. К Тексту,
помещаемому на обложке, каждым лицом непосредственно или от имени этого
лица на основании соглашения с ним может быть добавлено только по одной
строке на первой и на последней страницах обложки. Если на обложке
Документа Лицензиатом от своего имени или от имени лица, в интересах
которого действует Лицензиат, уже был помещен Текст, помещаемый на
обложке, Лицензиат не вправе добавить другой Текст. В этом случае
Лицензиат вправе заменить старый текст на новый с разрешения предыдущего
издателя, который включил старый текст в издание.

По настоящей Лицензии автор(ы) и издатель(и) Документа не передают право
использовать их имена и/или наименования в целях рекламы или заявления
или предположения, что любая из Модифицированных Версий получила их
одобрение.


\begin{center}
{\Large\bf 5. ОБЪЕДИНЕНИЕ ДОКУМЕНТОВ\par}
\end{center}


Лицензиат с соблюдением условий п.4 настоящей Лицензии вправе объединить
Документ с другими документами, которые опубликованы на условиях
настоящей Лицензии, при этом Лицензиат должен включить в произведение,
возникшее в результате объединения, все Неизменяемые разделы из всех
первоначальных документов без внесения в них изменений, а также указать
их в качестве Неизменяемых разделов данного произведения в списке
Неизменяемых разделов, который содержится в уведомлении, определяющем
лицензионные условия для произведения, и сохранить без изменений все
Отказы от отвтетственности.

Произведение, возникшее в результате объединения, должно содержать
только один экземпляр настоящей Лицензии. Повторяющиеся в произведении
одинаковые Неизменяемые разделы могут быть заменены единственной копией
таких разделов. Если произведение содержит несколько Неизменяемых
Разделов с одним и тем же названием, но с разным содержанием, Лицензиат
обязан сделать название каждого такого раздела уникальным путем
добавления после названия в скобках уникального номера данного раздела
или имени первоначального автора или издателя данного раздела, если
автор или издатель известны Лицензиату. Лицензиат обязан соответственно
изменить названия Неизменяемых разделов в списке Неизменяемых разделов
в уведомлении, определяющем лицензионные условия для произведения,
возникшего в результате объединения.

В произведении, возникшем в результате объединения, Лицензиат обязан
объединить все разделы, озаглавленные <<История>> из различных
первоначальных Документов в один общий раздел, озаглавленный <<История>>.
Подобным образом Лицензиат обязан объединить все разделы, озаглавленные
<<Благодарности>> и <<Посвящения>>. Лицензиат обязан исключить из
произведения все разделы, озаглавленные <<Одобрения>>.


\begin{center}
{\Large\bf 6. СБОРНИКИ ДОКУМЕНТОВ\par}
\end{center}


Лицензиат вправе издать сборник, состоящий из Документа и других
документов, публикуемых в соответствии с условиями настоящей Лицензии.
В этом случае Лицензиат вправе заменить все экземпляры настоящей
Лицензии в документах одним экземпляром, включенным в сборник, при
условии, что остальной текст каждого документа включен в сборник с
соблюдением условий по осуществлению копирования без внесения изменений.

Лицензиат вправе выделить какой-либо документ из сборника и издать его
отдельно в соответствии с настоящей Лицензией, при условии, что
Лицензиатом в данный документ включен текст настоящей Лицензии и им
соблюдены условия Лицензии по осуществлению копирования без внесения
изменений в отношении данного документа.


\begin{center}
{\Large\bf 7. ПОДБОРКА ДОКУМЕНТА И САМОСТОЯТЕЛЬНЫХ ПРОИЗВЕДЕНИЙ\par}
\end{center}


Размещение Документа или произведений, производных от Документа, с
другими самостоятельными документами или произведениями на одном
устройстве для хранения или распространения информации или носителе,
называется <<подборкой>>, если авторское право, возникоющее в результате
такой компиляции не используется для ограничения пользователей
компиляции сильнее, чем указано в лицензии каждого из отдельных
произведений. При включении Документа в <<подборку>>, условия настоящей
Лицензии не применяются к самостоятельным произведениям, размещенным
вышеуказанным способом вместе с Документом, при условии, что они не
являются произведениями, производными от Документа.

Если условия пункта 3 настоящей Лицензии относительно Текста,
помещаемого на обложке, могут быть применены к экземплярам Документа в
Подборке, то в этом случае Текст с обложки Документа может быть помещен
на обложке только собственно Документа внутри подборки при условии, что
Документ занимает менее половины объема всей Подборки. Если Документ
занимает более четвертой части объема Подборки, в этом случае Текст с
обложки Документа должен быть помещен на печатной обложке всей Подборки.


\begin{center}
{\Large\bf 8. ПЕРЕВОД\par}
\end{center}


Перевод является одним из способов модификации Документа, в силу чего
Лицензиат вправе распространять экземпляры перевода Документа в
соответствии с пунктом 4 настоящей Лицензии. Замена Неизменяемых
разделов их переводами может быть осуществлена только с разрешения
соответствующих правообладателей, однако Лицензиат вправе в дополнение
к оригинальным версиям таких Неизменяемых разделов включить в текст
экземпляра перевод всех или части таких Разделов. Лицензиат вправе
включить в текст экземпляра перевод настоящей Лицензии, всех
лицензионных уведомлений, включённых в Документ и всех Отказов от
ответственности при условии, что в него включен также и оригинальный
текст настоящей Лицензии на английском языке и оригинальные тексты всех
уведомлений и отказов. В случае разногласий в толковании текста перевода
и оригинального текста Лицензии, уведомлений или отказов, предпочтение
отдается оригинальному тексту.

Если в Документе есть разделы, Озаглавленные <<Благодарности>>,
<<Посвящения>> или <<История>>, требования (см. раздел 4) сохранять без
изменения их Названия (см. раздел 1) часто требует изменения названия
Документа.


\begin{center}
{\Large\bf 9. РАСТОРЖЕНИЕ ЛИЦЕНЗИИ\par}
\end{center}


Лицензиат вправе воспроизводить, модифицировать, распространять или
передавать права на использование Документа только на условиях настоящей
Лицензии. Любое воспроизведение, модификация, распространение или
передача прав на иных условиях являются недействительными и
автоматически ведут к расторжению настоящей Лицензии и прекращению всех
прав Лицензиата, предоставленных ему настоящей Лицензией. При этом права
третьих лиц, которым Лицензиат в соответствии с настоящей Лицензией
передал экземпляры Документа или права на него, сохраняются в силе при
условии полного соблюдения ими настоящей Лицензии.


\begin{center}
{\Large\bf 10. ПЕРЕСМОТР УСЛОВИЙ ЛИЦЕНЗИИ\par}
\end{center}


Free Software Foundation может публиковать новые исправленные версии
GFDL. Такие версии могут быть дополнены различными нормами,
регулирующими правоотношения, которые возникли после опубликования
предыдущих версий, однако в них будут сохранены основные принципы,
закрепленные в настоящей версии (смотри \url{http://www/gnu.org/copyleft/}).

Каждой версии присваивается свой собственный номер. Если указано, что
Документ распространяется в соответствии с определенной версией, то есть
указан ее номер, или любой более поздней версией настоящей Лицензии,
Лицензиат вправе присоединиться к любой из этих версий Лицензии,
опубликованных Free Software Foundation (при условии, что ни одна из
версий не является проектом Лицензии). Если Документ не содержит такого
указания на номер версии Лицензии, Лицензиат вправе присоединиться к
любой из версий Лицензии, опубликованных когда-либо Free Software
Foundation (при условии, что ни одна из версий не является Проектом
Лицензии).


\begin{center}
{\Large\bf Приложение: Порядок применения условий настоящей Лицензии к вашей документации\par}
\end{center}

Чтобы применить условия настоящей Лицензии к созданному вами документу,
вам следует включить в документ текст настоящей Лицензии, а также знак
охраны авторского права и уведомление, определяющее лицензионные условия,
сразу после титульного листа документа в соответствии с нижеприведенным
образцом:


\bigskip
\begin{quote}
    Copyright \copyright{}  ГОД  ИМЯ (НАИМЕНОВАНИЕ) АВТОРА.

    Каждый имеет право воспроизводить, распространять и/или вносить
    изменения в настоящий Документ в соответствии с условиями GNU Free
    Documentation License, Версией 1.2 или любой более поздней версией,
    опубликованной Free Software Foundation;
    Данный документ не содержит Неизменяемых разделов; не содержит Текста,
    помещаемого на первой странице обложки и не содержит Текста,
    помещаемого на последней странице обложки.
    Копия настоящей Лицензии включена в раздел под названием <<GNU Free
    Documentation License>>.
\end{quote}
\bigskip

Если документ содержит Неизменяемые разделы, Текст, помещаемый на первой
странице обложки либо Текст, помещаемый на последней странице обложки,
замените три строки <<данный \dots\ обложки.>> на нижеследущее

\bigskip
\begin{quote}
    данный Документ содержит следующие Неизменяемые разделы (УКАЗАТЬ НАЗВАНИЯ);
    данный документ содержит следующий Текст, помещаемый на первой странице
    обложки (ПЕРЕЧИСЛИТЬ), данный документ содержит следующий Текст,
    помещаемый на последней странице обложки (ПЕРЕЧИСЛИТЬ).
\end{quote}
\bigskip

Если документ содержит Неизменяемые разделы, но не содержит Текстов,
помещаемых на обложке, либо какую-нибудь другую комбинацию этих трёх
утверждений, скомпонуйте две предложенные альтернативы так, чтобы они
подходили к ситуации.

Если ваш документ содержит имеющие существенное значение примеры
программного кода, мы рекомендуем вам выпустить их отдельно в
соответствии с условиями одной из лицензий на свободное программное
обеспечение, например GNU General Public License, чтобы их можно было
использовать как свободное программное обеспечение.
