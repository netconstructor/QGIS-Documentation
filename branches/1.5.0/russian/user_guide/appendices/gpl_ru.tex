\chapter{Стандартная Общественная Лицензия GNU}\label{gpl_appendix_ru}
\index{license!GPL}

\begin{small}
This is an unofficial translation of the GNU General Public License into
Russian. It was not published by the Free Software Foundation, and does
not legally state the distribution terms for software that uses the GNU
GPL"--- only the original English text of the GNU GPL does that. However,
we hope that this translation will help Russian speakers understand the
GNU GPL better.

Настоящий перевод Стандартной Общественной Лицензии GNU на русский язык
не является официальным. Он не публикуется Free Software Foundation и не
устанавливает имеющих юридическую силу условий для распространения
программного обеспечения, которое распространяется на условиях Стандартной
Общественной Лицензии GNU. Условия, имеющие юридическую силу, закреплены
исключительно в аутентичном тексте Стандартной Общественной Лицензии GNU
на английском языке. Я надеюсь, что настоящий перевод поможет
русскоязычным пользователям лучше понять содержание Стандартной
Общественной Лицензии GNU.

Автор перевода Елена Тяпкина (tiapkina@hotmail.com), 20 марта 2002 г.


\begin{center}
GNU GENERAL PUBLIC LICENSE

Версия 2, июнь 1991г.


Copyright (C) 1989, 1991 Free Software Foundation, Inc.
59 Temple Place - Suite 330, Boston, MA  02111-1307, USA


Каждый вправе копировать и распространять экземпляры настоящей Лицензии
без внесения изменений в ее текст
\end{center}
Преамбула

Большинство лицензий на программное обеспечение лишаeт вас права
распространять и вносить изменения в это программное обеспечение.
Стандартная Общественная Лицензия GNU, напротив, разработана с целью
гарантировать вам право совместно использовать и вносить изменения в
свободное программное обеспечение, т.е. обеспечить свободный доступ к
программному обеспечению для всех пользователей. Условия настоящей
Стандартной Общественной Лицензии применяются к большей части
программного обеспечения Free Software Foundation, а также к любому
другому программному обеспечению по желанию его автора. (К некоторому
программному обеспечению Free Software Foundation применяются условия
Стандартной Общественной Лицензии GNU для Библиотек). Вы также можете
применять Стандартную Общественную Лицензию к разработанному вами
программному обеспечению.

Говоря о свободном программном обеспечении, мы имеем в виду свободу, а
не безвозмездность. Настоящая Стандартная Общественная Лицензия
разработана с целью гарантировать вам право распространять экземпляры
свободного программного обеспечения (и при желании получать за это
вознаграждение), право получать исходный текст программного обеспечения
или иметь возможность его получить, право вносить изменения в
программное обеспечение или использовать его части в новом свободном
программном обеспечении, а также право знать, что вы имеете все
вышеперечисленные права.

Чтобы защитить ваши права, мы вводим ряд ограничений с тем, чтобы никто
не имел возможности лишить вас этих прав или обратиться к вам с
предложением отказаться от этих прав. Данные ограничения налагают на вас
определенные обязанности в случае, если вы распространяете экземпляры
программного обеспечения или модифицируете программное обеспечение.

Например, если вы распространяете экземпляры такого программного
обеспечения за плату или бесплатно, вы обязаны передать новым обладателям
все права в том же объеме, в каком они принадлежат вам. Вы обязаны
обеспечить получение новыми обладателями программы ее исходного текста
или возможность его получить. Вы также обязаны ознакомить их с условиями
настоящей Лицензии.

Для защиты ваших прав мы: (1) оставляем за собой авторские права на
программное обеспечение и (2) предлагаем вам использовать настоящую
Лицензию, в соответствии с условиями которой вы вправе воспроизводить,
распространять и/или модифицировать программное обеспечение.

Кроме того, для защиты как нашей репутации, так и репутации других
авторов программного обеспечения, мы уведомляем всех пользователей, что
на данное программное обеспечение никаких гарантий не предоставляется.
Те, кто приобрел программное обеспечение, с внесенными в него третьими
лицами изменениями, должны знать, что они получают не оригинал, в силу
чего автор оригинала не несет ответственности за ошибки в работе
программного обеспечения, допущенные третьими лицами при внесении
изменений.

Наконец, программное обеспечение перестает быть свободным в случае, если
лицо приобретает на него исключительные права. Недопустимо, чтобы лица,
распространяющие свободное программное обеспечение, могли приобрести
исключительные права на использование данного программного обеспечения и
зарегистрировать их в Патентном ведомстве. Чтобы избежать этого, мы
заявляем, что обладатель исключительных прав обязан предоставить любому
лицу права на использование программного обеспечения либо не приобретать
исключительных прав вообще.

Ниже изложены условия воспроизведения, распространения и модификации
программного обеспечения.
УСЛОВИЯ ВОСПРОИЗВЕДЕНИЯ, РАСПРОСТРАНЕНИЯ И МОДИФИКАЦИИ

0. Условия настоящей Лицензии применяются ко всем видам программного
обеспечения или любому иному произведению, которое содержит указание
правообладателя на то, что данное произведение может распространяться на
условиях Стандартной Общественной Лицензии. Под термином <<Программа>>
далее понимается любое подобное программное обеспечение или иное
произведение. Под термином <<произведение, производное от Программы>>
понимается Программа или любое иное производное произведение в
соответствии с законодательством об авторском праве, т.\,е. произведение,
включающее в себя Программу или ее часть, как с внесенными в ее текст
изменениями, так и без них и/или переведенную на другой язык. (Здесь и
далее, понятие <<модификация>> включает в себя понятие перевода в самом
широком смысле). Каждый приобретатель экземпляра Программы именуется в
дальнейшем <<Лицензиат>>.

Действие настоящей Лицензии не распространяется на осуществление иных
прав, кроме воспроизведения, распространения и модификации программного
обеспечения. Не устанавливается ограничений на запуск Программы. Условия
Лицензии распространяются на выходные данные из Программы только в том
случае, если их содержание составляет произведение, производное от
Программы (независимо от того, было ли такое произведение создано в
результате запуска Программы). Это зависит от того, какие функции
выполняет Программа.

1. Лицензиат вправе изготовлять и распространять экземпляры исходного
текста Программы в том виде, в каком он его получил, без внесения в него
изменений на любом носителе, при соблюдении следующих условий: на каждом
экземпляре помещен знак охраны авторского права и уведомление об
отсутствии гарантий; оставлены без изменений все уведомления,
относящиеся к настоящей Лицензии и отсутствию гарантий; вместе с
экземпляром Программы приобретателю передается копия настоящей Лицензии.

Лицензиат вправе взимать плату за передачу экземпляра Программы, а также
вправе за плату оказывать услуги по гарантийной поддержке Программы.

2. Лицензиат вправе модифицировать свой экземпляр или экземпляры
Программы полностью или любую ее часть. Данные действия Лицензиата
влекут за собой создание произведения, производного от Программы.
Лицензиат вправе изготовлять и распространять экземпляры такого
произведения, производного от Программы, или собственно экземпляры
изменений в соответствии с пунктом 1 настоящей Лицензии при соблюдении
следующих условий:

    a) файлы, измененные Лицензиатом, должны содержать хорошо заметную
пометку, что они были изменены, а также дату внесения изменений;

    b) при распространении или публикации Лицензиатом любого произведения,
которое содержит Программу или ее часть или является производным от
Программы или от ее части, Лицензиат обязан передавать права на
использование данного произведения третьим лицам на условиях настоящей
Лицензии, при этом Лицензиат не вправе требовать уплаты каких-либо
лицензионных платежей. Распространяемое произведение лицензируется как
одно целое;

    c) если модифицированная Программа при запуске обычно читает команды
в интерактивном режиме, Лицензиат обязан обеспечить вывод на экран
дисплея или печатающее устройство сообщения, которое должно включать в
себя: знак охраны авторского права; уведомление об отсутствии гарантий
на Программу (или иное, если Лицензиат предоставляет гарантии); указание
на то, что пользователи вправе распространять экземпляры Программы в
соответствии с условиями настоящей Лицензии, а также на то, каким
образом пользователь может ознакомиться с текстом настоящей Лицензии.
(Исключение: если оригинальная Программа является интерактивной, но не
выводит в своем обычном режиме работы сообщение такого рода, то вывод
подобного сообщения произведением, производным от Программы, в этом
случае не обязателен).

Вышеуказанные условия применяются к модифицированному произведению,
производному от Программы, в целом. В случае если отдельные части данного
произведения не являются производными от Программы, являются результатом
творческой деятельности и могут быть использованы как самостоятельное
произведение, Лицензиат вправе распространять отдельно такое произведение
на иных лицензионных условиях. В случае если Лицензиат распространяет
вышеуказанные части в составе произведения, производного от Программы, то
условия настоящей Лицензии применяются к произведению в целом, при этом
права, приобретаемые сублицензиатами на основании Лицензии, передаются им
в отношении всего произведения, включая все его части, независимо от того,
кто является их авторами.

Целью настоящего пункта 2 не является заявление прав или оспаривание прав
на произведение, созданное исключительно Лицензиатом. Целью настоящего
пункта является обеспечение права контролировать распространение
произведений, производных от Программы, и составных произведений,
производных от Программы.

Размещение произведения, которое не является производным от Программы, на
одном устройстве для хранения информации или носителе вместе с Программой
или произведением, производным от Программы, не влечет за собой
распространения условий настоящей Лицензии на такое произведение.

3. Лицензиат вправе воспроизводить и распространять экземпляры Программы
или произведения, которое является производным от Программы, в
соответствии с пунктом 2 настоящей Лицензии, в виде объектного кода или
в исполняемой форме в соответствии с условиями п.п.1 и 2 настоящей
Лицензии при соблюдении одного из перечисленных ниже условий:

    a) к экземпляру должен прилагаться соответствующий полный исходный
текст в машиночитаемой форме, который должен распространяться в
соответствии с условиями п.п. 1 и 2 настоящей Лицензии на носителе,
обычно используемом для передачи программного обеспечения, либо

    b) к экземпляру должно прилагаться действительное в течение трех лет
предложение в письменной форме к любому третьему лицу передать за плату,
не превышающую стоимость осуществления собственно передачи, экземпляр
соответствующего полного исходного текста в машиночитаемой форме в
соответствии с условиями п.п. 1 и 2 настоящей Лицензии на носителе, обычно
используемом для передачи программного обеспечения, либо

    c) к экземпляру должна прилагаться полученная Лицензиатом информация
о предложении, в соответствии с которым можно получить соответствующий
исходный текст. (Данное положение применяется исключительно в том случае,
если Лицензиат осуществляет некоммерческое распространение программы, при
этом программа была получена самим Лицензиатом в виде объектного кода или
в исполняемой форме и сопровождалась предложением, соответствующим
условиям пп.b п.3 настоящей Лицензии).

Под исходным текстом произведения понимается такая форма произведения,
которая наиболее удобна для внесения изменений. Под полным исходным
текстом исполняемого произведения понимается исходный текст всех
составляющих произведение модулей, а также всех файлов, связанных с
описанием интерфейса, и сценариев, предназначенных для управления
компиляцией и установкой исполняемого произведения. Однако, в качестве
особого исключения, распространяемый исходный текст может не включать
того, что обычно распространяется (в виде исходного текста или в
бинарной форме) с основными компонентами (компилятор, ядро и т.д.)
операционной системы, в которой работает исполняемое произведение, за
исключением случаев, когда исполняемое произведение сопровождается таким
компонентом.

В случае если произведение в виде объектного кода или в исполняемой форме
распространяется путем предоставления доступа для копирования его из
определенного места, обеспечение равноценного доступа для копирования
исходного текста из этого же места удовлетворяет требованиям
распространения исходного текста, даже если третьи лица при этом не
обязаны копировать исходный текст вместе с объектным кодом произведения.

4. Лицензиат вправе воспроизводить, модифицировать, распространять или
передавать права на использование Программы только на условиях настоящей
Лицензии. Любое воспроизведение, модификация, распространение или
передача прав на иных условиях являются недействительными и автоматически
ведут к расторжению настоящей Лицензии и прекращению всех прав Лицензиата,
предоставленных ему настоящей Лицензией. При этом права третьих лиц,
которым Лицензиат в соответствии с настоящей Лицензией передал экземпляры
Программы или права на нее, сохраняются в силе при условии полного
соблюдения ими настоящей Лицензии.

5. Лицензиат не обязан присоединяться к настоящей Лицензии, поскольку он
ее не подписал. Однако только настоящая Лицензия предоставляет право
распространять или модифицировать Программу или произведение, производное
от Программы. Подобные действия нарушают действующее законодательство,
если они не осуществляются в соответствии с настоящей Лицензией. Если
Лицензиат внес изменения или осуществил распространение экземпляров
Программы или произведения, производного от Программы, Лицензиат тем
самым подтвердил свое присоединение к настоящей Лицензии в целом, включая
условия, определяющие порядок воспроизведения, распространения или
модификации Программы или произведения, производного от Программы.

6. При распространении экземпляров Программы или произведения,
производного от Программы, первоначальный лицензиар автоматически
передает приобретателю такого экземпляра право воспроизводить,
распространять и модифицировать Программу в соответствии с условиями
настоящей Лицензии. Лицензиат не вправе ограничивать каким-либо способом
осуществление приобретателями полученных ими прав. Лицензиат не несет
ответственности за несоблюдение условий настоящей Лицензии третьими
лицами.

7. Лицензиат не освобождается от исполнения обязательств в соответствии
с настоящей Лицензией в случае, если в результате решения суда или
заявления о нарушении исключительных прав или в связи с наступлением
иных обстоятельств, не связанных непосредственно с нарушением
исключительных прав, на Лицензиата на основании решения суда, договора
или ином основании возложены обязательства, которые противоречат
условиям настоящей Лицензии. В этом случае Лицензиат не вправе
распространять экземпляры Программы, если он не может одновременно
исполнить условия настоящей Лицензии и возложенные на него указанным
выше способом обязательства. Например, если по условиям лицензионного
соглашения сублицензиатам не может быть предоставлено право бесплатного
распространения экземпляров Программы, которые они приобрели напрямую
или через третьих лиц у Лицензиата, то в этом случае Лицензиат обязан
отказаться от распространения экземпляров Программы.

Если любое положение настоящего пункта при наступлении конкретных
обстоятельств будет признано недействительным или неприменимым, настоящий
пункт применяется за исключением такого положения. Настоящий пункт
применяется в целом при прекращении вышеуказанных обстоятельств или их
отсутствии.

Целью данного пункта не является принуждение Лицензиата к нарушению
патента или заявления на иные права собственности или к оспариванию
действительности такого заявления. Единственной целью данного пункта
является защита неприкосновенности системы распространения свободного
программного обеспечения, которая обеспечивается за счет общественного
лицензирования. Многие люди внесли свой щедрый вклад в создание большого
количества программного обеспечения, которое распространяется через
данную систему в надежде на ее длительное и последовательное применение.
Лицензиат не вправе вынуждать автора распространять программное
обеспечение через данную систему. Право выбора системы распространения
программного обеспечения принадлежит исключительно его автору.

Настоящий пункт 7 имеет целью четко определить те цели, которые
преследуют все остальные положения настоящей Лицензии.

8. В том случае если распространение и/или использование Программы в
отдельных государствах ограничено соглашениями в области патентных или
авторских прав, первоначальный правообладатель, распространяющий
Программу на условиях настоящей Лицензии, вправе ограничить территорию
распространения Программы, указав только те государства, на территории
которых допускается распространение Программы без ограничений,
обусловленных такими соглашениями. В этом случае такое указание в
отношении территорий определенных государств признается одним из условий
настоящей Лицензии.

9. Free Software Foundation может публиковать исправленные и/или новые
версии настоящей Стандартной Общественной Лицензии. Такие версии могут
быть дополнены различными нормами, регулирующими правоотношения, которые
возникли после опубликования предыдущих версий, однако в них будут
сохранены основные принципы, закрепленные в настоящей версии.

Каждой версии присваивается свой собственный номер. Если указано, что
Программа распространяется в соответствии с определенной версией, т.\,е.
указан ее номер, или любой более поздней версией настоящей Лицензии,
Лицензиат вправе присоединиться к любой из этих версий Лицензии,
опубликованных Free Software Foundation. Если Программа не содержит
такого указания на номер версии Лицензии Лицензиат вправе присоединиться
к любой из версий Лицензии, опубликованных когда-либо Free Software
Foundation.

10. В случае если Лицензиат намерен включить часть Программы в другое
свободное программное обеспечение, которое распространяется на иных
условиях, чем в настоящей Лицензии, ему следует испросить письменное
разрешение на это у автора программного обеспечения. Разрешение в
отношении программного обеспечения, права на которое принадлежат Free
Software Foundation, следует испрашивать у Free Software Foundation. В
некоторых случаях Free Software Foundation делает исключения. При
принятии решения Free Software Foundation будет руководствоваться двумя
целями: сохранение статуса свободного для любого произведения,
производного от свободного программного обеспечения Free Software
Foundation и обеспечение наиболее широкого совместного использования
программного обеспечения.

ОТСУТСТВИЕ ГАРАНТИЙНЫХ ОБЯЗАТЕЛЬСТВ

11. ПОСКОЛЬКУ НАСТОЯЩАЯ ПРОГРАММА РАСПРОСТРАНЯЕТСЯ БЕСПЛАТНО, ГАРАНТИИ
НА НЕЕ НЕ ПРЕДОСТАВЛЯЮТСЯ В ТОЙ СТЕПЕНИ, В КАКОЙ ЭТО ДОПУСКАЕТСЯ
ПРИМЕНИМЫМ ПРАВОМ. НАСТОЯЩАЯ ПРОГРАММА ПОСТАВЛЯЕТСЯ НА УСЛОВИЯХ
<<КАК ЕСТЬ>>. ЕСЛИ ИНОЕ НЕ УКАЗАНО В ПИСЬМЕННОЙ ФОРМЕ, АВТОР И/ИЛИ ИНОЙ
ПРАВООБЛАДАТЕЛЬ НЕ ПРИНИМАЕТ НА СЕБЯ НИКАКИХ ГАРАНТИЙНЫХ ОБЯЗАТЕЛЬСТВ,
КАК ЯВНО ВЫРАЖЕННЫХ, ТАК И ПОДРАЗУМЕВАЕМЫХ, В ОТНОШЕНИИ ПРОГРАММЫ, В ТОМ
ЧИСЛЕ ПОДРАЗУМЕВАЕМУЮ ГАРАНТИЮ ТОВАРНОГО СОСТОЯНИЯ ПРИ ПРОДАЖЕ И
ПРИГОДНОСТИ ДЛЯ ИСПОЛЬЗОВАНИЯ В КОНКРЕТНЫХ ЦЕЛЯХ, А ТАКЖЕ ЛЮБЫЕ ИНЫЕ
ГАРАНТИИ. ВСЕ РИСКИ, СВЯЗАННЫЕ С КАЧЕСТВОМ И ПРОИЗВОДИТЕЛЬНОСТЬЮ
ПРОГРАММЫ, НЕСЕТ ЛИЦЕНЗИАТ. В СЛУЧАЕ ЕСЛИ В ПРОГРАММЕ БУДУТ ОБНАРУЖЕНЫ
НЕДОСТАТКИ, ВСЕ РАСХОДЫ, СВЯЗАННЫЕ С ТЕХНИЧЕСКИМ ОБСЛУЖИВАНИЕМ, РЕМОНТОМ
ИЛИ ИСПРАВЛЕНИЕМ ПРОГРАММЫ, НЕСЕТ ЛИЦЕНЗИАТ.

12. ЕСЛИ ИНОЕ НЕ ПРЕДУСМОТРЕНО ПРИМЕНЯЕМЫМ ПРАВОМ ИЛИ НЕ СОГЛАСОВАНО
СТОРОНАМИ В ДОГОВОРЕ В ПИСЬМЕННОЙ ФОРМЕ, АВТОР И/ИЛИ ИНОЙ ПРАВООБЛАДАТЕЛЬ,
КОТОРЫЙ МОДИФИЦИРУЕТ И/ИЛИ РАСПРОСТРАНЯЕТ ПРОГРАММУ НА УСЛОВИЯХ НАСТОЯЩЕЙ
ЛИЦЕНЗИИ, НЕ НЕСЕТ ОТВЕТСТВЕННОСТИ ПЕРЕД ЛИЦЕНЗИАТОМ ЗА УБЫТКИ, ВКЛЮЧАЯ
ОБЩИЕ, РЕАЛЬНЫЕ, ПРЕДВИДИМЫЕ И КОСВЕННЫЕ УБЫТКИ (В ТОМ ЧИСЛЕ УТРАТУ ИЛИ
ИСКАЖЕНИЕ ИНФОРМАЦИИ, УБЫТКИ, ПОНЕСЕННЫЕ ЛИЦЕНЗИАТОМ ИЛИ ТРЕТЬИМИ ЛИЦАМИ,
НЕВОЗМОЖНОСТЬ РАБОТЫ ПРОГРАММЫ С ЛЮБОЙ ДРУГОЙ ПРОГРАММОЙ И ИНЫЕ УБЫТКИ).
АВТОР И/ИЛИ ИНОЙ ПРАВООБЛАДАТЕЛЬ В СООТВЕТСТВИИ С НАСТОЯЩИМ ПУНКТОМ НЕ
НЕСУТ ОТВЕТСТВЕННОСТИ ДАЖЕ В ТОМ СЛУЧАЕ, ЕСЛИ ОНИ БЫЛИ ПРЕДУПРЕЖДЕНЫ О
ВОЗМОЖНОСТИ ВОЗНИКНОВЕНИЯ ТАКИХ УБЫТКОВ.
\end{small}
