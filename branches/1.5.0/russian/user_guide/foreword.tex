%  !TeX  root  =  user_guide.tex
\mainmatter
\pagestyle{scrheadings}
\addchap{Предисловие}\label{label_forward}


% when the revision of a section has been finalized,
% comment out the following line:
% \updatedisclaimer

Добро пожаловать в удивительный мир Географических Информационных Систем
(ГИС)! Quantum GIS (QGIS) является Географической Информационной
Системой с открытым исходным кодом. Проект был запущен в мае 2002 года,
а в июне того же года "--- создан проект на площадке SourceForge. Мы
много работали, чтобы сделать программное обеспечение ГИС (которое
традиционно является дорогим проприетарным программным обеспечением)
доступным любому, кто имеет доступ к персональному компьютеру.
В настоящее время QGIS работает на большинстве платформ Unix, Windows, и
OS X. QGIS разработан с использованием инструментария Qt
(\url{http://qt.nokia.com}) и C++. Это означает, что QGIS легок в
использовании, имеет приятный и простой графический интерфейс (GUI).

QGIS стремится быть легкой в использовании ГИС, предоставляя общую
функциональность. Первоначальная цель заключалась в обеспечении
просмотра ГИС данных. QGIS достиг той стадии в своем развитии, когда
многие используют ее в своих ежедневных задачах просмотра ГИС данных.
QGIS поддерживает множество растровых и векторных форматов данных, а
поддержка новых форматов реализуется с помощью модулей (полный список
поддерживаемых форматов данных см. в Приложении~\ref{appdx_data_formats}).

QGIS выпускается на условиях лицензии GNU General Public License (GPL).
Разработка QGIS под этой лицензией означает, что вы можете просмотреть и
изменить исходный код, и гарантирует, что вы, наш счастливый
пользователь, всегда будете иметь доступ к программному обуспечению ГИС,
которое является бесплатным и может быть свободно изменено. Вы должны
были получить полную копию лицензии с вашей копией QGIS, но также вы
можете найти ее в Приложении~\ref{gpl_appendix}.

\begin{Tip}\caption{\textsc{Актуальная версия документации}}\index{documentation}
Актуальную версию данного документа всегда можно найти на странице
\url{http://download.osgeo.org/qgis/doc/manual/}, или в разделе
документации на веб-сайте QGIS \url{http://qgis.osgeo.org/documentation/}
\end{Tip}

\addsec{Возможности}\label{label_majfeat}

\qg предоставляет большое количество распространенных ГИС функций,
обеспечиваемых встроенными инструментами и модулями. Чтобы получить
первое представление, они могут быть представлены шестью категориями в
виде краткого резюме.

\minisec{Просмотр данных}

Можно просматривать и комбинировать векторные и растровые данные в
различных форматах и проекциях без преобразования во внутренний или
общий формат. Поддерживаются следующие форматы:

\begin{itemize}[label=--]
\item пространственные таблицы PostgreSQL используя PostGIS, векторные
форматы
%\footnote{форматы баз данных поддерживаемые библиотекой OGR, такие как Oracle или
%mySQL, пока еще не поддерживаются в QGIS.}
поддерживаемые установленной библиотекой OGR, включая shape-файлы ESRI,
MapInfo, SDTS (Spatial Data Transfer Standard) и GML (Geography Markup
Language) (см. Приложение~\ref{appdx_ogr} для получения полного списка).
\item Форматы растров и изображений поддерживаемых установленной
библиотекой GDAL (Geospatial Data Abstraction Library), такие как
GeoTiff, Erdas Img., ArcInfo Ascii Grid, JPEG, PNG (см. Приложение~\ref{appdx_gdal}
для получения полного списка).
\item базы данных SpatiaLite (см. Раздел~\ref{label_spatialite})
\item растровые и векторные форматы баз данных GRASS (область/набор данных),
см. Раздел~\ref{sec:grass}.
\item Пространственные данные публикуемые в сети Интернет с помощью
OGC-совместимых (Open Geospatial Consortium) сервисов Web Map Service
(WMS) или Web Feature Service (WFS), см. Раздел~\ref{working_with_ogc},
\item данные OpenStreetMap (OSM) (см. Раздел~\ref{plugins_osm}).
\end{itemize}

\minisec{Изучение данных и компоновка карт}

С помощью удобного графического интерфейса можно компоновать карты и
интерактивно изучать пространственные данные. Графический интерфейс
включает в себя множество полезных инструментов:

\begin{itemize}[label=--]
\item перепроецирование налету
\item компоновщик карт
\item панель обзора
\item пространственные закладки
\item определение/выборка объектов
\item редактирование/просмотр/поиск атрибутов
\item подписывание объектов
\item изменение символики векторных и растровых слоев
\item добавление слоя координатной сетки "--- теперь средствами плагина
fTools
\item декорирование карты с помощью стрелки на север, линейки масштаба
и знака авторского права
\item сохранение и возобновление проектов
\end{itemize}

\minisec{Создание, редактирование, управление и экспорт данных}

Можно создавать, редактировать, управлять и экспортировать векторные
данные в нескольких форматах. Чтобы иметь возможность редактировать и
экпортировать растровые данные в другие форматы, они должны быть
импортированы в GRASS. QGIS предоставляет следующие возможности:

\begin{itemize}[label=--]
\item инструменты цифрования для форматов поддерживаемых библиотекой OGR
и векторных слоев GRASS
\item создание и редактирование shape-файлов и векторных слоев GRASS
\item геокодирование изображений с помощью модуля пространственной
привязки
\item инструменты GPS для импорта и экспорта данных в формате GPX,
преобразования прочих форматов GPS в формат GPX или скачивание/загрузка
непосредственно в прибор GPS (в Linux, usb: был добавлен в список
устройств GPS)
\item визуализация и редактирование данных OpenStreetMap
\item создание из shape-файлов слоев PostGIS с помощью плагина SPIT
\item улучшенная обработка таблиц PostGIS
\item управление атрибутами векторных данных с помощью новой таблицы
атрибутов (см. Раздел~\ref{sec:attribute table}) или модуля Table Manager
\item сохранение снимков экрана как изображениий с пространственной
привязкой
\end{itemize}

\minisec{Анализ данных}

Вы можете анализировать пространственные данные PostgreSQL/PostGIS и
других форматов поддерживаемых OGR используя модуль fTools написанный на
языке Python. В настоящее время QGIS предоставляет возможность
использовать инструменты векторного анализа, выборки, геопроцессинга,
управления геометрией и базами данных. Также можно использовать
интегрированные инструменты GRASS, которые включают в себя
функциональность более, чем 300 модулей GRASS (см. Раздел~\ref{sec:grass}).

\minisec{Публикация карт в сети Интернет}

QGIS может использоваться для экспорта данных в map-файл и публикации
его в сети Интернет используя установленный веб-сервер UMN MapServer.
QGIS также может использоваться как WMS или WFS клиент, а также как WMS
сервер.

\minisec{Расширение функциональности QGIS с помощью плагинов}

QGIS может быть адаптирован к особым потребностям с помощью расширяемой
архитектуры модулей. QGIS предоставляет библиотеки, которые могут быть
использованы для создания модулей. Можно даже создавать новые приложения
используя языки программирования C++ или Python!

\minisec{Основные модули}

\begin{enumerate}
\item Добавить слой из текста с разделителями (Загружает и выводит
текстовые файлы, содержащие координаты x, y)
\item Захват координат (Захватывает координаты мыши в различных CRS
(Coordinate Reference System))
\item Оформление (Знак авторского права, стрелка на север, масштабная
линейка)
\item Наложение диаграмм (Наложение диаграмм на векторные слои)
\item Преобразователь Dxf2Shp (Преобразование файлов DXF в shape-файлы)
\item Инструменты GPS (Загрузка и импорт данных GPS)
\item GRASS (Поддержка ГИС GRASS)
\item Привязка растров GDAL (Добавляет в растровый файл информацию о
проекции используя GDAL)
\item Модуль интерполяции (Интерполяция данных по вершинам в векторном
слое)
\item Экспорт в Mapserver (Экспорт проекта QGIS в map-файл MapServer)
\item Преобразователь слоев OGR (Преобразует векторные слои в форматы,
поддерживаемые библиотекой OGR)
\item Модуль OpenStreetMap (Просмотр и редактирование данных
OpenStreetMap)
\item Доступ к данным Oracle Spatial GeoRaster
\item Установщик модулей Python (Загрузка и установка модулей QGIS)
\item Быстрая печать (Печать карты с минимумом параметров)
\item Морфометрический анализ (Морфометрический анализ растровых слоев)
\item SPIT (Инструмент импорта shape-файлов в PostgreSQL/PostGIS)
\item Модуль WFS (Добавляет возможность загрузки слоев WFS)
\item eVIS (Инструмент визуализации событий-показ изображений, связанных
с векторными объектами)
\item fTools (Инструменты для управления и анализа векторных данных)
\item Консоль Python (Доступ к среде разработки QGIS)
\item Инструменты GDAL
\end{enumerate}

\minisec{Внешние модули Python}

QGIS предлагает все большее число внешних плагинов Python, которые
предоставлены сообществом. Эти плагины находятся в официальном
репозитории PyQGIS, и могут быть легко установлены с помощью Установщика
модулей Python (см. Раздел~\ref{sec:plugins}).

\subsubsection{Что нового в версии \CURRENT}

Наиболее важные добавления и улучшения для пользователей QGIS:
\begin{itemize}[label=--]
%  \item TODO(anne): rearrange the list on ManualTasks wikipage!
 \item Добавлено контекстное меню справки в диалоги
 \item Инструмент измерения углов
 \item Виджет GPS-слежение
 \item Несколько новых внешних модулей: <<Пространственные запросы>>,
 <<Инструменты GDAL>>, <<Google Earth>>\dots
 \item Добавление таблицы атрибутов к компоновке карты
 \item Сохранение векторного слоя в режиме редактирования
 \item Перемещение нового модуля подписей в ядро и показ подписей
 генерируемых новым модулем в компоновке карты (однако, без корректного
 масштабирования)
 \item Добавлена поддержка <<Drag and drop>> для создания легенды в
 компоновке карты
 \item Расширенный модуль привязки растров с новым интерфейсом
 \item Добавлена возможность ограничить запрос объектов текущим охватом
 в модуле добавления слоя WFS
 \item Инструмент связывания атрибутов теперь поддерживает файлы dbf
 и csv % table join rulez!
 \item Показ количества объектов в заголовке атрибутивной таблицы
 \item Консоль Python может запускаться вне QGIS
 \item Новый инструмент текстовой аннотации
 \item Новый ползунок масштаба WMS
 \item Улучшения в поиске систем координат (показать/скрыть устаревшие
 системы координат)
 \item Добавлены GDAL-совместимые системы координат (без параметров
 +towgs84) для польской epsg: 2172-2180, 3120, 3328-3335 и 4179 как
 epsg+40000
 \item Автоматическое изменение единиц измерения в диалоговом окне
 выбора системы координат, когда пользователь выбирает новую систему
 координат
 \item Новое окно соединения с Oracle Spatial GeoRaster
 \item Новое окно добавления слоя/базы данных Spatialite и возможность
 создания нескольких слоев Spatialite

\end{itemize}
