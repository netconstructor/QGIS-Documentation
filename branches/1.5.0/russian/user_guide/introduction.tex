%  !TeX  root  =  user_guide.tex
\pagestyle{scrheadings}
\chapter{Введение в ГИС}\label{label_intro}

% when the revision of a section has been finalized,
% comment out the following line:
%\updatedisclaimer

Географическая Информационная Система (ГИС) (\cite{mitchel05}
\footnote{Эта глава написана Tyler Mitchell
(\url{http://www.oreillynet.com/pub/wlg/7053}) и подлежит лицензированию
в соответствии с Creative Commons License. Tyler автор книги
\textit{Web Mapping Illustrated}, опубликованной издательством O'Reilly
в 2005 году.}) это набор программного обеспечения, которое позволяет
создавать, визуализировать, строить запросы и анализировать
пространственные данные. Пространственные данные относятся к информации
о географическом положении объекта. Зачастую это предполагает
использование географических координат, таких как значение широты и
долготы. Наряду с термином пространственные данные, часто используются
другие термины, такие как: географические данные, ГИС данные,
картографические данные, данные о местоположении, данные о координатах
и данные о пространственной геометрии.

Приложения, использующие пространственные данные, выполняют различные
функции. Производство карт"---наиболее простая в понимании функция
геоинформационных приложений. Картографические программы отображают
пространственные данные в форме, которая пригодна для просмотра на
экране компьютера или на распечатанной странице. Приложения могут
представлять данные в виде статической карты (простое изображение) или
в виде динамических карт, которые настроены для просмотра посредством
настольного приложения или на веб-странице.

Многие люди ошибочно полагают, что геоинформационные системы просто
создают карты, но анализ пространственных данных"---другая важнейшая
функция геоинформационных систем. Некоторые типичные виды анализа
включают вычисления:

\begin{enumerate}
\item расстояний между географическими местоположениями
\item площадей (например, в квадратных метрах) в пределах определенной
территории
\item того, что географические объекты перекрывают другие объекты
\item размера перекрытия между объектами
\item числа объектов в пределах расстояния от других объектов
\item и так далее...
\end{enumerate}

Это может показаться излишне простым, но может применяться в самых
различных направлениях многих областей науки. Результаты анализа могут
быть показаны на карте, но зачастую оформляются в виде отчета для
поддержки принятия управленческих решений.

Последние события в сфере услуг на основе определения местоположения,
предвещают появление новых возможностей, основанных на комбинировании
функций карт и анализа. Например, у вас есть телефон, который
отслеживает свое местоположение. При наличии соответствующего
программного обеспечения, ваш телефон может подсказать вам какие
рестораны находятся в пределах пешей досягаемости. Несмотря на то, что
это новое применение геоинформационных технологий, по существу это дает
вам результаты анализа геопространственных данных.

\section{Почему все это так ново?}\label{label_whynew}

Ну, это не так и ново. Существует множество новых устройств, которые
поддерживают мобильные геоинформационные услуги. Также доступны многие
геоинформационные приложения с открытым исходным кодом, но в
существовании пространственно ориентированных устройств и приложений нет
ничего нового. Приемники глобальной системы позиционирования (GPS)
являются обычным явлением, но они использовались в различных отраслях
более десятка лет. Кроме того, настольные картографические системы
и аналитические инструменты также были одним из основных коммерческих
рынков, особенно в сфере управления природными ресурсами.

Новым является то, как и кем используется новейшее оборудование и
программное обеспечение. Традиционные пользователи инструментов
картирования и анализа были высококвалифицированными ГИС-аналитиками или
специалистами в цифровой картографии, подготовленными к использованию
CAD-подобных инструментов. Теперь же, вычислительные возможности
домашних компьютеров и программного обеспечения с открытым исходным
кодом предоставляют армии любителей, профессионалов, веб-разработчиков
и так далее, возможности для работы с пространственными данными. Кривая
обученности устремилась вниз. Цены устремились вниз. Возросла значимость
геоинформационных технологий.

В каком виде хранятся пространственные данные? Вкратце, существует два
типа пространственных данных широко используемых сегодня. Это в
дополнение к традиционным табличным данным, которые также широко
используются геоинформационными приложениями.

\subsection{Растровые данные}\label{label_rasterdata}

Один из типов геоинформационных данных называется растровыми данными,
или просто <<растр>>. Наиболее распространенными видами растровых данных
являются цифровые спутниковые снимки или аэрофотоснимки. Высотные
отмывки или цифровые модели рельефа также представляются в виде растровых
данных. В виде растровых данных могут быть представлены любые объекты
карты, но существуют и ограничения.

Растр представляет собой регулярную сетку ячеек, или, в случаях когда
говорят об изображении, пикселей. Сетка имеет фиксированное количество
строк и столбцов. Каждая ячейка имеет числовое значение и определенное
пространственное разрешение (например, 30x30 метров).

Несколько перекрывающихся растров используются для получения изображений
с более, чем одним значением цвета (то есть, один растр для каждого
набора значений красного, зеленого и синего, комбинируется для создания
цветного изображения). Спутниковые изображения также отображают данные
в виде нескольких <<каналов>>. Каждый канал, это, по-существу, отдельный,
пространственно перекрывающийся растр, содержащий значения определенной
длины световой волны. Как можно себе представить, большие растры имеют
больший размер файла. Растр с меньшим размером ячейки передает более
детальное изображение, но занимает больше места. Хитрость заключается в
нахождении баланса между размером ячейки для целей хранения, и размером
ячейки для исследовательских или картографических целей.

\subsection{Векторные данные}\label{label_vectordata}

Векторные данные также используются в геоинформационных системах. Если
вы не прогуливали занятия по геометрии и тригонометрии, то уже знакомы
с некоторыми характеристиками векторных данных. В самом простом смысле,
вектор"---это способ описания местоположения с помощью набора координат.
Каждая координата соотносится с географическим местоположением с помощью
системы значений X и Y.

Это можно рассматривать со ссылкой на декартову плоскость"---диаграммы,
отображающие оси X и Y. Их можно использовать для создания графика
снижений пенсионных накоплений или расчета процентов по ипотеке, но при
анализе пространственных данных и их картировании, важны концепция и
общие понятия.

В зависимости от целей, существуют различные способы представления
географических координат. Это еще одна целая область исследований"---
картографические проекции.

Векторные данные могут представляться в трех формах, каждая из которых,
более сложная и основана на предыдущей.

\begin{enumerate}
\item Точки"--- Одна пара координат (x y) определяет отдельное
географическое местоположение
\item Линии"--- Множество пар координат (x1 y1, x2 y2, x3 y4, ... xn yn)
следующих в определенном порядке, подобно рисованию линии из точки
(x1 y1) в точку (x2 y2) и так далее. Эти части между каждой точкой
считаются сегментом линии. Они имеют длину и направление, которое
определяется порядком следования точек. Технически, линия представляет
собой две пары координат соединенных вместе, в то время как ломаная
линия образуется объединением сегментов.
\item Полигоны"--- Если линии образуются последовательностью из более
чем двух точек, с последней точкой в том же положении, что и первая,
это называется полигоном. Треугольник, круг, прямоугольник и так далее"---
все это полигоны. Ключевая особенность полигонов"--- это фиксированная
область внутри них.
\end{enumerate}
