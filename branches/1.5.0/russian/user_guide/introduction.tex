%  !TeX  root  =  user_guide.tex
\pagestyle{scrheadings}
\chapter{Введение в ГИС}\label{label_intro}

% when the revision of a section has been finalized,
% comment out the following line:
%\updatedisclaimer

Географическая Информационная Система (ГИС) (\cite{mitchel05}
) представляет собой пакет программного обеспечения, предназначенный для
создания, визуализации, поиска и анализа пространственных данных.
\footnote{Эта глава написана Тайлером Митчеллом (Tyler Mitchell, 
\url{http://www.oreillynet.com/pub/wlg/7053}) и публикуется на условиях
лицензии Creative Commons. Т. Митчелл является автором книги
\textit{Web Mapping Illustrated}, опубликованной издательством O'Reilly
в 2005 году.}
Пространственные данные относятся к информации
о географическом положении объекта. Зачастую это предполагает
использование географических координат, таких как широта и долгота. Наряду с термином пространственные данные часто используются
другие термины, например: географические данные, ГИС-данные,
картографические данные, данные о местоположении, данные о координатах
и данные о пространственной геометрии.

Круг задач приложений для работы с пространственными данными достаточно широк.
Производство карт "--- наиболее простая в понимании функция
геоинформационных приложений. Картографические программы выводят
пространственные данные в пригодном для просмотра на экране или распечатки виде.
Приложения могут представлять данные в виде статических (простое изображение) или
динамических карт, которые предназначены для просмотра посредством
настольного приложения или на веб-странице.

Многие люди ошибочно полагают, что геоинформационные системы просто
создают карты, но анализ пространственных данных "--- другая важнейшая
задача геоинформационных систем. Примерами геоинформационного анализа
могут быть вычисления:

\begin{enumerate}
\item расстояний между географическими объектами;
\item площадей (например, в квадратных метрах) определенной территории;
\item количества пересечений одних географических объектов другими;
\item площадей перекрытия объектов;
\item количества объектов в пределах определённого расстояния от заданной точки
\item и так далее...
\end{enumerate}

Эти функции кажутся очень простыми, однако они применяются в самых
различных направлениях многих областей науки. Результаты анализа могут
быть показаны на карте, но зачастую оформляются в виде отчётов для
поддержки принятия управленческих решений.

Последние события в сфере услуг на основе определения местоположения,
предвещают появление новых возможностей, основанных на комбинировании
функций карт и анализа. Например, у вас есть телефон, который
отслеживает своё местоположение. При наличии соответствующего
программного обеспечения, телефон может подсказать вам, какие
рестораны находятся в пределах пешей досягаемости. Подобные
прикладные реализации геоинформационных технологий по существу выполняют
анализ пространственных данных и вывод результатов в удобной для пользователя
форме.

\section{В чём новизна?}\label{label_whynew}

Как таковой, новизны в этом нет. Существует множество новых устройств, которые
поддерживают мобильные геоинформационные услуги. Также доступны многие
геоинформационные приложения с открытым исходным кодом, но в
существовании пространственно-ориентированных устройств и приложений нет
ничего нового. Приёмники глобальной системы позиционирования (GPS) "--- обычное
явление, они использовались в различных отраслях более десятка лет.
Настольные картографические системы и инструменты анализа также были одним из
основных коммерческих рынков, особенно в сфере управления природными ресурсами.

Новизна заключается в том, как и кем используется новейшее оборудование и
программное обеспечение. Традиционными пользователями инструментов
картирования и анализа были высококвалифицированные инженеры или
специалисты в цифровой картографии, подготовленные к работе с
САПР и подобными системами. Теперь же, вычислительные возможности
домашних компьютеров и программного обеспечения с открытым исходным
кодом дают возможность работы с пространственными данными любителям, профессионалам,
веб-разработчикам и так далее. Кривая обучаемости устремляется вниз. Цены устремляются вниз. Значимость геоинформационных технологий возрастает.

В каком виде хранятся пространственные данные? В дополнение к традиционным
табличным данным (которые также широко используются в геоинформационных
приложениях), существует два основных тип пространственных данных: растровые и
векторные.

\subsection{Растровые данные}\label{label_rasterdata}

Первый тип геоинформационных данных "--- растровые данные, которые чаще называют
просто <<растр>>. Наиболее распространенными видами растровых данных являются цифровые спутниковые снимки или аэрофотоснимки. Высотные отмывки или цифровые модели рельефа также
представляются в виде растровых данных. В виде растровых данных могут быть
представлены любые объекты карты, но в их применении существуют определённые
ограничения.

Растр представляет собой регулярную сетку ячеек, или, в случаях когда
говорят об изображении, пикселей. Сетка имеет фиксированное количество
строк и столбцов. Каждая ячейка имеет числовое значение и определённое
пространственное разрешение (например, 30x30 метров).

Несколько перекрывающихся растров используются для получения изображений
с более чем одним значением цвета (то есть, набор растров по одному для каждого
значения красного, зеленого и синего комбинируется для создания
цветного изображения). Спутниковые изображения также представлены в виде данных,
состоящих из нескольких <<каналов>>. Канал по существу являются отдельными растрами,
покрывающими одну и ту же область, которые содержат значения определенной
длины световой волны.

Очевидно, что большие растры имеют больший размер файла. Растр с меньшим
размером ячейки передает более детальное изображение, но занимает больше места.
Хитрость заключается в нахождении баланса между размером ячейки для целей
хранения, и размером ячейки для исследовательских или картографических целей.

\subsection{Векторные данные}\label{label_vectordata}

В геоинформационных системах также используются векторные данные. Если
вы не прогуливали занятия по геометрии и тригонометрии, то уже знакомы
с некоторыми характеристиками векторных данных. В самом простом смысле,
вектор "--- это способ описания местоположения с помощью набора координат.
Каждая координата соотносится с географическим местоположением с помощью
системы значений X и Y.

Векторные данные можно рассматривать со ссылкой на декартову плоскость
"--- систему координат, образованную двумя осями "--- X и Y, которую можно встретить,
например, в графике снижения пенсионных накоплений или расчета процентов по ипотеке.
Система координат "--- одно из основных понятий в картировании и анализе пространственных
данных.

В зависимости от целей, существуют различные способы представления
географических координат. Это ещё одна большая область знаний "---
картографические проекции.

Векторные данные могут представляться в трех формах, каждая из которых
более сложная и основана на предыдущей.

\begin{enumerate}
\item Точки "--- одна пара координат (x y) определяет отдельное
географическое местоположение
\item Линии "--- множество пар координат (x1 y1, x2 y2, x3 y4, ... xn yn),
следующих в определенном порядке, задают линию, проведённую из точки
(x1 y1) в точку (x2 y2) и так далее. Части линии между двумя соседними точками
называются сегментом линии. Они имеют длину и направление, которое
определяется порядком следования точек. Технически, линия представляет
собой две пары координат соединённых вместе, в то время как ломаная
линия образуется объединением сегментов.
\item Полигоны "--- если линии образуются последовательностью из более
чем двух точек, с последней точкой в том же положении, что и первая,
такая фигура называется полигоном. Треугольник, круг, прямоугольник и т.д. "---
всё это полигоны. Ключевая особенность любого полигона "--- это замкнутая область,
находящаяся в пределах его границ.
\end{enumerate}
