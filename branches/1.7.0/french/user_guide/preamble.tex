%  !TeX  root  =  user_guide.tex
\frontmatter
\pagestyle{scrplain}
\addchap{Préambule}
\vspace{1cm}
Ce document est le manuel officiel d'utilisation du logiciel \QG. Les logiciels et le matériel décrits dans ce document sont pour la plupart des marques déposées et donc soumises à des obligations légales. \QG est distribué sous la Licence publique générale GNU (GPL). Vous trouverez plus d'informations sur la page internet de \QG \url{http://qgis.osgeo.org}.
\par\bigskip
Les détails, données, résultats, etc. inclus dans ce document ont été écrits et vérifiés au mieux des connaissances des auteurs et des éditeurs. Néanmoins, des erreurs dans le contenu sont possibles.
\par\bigskip
Ainsi l'ensemble des données ne saurait faire l'objet d'une garantie. Les auteurs et les éditeurs ne sauraient être responsables de tout dommage direct, indirect, secondaire ou accessoire découlant de l'utilisation de ce manuel. Les éventuelles corrections sont toujours les bienvenues.
\par\bigskip
Ce document a été rédigé avec \LaTeX. Les sources sont disponibles en code \LaTeX{} via\\ \url{https://svn.osgeo.org/qgis/docs/tags/1.3.0_user_guide} et en PDF via \url{http://qgis.osgeo.org/documentation/manuals.html}. 
Des versions traduites peuvent être téléchargées via la section de documentation du projet QGIS. Pour plus d'informations sur les manières de contribuer à ce document et à sa traduction, veuillez visiter \url{http://www.qgis.org/wiki/} 

\vspace{1cm}
\noindent
\textbf{Références de ce document}
\par\bigskip
Ce document contient des références internes et externes sous forme de lien. Cliquer sur un lien interne provoque un déplacement dans le document, tandis que cliquer sur un lien externe ouvrira une adresse internet dans le navigateur par défaut. En PDF, les liens internes seront indiqués en bleu et les externes en rouge. En HTML, le navigateur affiche et gère les deux types de liens de la même fa\c{c}on.

\newpage

\begin{flushleft}
\textbf{Auteurs et éditeurs :}
 \par\bigskip\noindent
\begin{tabular}{p{4cm} p{4cm} p{4cm}}
Tara Athan & Radim Blazek & Godofredo Contreras   \\
 Otto Dassau & Martin Dobias & Claudia A. Engel \\ 
 Carson J.Q. Farmer &J\"urgen E. Fischer & Anne Ghisla \\
Stephan Holl &  Magnus Homann& Marco Hugentobler \\ 
 Lars Luthman & Gavin Macaulay & Werner Macho \\
  Tyler Mitchell & Brendan Morely&Gary E. Sherman\\
  Tim Sutton & David Willis \\ \
\end{tabular}
\end{flushleft}

\begin{flushleft}
\textbf{Traducteurs version francophone:}
  \par\bigskip\noindent
\begin{tabular}{p{4cm} p{4cm} p{4cm}}
Benjamin Bohard & Jeremy Garniaux & Yves Jacolin \\
Stéphane Morel & Jean Roc Morreale & Marie Silvestre \\
Tahir Tamba & Xavier M. & Cyril de Runz \\
Benjamin Lerre \\
\end{tabular}
\end{flushleft}

Nos remerciements vont à Bertrand Masson pour son aide précieuse quant à la mise en page de ce document, Tisham Dhar pour avoir préparé l'environnement initial de documentation pour MS Windows, à Tom Elwertowski et William Kyngesburye pour la section d'installation sur Mac OS X et à Carlos  D\'{a}vila, Paolo Cavallini et Christian Gunning pour les révisions. Si nous avons négligé de citer ici le nom d'un contributeur, veuillez accepter nos excuses pour cet oubli et nous le signaler pour correction.
\par\bigskip\noindent
\textbf{Copyright \copyright~2004 - 2010 \QG Development Team}
\par\bigskip\noindent
\textbf{Internet :} \url{http://qgis.osgeo.org}

\addsec{Licence de ce document}

La permission de copier, distribuer, modifier ce document est accordée sous les termes de la GNU Free Documentation License, dans sa version 1.3 ou plus récente telle que publiée par la Free Software Foundation; sans modification de son contenu, sans ajouts la précédant ou la suivant. Une copie peut être lue dans la section \ref{label_fdl} nommée "GNU Free Documentation License".

\newpage

