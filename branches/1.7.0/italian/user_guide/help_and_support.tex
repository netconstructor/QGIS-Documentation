%  !TeX  root  =  user_guide.tex  

\chapter{Aiuto e Supporto}\label{label_helpsupport}

% when the revision of a section has been finalized, 
% comment out the following line:
% \updatedisclaimer

\section{Mailinglist}
QGIS è in continuo sviluppo e pertanto non sempre funziona come ci si aspetterebbe.
La maniera migliore per ottenere aiuto e suggerimento in questi casi è quella 
di iscriversi alla mailinglist 'qgis-users': le tue domande raggiungeranno 
molte persone e tutti trarranno beneficio dalle risposte fornite.

\minisec{qgis-users}
Questa lista è utilizzata per discussioni sia generiche su QGIS che 
specifiche su installazione ed utilizzo. 
Per iscriversi a qgis-users visitare la pagina web: \\
\url{http://lists.osgeo.org/mailman/listinfo/qgis-user}

\minisec{fossgis-talk-liste}
Questa lista è dedicata a chi parla tedesco e tratta di GIS Open Source, 
compreso QGIS.
Per iscriversi a fossgis-talk-liste visitare la pagina web: \\
\url{https://lists.fossgis.de/mailman/listinfo/fossgis-talk-liste}

\minisec{qgis-developer}
Questa lista è dedicata agli sviluppatori alle prese con aspetti di natura più tecnica.
Per iscriversi a qgis-developer visitare la pagina web: \\
\url{http://lists.osgeo.org/mailman/listinfo/qgis-developer}

\minisec{qgis-commit}
Ogni volta che viene eseguito un commit negli archivi del codice di QGIS viene 
inviata una email a questa lista.
Per iscriversi a qgis-commit visitare la pagina web: \\
\url{http://lists.osgeo.org/mailman/listinfo/qgis-commit}

\minisec{qgis-trac}
Questa lista fornisce notifiche e-mail in relazione alla gestione del progetto, inclusi 
rapporti di malfunzionamenti, obiettivi, e richieste di nuove funzionalità.
Per iscriversi a qgis-trac visitare la pagina web: \\
\url{http://lists.osgeo.org/mailman/listinfo/qgis-trac}

\minisec{qgis-community-team}
Questa lista tratta di argomenti relativi alla documentazione di
aiuto contestuale, guida utente, esperienza online incluso siti web, 
blog, mailing list, forum, e traduzione. Se si vuole lavorare ad una guida utente, questa lista è un buon punto di partenza 
per far domande.
Per iscriversi a qgis-community-team visitare la pagina web: \\
\url{http://lists.osgeo.org/mailman/listinfo/qgis-community-team}

\minisec{qgis-release-team}
Questa lista si occupa di argomenti relativi al processo di rilascio, 
alla creazione di pacchetti dei binari per i diversi sistemi operativi 
e all'annuncio del rilascio delle nuove versioni.
Per iscriversi a qgis-release-team visitare la pagina web: \\
\url{http://lists.osgeo.org/mailman/listinfo/qgis-release-team}

\minisec{qgis-tr}
Questa lista è dedicata alle attività di traduzione dei manuali e dell'interfaccia grafica (GUI).
Per iscriversi a qgis-tr visitare la pagina web: \\
\url{http://lists.osgeo.org/mailman/listinfo/qgis-tr}

\minisec{qgis-edu}
Questa lista è dedicata alle attività di formazione, inclusa la realizzazione 
di materiale formativo.
Per iscriversi a qgis-edu visitare la pagina web: \\
\url{http://lists.osgeo.org/mailman/listinfo/qgis-edu}

\minisec{qgis-psc}
Questa lista è usata dal Comitato Direttivo per attività concernenti 
la gestione e la direzione di Quantum GIS. 
Per iscriversi a qgis-psc visitare la pagina web: \\
\url{http://lists.osgeo.org/mailman/listinfo/qgis-psc}

Siete invitati ad iscrivervi ed a contribuire 
alle liste fornendo risposte e condividendo le vostre esperienze. Tenete 
presente che qgis-commit e qgis-trac sono progettate come mezzo di notifica 
e non per accogliere post degli utenti. 

\section{IRC}
QGIS è presente anche su IRC: visitateci registrandovi al canale \#qgis su
\url{irc.freenode.net}. Si prega di attendere un po' per le risposte, dato 
che molte persone sul canale fanno anche altre cose e quindi ci può volere 
un po' di tempo prima che la vostra domanda sia notata.
È disponibile anche un supporto commerciale per QGIS: si veda il sito web 
\url{http://qgis.org/en/commercial-support.html} per maggiori informazioni.

Tutte le discussioni sul canale \#qgis sono registrate in un registro a beneficio
di tutti: \url{http://logs.qgis.org}.

\section{BugTracker}
La lista qgis-users è sopratutto dedicata a domande generiche tipo ``come faccio a
fare questo in QGIS?'', ma potrebbe succedere di trovarsi di fronte ad un bug (malfunzionamento) di QGIS. 
È possibile sottoporre un bug tramite il 'bug tracker' \url{http://hub.qgis.org/projects/quantum-gis/issues}. 
Quando si apre un ticket per un bug, si prega di fornire il proprio indirizzo email da usare
per richiedere eventuali informazioni aggiuntive.

Per favore considerate che i bug da voi sottoposti non sempre avranno la priorità
da voi desiderata. Alcuni bug potrebbero richiedere un notevole lavoro di sviluppo
e non sempre si hanno persone disponibili.
Anche la richiesta di funzionalità aggiuntive può essere inoltrata usando lo stesso sistema di ticket come per 
i malfunzionamenti: selezionate \usertext{enhancement} come tipo di ticket.

Potete anche fornire una vostra soluzione ad un bug, utilizzando lo stesso sistema di ticket: selezionate 
\usertext{patch} come tipo di ticket. Qualcuno degli sviluppatori ne farà una revisione e la 
applicherà a QGIS. \\

Prima di vedere la propria soluzione applicata a QGIS potrebbe trascorrere del tempo: gli sviluppatori 
possono essere impegnati in altri lavori.

% unused, since community.qgis.org seems to be lost. (SH)
% There is also a community site for QGIS where we encourage QGIS users to share
% their experiences and provide case studies about how they are using QGIS. The
% community site is available at: http://community.qgis.org 

\section{Blog}
La comunità di QGIS mantiene anche un blog all'indirizzo \url{http://blog.qgis.org}: sono disponibili 
alcuni articoli interessanti per gli utenti come per gli sviluppatori. 
Siete invitati a contribuire al blog dopo esservi registrati!

\section{Wiki}
Infine, manteniamo anche un sito wiki all'indirizzo \url{http://wiki.qgis.org} dove potete trovare una varietà 
di utili informazioni correlate allo sviluppo di QGIS, ai piani di rilascio di nuove versioni, collegamenti a 
siti, consigli di traduzione e così via. 
