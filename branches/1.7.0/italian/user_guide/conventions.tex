%  !TeX  root  =  user_guide.tex

% when the revision of a section has been finalized, 
% comment out the following line:
% \updatedisclaimer

\addchap{Convenzioni}\label{label_conventions}

Questa sezione descrive le convenzioni di rappresentazione grafica usate nel manuale.

\addsec{Convenzioni per l'interfaccia grafica (GUI)}

Le convenzioni stilistiche per l'interfaccia grafica hanno lo scopo di imitarne l'effettivo aspetto. In generale, 
si è evitato di usare immagini o indicazioni che compaiono solo al passaggio del mouse sopra l'indicazione stessa, 
in modo che l'utente possa scorrere visivamente l'interfaccia grafica per trovare quello che più assomiglia 
all'istruzione rappresentata nel manuale.

\begin{itemize}[label=--,itemsep=5pt]
\item  Opzioni da menu: \mainmenuopt{Layer} \arrow
\dropmenuopttwo{mActionAddRasterLayer}{Aggiungi un layer raster}

oppure

\mainmenuopt{Impostazioni} \arrow
\dropmenuopt{Barre degli strumenti} \arrow \dropmenucheck{Digitalizzazione}
\item Strumenti: \toolbtntwo{mActionAddRasterLayer}{Aggiungi raster}
\item Pulsante: \button{Salva come predefinito}
\item Titolo casella di dialogo: \dialog{Proprietà layer}
\item Scheda (tab): \tab{Generale}
\item Oggetto della toolbox: \toolboxtwo{nviz}{nviz - Apri vista 3D in NVIZ}
\item Casella di controllo: \checkbox{Render}
\item Pulsante di scelta:  \radiobuttonon{Postgis SRID} \radiobuttonoff{EPSG ID}
\item Scelta numerica: \selectnumber{Hue}{60}
\item Scelta testuale: \selectstring{Stile del bordo}{---Linea continua}
\item Cerca file: \browsebutton
\item Scelta colore: \selectcolor{Colore del bordo}{yellow}
\item Barra di scorrimento: \slider{Trasparenza}{0}{20mm}
\item Inserimento testo: \inputtext{Nome da mostrare}{lakes.shp}
\end{itemize}
L'ombreggiatura caratterizza un componente della GUI cliccabile.

\addsec{Convenzioni per il testo o la tastiera}

Il manuale include anche convenzioni relative al testo, all'inserimento da tastiera e alle parti di codice per indicare diverse entità come classi o metodi. Non hanno alcuna corrispondenza visuale con l'applicativo.

\begin{itemize}[label=--]
%Use for all urls. Otherwise, it is not clickable in the document.
\item Collegamenti web: \url{http://qgis.org}
%\item Single Keystroke: press \keystroke{p}
\item Combinazioni di tasti: premere \keystroke{Ctrl+B} significa la pressione del tasto B mentre si tiene premuto il tasto Ctrl.
\item Nome di un file: \filename{lakes.shp}
%\item Name of a Field: \fieldname{NAMES}
\item Nome di una classe: \classname{NewLayer}
\item Metodo: \method{classFactory}
\item Server: \server{myhost.it}
%\item SQL Table: \sqltable{example needed here}
\item Inserimento di testo da parte dell'utente al prompt dei comandi: \usertext{qgis ---help}
\end{itemize}

Le porzioni di codice sono indicate con un font a spaziatura fissa:
\begin{verbatim}
PROJCS["NAD_1927_Albers",
  GEOGCS["GCS_North_American_1927",
\end{verbatim}

\addsec{Istruzioni specifiche per sistema operativo}

Sequenze della GUI e piccole porzioni di testo possono essere formattate in sequenza lineare, ad es.: cliccare \{\nix{}\win{File} \osx{QGIS}\} \arrow Esci per chiudere QGIS.

Questa notazione indica che sui sistemi operativi Linux, Unix e Windows, bisogna fare click innanzitutto sull'opzione di menu File quindi su Esci nel menu a tendina, mentre su 
Macintosh OSX bisogna fare click dapprima sull'opzione di menu QGIS, poi su Esci dal menu a tendina. Grosse porzioni di testo possono essere formattate come elenco:

\begin{itemize}
\item \nix{fai questo}
\item \win{fai quello}
\item \osx{fai qualcos'altro}
\end{itemize}

oppure in paragrafi.

\nix{} \osx{}Fai questo e questo e questo. Quindi fai questo e questo e questo e questo e questo e questo e questo e questo e questo.

\win{}Fai quello. Poi fai quello e quello e quello e quello e quello e quello e quello e quello e quello e quello e quello e quello e quello e quello e quello.

Le schermate riportate nella guida sono state create su diversi sistemi operativi, indicati da apposite icone alla fine della didascalia.