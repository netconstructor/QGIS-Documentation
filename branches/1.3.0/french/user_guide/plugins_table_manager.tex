%  !TeX  root  =  user_guide.tex  
\section{Extension de gestion de tables}\label{sec:ftools}

Cette extension fournit des outils pour gérer les tables attributaires et leurs champs directement depuis QGIS. Pour le moment, cela regroupe les possibilité d'insertion, de suppression, de déplacement, de copie et d'enregistrement des résultats dans un nouveau fichier shapefile. Cette extension ne permet pas encore l'écriture dans la table originelle, tout les changements sont exportés dans un nouveau fichier pour des raisons de sécurité.

\minisec{Installer l'extension de gestion des tables}

Pour utiliser ces fonctionnalités dans QGIS, vous devez installer l'extension via le\\ \mainmenuopt{Gestionnaire d'extensions python} (voir la section 
\ref{sec:load_external_plugin}). Sélectionnez le menu \mainmenuopt{Extensions} > \mainmenuopt{Gestionnaire d'extensions}, select \dropmenuopt{Gestionnaire de table} et cliquez sur \button{OK}. Un nouvel icône \toolbtntwo{tableManagerIcon}{Gestionnaire de table} apparaît dans la barre d'outils.
