% vim: set textwidth=78 autoindent:

% QGIS Tips
% define tip float
% doesn't work if written in qgis_style.sty
% please keep the style definitions here and 
% and load float package in qgis_style.sty
\floatstyle{ruled}
\newfloat{Tip}{ht}{lox}
\floatname{Tip}{Hinweis}
\newcommand\qgistip[1]{\raggedright\small{#1}}
\renewcommand{\topfraction}{0.85}
\renewcommand{\textfraction}{0.1}
\renewcommand{\floatpagefraction}{0.75}

\thispagestyle{empty}
\addcontentsline{toc}{section}{Pr�ambel}

%%%%%%%%%%% nothing to change above %%%%%%%%%%

\section*{Pr�ambel}

% when the revision of a section has been finalized, 
% comment out the following line:
% \updatedisclaimer

Dieses Werk ist das offizielle Handbuch zur Benutzung und Installation der
Software Quantum GIS. Die in diesem Werk genannten Soft- und
Hardwarebezeichnungen sind in den meisten F�llen auch eingetragene
Warenzeichen und unterliegen als solche den gesetzlichen Bestimmungen.
Quantum GIS ist unter der GNU General Public License ver�ffentlicht. Weitere
Informationen finden Sie auf der Quantum GIS Homepage \url{http://qgis.osgeo.org}.

Die in diesem Werk enthaltenen Angaben, Daten, Ergebnisse usw. wurden von den
Autoren nach bestem Wissen erstellt und mit Sorgfalt �berpr�ft. Dennoch sind
inhaltliche Fehler nicht v�llig auszuschlie�en. Daher erfolgen alle Angaben
ohne jegliche Verpflichtung oder Garantie. Die Autoren und Herausgeber
�bernehmen aus diesem Grund auch keinerlei Verantwortung oder Haftung f�r
Fehler und deren Folgen. Hinweise auf eventuelle Irrt�mer werden gerne
entgegengenommen.

Dieses Dokument wurde mit \LaTeX~gesetzt. Es ist als \LaTeX-Quelltext
erh�ltlich unter
\href{http://www.qgis.org/wiki/index.php/Manual_Writing}{subversion},
und kann online als PDF Dokument unter
\url{http://www.qgis.org/de/dokumentation/handbuecher.html} heruntergeladen werden.

\textbf{Verweise in diesem Dokument}

Das Dokument enth�lt interne und externe Verweise. Wenn Sie auf einen
internen im PDF in blau gehaltenen Verweis klicken, springen Sie innerhalb
des Dokuments. Klicken Sie auf einen externen im PDF in rot gehaltenen 
Verweis, dann wird mit Ihrem Webbrowser eine Seite im Internet ge�ffnet. In
der HTML Version sind die Farben der Verweise identisch.  

\begin{flushleft}
\textbf{Autoren des englischsprachigen User Guides:}
 
\begin{tabular}{p{4cm} p{4cm} p{4cm} p{4cm}}
Tara Athan & Radim Blazek & Godofredo Contreras & Claudia A. Engel \\
Otto Dassau & Martin Dobias & J\"urgen E. Fischer & Anne Ghisla \\
Stephan Holl & Marco Hugentobler & Magnus Homann & Tim Sutton \\
Lars Luthman & Gavin Macaulay & Werner Macho & Carson J.Q. Farmer \\
Tyler Mitchell & Brendan Morely & Gary E. Sherman & David Willis \\
\end{tabular}

Herzlichen Dank an Tisham Dhar f�r die Vorbereitung der MSYS-Umgebung, an Tom
Elwertowski und William Kyngesburye f�r ihre Hilfe bei den MAC OSX Kapiteln
und an Carlos D\'{a}vila, Paolo Cavallini and Christian Gunning f�r die
�berarbeitung und Kontrolle.

\textbf{�bersetzung ins Deutsche:}

Die �bersetzung des Benutzerhandbuches wurde vom Kanton Solothurn gesponsort
und von Otto Dassau durchgef�hrt.

\vspace{0.5cm}

\textbf{Copyright \copyright~2004 - 2009 Quantum GIS Development Team} \\
\textbf{Internet:} \url{http://qgis.osgeo.org}
\end{flushleft}

