% vim: set textwidth=78 autoindent:
% QGIS Tips
% define tip float
% doesn't work if written in qgis_style.sty
% please keep the style definitions here and 
% and load float package in qgis_style.sty
\floatstyle{ruled}
\newfloat{Tip}{ht}{lox}
\floatname{Tip}{Tip}
\newcommand\qgistip[1]{\raggedright\small{#1}}
\renewcommand{\topfraction}{0.85}
\renewcommand{\textfraction}{0.1}
\renewcommand{\floatpagefraction}{0.75}

\thispagestyle{empty}
\addcontentsline{toc}{section}{Preambolo}

%%%%%%%%%%% nothing to change above %%%%%%%%%%

\section*{Preambolo}

% when the revision of a section has been finalized, 
% comment out the following line:
%\updatedisclaimer

Questo documento costituisce la traduzione italiana dell'originale
guida all'uso, installazione e programmazione del programma Quantum
GIS. Software e hardware citati in questo documento sono in
molti casi marchi registrati e quindi soggetti a restrizioni
legali. Quantum GIS è soggetto alla GNU General Public License. Maggiori
informazioni alla homepage di Quantum GIS
\url{http://qgis.osgeo.org}.

Dettagli, dati, risultati ecc. presenti in questo documento sono stati
scritti e verificati con la miglior diligenza possibile da parte di autori
ed editori. Non si escludono, tuttavia, errori inerenti il contenuto.

Di conseguenza nessun dato è da ritenere adatto ad alcuno scopo specifico
né tanto meno viene garantito. Gli autori e gli editori non si assumono alcuna responsabilità per eventuali
danni e per le loro conseguenze. Sono comunque ben accette le segnalazioni
di possibili errori.

Questo documento è stato formattato con \LaTeX~ed è disponibile sia come codice sorgente \LaTeX~scaricabile
tramite \href{http://wiki.qgis.org/qgiswiki/DocumentationWritersCorner}{subversion} sia
come documento PDF disponibile online all'indirizzo \url{http://qgis.osgeo.org/documentation/manuals.html}.
Anche le versioni tradotte di questo documento possono essere scaricate dall'area documentazione
del progetto QGIS. Per ulteriori informazioni sul come contribuire a questo documento e alla sua
traduzione, si prega di visitare questo link: \url{http://wiki.qgis.org/qgiswiki/DocumentationWritersCorner}

\vspace{0.5cm}

\textbf{Collegamenti in questo documento}

Questo documento contiene link interni ed esterni. Cliccando su un
link interno ci si muove all'interno del documento stesso, mentre
cliccando su un link esterno si aprirà un indirizzo internet. Nel
formato PDF, i collegamenti interni sono mostrati in colore blu,
mentre quelli esterni sono mostrati in colore rosso e gestiti dal
browser di sistema. In formato HTML il browser mostra e gestisce in
maniera identica entrambi i tipi di collegamento.

\begin{flushleft}
\textbf{Autori ed editori della guida all'uso, installazione e programmazione:}

\begin{tabular}{p{4cm} p{4cm} p{4cm} p{4cm}}
Tara Athan & Radim Blazek & Godofredo Contreras & Claudia A. Engel \\
Otto Dassau & Martin Dobias & Peter Ersts & J\"urgen E. Fischer \\ 
Anne Ghisla & Stephan Holl & N. Horning & Marco Hugentobler \\ 
Magnus Homann & K. Koy & Lars Luthman & Gavin Macaulay \\
Werner Macho & Carson J.Q. Farmer & Tyler Mitchell & Brendan Morely \\
Gary E. Sherman & Tim Sutton & David Willis \\
\end{tabular}

Con i ringraziamenti a Tisham Dhar per aver preparato la documentazione
iniziale dell'ambiente msys (MS Windows), a Tom Elwertowski e William
Kyngesburye per l'aiuto alla sezione di installazione su MAC OSX e
a Carlos Dàvila, Paolo Cavallini e Christian Gunning per le revisioni.
Se avessimo dimenticato di menzionare qualche collaboratore, lo preghiamo
di accettare le nostre scuse per la svista.

\textbf{Copyright \copyright~2004 - 2010 Quantum GIS Development Team} \\
\textbf{Internet:} \url{http://qgis.osgeo.org}
\end{flushleft}

\newpage

\minisec{License of this document}

Permission is granted to copy, distribute and/or modify this document under
the terms of the GNU Free Documentation License, Version 1.3 or any later
version published by the Free Software Foundation; with no Invariant
Sections, no Front-Cover Texts and no Back-Cover Texts.  A copy of the
license is included in section \ref{label_fdl} entitled "GNU Free Documentation
License".


