%  !TeX  root  =  user_guide.tex  
\mainmatter
\pagestyle{scrheadings}
\addchap{Avant-propos}\label{label_forward}

Bienvenue dans le monde merveilleux des Systèmes d'Informations Géographiques (SIG) ! Quantum GIS est un SIG libre qui a débuté en mai 2002 et s'est établi en tant que projet en juin 2002 sur SourceForge. Nous avons travaillé dur pour faire de ce logiciel SIG (qui sont traditionnellement des logiciels propriétaires assez coûteux) un choix viable pour toute personne ayant un ordinateur. \qg est utilisable sur la majorité des Unix, Mac OS X et Windows. \qg utilise la bibliothèque logicielle Qt 4 (\url{http://www.nokia.com}) et le langage C++, ce qui ce traduit par une interface graphique simple et réactive.

\qg se veut simple à utiliser, fournissant des fonctionnalités courantes. Le but initial était de fournir un visualisateur de données SIG. \qg a depuis atteint un stade dans son évolution où beaucoup y recourent pour leurs besoins quotidiens. \qg supporte un grand nombre de formats raster et vecteur, avec un support de nouveaux formats facilités par l'architecture des modules d'extension (lisez l'Annexe \ref{appdx_data_formats} pour une liste complète des formats actuellement supportés)

\qg est distribué sous la licence GPL. Ceci vous permet de pouvoir regarder et modifier le code source, tout en vous garantissant un accès à un programme SIG sans coût et librement modifiable. Vous devez avoir reçu une copie complète de la licence avec votre exemplaire de \qg, vous la trouverez également dans l'Annexe \ref{gpl_appendix}.

\begin{Tip}\caption{\textsc{Documentation à jour}}\index{documentation}
La dernière version de ce document est disponible sur \url{http://download.osgeo.org/qgis/doc/manual/}, ou dans la section documentation du site de \qg \url{http://qgis.osgeo.org/documentation/}
\end{Tip}

\addsec{Fonctionnalités}\label{label_majfeat}

\qg offre beaucoup d'outils SIG standards par défaut et via les extensions. Voici un bref résumé en six catégories qui vous donnera un premier aperçu.

\minisec{Visualiser des données}

Vous pouvez afficher et superposer des couches de données rasters et vecteurs dans différents formats et projections \footnote{\qg ne proposant actuellement de projection à la volée que pour les données de type vecteur, les données de type raster doivent être dans la même projection pour pouvoir être associées entre elles.} sans avoir à faire de conversion dans un format commun. Les formats supportés incluent :

\begin{itemize}[label=--]
\item les tables spatiales de \ppg, les formats vecteurs supportés par la bibliothèque OGR installée, ce qui inclut les fichiers de forme ESRI (shapefiles), MapInfo, STDS et GML (voir l'Annexe \ref{appdx_ogr} pour la liste complète) .
\item les formats raster supportés par la bibliothèque GDAL (Geospatial Data Abstraction Library) tels que GeoTiff, Erads Img., ArcInfo Ascii Grid, JPEG, PNG (voir l'Annexe \ref{appdx_gdal} pour la liste complète).

\item les formats raster et vecteur provenant des bases de données GRASS. 
\item les données spatiales provenant des services réseaux compatibles OGC comme le Web Map Service (WMS) ou le Web Feature Service (WFS) (voir la  section \ref{working_with_ogc}),
\item les données OpenStreetMap (voir la section \ref{plugins_osm}).

\item les bases de données SpatiaLite (lire la section \ref{label_spatialite}) 
\end{itemize}
\minisec{Parcourir les données et créer des cartes} 

Vous pouvez créer des cartes et les parcourir de manière interactive avec une interface abordable. Les outils disponibles dans l'interface sont :

\begin{itemize}[label=--]
\item projection à la volée (adapte les unités de mesure et reprojette automatiquement les données vectorielles)
\item composition de carte
\item panneau de navigation
\item signet géospatial
\item identification et sélection des entités
\item affichage, édition et recherche des attributs
\item étiquetage des entités
\item personnalisation de la symbologie des données raster et vecteur
\item ajout d'une couche de graticule lors de la composition
\item ajout d'une barre d'échelle, d'une flèche indiquant le nord et d'une étiquette de droits d'auteur
\item sauvegarde et chargement de projets
\end{itemize}

\minisec{Créer, éditer, gérer et exporter des données}

Vous pouvez créer, éditer, gérer et exporter des données vectorielles dans plusieurs formats. \qg permet ce qui suit :

\begin{itemize}[label=--]
\item outils de numérisation pour les formats d'OGR et les couches vecteurs de GRASS
\item créer et éditer des fichiers de forme (shapefiles) et les couches vecteur de GRASS
\item géoréférencer des images avec l'extension de géoréférencement
\item outils d'import/export du format GPX pour les données GPS, avec la conversion des autres formats GPS vers le GPX ou l'envoi/réception directement vers une unité GPS
\item créer des couches \pg à partir de fichiers shapefiles avec l'extension SPIT
\item gérer les attributs de tables des couches vecteur grâce à l'extension de gestion des tables ou celle de tables attributaires (voir la section \ref{sec:attribute table})
\item enregistrer des captures d'écran en tant qu'images géoréférencées
\end{itemize}

\minisec{Analyser les données} 

Vous pouvez opérer des analyses spatiales sur des données \ppg et autres formats OGR en utilisant l'extension ftools. \qg permet actuellement l'analyse vectorielle, l'échantillonnage, la gestion de la géométrie et des bases de données. Vous pouvez aussi utiliser les outils GRASS intégrés qui comportent plus de 300 modules (voir la section \ref{sec:grass})

\minisec{Publier une carte sur Internet}

\qg peut être employé pour exporter des données vers un mapfile et le publier sur Internet via un serveur web employant l'UMN MapServer. \qg peut aussi servir de client WMS/WFS ou de serveur WMS.

\minisec{Étendre les fonctionnalités de \qg grâce à des extensions} 

\qg peut être adapté à vos besoins particuliers du fait de son architecture d'extensions. \qg fournit des bibliothèques qui peuvent être employées pour créer des extensions, vous pouvez même créer de nouvelles applications en C++ ou python !

\minisec{Extensions principales}

\begin{enumerate}
\item Ajouter une couche de texte délimité (charge et affiche des fichiers texte ayant des colonnes contenant des coordonnées X/Y)
\item Capture de coordonnées (Enregistre les coordonnées sous la souris dans un SCR différent)
\item Décorations (Étiquette de droit d'auteur, flèche indiquant le nord et barre d'échelle)
\item Insertion de diagrammes (place des diagrammes sur une couche vectorielle)
\item Convertisseur Dxf2Shp (convertit les fichiers DXF en fichier SHP)
\item Outils GPS Tools (importe et exporte des données GPS)
\item GRASS (intégration du SIG GRASS)
\item Géoréférenceur GDAL (ajoute une projection à un raster)
\item Extension d'Interpolation (interpole une surface en utilisant les sommets d'une couche vectorielle)
\item Export Mapserver (exporte un fichier de projet QGIS dans le format de carte de MapServer)
\item Convertisseur de couche OGR (convertit un fichier vectoriel dans plusieurs formats)
\item Extenstion OpenStreetMap (permet de visualiser et d'éditer des données OSM)
\item support des GeoRaster d'Oracle Spatial
\item Installateur d'extension Python (télécharge et installe des extensions python pour \qg)
\item Impression rapide (Imprimer une carte en un minimum d'effort)
\item Analyse de terrain raster
\item SPIT, outil d'importation de Shapefile vers \ppg
\item Ajouter une couche WFS 
\item eVIS (visualisation d'évènements multimédias)
\item fTools (outils d'analyse et de gestion de vecteurs)
\item Console Python (accédant à l'environnement QGIS)
\item Outils graphiques pour GDAL
\end{enumerate}

%\minisec{External Python Plugins}
\minisec{Extensions Python externes}

\qg offre un nombre croissant d'extensions complémentaires en Python fournies par la communauté. Ces extensions sont entreposées dans le répertoire UTILISATEUR\footnote{L'emplacement change selon le système d'exploitation, ainsi sous \nix{} il s'agit du répertoire HOME tandis que sous \win{} il s'agit du répertoire utilisateur se situant dans Document And Settings}/.qgis/python/plugins et peuvent être facilement installées en utilisant l'extension d'installation Python (voir la section \ref{sec:plugins}). 

\subsubsection{Quoi de neuf dans la version ~\CURRENT} 

Voici les ajouts et améliorations les plus notables :
\begin{itemize}[label=--]
%  \item TODO(anne): rearrange the list on ManualTasks wikipage!
 \item Ajout d'une aide contextuelle dans les dialogues
 \item Outil de mesure d'angle
 \item Outil de suivi GPS 
 \item Plusieurs nouvelles extensions de qualité 'Spatial Query, GDAL Tools, Google Earth \dots"
 \item Ajout d'une table d'attributs sur la composition de carte
 \item Enregistrer une couche vectorielle pendant l'édition
 \item Intégration du nouveau moteur d'étiquetage et affichage des étiquettes dans une composition
 \item Ajout d'un mode "Glisser-Déposer" pour les légendes de carte
 \item Nouvelle interface pour le géoréférenceur
 \item Option BBOX pour le WFS 
 \item L'outil de jointure d'attribut supporte maintenant les fichiers dbf et csv
 \item Indique le nombre d'entités dans la table attributaire
 \item La console Python fonctionne aussi en-dehors de \qg
 \item Nouvel outil d'annotation
 \item Barre de zoom pour les couches WMS
 \item Amélioration de la recherche de projections
 \item Ajout de SCR compatibles avec GDAL (sans les paramètres +towgs84) pour les codes epsg polonais : 2172-2180, 3120, 3328-3335 et 4179 comme epsg+40000
 \item Changement automatique des unités de la carte dans la fenêtre de projet lors de la sélection d'un SCR
 \item Nouveau dialogue de connexion aux rasters d'Oracle
 \item Ajout du support SpatiaLite et de la création de multiples couches SpatiaLites via \qg
\end{itemize}
