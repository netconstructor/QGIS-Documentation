%  !TeX  root  =  user_guide.tex  
\mainmatter
\pagestyle{scrheadings}
\addchap{写在前面}\label{label_forward}
\pagenumbering{arabic}
\setcounter{page}{1}

% when the revision of a section has been finalized, 
% comment out the following line:
% \updatedisclaimer

欢迎来到地理信息系统(GIS)的迷人世界!Quantum GIS(QGIS)是一个开源GIS软件。本项目开始于2002年5月并于同年6月在SourceForge落户。我们努力使GIS软件(在传统上是非常贵的专属软件)成为一个对任何拥有个人电脑的人都可用的软件。QGIS目前可运行在Unix、Windows和OS X平台之上。QGIS是基于跨平台的图形工具Qt软件包(\url{http://qt.nokia.com}),并采用C++语言开发的一个GIS软件。这意味着QGIS反应迅速并且拥有一个令人愉悦和易于操作的图形用户界面(GUI)。

QGIS致力于成为简洁易用的GIS平台,提供了通用的功能和特性。QGIS最初的目的是提供一个GIS数据浏览器。经过不断改进,QGIS已经可以满足大部分用户日常的浏览GIS数据的需求。QGIS支持许多栅格和矢量数据格式,通过插件机制可以很容易地增加对新数据格式的支持。(目前所支持的所有数据格式参见\ref{appdx_data_formats})。

QGIS是在GNU通用公共许可证(GPL)授权下发布的。在该许可证下发布意味着用户可以查看和修改源代码,并且可以保证我们的用户能够使用一个免费并且可自由修改的GIS软件。用户的QGIS软件会附带一份完整的GPL许可证副本,你也可在附录\ref{gpl_appendix}中查看该许可证。

\begin{Tip}\caption{\textsc{最新文档}}\index{documentation}
本文档的最新版本发布在:\url{http://download.osgeo.org/qgis/doc/manual/},或者在QGIS网站文档区:\url{http://qgis.osgeo.org/documentation/}
\end{Tip}

\addsec{特性}\label{label_majfeat}
\qg 通过核心组件和插件提供了许多通用的GIS功能。这些功能可以大致分为以下六类:

\minisec{数据查看}

用户可以查看并叠置不同格式的矢量和栅格数据,并且可以不经转换为内部或通用格式而查看地图投影。所支持的格式包括:

\begin{itemize}[label=--]
\item 可空间操作的PostgreSQL表数据,主要是PostGIS和其他OGR库支持的矢量格式,包括ESRI的shape文件、Mapinfo文件、SDT文件和GML文件(详情参阅附录\ref{appdx_ogr})。
\item GDAL(地理空间数据抽象库)库支持的栅格影像数据,如GeoTiff格式、Erdas Img格式、ArcInfo Ascii Grid格式、JPEG格式、PNG格式等(详情参阅附录\ref{appdx_gdal})。
\item 空间数据库格式(参见\ref{label_spatialite}) 
\item GRASS数据库(区位/图集)里的GRASS栅格及矢量数据格式,参见\ref{sec:grass}, 
\item 遵循OGC网络地图服务标准(WMS)或网络图元服务标准(WFS)的在线空间数据,参见\ref{working_with_ogc},
\item OpenStreetMap数据格式(参见\ref{plugins_osm}).
\end{itemize}

\minisec{数据漫游和地图合成} 

用户可通过友好的图形用户界面(GUI)交互地合成地图和漫游数据。在GUI中有很多有用的工具,包括:

\begin{itemize}[label=--]
\item 实时地图投影
\item 地图生成器
\item 全局面板
\item 空间位置书签
\item 标识/选中图元
\item 编辑/查看/搜索属性数据
\item 图元标签
\item 改变矢量或栅格符号
\item 增加方格图图层 - 通过fTools插件实现
\item 使用带指北针的比例尺或版权标志来装饰地图
\item 保存和重新载入工程
\end{itemize}

\minisec{数据创建、编辑、管理和导出}

用户可创建、编辑、管理和出口几种不同格式的矢量地图。栅格数据必须输入到GRASS中才能以其他格式编辑和输出。QGIS提供以下功能:

\begin{itemize}[label=--]
\item 数字化OGR支持的格式和GRASS矢量图层
\item 创建和编辑shape文件及GRASS矢量图层
\item 通过Georeferencer插件对图像进行地理编码
\item 用GPS工具输入输出GPX格式,并把其他GPS格式转化成GPX或者直接下载或上传到一个GPS装置(Linux系统已经对GPS设备列表增加了usb支持)
\item 使OpenStreetMap数据可视化并对其进行编辑
\item 通过SPIT插件对shp文件创建PostGIS图层
\item 对PostGIS桌面处理进行了改进
\item 使用新的属性表(参见\ref{sec:attribute table})或者Table Manager插件管理矢量属性表
\item 把截图存为地标图像
\end{itemize}

\minisec{数据分析} 

用户可以通过PostgreSQL/PostGIS数据库管理系统完成空间数据分析,而且也可以通过使用fTools Python插件来执行其他OGR支持的格式。目前OGIS提供了矢量分析,取样,地理处理,几何和数据库管理工具箱。用户也可以使用完整的GRASS工具箱,包括了GRASS的300多个模块的全部功能(参见\ref{sec:grass})。

\minisec{地图数据的网络发布}

QGIS可用于把数据输出到一个映像文件或者通过UMN Mapserver服务器把数据发布到因特网。QGIS也可作为WMS或WFS的一个客户端,还可作为WMS的服务器。

\minisec{QGIS功能的插件式扩展} 

QGIS可通过扩展插件构架来适应用户的特殊需求。QGIS提供了可用于创建插件的函数库。用户甚至可以用C++或插件创建新的应用程序!

\minisec{核心插件}

\begin{enumerate}
\item 添加界定文本层插件(加载并显示包含x、y坐标的界定文本文件)
\item 坐标捕捉插件(在不同的CRS中捕捉鼠标坐标)
\item 装饰插件(版权标签、指北针和比例尺)
\item 图表覆盖插件(把图表覆盖到矢量图层上)
\item Dxf2Shp转换器插件(把DXF转换成Shape)
\item GPS工具箱插件(装载并输入GPS数据)
\item GRASS插件(集成GRASS GIS)
\item Georeference GDAL插件(GDAL将投影信息添加到栅格数据)
\item 插入插件(基于矢量图层最顶层进行插入)
\item Mapserver输出插件(将QGIS工程文件转换成一个Mapserver地图文件)
\item OGR图层转换器插件(矢量图层格式的转换)
\item OpenStreetMap插件(对openstreetmap数据的查看和编辑)
\item Oracle Spatial GeoRaster支持插件
\item Python插件安装插件(下载并安装QGIS的python插件)
\item 快速打印插件(高效打印出一张地图)
\item 栅格地形分析插件(栅格基准地形分析)
\item SPIT插件(将形文件输入到PostgreSQL/PostGIS)
\item WFS插件插件(将WFS图层添加到QGIS界面canvas)
\item eVIS插件(事件可视化工具)
\item fTools插件(矢量数据分析和管理的工具)
\item Python控制台插件(进入QGIS环境)
\item GDAL工具箱插件
\end{enumerate}

\minisec{Python外部插件}

QGIS提供了越来越多的由社区提供的外部python插件。这些插件存在于官方的PyQGIS插件仓库中,使用Python插件安装插件可以很容易地进行安装(参见\ref{sec:plugins})。

\subsubsection{\qg \CURRENT 中的新特性} 

请注意,本版本是QGIS系列版本的最新发行版。因此其在QGIS 1.0.x和QGIS 1.5.0基础上添加了许多新特性,并扩展了编程接口。我们推荐您使用当前版本代替以前的旧版本。

本发行版添加了许多新特性和增强功能,并修复了177处错误。

\textbf{通用特性变化}

\begin{itemize}[label=--]
\item 添加动态GPS跟踪的gpsd支持
\item 增加一个新的插件以支持线下编辑
\item 字段计算器在出现计算错误时会插入空(NULL)图元值,而不是停止计算并反转所有图元的计算值
\item 更新了srs.db文件以包含网格引用
\item 增加一个原生的栅格计算实现(C++)以有效地处理巨量栅格数据
\item 状态条与扩展部件更好地集成,以实现部件文字内容的复制粘贴
\item 字段计算器的诸多增强和新操作功能的添加,如字段连接、行计数器等
\item 添加了--configpath选项覆盖默认的配置路径(~/.qgis)支持并强制QSettings使用该目录以支持用户自定义设置。这样,用户可以,比如,将QGIS连同所有插件和设置转移到闪存盘
\item 试验性的WFS-T支持。可选地将WFS转移至网络管理
\item Georeferencer插件的诸多细微改进和增强
\item 属性对话框及编辑器对长整型数据(long int)的支持
\item QGIS Mapserver项目已经合并到主SVN仓库中,软件包已经可用。QGIS Mapserver使得您能够通过OGC的WMS标准协议来发布您自己的QGIS项目
\item 选择并度量工具条的弹出窗口和子菜单项
\item 新增对非空间数据表(当前主要是OGR、文本数据以及PostgreSQL数据)的支持。这些数据表支持字段查找,或在数据表视图下进行一般性地浏览和编辑。
\item 对图元ID查询字符串的支持以及其他多种查询相关的增强
\item 支持地图图层和数据提供者的重新载入操作。比如缓冲数据提供者(当前主要是WMS和WFS)可以与数据源的修改同步。
\end{itemize}

\textbf{图层列表(TOC)变化}

\begin{itemize}[label=--]
\item 栅格图例菜单项新增加了一个选项以支持在当前范围内对当前图层通过最小、最大像素值进行缩放。
\item 当使用图层列表上下文菜单的“另存为”按钮保存矢量数据图层时,您可以设置OGR的创建选项。
\item 在图层列表现在可以直接选择、移除或移动多个图层。
\end{itemize}

\textbf{标记(仅适用于新版本)}

\begin{itemize}[label=--]
\item 数据限定的标记位置
\item 自动断行、数据限定的字体和缓冲区设置
\end{itemize}

\textbf{图层属性和符号}

\begin{itemize}[label=--]
\item 添加三种新的渐变符号渲染(版本2)分类模式,包括自然断点(Natural Breaks, Jenks)、标准差和Pretty Breaks(基于R统计环境的优异)。 
\item 符号属性对话框更快的加载速度。
\item 用于分组式和渐进式渲染(symbology-ng)的数据限定的旋转度和大小。
\item 直线符号可以使用大小比例来改变线宽。
\item 新的基于Qwt的栅格直方图实现。直方图可以保存为图像文件。栅格直方图的x轴上显示像素值。
\item 交互地从图层显示区选中一些像素以在栅格图层属性对话框中生成透明度表。
\item 能够通过矢量图层的颜色带多选框创建颜色带。
\item 在符号选择页添加了一个“样式管理器”以方便用户更快地找到央视管理入口
\end{itemize}

\textbf{地图合成}

\begin{itemize}[label=--]
\item 在部件位置对话框可以显示和操作地图合成器部件的宽度和高度。 
\item 现在可以使用回退键(backspace)删除地图合成器的部件。
\item 地图合成器属性表的排序支持(多列以及升、降序)。
\end{itemize}

