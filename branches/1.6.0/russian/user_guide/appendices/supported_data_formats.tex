%  !TeX  root  =  user_guide.tex
\chapter{Поддерживаемые форматы данных}\label{appdx_data_formats}
% when the revision of a section has been finalized,
% comment out the following line:
% \updatedisclaimer

Для чтения и записи векторных и растровых форматов данных QGIS
использует библиотеки GDAL/OGR. Обратите внимание, что, по различным
причинам, не все из перечисленных ниже форматов могут работать в
QGIS. Например, некоторые требуют установку внешних коммерческих
библиотек или библиотека GDAL в вашей операционной системе (ОС) не
поддерживает формат, который вы хотите использовать. При загрузке в
QGIS данных векторных или растровых форматов в списке типов файлов
будут отображаться только те форматы, которые были проверены.
Остальные (непроверенные) форматы могут быть загружены, если выбрать
*.*.

%\section{OGR Vector Formats}\label{appdx_ogr}
\section{Векторные форматы OGR}\label{appdx_ogr}
\index{OGR!поддерживаемые форматы}

На время составления настоящего документа библиотекой OGR
\cite(OGRweb) поддерживаются следующие форматы. Полный список
доступен \url{http://www.gdal.org/ogr/ogr_formats.html}.

\begin{itemize}
\item Arc/Info Binary Coverage
\item Comma Separated Value (.csv)
\item DODS/OPeNDAP
\item ESRI Personal GeoDatabase
\item ESRI ArcSDE
\item Shape-файл ESRI
\item FMEObjects Gateway
\item GeoJSON
\item Geoconcept Export
\item GeoRSS
\item GML
\item GMT
\item GPX
\item GRASS Vector \footnote{Поддержка GRASS обеспечивается
    расширением <<GRASS>>}
\item Informix DataBlade
\item INTERLIS
\item IHO S-57 (ENC)
\item Файл Mapinfo %File
\item Microstation DGN
\item OGDI Vectors
\item ODBC
\item Oracle Spatial
\item PostgreSQL\footnote{В QGIS реализованы собственные функции для
работы с PostgreSQL. OGR должна быть установлена без поддержки PostgreSQL}
\item SDTS
\item SQLite
\item UK .NTF
\item U.S. Census TIGER/Line
\item VRT - Virtual Datasource
\item X-Plane/Flighgear aeronautical data
\end{itemize}

%\section{GDAL Raster Formats}\label{appdx_gdal}
\section{Растровые форматы GDAL}\label{appdx_gdal}
\index{растровые слои!поддерживаемые форматы} \index{GDAL!поддерживаемые форматы}

На время составления настоящего документа библиотекой GDAL
\cite(GDALweb) поддерживаются следующие форматы. Полный список
доступен \url{http://www.gdal.org/formats_list.html}.

\begin{itemize}
\item Arc/Info ASCII Grid
\item ADRG/ARC Digitilized Raster Graphics
\item Arc/Info Binary Grid (.adf)
\item Magellan BLX Topo (.blx, .xlb)
\item Microsoft Windows Device Independent Bitmap (.bmp)
\item BSB Nautical Chart Format (.kap)
\item VTP Binary Terrain Format (.bt)
\item CEOS (Spot for instance)
\item First Generation USGS DOQ (.doq)
\item New Labelled USGS DOQ (.doq)
\item Military Elevation Data (.dt0, .dt1)
\item ERMapper Compressed Wavelets (.ecw)
\item ESRI .hdr Labelled
\item ENVI .hdr Labelled Raster
\item Envisat Image Product (.n1)
\item EOSAT FAST Format
\item FITS (.fits)
\item Graphics Interchange Format (.gif)
\item GMT compatible netCDF
\item GRASS Rasters \footnote{Поддержка GRASS обеспечивается расширением <<GRASS>>}
\item Golden Software Binary Grid
\item TIFF / BigTIFF / GeoTIFF (.tif)
\item Hierarchical Data Format Release 4 (HDF4)
\item Hierarchical Data Format Release 5 (HDF5)
\item ILWIS Raster Map (.mpr,.mpl)
\item Intergraph Raster
\item Erdas Imagine (.img)
\item Atlantis MFF2e
\item Japanese DEM (.mem)
\item JPEG JFIF (.jpg)
\item JPEG2000 (.jp2, .j2k)
\item NOAA Polar Orbiter Level 1b Data Set (AVHRR)
\item Erdas 7.x .LAN and .GIS
\item In Memory Raster
\item Vexcel MFF
\item Vexcel MFF2
\item Atlantis MFF
\item Multi-resolution Seamless Image Database  MrSID
\item NITF
\item NetCDF
\item OGDI Bridge
\item Oracle Spatial Georaster
\item OGC Web Coverage Server
\item OGC Web Map Server
\item PCI .aux Labelled
\item PCI Geomatics Database File
\item PCRaster
\item Portable Network Graphics (.png)
\item Netpbm (.ppm,.pgm)
\item USGS SDTS DEM (*CATD.DDF)
\item SAR CEOS
\item USGS ASCII DEM (.dem)
\item X11 Pixmap (.xpm)

\end{itemize}
\clearpage
