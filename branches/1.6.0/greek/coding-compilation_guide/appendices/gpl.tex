\section{Γενική άδεια χρήσης GNU}\label{gpl_appendix}
\index{license!GPL}

\begin{small}
\begin{center}
ΓΕΝΙΚΗ ΑΔΕΙΑ ΧΡΗΣΗΣ GNU

Έκδοση 2, Ιούνιος 1991


Copyright (C) 1989, 1991 Free Software Foundation, Inc.  
59 Temple Place - Suite 330, Boston, MA  02111-1307, USA


Οποιοσδήποτε έχει την άδεια να αντιγράψει και να διανείμει αντίγραφα αυτής της άδειας κειμένου, αλλά η αλλαγή δεν επιτρέπεται.
\end{center}
Εισαγωγή

Οι άδειες για τα περισσότερα λογισμικά ειναι σχεδιασμένες να σας παίρνουν την ελευθερία να τα αλλάζετε και να τα μοιράζεστε. Αντίθετα, η Γενική Άδεια Χρήσης GNU προορίζεται να σας εγγυηθεί τη ελευθερία να μοιράζεστε και να ανταλλάζετε λογισμικό – για να είναι βέβαιο ότι το λογισμικό θα είναι ελευθερο προς όλους τους χρήστες. Αυτή η Γενική Άδεια Χρήσης ισχύει για τα περισσότερα λογισμικά του Free Software Foundation και για οποιοδήποτε άλλο λογισμικό του οποίου το άτομο που το έφτιαξε δεσμεύεται να τη χρησιμοποιεί (κάποια άλλα λογισμικά του Free Software Foundation αντιθέτως καλύπτονται απο τη Γενική Άδεια Χρήσης GNU). Μπορείτε να το εφαρμόσετε επίσης και στα δικά σας προγράμματα.

Όταν μιλάμε για ελεύθερο λογισμικό, αναφερόμαστε στην ελευθερία, όχι στην τιμή. Οι Γενικές Άδειες Χρήσης μας έχουν σχεδιαστεί για να γίνεται βέβαιο ότι έχετε την ελευθερία να διανέμετε αντίγραφα του ελεύθερου λογισμικού (και να χρεώσετε για αυτή την υπηρεσία αν επιθυμείτε), ότι λαμβάνετε πηγαίο κώδικα ή μπορείτε να τον πάρετε αν θέλετε, ότι μπορείτε να αλλάξετε το λογισμικό ή να χρησιμοποιήσετε κομμάτια του σε καινούργια νέα λογισμικά; και για να ξέρετε ότι μπορείτε να χρησιμποιήσετε αυτά τα πράγματα.

Για να προστατέψετε τα διακαιώματά σας, χρειάζεται να βάλουμε περιορισμούς και να απαγορέψουμε απο οποιονδήποτε σας αρνηθεί αυτά τα δικαιώματα ή σας ζητήσει να παραιτηθείτε απο αυτά. Αυτοί οι περιορισμοί μεταφράζονται σε βέβαιες ευθύνες για εσάς αν διανέμεται αντίγραφα του λογισμικού, ή το αλλάξετε.

Για παράδειγμα, αν διενείμετε αντίγραφα ενός τέτοιου προγράμματος, είτε δωρεάν είτε για κάποιο αντίτιμο, πρέπει να δώσετε στους παραλαμβάντοντες όλα τα διακαιώματα που έχετε. Πρέπει να σιγουρευτείτε ότι και αυτοί, επίσης, λαμβάνουν ή μπορούν να πάρουν τον πηγαίο κώδικα. Και πρέπει να τους δείξετε αυτούς τους όρους έτσι ώστε να γνωρίζουν τα διακιώματά τους.

Προστατέυουμε τα διακαιώματα σας με δύο τρόπους: (1) βάζοντας πνευματική ιδιοκτησία στο λογισμικό και (2) προσφέροντάς σας αυτή την άδεια που σας δίνει ένομη άδεια να αντιγράψετε, να διανέμετε και/ή να αλλάξετε το λογισμικό.

Επίσης, για την προστασία του κάθε συγγραφέα και τη δική μας, θέλουμε να σιγουρευτούμε ότι όλοι καταλαβαίνουν ότι δεν υπάρχει εγγύηση γι'αυτό το ελεύθερο λογισμικό. Αν το λογισμικό αλλαχθεί απο κάποιον άλλον και διαβιβαστεί, θέλουμε οι παραλαμβάνοντες να ξέρουν ότι αυτό που έχουν δεν είναι το αυθεντικό, οπότε οποιαδήποτε προβλήματα που δημιουργούνται απο άλλους δεν αντικατοπτρίζουν την αυθεντική φήμη των συγγραφέων.

Τελικά, οποιοδήποτε ελεύθερο πρόγραμμα απειλείται συνεχώς απο τις πατέντες λογισμικών. Θέλουμε να αποφύγουμε τον κίνδυνο ότι αναδιανομείς ενός ελεύθερου προγράμματος θα αποκτήσουν τελικά άδειες πατέντας, κάνοντας έτσι το πρόγραμμα ιδιωτικό. Για να αποφευχθεί αυτό, έχουμε ξεκαθαρίσει ότι οποιαδήποτε πατέντα πρέπει να αδειοδοτηθεί για την ελεύθερη χρήση οποιουδήποτε ή να μην αδειοδοτηθεί καθόλου.

Οι ακριβείς όροι και προϋποθέσεις για αντιγραφή, διανομή και τροποποίηση ακολουθούν. ΌΡΟΙ ΚΑΙ ΠΡΟΫΠΟΘΕΣΕΙΣ ΓΙΑ ΑΝΤΙΓΡΑΦΗ, ΔΙΑΝΟΜΗ ΚΑΙ ΤΡΟΠΟΙΗΣΗ

0. Αυτή η Άδεια ισχύει για οποιοδήποτε πρόγραμμα ή άλλη δουλειά η οποία εμπεριέχει μια σημείωση τοποθετημένη απο τον  ιδιοκτήτη της πνευματικής ιδιοκτησίας λέγοντας ότι μπορεί να διανεμηθεί υπο τους όρους αυτής της Γενικής Άδειας Χρήσης. Το “Πρόγραμμα”, παρακάτω, αναφέρεται σε οποιοδήποτε τέτοιο πρόγραμμα ή εργασία, και μια “εργασία που βασίζεται στο Πρόγραμμα” σημαίνει είτε το πρόγραμμα είτε οποιαδήποτε παράγωγη εργασία υπο νόμους πνευματικής ιδιοκτησίας: αυτό σημαίνει, μια εργασία που εμπεριέχει το πρόγραμμα ή ένα μέρος του, είτε αυτολεξεί είτε με μετατροπές και/η μεταφρασμένο σε άλλη γλώσσα. (Εδώ, η μετάφραση συμπεριλαμβάνεται χωρίς περιορισμούς στον όρο “μετατροπή”). Κάθε άδεια αναφέρεται ως “εσείς”.

Δραστηριότητες πέρα απο την αντιγραφή, τη διανομή και τη μετατροπή δεν καλύπτονται απο αυτή την Άδεια; είναι πέρα απο το σκοπό της. Η ενέργεια της εκτέλεσης του Προγράμματος δεν απαγορεύεται και το αποτέλεσμα του Προγράμματος καλύπτεται μόνο αν τα περιεχόμενά του αποτελούν δουλειά βασισμένη στο Πρόγραμμα (ανεξάρτητα απο το αν έχουν δημιουργηθεί απο την εκτέλεση του Προγράμματος). Το αν αυτό είναι αλήθεια εξαρτάται απο το τι κάνει το Πρόγραμμα.

1. Μπορείτε να αντιγράψετε και να διανέμετε αυτολεξεί αντίγραφα του πηγαίου κώδικα του Προγράμματος καθώς τον λαμβάνετε, με οποιοδήποτε μέσο, δεδομένου ότι ευδιάκριτα και κανονικά εκδίδετε σε κάθε αντίγραφο μια κανονική σημείωση πνευματικής ιδιοκτησίας και αποποίησης εγγύησης; κρατήστε ανέπαφες όλες τις σημειώσεις που αναφέρονται σε αυτή την Άδεια και στην απουσία κάθε εγγύησης; και δώστε σε όλους τους άλλους παραλαμβάνοντες του Προγράμματος ένα αντίγραφο αυτής της Άδειας μαζί με το Πρόγραμμα.

  Μπορεί να χρεωθείτε κάποιο ποσό ή τη φυσική πράξη της μεταβίβασης ενός αντιγράφου, και μπορείτε απο επιλογή σας να προσφέρετε εγγύηση προστασίας σε ανταλαγή για το χρηματικό ποσό.

2. Μπορεί να τροποποιήσετε το αντίγραφο ή τα αντίγραφα του Προγράμματος σας ή οποιοδήποτε μέρος του σχηματίζοντας έτσι μια δουλειά βασισμένη στο πρόγραμμα, και να αντιγράψετε και να διανέιμετε τέτοιες μετατροπές ή να δουλέψετε υπο τους όρους του Κεφαλαίου 1 παραπάνω, δεδομένου ότι ικανοποιούνται όλοι οι παρακάτω όροι:

    α) Πρέπει να κάνετε τα αλλαγμένα αρχεία να να φέρουν εμφανείς σημειώσεις που να υποδηλώνουν οτι αλλάξατε τα αρχεία καθώς και την ημερομηνία της κάθε αλλαγής. 

    β) Πρέπει να κάνετε οποιαδήποτε δουλειά διανέμετε ή εκδίδετε, ό,τι ολοκληρωτικά ή ό,τι κατα ένα μέρος εμπεριέχει ή προέρχεται απο το Πρόγραμμα ή οποιοδήποτε μέρος, να αδειοδοτηθεί ως ενιαίο χωρίς χρέωση σε τρίτους υπο τους όρους της Άδειας. 

    γ) Αν το τροποποιημένο πρόγραμμα διαβάζει κανονικά εντολές διαδραστικά όταν τρέχει, πρέπει να το προκαλέσετε, όταν αρχίσει να τρέχει για τόσο διαδραστική χρήση στον πιο συνηθισμένο τρόπο, να τυπώνει ή να εμφανίζει μία ανακοίνωση συμπεριλαμβάνοντας μια κανονική σημείωση πνευματικής ιδιοκτησίας και μια σημείωση ότι δεν υπάρχει εγγύηση (ή αλλιώς λέγοντας ότι θα παρέχετε εγγύηση) και ότι οι χρήστες μπορούν να αναδιανέμουν το πρόγραμμα υπο αυτούς τους όρους, και λέγοντας στο χρήστη πως να διαβάσει ένα αντίγραφο της Άδειας. (Εξαίρεση: αν το πρόγραμμα απο μόνο του είναι διαδραστικό αλλα δεν εμφανίζει κανονικά μία ανακοίνωση, δουλέψτε βασιζόμενοι στο ότι το Πρόγραμμα δεν χρειάζεται να εμφανίσει κάποια ανακοίνωση). 

Αυτές οι απαιτήσεις ισχύουν για την τροποποιημένη δουλειά ως σύνολο. Αν αναγνωρίσιμα κεφάλαια αυτής της δουλειάς δεν παράγονται απο το πρόγραμμα και μπορούν λογικά να θεωρηθούν ανεξάρτητες και ξεχωριστές δουλειές απο μόνες τους, τότε αυτή η άδεια, και οι όροι της, δεν ισχύουν σε εκείνα τα κεφάλαια όταν τα διανέμεται ως ξεχωριστές δουλειές. Αλλά όταν διανέμετε τα ίδια κεφάλαια ως μέρος κάτι μεγαλυτέρου το οποίο είναι μια δουλειά βασισμένη στο Πρόγραμμα, η διανομή του μεγαλυτέρου πρέπει να είναι υπο τους όρους αυτής της Άδειας, της οποίας το ελεύθερο για άλλες άδειες επεκτείνεται στο ολόκληρο, οπότε και στο κάθε κομμάτι ξεχωριστά ανεξάρτητα απο το ποιός το έγραψε.

Συμπερασματικά, δεν είναι η πρόθεση αυτού του κεφαλαίο να διεκδικήσει δικαιώματα ή να συναγωνιστεί τα δικαιώματα σας να δουλέψετε ότι είναι γραμμένο εντελώς απο εσάς; η πρόθεση είναι να ασκήσετε το διακαίωμα να ελέγξετε τη διανομή παραγώγων ή συλλεκτικών εργασιών βασισμένες στο Πρόγραμμα.

Επιπρόσθετα, αμιγής συνάθροιση μιας άλλης εργασίας που δεν βασίζεται στο Πρόγραμμα με το Πρόγραμμα (ή με μια εργασία που βασίζεται στο Πρόγραμμα) σε έναν όγκο αποθήκευσης ή μέσο διανομής δεν φέρνει την άλλη εργασία υπο τα πλαίσια αυτής της Άδειας.

3. Μπορείτε να αντιγράψετε και να διανείμετε το Πρόγραμμα (ή μια εργασία βασισμένη σε αυτή, υπό το Κεφάλαιο 2) σε κώδικα αντικειμένου ή εκτελέσιμη μορφή υπό τους όρους του Κεφαλαίου 1 και 2 παραπάνω δεδομένου ότι θα κάνετε ένα απο τα παρακάτω:

    α) να το συνοδέψετε με τον ολόκληρο αντίστοιχο πηγαίο κώδικα που είναι αναγνώσιμο απο το μηχάνημα, ο οποίος πρέπει να έχει διανεμηθεί υπό του όρους του Κεφαλαίου 1 και 2 παραπάνω σε ένα μέσο που χρησιμοποιείται συνήθως για ανταλλαγή λογισμικού, ή, 

    β)  Να το συνοδέψετε με μια γραπτή εισήγηση, που θα ισχύει για τουλάχιστον τρία χρόνια, να δίνει σε οποιονδήποτε τρίτο, για μια χρέωση όχι περισσότερη απο το κόστος σας του να διανέμετε φυσικά τον πηγαίο κώδικα, ένα ολοκληρωμένο αναγνώσιμο-απο-μηχάνημα αντίγραφο του αντίστοιχου πηγαίου κώδικα, να διανεμηθεί υπό τους όρους του Κεφαλαίου 1 και 2 παραπάνω σε ένα μέσο που συνήθως χρησιμοποιείται για ανταλλαγή λογισμικού, ή, 

    γ)  Συνοδέψτε το με την πληροφορία που λάβατε ως προς την εισήγηση να διανείμετε τον αντίστοιχο πηγαίο κώδικα (αυτή η εναλακτική επιτρέπεται μόνο για μη εμπορική διανομή και μόνο αν έχετε λάβει το πρόγραμμα σε κώδικα αντικειμένου ή εκτελέσιμη μορφή με μια τέτοια προσφορά, σε συμφωνία με το Υποκεφάλαιο β παραπάνω). 

Ο πηγαίος κώδικας για μια εργασία σημαίνει την προτιμόμενη μορφή της εργασίας για να γίνονται μετατροπές. Για μια εκτελέσιμη εργασία, ο ολοκληρωμένος πηγαίος κώδικας σημαίνει όλο τον πηγαίο κώδικα για κάθε αυτοτελή λογισμική μονάδα που περιέχει, συν οποιαδήποτε σχετιζόμενα αρχεία περιβάλλοντος, συν τον κώδικα που χρησιμποιείται για τον έλεγχο της σύνταξης και της εγκατάστασης του εκτελέσιμου. Παρ'όλα αυτά, ως μια ειδική εξαίρεση, ο διανεμημένος πηγαίος κώδικας δεν χρειάζεται να συμπαριλάβει οτιδήποτε είναι κανονικά διανεμημένο (είτε σε πηγαία είτε σε δυαδική μορφή) με τα κύρια στοιχεία (συντάκτης πυρήνας κλπ) του λειτουργικού συστήματος πάνω στο οποίο το εκτελέσιμο τρέχει, εκτός αν το στοιχείο αυτό απο μόνο του συνοδεύει το εκτελέσιμο.

Αν η διανομή του εκτελέσιμου ή του κώδικα αντικειμένου γίνεται δίνοντας πρόσβαση για αντιγραφή απο ένα αναδειγμένο μέρος, μετά η προσφορά ισοδύναμης πρόσβασης για αντιγραφή του πηγαίου κώδικα απο το ίδιο μέρος μετράει ως διανομή του πηγαίου κώδικα, παρ'όλο που τρίτα άτομα δεν υποχρεώνονται να αντιγράψουν τον πηγαίο μαζί με τον κώδικα αντικειμένου.

4. Δεν μπορείτε να αντιγράψετε, να μετατρέψετε, να δημιουργήσετε υπο-άδειες, ή να διανέμετε το Πρόγραμμα παρά μόνο όπως ρητά παρέχεται υπο την συγκεκριμένη άδεια. Οποιαδήποτε προσπάθεια αντιθέτως να αντιγράψετε, να τροποποιήσετε, να δημιουργήσετε υπο-άδειες ή να διανείμεντε το Πρόγραμμα είναι άκυρη, και αυτόματα θα τερματίσει τα δικαιώματα σας υπο αυτή την Άδεια. Παρ'όλα αυτά, άτομα που έχουν λάβει αντίγραφα, ή δικαιώματα απο εσάς υπο τη συγκεκριμένη Άδεια δεν θα τους τερματιστεί η άδεια καθώς τέτοια άτομα παραμένουν σε πλήρη συμμόρφωση.

5. Δεν σας απαιτείται να δεχτείτε την Άδεια, καθώς δεν την έχετε υπογράψει. Παρ'όλα αυτά τίποτα άλλο δεν σας δίνει την άδεια να τροποποιήσετε ή να διανείμεντε το Πρόγραμμα ή τις παράγωγες εργασίες του.  Αυτές οι ενέργειες απαγορεύονται απο το νόμο αν δεν δέχεστε αυτή την Άδεια. Επομένως, τροποποιώντας ή διανέμοντας το Πρόγραμμα (ή οποιαδήποτε δουλειά βασίζεται στο Πρόγραμμα), δείχνετε την αποδοχή σας σε αυτή την άδεια, και όλους τους όρους της και τις προϋποθέσεις για αντιγραφή, διανομή ή τροποποίηση του Προγράμματος ή εργασιών που βασίζονται σε αυτό.

6. Κάθε φορά που αναδιανέμετε το Πρόγραμμα ( ή οποιαδήποτε εργασία βασίζεται στο Πρόγραμμα), ο αποδέκτης αυτόματα λαμβάνει μια άδεια απο τον αυθεντικό δημιουργό της άδεια για αντιγραφή, διανομή ή τροποποίηση του Προγράμματος υπο τους όρους και προϋποθέσεις. Δεν μπορείτε να επιβάλετε περισσότερους περιορισμούς στην εξάσκηση διακαιωμάτων που παρέχονται. Δεν είστε υπεύθυνοι για την επιβολή συμμόρφωσης απο τρίτους σε αυτή την Άδεια.

7. Αν, ως αποτέλεσμα απόφασης ενός δικαστηρίου ή ισχυρισμό για παράβαση πατέντας ή για οποιοδήποτε άλλο λόγο (που δεν περιορίζεται σε θέματα πατέντας), προϋποθέσεις σας επιβάλονται (είτε απο απόφαση δικαστηρίου, συμφωνία ή κάτι άλλο) που αντιτίθενται στις προϋποθέσεις αυτής της Άδειας, δεν σας δίνεται άφεση απο τις προϋποθέσεις αυτή της άδειας. Αν δεν μπορείτε να διανείμετε έτσι ώστε να ικανοποιείτε ταυτόχρονα τις υποχρεώσεις σας υπο αυτή την Άδεια και οποιαδήποτε άλλη συναφή υποχρέωση, τότε ως αποτέλεσμα δεν μπορείτε να διανέμετε το Πρόγραμμα καθόλου. Για παράδειγμα, αν μια άδεια πατέντας δεν επέτρεπε δωρεάν αναδιανομή του Προγράμματος απο όλους εκείνουν που λαμβάνουν αντίγραφα απ'ευθείας ή έμμεσα μέσω εσας, τότε ο μόνος τρόπος που θα μπορούσατε να ικανοποιήσετε και αυτό και την Άδεια θα ήταν να απέχετε εντελώς απο τη διανομή του Προγράμματος.

 Αν οποιοδήποτε μέσος αυτού του κεφαλαίου είναι λανθασμένο ή μη εκτελέσιμο υπό οποιαδήποτε συγκυρία, η ισορροπία αυτού του κεφαλαίου προορίζεται να ισχύει και το κεφάλαιο ως πληρότητα προορίζεται να ισχύει σε άλλες καταστάσεις.

Δεν είναι σκοπός αυτού του κεφαλαίου να σας πείσει να παραβιάσετε οποιαδήποτε πατέντα ή διεκδικήσεις δικαιωμάτων ιδιοκτησίας ή να αμφισβητήσετε την εγκυρότητα απο οποιαδήποτε τέτοια διεκδίκηση; αυτό το κεφάλαιο έχει ως μοναδικό ρόλο την προστασία της ακεραιότητας του συστήματος διανομής του ελεύθερου λογισμικού, που εφαρμόζεται απο πράξεις δημόσιας άδειας. Πολλοί άνθρωποι έχουν κάνει γενναιόδωρες συνεισφορές στο ευρύ φάσμα των λογισμικών που διανέμονται μέσω αυτού του συστήματος σε εξάρτηση συμφώνων εφαρμογών πάνω σε αυτό το σύστημα; εξαρτάται απο το συγγραφέα/δωρητή να αποφασίσει αν προτίθεται να διανέμει λογισμικό δια μέσου οποιουδήποτε άλλου συστήματος και μια άδεια δεν μπορεί να επιβάλει αυτή την επιλογή.

Αυτό το κεφάλαιο πρόκειται να κάνει ολοκληρωτικά ξεκάθαρο του τι πιστεύεται μια συνέπεια του υπολοίπου αυτής της άδειας.

8. Αν η διανομή και/ή η χρήση αυτού του Προγράμματος περιορίζεται σε συγκεκριμένες χώρες είτε απο πατέντες ή περιβάλλοντα με πνευματική ιδιοκτησία, ο αρχικός κάτοχος της πνευματικής ιδιοκτησίας που τοποθετεί το Πρόγραμμα υπο αυτή την άδεια μπορεί να προσθέσει ένα κατηγορηματικό περιορισμό γεωγραφικής διανομής αποκλείοντας αυτές τις χώρες, έτσι ώστε η διανομή να επιτρέπεται μόνο μέσα ή ανάμεσα στις χώρες έτσι ώστε να μην αποκλείεται. Σε μία τέτοια περίπτωση, αυτή η Άδεια ενσωματώνει τον περιορισμό όπως αν ήταν γραμμένος στο κυρίως τμήμα αυτής της Άδεια.

9. Το Ίδρυμα Ελεύθερου Λογισμικού μπορεί να εκδόσει ανανεωμένες και/ή καινούργιες εκδόσεις της Γενικής Άδειας Χρήσης (GPL) ανα διάφορα χρονικά διαστήματα. Τέτοιες καινούργιες εκδόσεις θα είναι παρεμφερείς στην τωρινή έκδοση, αλλά μπορεί να διαφέρουν σε λεπτομέρεια να αναδείξουν καινούργια προβλήματα ή ανησυχίες.

Σε κάθε έκδοση δίνεται ένας χαρακτηριστικός αριθμός έκδοσης. Αν το Πρόγραμμα καθορίζει έναν αριθμό έκδοσης αυτής της Άδειας που ισχύει για αυτή και “για οποιαδήποτε μεταγενέστερη έκδοση” μπορείτε να επιλέξετε να ακολουθήσετε τους όρους και τις προϋποθεσεις είτε αυτής της έκδοσης είτε απο οποιαδήποτε μεταγενέστερη έκδοση που εκδίδεται απο το Ιδρυμα Ελεύθερου Λογισμικού. Αν το Πρόγραμμα δεν καθορίζει έναν αριθμό έκδοσης αυτής της Άδειας, μπορείτε να διαλέξετε οποιαδήποτε άδεια έχει εκδοθεί απο το Ίδρυμα Ελεύθερου Λογισμικού.

10. Αν επιθυμείτε να ενσωματώσετε μέρη αυτού του Προγράμματος μέσα σε άλλα ελεύθερα προγράμματα των οποίων οι προϋποθέσεις διανομής είναι διαφορετικές γράψτε στο συγγραφέα και ζητήστε άδεια. Για λογισμικό το οποίο έχει πνευματική ιδιοκτησία απο το 'Ιδρυμα Ελεύθερου Λογισμικού, γράψτε στο Ίδρυμα το ίδιο; μερικές φορές κάνουμε εξαιρέσεις για αυτό. Η απόφασή μας θα οδηγηθεί απο τους δύο στοχους της διατήρησης της ελεύθερης κατάστασης όλων των παραγομένων του ελεύθερου λογισμικού μας και της προώθησης και του μοιράσματος και επαναχρησιμοποίησης του λογισμικού γενικά.

ΔΕΝ ΥΠΑΡΧΕΙ ΕΓΓΥΗΣΗ

11. ΕΠΕΙΔΗ ΤΟ ΠΡΟΓΡΑΜΜΑ ΕΧΕΙ ΑΔΕΙΟΔΟΤΗΘΕΙ ΜΕ ΜΗΔΕΝΙΚΟ ΚΟΣΤΟΣ, ΔΕΝ ΥΠΑΡΧΕΙ ΕΓΓΥΗΣΗ ΓΙΑ ΤΟ ΠΡΟΓΡΑΜΜΑ, ΣΤΗΝ ΕΚΤΑΣΗ ΠΟΥ ΕΠΙΤΡΕΠΕΤΑΙ ΑΠΟ ΤΗΝ ΕΦΑΡΜΟΓΗ ΤΟΥ ΝΟΜΟΥ. ΕΚΤΟΣ ΑΝ ΕΧΕΙ ΔΗΛΩΘΕΙ ΔΙΑΦΟΡΕΤΙΚΑ ΣΤΗ ΓΡΑΦΗ ΤΩΝ ΚΑΤΟΧΩΝ ΤΩΝ ΠΝΕΥΜΑΤΙΚΩΝ ΔΙΚΑΙΩΜΑΤΩΝ ΚΑΙ/'Η ΑΛΛΑ ΑΤΟΜΑ ΠΑΡΕΧΟΥΝ ΤΟ ΠΡΟΓΡΑΜΜΑ “ΟΠΩΣ ΕΙΝΑΙ” ΧΩΡΙΣ ΕΓΓΥΗΣΗ ΚΑΠΟΙΟΥ ΕΙΔΟΥΣ, ΕΙΤΕ ΓΡΑΜΜΕΝΟ ΕΙΤΕ ΥΠΟΔΗΛΩΜΕΝΟ,  ΣΥΜΠΕΡΙΛΑΜΒΑΝΟΝΤΑΣ, ΑΛΛΑ ΟΧΙ ΠΕΡΙΟΡΙΣΜΕΝΟ ΣΤΙΣ ΥΠΟΔΗΛΩΜΕΝΕΣ ΕΓΓΥΗΣΕΙΣ ΑΓΟΡΑΣΙΜΌΤΗΤΑΣ ΚΑΙ ΚΑΤΑΛΗΛΟΤΗΤΑΣ ΓΙΑ ΕΝΑ ΣΥΓΚΕΚΡΙΜΕΝΟ ΣΚΟΠΟ. ΟΛΟ ΤΟ ΡΙΣΚΟ ΟΠΩΣ ΚΑΙ Η ΠΟΙΟΤΗΤΑ ΚΑΙ Η ΑΠΟΔΟΣΗ ΤΟΥ ΠΡΟΓΡΑΜΜΑΤΟΣ ΕΙΝΑΙ ΠΑΝΩ ΣΑΣ. ΑΝ ΤΟ ΠΡΟΓΡΑΜΜΑ ΑΝΑΔΕΙΧΘΕΙ ΕΛΑΤΩΜΑΤΙΚΟ, ΑΝΑΛΑΒΕΤΕ ΤΟ ΚΟΣΤΟΣ ΟΛΩΝ ΤΩΝ ΑΠΑΡΑΙΤΗΤΩΝ ΕΠΙΔΙΟΡΘΩΣΕΩΝ Ή ΡΥΘΜΙΣΕΩΝ.

12. ΣΕ ΚΑΜΙΑ ΠΕΡΙΠΤΩΣΗ ΕΚΤΟΣ ΑΝ ΖΗΤΗΘΕΙ ΑΠΟ ΕΦΑΡΜΟΣΙΜΟ ΝΟΜΟ Ή ΣΥΜΦΩΝΗΘΕΙ ΜΕ ΕΓΓΡΑΦΗ ΘΕΛΗΣΗ ΑΠΟ ΚΑΘΕ ΚΑΤΟΧΟ ΠΝΕΥΜΑΤΙΚΩΝ ΔΙΚΑΙΩΜΑΤΩΝ, Ή ΑΠΟ ΟΠΟΙΟΝΔΗΠΟΤΕ ΑΛΛΟΝ ΜΠΟΡΕΙ ΝΑ ΤΡΟΠΟΠΟΙΗΣΕΙ ΚΑΙ/Ή ΝΑ ΑΝΑΔΙΑΝΕΜΕΙ ΤΟ ΠΡΟΓΡΑΜΜΑ ΟΠΩΣ ΑΔΕΙΟΔΟΤΕΙΤΑΙ ΠΑΡΑΠΑΝΩ, ΕΙΝΑΙ ΥΕΠΕΥΘΥΝΟΣ ΣΕ ΣΑΣ ΓΙΑ ΖΗΜΙΕΣ, ΣΥΜΠΕΡΙΛΑΜΒΑΝΟΝΤΑΣ ΟΠΟΙΕΣΔΗΠΟΤΕ ΓΕΝΙΚΕΣ, ΕΙΔΙΚΕΣ, ΣΥΜΠΤΩΜΑΤΙΚΕΣ Ή ΑΠΟΘΕΤΙΚΕΣ ΖΗΜΙΕΣ ΠΡΟΕΡΧΟΜΕΝΕΣ ΑΠΟ ΤΗ ΧΡΗΣΗ Ή ΤΗΝ ΑΝΙΚΑΝΟΤΗΤΑ ΝΑ ΧΡΗΣΙΜΟΠΟΙΗΣΕΤΕ ΤΟ ΠΡΟΓΡΑΜΜΑ (ΣΥΜΠΕΡΙΛΑΜΒΑΝΟΝΤΑΣ ΑΛΛΑ ΧΩΡΙΣ ΝΑ ΥΠΑΡΧΕΙ ΠΕΡΙΟΡΙΖΕΤΑΙ ΑΠΛΑ ΣΤΗΝ ΑΠΩΛΕΙΑ ΔΕΔΟΜΕΝΩΝ Ή ΣΤΗΝ ΑΝΑΚΡΙΒΗ ΑΝΤΑΠΟΔΟΣΗ ΔΕΔΟΜΕΝΩΝ Ή ΣΤΙΣ ΑΠΩΛΕΙΕΣ ΠΟΥ ΠΡΟΚΑΛΟΥΝΤΑΙ ΑΠΟ ΕΣΑΣ Ή ΤΡΙΤΟΥΣ Ή ΣΕ ΜΙΑ ΑΝΕΠΑΡΚΕΙΑ ΤΟΥ ΠΡΟΓΡΑΜΜΑΤΟΣ ΝΑ ΔΟΥΛΕΨΕΙ ΠΑΡΑΛΛΗΛΑ ΜΕ ΑΛΛΑ ΠΡΟΓΡΑΜΜΑΤΑ), ΑΚΟΜΗ ΚΑΙ ΑΝ ΚΑΠΟΙΟΣ ΤΕΤΟΙΟΣ ΧΡΗΣΤΗΣ Ή ΤΡΙΤΟΣ ΕΧΕΙ ΕΝΗΜΕΡΩΘΕΙ ΓΙΑ ΤΗΝ ΠΙΘΑΝΟΤΗΤΑ ΤΕΤΟΙΩΝ ΖΗΜΙΩΝ. 
\end{small}
