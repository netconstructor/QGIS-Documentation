% vim: set textwidth=78 autoindent:

\section{Vorwort}\label{label_forward}
\pagenumbering{arabic}
\setcounter{page}{1}

Willkommen in der wunderbaren Welt der Geographischen Informationssysteme (GIS)!
Quantum GIS ist ein Freies (Open Source) GIS. Die Idee zu dem Projekt wurde im
Mai 2002 geboren und bereits im Juni desselben Jahres bei SourceForge etabliert.  
Wir haben hart daran gearbeitet, traditionell sehr teure GIS Software kostenfrei 
f�r jeden, der Zugang zu einem PC hat, bereitzustellen. 
QGIS kann unter den meisten Unices, Windows und MacOSX betrieben werden, QGIS
wurde mit Hilfe des Qt toolkit (\url{http://www.trolltech.com}) und C++
entwickelt. Dadurch ist QGIS sehr benutzerfreundlich und besitzt eine einfach zu
bedienende und intuitive grafische Benutzeroberfl�che.

QGIS soll ein einfach zu benutzendes GIS sein und grundlegende
GIS-Funktionalit�ten bieten. Das anf�ngliche Ziel bestand darin, einen 
einfachen Geo-Datenviewer zu entwickeln. Dieses Ziel wurde bereits mehr als 
erreicht, so dass QGIS mittlerweile von vielen Anwendern f�r ihre t�gliche 
Arbeit eingesetzt wird. 
QGIS unterst�tzt eine Vielzahl von Raster- und Vektorformaten. Mit Hilfe der
Plugin-Architektur k�nnen weitere Funktionalit�ten einfach erg�nzt werden (vgl. 
Appendix \ref{appdx_data_formats} f�r eine vollst�ndige Liste derzeit
unterst�tzter Datenformate).

QGIS wird unter der GNU Public License (GPL) herausgegeben. F�r die Entwicklung
des Programms bedeutet dies das Recht, den Quellcode einzusehen und
entsprechend der Lizens ver�ndern zu d�rfen. F�r die Anwendung der Software ist
damit garantiert, dass QGIS kostenfrei aus dem Internet heruntergeladen, genutzt
und weitergegeben werden kann. Eine vollst�ndige Kopie der Lizens ist dem
Programm beigef�gt und kann auch im Appendix \ref{gpl_appendix} eingesehen
werden.   

\begin{quote}
\begin{center}
\textbf{Bemerkung:} Die aktuellste englische Version dieser Dokumentation finden 
Sie hier: 
\newline http://qgis.org/docs/userguide.pdf 
\end{center}
\end{quote}

\subsection{Hauptfunktionalit�ten}\label{label_majfeat}

Quantum GIS bietet zahlreiche GIS Funktionalit�ten. Die wichtigsten sind hier
aufgelistet. 

\begin{enumerate}
\item Support f�r die r�umliche Datenbank PostgreSQL/PostGIS 
\item Support f�r Shapes und andere Vektorformate, die von OGR unterst�tzt werden
\item Integration von GRASS GIS zur Visualisierung, Editierung und Analyse
\item On-the-fly (OTF) Projektion von Vektorebenen
\item Kartenerstellung
\item Objektabfrage 
\item Anzeige von Attributtabellen 
\item Objekte selektieren 
\item Objekte labeln
\item Persistente Selektion 
\item Projekte speichern und wieder �ffnen
\item Support f�r Rasterformate, die von der GDAL-Bibliothek unterst�tzt werden
\item Vektorbeschriftung �ndern  
\item SVG Symbole darstellen 
\item Darstellen von Rasterdaten wie Digitalen Gel�ndemodellen oder Luft- und 
Satellitenbilddaten 
\item �ndern der Raster Symbologie (Graustufen, Pseudofarben und Mehrkanal RGB) 
\item Export als Mapserver Mapfile 
\item Digitalisierung
\item Karten�bersicht
\item Plugins 
\end{enumerate}


\subsection{Was ist neu in \CURRENT}\label{label_whatsnew}
Die aktuelle QGIS Version \CURRENT bringt einige sehr interessante neue
Funktionalit�ten mit.
%% FIXME: the following list must be varified, not sure about everything: SH(16.05.2006)
\begin{itemize}
\item WMS Support
\item Verbessertes Editieren von Vektorgeometrien und -attributen
\item Verbessertes Messwerkzeug auch f�r Fl�chengr��e
\item Attributsuche 
\item Neue Struktur der Legende
\item �berarbeitung der API, um die QGIS-Bibliothek f�r Mapping-Applikationen nutzen
zu k�nnen
\item Verbessertes MapServer Exporttool
\item Vektor Transparenz und Antialiasing
\item GRASS-Support f�r alle Plattformen
\item Erweiterter GRASS-Support und zus�tzliche Toolbox Module
\item Erweitertes Vektoreditieren (copy, cut, paste, snapping und Knotenpunkt editieren)
\item Editieren von Shapefile/OGR-Ebenen
\end{itemize}

%%% This is old version 0.7.4
%\subsection{What was new in \OLD}\label{label_whatsnewinold}
%Version \OLD brings several important features, including projection
%support, a map composer, and better integration with GRASS. The major new
%features in this release include:

%\begin{enumerate}
%  \item On the fly projection for reprojecting layers in different coordinate systems
%  \item Map Composer for creating print layouts
%  \item Toolbox for running GRASS tools from QGIS
%  \item Raster graphing tool to produce a histogram for a raster layer.
%  \item Raster query using the identify tool to get the pixel values from a raster 
%  \item New customizable settings for the digitizing line width, color, and selection color  
%  \item New symbols for use with point layers are available from the layer properties dialog 
%  \item Spatial bookmarks allow you to create and manage bookmarks for an area on the map
%  \item Measure tool to measure distances on the map with both
%  segment length and total length displayed as you click
%  \item GPX loading times and memory consumption for large GPX (GPS) files
%  has been drastically reduced.  
%  \item Digitizing enhancements, including the ability to capture data straight
%  into PostgreSQL/PostGIS, and improvements to the definition of attribute tables
%  for newly created layers.
%  \item Raster Georeferencing plugin can be used
%  to generate a world file for a raster by defining known
%  control points in the raster coordinate system.
%\end{enumerate}

